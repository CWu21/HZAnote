%% Int note: https://cds.cern.ch/record/2654594

%-------------------------------------------------------------------------------
% This file provides a skeleton ATLAS note.
% \pdfinclusioncopyfonts=1
% This command may be needed in order to get \ell in PDF plots to appear. Found in
% https://tex.stackexchange.com/questions/322010/pdflatex-glyph-undefined-symbols-disappear-from-included-pdf
%-------------------------------------------------------------------------------
% Specify where ATLAS LaTeX style files can be found.
\newcommand*{\ATLASLATEXPATH}{latex/}
% Use this variant if the files are in a central location, e.g. $HOME/texmf.
% \newcommand*{\ATLASLATEXPATH}{}
%-------------------------------------------------------------------------------
\documentclass[NOTE, atlasdraft=true, texlive=2017, UKenglish]{\ATLASLATEXPATH atlasdoc}
% The language of the document must be set: usually UKenglish or USenglish.
% british and american also work!
% Commonly used options:
%  atlasdraft=true|false This document is an ATLAS draft.
%  texlive=YYYY          Specify TeX Live version (2016 is default).
%  coverpage             Create ATLAS draft cover page for collaboration circulation.
%                        See atlas-draft-cover.tex for a list of variables that should be defined.
%  cernpreprint          Create front page for a CERN preprint.
%                        See atlas-preprint-cover.tex for a list of variables that should be defined.
%  NOTE                  The document is an ATLAS note (draft).
%  PAPER                 The document is an ATLAS paper (draft).
%  CONF                  The document is a CONF note (draft).
%  PUB                   The document is a PUB note (draft).
%  BOOK                  The document is of book form, like an LOI or TDR (draft)
%  txfonts=true|false    Use txfonts rather than the default newtx
%  paper=a4|letter       Set paper size to A4 (default) or letter.

%-------------------------------------------------------------------------------
% Extra packages:
\usepackage[subfigure=true, biblatex=true, backend=bibtex]{\ATLASLATEXPATH atlaspackage}
% Commonly used options:
%  biblatex=true|false   Use biblatex (default) or bibtex for the bibliography.
%  backend=bibtex        Use the bibtex backend rather than biber.
%  subfigure|subfig|subcaption  to use one of these packages for figures in figures.
%  minimal               Minimal set of packages.
%  default               Standard set of packages.
%  full                  Full set of packages.
%-------------------------------------------------------------------------------
% Style file with biblatex options for ATLAS documents.
\usepackage{\ATLASLATEXPATH atlasbiblatex}

% Package for creating list of authors and contributors to the analysis.
\usepackage{\ATLASLATEXPATH atlascontribute}

% Useful macros
\usepackage{\ATLASLATEXPATH atlasphysics}
% See doc/atlas_physics.pdf for a list of the defined symbols.
% Default options are:
%   true:  journal, misc, particle, unit, xref
%   false: BSM, heppparticle, hepprocess, hion, jetetmiss, math, process,
%          other, snippets, texmf
% See the package for details on the options.

% Files with references for use with biblatex.
% Note that biber gives an error if it finds empty bib files.
\addbibresource{HZXAnalysisSupportNote.bib}
\addbibresource{bib/ATLAS.bib}
\addbibresource{bib/CMS.bib}
\addbibresource{bib/ConfNotes.bib}
\addbibresource{bib/PubNotes.bib}

% Paths for figures - do not forget the / at the end of the directory name.
\graphicspath{{logos/}{figures/}}

% Add you own definitions here (file HZXAnalysisSupportNote-defs.sty).
\usepackage{HZXAnalysisSupportNote-defs}

\usepackage{multirow}

\usepackage{multicol}

\usepackage{bm}

%% \usepackage{array}
%% \newcolumntype{L}[1]{>{\raggedright\let\newline\\\arraybackslash\hspace{0pt}}m{#1}}
%% \newcolumntype{C}[1]{>{\centering\let\newline\\\arraybackslash\hspace{0pt}}m{#1}}
%% \newcolumntype{R}[1]{>{\raggedleft\let\newline\\\arraybackslash\hspace{0pt}}m{#1}}

\newcolumntype{L}[1]{>{\raggedright\arraybackslash}p{#1}}
\newcolumntype{C}[1]{>{\centering\arraybackslash}p{#1}}
\newcolumntype{R}[1]{>{\raggedleft\arraybackslash}p{#1}}

%-------------------------------------------------------------------------------
% Generic document information
%-------------------------------------------------------------------------------

% Title, abstract and document
%-------------------------------------------------------------------------------
% This file contains the title, author and abstract.
% It also contains all relevant document numbers used for an ATLAS note.
%-------------------------------------------------------------------------------

% URL: https://cds.cern.ch/record/2654594

% Title
\AtlasTitle{Search for Decays of the Higgs Boson into a $Z$ Boson and a Light Hadronically Decaying Resonance}

% Draft version:
% Should be 1.0 for the first circulation, and 2.0 for the second circulation.
% If given, adds draft version on front page, a 'DRAFT' box on top of each other page, 
% and line numbers.
% Comment or remove in final version.
\AtlasVersion{0.18}

% Abstract - % directly after { is important for correct indentation
\AtlasAbstract{%
  A search for decays of the Higgs boson to a $Z$ boson and a light resonance in two lepton plus jet events, using the full Run 2 (2015-2018) dataset of 139~fb$^{-1}$ is presented. The target resonance is an $a^0$ from the Two Higgs Doublet Model, or a $J/\psi$ or $\eta_c$, which decays hadronically. Due to its low mass and high boost, the resonance is reconstructed as a single jet of hadrons, while the $Z$ boson is reconstructed as a dilepton system. A multivariate event selection based on a Multi-Layer Perceptron is followed by a single-bin profile likelihood fit. In the absence of a signal, 95\% $CL_s$ upper limits on $\sigma(H) \text{BR}(H\to Za^0) / \sigma_\text{SM}(H)$ are set, with values starting from $16.8~\text{pb}$ for the signal hypothesis of a 0.5~GeV $a^0$ decaying to gluons, and $105~\text{pb}$ and $107~\text{pb}$ for the $J/\psi$ and $\eta_c$, respectively.
}

% Author - this does not work with revtex (add it after \begin{document})
%% \author{The ATLAS Collaboration}

% Authors and list of contributors to the analysis
% \AtlasAuthorContributor also adds the name to the author list
% Include package latex/atlascontribute to use this
% Use authblk package if there are multiple authors, which is included by latex/atlascontribute
% \usepackage{authblk}
% Use the following 3 lines to have all institutes on one line
% \makeatletter
% \renewcommand\AB@affilsepx{, \protect\Affilfont}
% \makeatother
% \renewcommand\Authands{, } % avoid ``. and'' for last author
% \renewcommand\Affilfont{\itshape\small} % affiliation formatting
\AtlasAuthorContributor{Andrew Chisholm}{a}{Analyser.}
\AtlasAuthorContributor{Konstantinos Nikolopoulos}{a}{Analyser.}
\AtlasAuthorContributor{Elliot Reynolds}{a}{Analyser.}
% \AtlasContributor{Fourth AtlasContributor}{Contribution to the analysis.}
%% \author[a]{Andrew Chisholm}
%% \author[b]{Konstantinos Nikolopoulos}
%% \author[b]{Elliot Reynolds}
\affil[a]{University of Birmingham}

% If a special author list should be indicated via a link use the following code:
% Include the two lines below if you do not use atlasstyle:
% \usepackage[marginal,hang]{footmisc}
% \setlength{\footnotemargin}{0.5em}
% Use the following lines in all cases:
% \usepackage{authblk}
% \author{The ATLAS Collaboration%
% \thanks{The full author list can be found at:\newline
%   \url{https://atlas.web.cern.ch/Atlas/PUBNOTES/ATL-PHYS-PUB-2017-007/authorlist.pdf}}
% }

% ATLAS reference code, to help ATLAS members to locate the paper
\AtlasRefCode{HDBS-2018-37}

% ATLAS note number. Can be an COM, INT, PUB or CONF note
% \AtlasNote{ATLAS-CONF-2017-XXX}
% \AtlasNote{ATL-PHYS-PUB-2017-XXX}
% \AtlasNote{ATL-COM-PHYS-2017-XXX}

% Author and title for the PDF file
\hypersetup{pdftitle={ATLAS document},pdfauthor={The ATLAS Collaboration}}

%-------------------------------------------------------------------------------
% Content
%-------------------------------------------------------------------------------
\begin{document}

\maketitle


%-------------------------------------------------------------------------------
% Changelog
%-------------------------------------------------------------------------------
\clearpage

\section*{Version 0.1 (CDS Version 7) - 17 June 19}

\textbf{Notes}
\begin{itemize}
\item First draft presented to EB at first meeting.
\item Comments from EB on CDS.
\end{itemize}

\textbf{To-do}
\begin{itemize}
\item Calculate and implement uncertainty to account for the fact that inclusive Higgs cross section is used to scale signal, while only ggF is modelled.
\item Calculate and implement alternative parton shower uncertainty (probably based on \textsc{Herwig}).
\item Calculate and implement jet uncertainties.
\item Calculate and implement PDF, factorisation and renormalisation SF uncertainties.
\item Calculate and implement tracking uncertainties.
\item Respond to comments from EB on CDS.
\end{itemize}

\section*{Version 0.2 (CDS Version 10) - 4 August 19}

\textbf{Notes}
\begin{itemize}
\item Iteration of note for second EB meeting.
\item Responded to comments on first iteration from EB on CDS.
\end{itemize}

\textbf{Main Documentation Changes}
\begin{itemize}
\item Changelog added at start of note
\item All results updated as per analysis changes, which are described below.
\item Background in plots of variables in Section~\ref{sec:selection} reweighted as per Section~\ref{sec:bkgdrw}.
\item Reweighting plots in Section~\ref{sec:bkgdrw} now shown with and without reweighting, side by side.
\item Three addition validation regions with $150<m_{\ell\ell\text{j}}<155$~GeV added to background validation tables of Section~\ref{sec:bkgdval}.
\item New plot (Figure~\ref{fig:bkgdvrs}) showing all validations regions in MC and data.
\item Numbers in Table~\ref{tab:mlpwps} now calculated using only events with $120<m_{\ell\ell\text{j}}<135$~GeV.
\item Added Section~\ref{sec:syssigdecaymodel}, describing the hadronisation uncertainty which will be added.
\item Updated Section~\ref{sec:syshiggsxsec}, describing the acceptance uncertainty which has been added.
\item Updated Section~\ref{sec:sysjet}, describing the jet uncertainties which have been added.
\item Table of correct $X$ selection efficiencies replaced with range given in main text.
\item Plots of toy limit distributions and profile likelihood scans updated to remove background MC statistical uncertainty from no-systematics plots.
\item New model-independent results section: Section~\ref{sec:modelindependentresults}.
\item Appendix on signal contamination added (now out of date).
\item Appendix~\ref{app:mlppileup} added.
\end{itemize}

\textbf{Analysis Changes}
\begin{itemize}
\item $ggZZ$ cross section corrected (it was set to the $ZZ$ cross section).
\item All cross sections updated for slight changes (previously using MC15 values). Changes all negligible.
\item Lepton SF updated to latest CP recommendations. Changes on $\sim 1\%$ level.
\item Updated luminosity to latest recommendations. $\sim 2\%$ decrease in 2018 luminosity.
\item Removed duplicated data events. Change negligible.
\item Introduced uncertainty to account for the fact that inclusive Higgs cross section is used, while only ggF is modelled.
\item Calculated, but not yet introduced, PDF, factorisation and renormalisation SF uncertainties.
\item Introduced JES uncertainty. Note the asymmetric effect on the likelihood, described in Section~\ref{sec:sysjet}.
\item Unblinded region with $150<m_{\ell\ell\text{j}}<155$~GeV as additional validation region.
\item Produced model-independent limits, as described in Section~\ref{sec:modelindependentresults}.
\end{itemize}

\textbf{To-do}%% (Pre 5$^\text{th}$ August 2019 EB Meeting)}
\begin{itemize}
\item Implement PDF, factorisation and renormalisation SF uncertainties.
\item Calculate and implement alternative parton shower uncertainty (based on \textsc{Herwig}).
\item Calculate and implement tracking uncertainties.
\item Recast limits in terms of $\sigma\times$BRs
\end{itemize}

%% \textbf{To-do (Post 5$^\text{th}$ August 2019 EB Meeting)}
%% \begin{itemize}
%% \item
%% \end{itemize}

\section*{Version 0.3 (CDS Version 13) - 10 September 19}

\textbf{Notes}
\begin{itemize}
\item Iteration of note for third EB meeting.
%% \item Responded to comments on second iteration from EB on CDS.
\end{itemize}

\textbf{Main Documentation Changes}
\begin{itemize}
\item Section added/adjusted for all Analysis Changes.
\item Text and figure added explaining signal distribution of three body mass.
\item Subsection on fits to toy datasets removed.
\end{itemize}

\textbf{Analysis Changes}
\begin{itemize}
\item Background model updated to use 10\% $MLP$ ABCD regions, as described in second EB meeting.
\item Showering/ME systematics added for signal and background.
\item Renormalisation scale systematics added for signal and background.
\end{itemize}

\textbf{To-do}
\begin{itemize}
\item Calculate and implement tracking uncertainties.
\item Recast limits in terms of $\sigma\times$BRs.
\end{itemize}

\section*{Version 0.4 (CDS Version 15) - 20 September 19}

\textbf{Notes}
\begin{itemize}
\item Iteration of note for unblinding circulation
%% \item Responded to comments on second iteration from EB on CDS.
\end{itemize}

\textbf{Main Documentation Changes}
\begin{itemize}
\item Section added/adjusted for all Analysis Changes.
\item Text added mentioning \textsc{Herwig} modelling issues.
\end{itemize}

\textbf{Analysis Changes}
\begin{itemize}
\item Tracking uncertainties calculated.
\item Signal showering uncertainty for 3.5~GeV $a$ signal hypothesis interpolated from values determined on signal hypothesis with close masses.
\item Bug fixed in \textsc{MadGraph} background reweighting
\end{itemize}

\textbf{To-do}
\begin{itemize}
\item Recast limits in terms of $\sigma\times$BRs.
\end{itemize}


\section*{Version 0.5 (CDS Version 16) - 27 September 19}

\textbf{Notes}
\begin{itemize}
\item Post-unblinding iteration of note.
\end{itemize}

\textbf{Main Documentation Changes}
\begin{itemize}
\item Results section updated with final results.
\item Track studies plots updated with remaining signal samples.
\item Various tables reformatted, including adding uncertainties.
\end{itemize}

\textbf{Analysis Changes}
\begin{itemize}
\item None.
\end{itemize}

\textbf{To-do}
\begin{itemize}
\item Recast limits in terms of $\sigma\times$BRs.
\end{itemize}


\section*{Version 0.6 (CDS Version 21) - 4 October 19}

\textbf{Notes}
\begin{itemize}
\item Pre HDBS approval iteration of note.
\end{itemize}

\textbf{Main Documentation Changes}
\begin{itemize}
\item Extended unblinded classification MLP output plot range.
\end{itemize}

\textbf{Analysis Changes}
\begin{itemize}
\item Added limits which have been recast in terms of $\sigma\times$BRs.
\end{itemize}

\textbf{To-do}
\begin{itemize}
\item None.
\end{itemize}


\section*{Version 0.7 (CDS Version 22) - 14 October 19}

\textbf{Notes}
\begin{itemize}
\item Pre HDBS approval second iteration of note, after responding to comments from Maximilian and Wade.
\end{itemize}

\textbf{Main Documentation Changes}
\begin{itemize}
\item New pages added between sections.
\item Motivation section expanded.
\item Appendix added with plots of MLP inputs in the SR.
\item Section added describing reweighting uncertainty studies.
\item Table added summarising background uncertainties.
\end{itemize}

\textbf{Analysis Changes}
\begin{itemize}
\item None.
\end{itemize}

\textbf{To-do}
\begin{itemize}
\item None.
\end{itemize}


\section*{Version 0.8 (CDS Version 23) - 14 October 19}

\textbf{Notes}
\begin{itemize}
\item Post HDBS approval iteration of note.
\end{itemize}

\textbf{Main Documentation Changes}
\begin{itemize}
\item Description of indirect constraints on non-SM Higgs decays from couplings measurements added to $a$ literature review section.
\end{itemize}

\textbf{Analysis Changes}
\begin{itemize}
\item None.
\end{itemize}

\textbf{To-do}
\begin{itemize}
\item None.
\end{itemize}


\section*{Version 0.9 (CDS Version 23) - 14 October 19}

\textbf{Notes}
\begin{itemize}
\item Second post HDBS approval iteration of note.
\end{itemize}

\textbf{Main Documentation Changes}
\begin{itemize}
\item Addition of tables detailing signal systematic uncertainties: Tables~\ref{tab:mursys} and \ref{tab:sighadunctable}.
\item Addition of descriptions of the treatment of any possible uncertainties arising from the reweighting procedure in Section~\ref{sec:sysrwproc}.
\item Plots added comparing the nominal background prediction, \textsc{MadGraph}-based background prediction, and data, in all analysis regions (validation and signal). These are Figures~\ref{fig:bkgdvrs} and \ref{fig:bkgdvrsmg}.
\end{itemize}

\textbf{Analysis Changes}
\begin{itemize}
\item None.
\end{itemize}

\textbf{To-do}
\begin{itemize}
\item None.
\end{itemize}


\section*{Version 0.10 (CDS Version 23) - 16 January 20}

\textbf{Notes}
\begin{itemize}
\item EB approved, HDBS approval iteration of the note.
\end{itemize}

\textbf{Main Documentation Changes}
\begin{itemize}
\item Small changes to reflect analysis changes.
\item Generator-level acceptance added.
\item Plots now include previously blinded data and MC.
\item Merged signal and background MC statistics systematics sections.
\end{itemize}

\textbf{Analysis Changes}
\begin{itemize}
%% \item Charm-quark $a$ decay interpretation removed.
\item Explicit 18~GeV lepton $p_\text{T}$ cut applied.
\item Removed high weight \textsc{Sherpa} events as per PMG recommendations.
\item NNLO signal scale factors added.
\item Signal parton shower systematic uses 2D reweighting in the track multiplicity and $U1(0.7)$, and does not require interpolation. Also, bug fixed in which inconsistent selection was being used for two generator-level samples.
\item Bug fixed in which changes to cross section were being included in the estimation of the signal renormalisation scale uncertainties.
\end{itemize}

\textbf{To-do}
\begin{itemize}
\item None.
\end{itemize}


\section*{Version 0.11 (CDS Version 24) - 20 January 20}

\textbf{Notes}
\begin{itemize}
\item EB approved, HDBS approval iteration of the note, with improved plots for auxiliary material.
\end{itemize}

\textbf{Main Documentation Changes}
\begin{itemize}
\item Various figures updated/neatened, ready to be used as auxiliary material.
\end{itemize}

\textbf{Analysis Changes}
\begin{itemize}
\item None.
\end{itemize}

\textbf{To-do}
\begin{itemize}
\item None.
\end{itemize}


\section*{Version 0.12 (CDS Version 25) - 22 January 20}

\textbf{Notes}
\begin{itemize}
\item EB approved, HDBS approval iteration of the note, with improved plots for auxiliary material.
\end{itemize}

\textbf{Main Documentation Changes}
\begin{itemize}
\item Various figures updated/neatened further, ready to be used as auxiliary material.
\end{itemize}

\textbf{Analysis Changes}
\begin{itemize}
\item None.
\end{itemize}

\textbf{To-do}
\begin{itemize}
\item None.
\end{itemize}


\section*{Version 0.13 (CDS Version 26) - 24 January 20}

\textbf{Notes}
\begin{itemize}
\item EB approved, HDBS approval iteration of the note, with improved plots for auxiliary material.
\end{itemize}

\textbf{Main Documentation Changes}
\begin{itemize}
\item Various figures updated/neatened further, ready to be used as auxiliary material.
\end{itemize}

\textbf{Analysis Changes}
\begin{itemize}
\item None.
\end{itemize}

\textbf{To-do}
\begin{itemize}
\item None.
\end{itemize}


\section*{Version 0.14 (CDS Version 27) - 27 January 20}

\textbf{Notes}
\begin{itemize}
\item EB approved, HDBS approval iteration of the note, with improved plots for auxiliary material.
\end{itemize}

\textbf{Main Documentation Changes}
\begin{itemize}
\item Labels fixed on limit plots.
\end{itemize}

\textbf{Analysis Changes}
\begin{itemize}
\item None.
\end{itemize}

\textbf{To-do}
\begin{itemize}
\item None.
\end{itemize}


\section*{Version 0.15 (CDS Version 28) - 31 January 20}

\textbf{Notes}
\begin{itemize}
\item Circulation version of note.
\end{itemize}

\textbf{Main Documentation Changes}
\begin{itemize}
\item Minor changes to plots to make them compatible with paper auxiliary plots.
\end{itemize}

\textbf{Analysis Changes}
\begin{itemize}
\item None.
\end{itemize}

\textbf{To-do}
\begin{itemize}
\item None.
\end{itemize}


\section*{Version 0.16 (CDS Version 29) - 6 February 20}

\textbf{Notes}
\begin{itemize}
\item Post first circulation version of note.
\end{itemize}

\textbf{Main Documentation Changes}
\begin{itemize}
\item Minor changes to plots to make them compatible with paper auxiliary plots.
\end{itemize}

\textbf{Analysis Changes}
\begin{itemize}
\item None.
\end{itemize}

\textbf{To-do}
\begin{itemize}
\item None.
\end{itemize}


\section*{Version 0.17 (CDS Version 30) - 3 March 20}

\textbf{Notes}
\begin{itemize}
\item Post first circulation version of note, with modified background modelling systematics.
\end{itemize}

\textbf{Main Documentation Changes}
\begin{itemize}
\item Documentation updated to account for modified background modelling systematics.
\item Plots and tables harmonised with those in paper.
\item Table~\ref{tab:asimovresults} fixed. It was showing $\Delta\mu$ for $\mu =0$ before.
\end{itemize}

\textbf{Analysis Changes}
\begin{itemize}
\item Modified background modelling systematics.
\end{itemize}

\textbf{To-do}
\begin{itemize}
\item None.
\end{itemize}


\section*{Version 0.18 (CDS Version 31) - 26 March 20}

\textbf{Notes}
\begin{itemize}
\item Cutflow added.
\end{itemize}

\textbf{Main Documentation Changes}
\begin{itemize}
\item Appendix~\ref{app:cutflow} added with cutflow.
\end{itemize}

\textbf{Analysis Changes}
\begin{itemize}
\item None.
\end{itemize}

\textbf{To-do}
\begin{itemize}
\item None.
\end{itemize}


\clearpage
\tableofcontents
% List of contributors - print here or after the Bibliography.
\PrintAtlasContribute{0.30}
\clearpage

%-------------------------------------------------------------------------------
\clearpage
\section{Introduction}
\label{sec:intro}
%-------------------------------------------------------------------------------

% Analysis overview
This analysis searches for decays of the Higgs boson to a $Z$ boson, and a light ($\leq 4$~GeV) resonance. The $Z$ boson is required to decay to leptons ($\ell$), specifically electrons ($e$) or muons ($\mu$), although the selection has some acceptance for decays to pairs of $\tau$ leptons if they both decay leptonically. Hadronic decays of the light resonance are targeted, and due to its low mass and large boost, they are reconstructed as a single jet of hadrons. The full Run 2 dataset is used. This analysis is focusing on the light pseudo-scalar resonance ($a$) from the Two Higgs Doublet Model (2HDM), or 2HDM with an additional singlet (2HDM$+$s).

%\subsection{Charmonium States}
%\label{sec:introcharmonium}

%% Interpretations and motivation (charmonium)\\
%The ad-hoc nature of the Yukawa sector makes the coupling of the Higgs boson to quarks an ideal place to search for new physics. The couplings of the Higgs boson to third generation quarks have been established experimentally~\cite{Aaboud:2018zhk,Sirunyan:2018kst,Aaboud:2018urx,Sirunyan:2018hoz}. However, the coupling of the Higgs boson to first and second generation quarks have not yet been measured. With the largest mass of the first or second generation quarks~\cite{PhysRevD.98.030001}, the charm quark provides the frontier of the couplings of the Higgs boson to quarks. Decays of the Higgs boson to a $Z$~boson and light SM resonances provide an effective probe of this part of the Higgs sector~\cite{Bodwin:2013gca,Kagan:2014ila,Chisholm:2016fzg}, but are still only loosely constrained. However, the potential of searches for decays of the Higgs boson to bosons and light SM resonances has been demonstrated by the ATLAS experiment~\cite{Aaboud:2018txb,Aaboud:2017xnb,Aaboud:2016rug,Aad:2015sda}, though only in exclusive decay modes of the resonance. Searches have also been performed for inclusive Higgs boson decays to pairs of charm quarks~\cite{Aaboud:2018fhh}, but never through a charmonium resonance. This is the first search at the LHC for decays of the Higgs boson to a final state containing a $J/\psi$ or an $\eta_c$ which decays to an inclusive hadronic final state. These decays provide a low $Q^2$ probe of $H\to ZZ^*$, in addition to probing the coupling of charm quark to the Higgs boson. With a SM branching ratio ($\mathcal{B}$) of $1.4\times 10^{-5}$ ($2.2\times 10^{-6}$)~\cite{Isidori:2013cla} the $H\to Z\eta_c$ ($H\to ZJ/\psi$) decay channels are highly sensitive to modifications from new physics~\cite{Isidori:2013cla,Alte:2016yuw}.

\subsection{Light Higgs Bosons}
\label{sec:introa0}

%% Interpretations and motivation (a0)\\
The SM Higgs sector is the simplest mechanism to generate the masses of the $W$ and $Z$ bosons, though extended Higgs sectors are also possible. Two such extensions are the Two Higgs Doublet Model (2HDM)~\cite{PhysRevD.90.075004} and 2HDM with an additional singlet (2HDM$+$S)~\cite{PhysRevD.90.075004,a0}, where the $a$ can have a large coupling to the observed Higgs boson. These models are necessary to generate the masses in the Minimal Supersymmetric Model (MSSM), and the Next-to-MSSM (NMSSM), respectively~\cite{Gunion:2002zf}. The extension of the MSSM by this additional scalar field elegantly solves the $\mu$-problem of the MSSM~\cite{Babu:2001gp}, and greatly reduces the fine-tuning and little hierarchy problems. Due to the narrow width of a SM Higgs boson with a mass of 125~GeV, even a small coupling to a non-SM Higgs boson could result in new decay modes with large branching ratios. The Yukawa-like couplings of these light Higgs bosons mean that for the mass range considered in this search, $m_{a}<4~\text{GeV}$, their dominant branching ratios lead to hadronic final states, as shown in Figure~\ref{fig:a0brs}~\cite{Curtin:2013fra}.

\begin{figure}[!htbp]
  \centering
  \subfigure[]{\includegraphics[width=0.475\textwidth]{figures/2HDMSplot_pseudoscalar_gtGeV_tb_0-5_type_2__16.pdf}}
  \subfigure[]{\includegraphics[width=0.475\textwidth]{figures/2HDMSplot_pseudoscalar_gtGeV_tb_5_type_2__17.pdf}}
  %% \includegraphics[width=0.65\textwidth]{figures/a0BRs.pdf}
  \caption{Branching ratios for the decay of the lightest pseudoscalar Higgs boson in the Type II 2HDM$+$s, for $tan\beta=0.5$ (a) and  $tan\beta=5$ (b).~\cite{Curtin:2013fra}.}
  \label{fig:a0brs}
\end{figure}

%% previous constraints\\
Previous searches for decays of the Higgs boson to light Higgs bosons at the LHC lead to upper limits from CMS in the $H\to aa\to \mu^+\mu^-\tau^+\tau^-$~\cite{CMSa0Limitmumutautau}, $H\to aa\to \mu^+\mu^-b\bar b$~\cite{CMSa0Limitmumubb}, $H\to aa\to \tau^+\tau^-\tau^+\tau^-$~\cite{CMSa0Limittautautautau1,CMSa0Limittautautautau2} and $H\to aa\to \mu^+\mu^-\mu^+\mu^-$~\cite{CMSa0Limitmumumumu} decay modes, and from ATLAS in the $H\to aa\to \mu^+\mu^-\tau^+\tau^-$~\cite{ATLASa0Limitmumutautau}, $H\to aa\to b\bar b b\bar b$~\cite{Aaboud:2016oyb}, $H\to (Z/a)a\to \ell_1^+\ell_1^-\ell_2^+\ell_2^-$~\cite{Aaboud:2018fvk}, $H\to aa\to b\bar b\mu^+\mu^-$~\cite{Aaboud:2018esj}, $H\to aa\to \gamma\gamma \text{jj}$~\cite{Aaboud:2018gmx}, and $H\to aa\to \gamma\gamma\gamma\gamma$~\cite{Aad:2015bua} decay modes. Previously, the D\O~experiment at the Tevatron set limits in the $H\to aa \to \mu^+\mu^-\mu^+\mu^-$ and $H\to aa \rightarrow\mu^+\mu^-\tau^+\tau^-$ decay modes~\cite{Abazov:2009yi}. These previous searches have mostly been limited to $H\to aa$ decays, and $a$ final states including leptons, photons or bottom quark pairs. This is the first search for hadronic decays of the $a$ over this mass range, and one of the few searches for $H\to Za$ decays, and therefore it provides complementary information to the above searches~\cite{Chisholm:2016fzg}. This search is particularly important in the low (high) $\tan\beta$ phase space of the Type-II and Type-III (Type-I and Type-IV) 2HDM and 2HDM$+$S, where $a$ decays to leptons are suppressed, leading to almost exclusively hadronic decays~\cite{PhysRevD.90.075004}. Therefore, these parts of the 2HDM($+$S) phase space are largely unconstrained by previous searches, but can be probed with this search.

Constraints on decays of the Higgs boson to light Higgs bosons exist from simultaneous fits to the observed Higgs boson decay modes, which set a limit on decays of the Higgs boson to non-SM final states at 22\%~\cite{Falkowski:2013dza}. However, this constraint assumes SM couplings to the SM particles. If this assumption is relaxed, then the constraint on non-SM decays of the Higgs boson relaxes to $\sim 50\%$~\cite{Falkowski:2013dza}.


%% Previous searches at the LHC include limits from CMS $a$ production~\cite{CMSa0Limitmumutautau,CMSa0Limitmumubb,CMSa0Limittautautautau1,CMSa0Limittautautautau2,CMSa0Limitmumumumu}, and the ATLAS limit in the $H\to aa\to 2\mu 2\tau$~\cite{ATLASa0Limitmumutautau}, $H\to aa\to 4b$~\cite{Aaboud:2016oyb}, $H\to (Z/a)a\to \ell_1^+\ell_1^-\ell_2^+\ell_2^-$~\cite{Aaboud:2018fvk}, $H\to aa\to b\bar b\mu^+\mu^-$~\cite{Aaboud:2018esj}, $H\to aa\to \gamma\gamma \text{jj}$~\cite{Aaboud:2018gmx}, and $H\to aa\to \gamma\gamma\gamma\gamma$~\cite{Aad:2015bua} channels. Previously, the D\O~experiment at the Tevatron set limits on $H\to aa \to 4\mu$ and $H\to aa \rightarrow\mu\mu\tau\tau$ in the low $m_{a$ range~\cite{Abazov:2009yi}.


%% %-------------------------------------------------------------------------------
%% \section{The ATLAS Detector}
%% \label{sec:atlasdetector}
%% %-------------------------------------------------------------------------------

%% %{\color{red}Section taken from ATLAS-CONF-2013-018}
%% The ATLAS experiment~\cite{1748-0221-3-08-S08003} deploys a multi-purpose particle physics detector with forward-backward symmetric cylindrical geometry and a near 4$\pi$ coverage in solid angle\footnote{The ATLAS experiment uses a right-handed coordinate system with its origin at the nominal interaction point. The $z$-axis is along the beam pipe, the $x$-axis points to the centre of the LHC ring and the $y$-axis is defined as pointing upwards. Polar coordinates $(r,\phi)$ are used in the transverse plane, $\phi$ being the azimuthal angle around the beam pipe. The pseudo-rapidity $\eta$ is defined as $\eta = -\ln[\tan(\theta/2)]$ where $\theta$ is the polar angle.}. The interaction point is surrounded by an inner detector (ID), a calorimeter system, and a muon spectrometer (MS). The ID covers $|\eta| < 2.5$ and consists of a silicon pixel detector, a silicon micro-strip detector, and a transition radiation tracker (TRT). The ID is surrounded by a thin superconducting solenoid providing a 2 T axial magnetic field. One significant upgrade for Run-2 is the presence of the Insertable B-Layer (IBL)~\cite{Capeans:1291633}, an additional pixel layer close to the interaction point, that provides high-resolution hits at small radius to improve the tracking performance. The calorimeter system has a high-granularity lead/liquid-argon (LAr) sampling calorimeter that measures the energy and the position of electromagnetic showers within $|\eta| < 3.2$.
%% LAr sampling calorimeters are also used to measure hadronic showers in the endcap ($1.5 < |\eta| < 3.2$) and forward ($3.1 < |\eta| < 4.9$) regions, while an iron/scintillator tile calorimeter measures hadronic showers in the central region ($|\eta| < 1.7$). The MS surrounds the calorimeters and consists of three large superconducting air-core toroid magnets, each with eight coils, a system of precision tracking chambers ($|\eta| < 2.7$), and fast trigger chambers ($|\eta| < 2.4$). For Run-2 the ATLAS detector has a two-level trigger system. The first-level trigger (Level-1 trigger) is implemented in hardware and uses a subset of the detector information to reduce the accepted rate to 100 kHz. This is followed by a software-based trigger (called higher-level trigger, HLT) that reduces the rate of events recorded to 1 kHz.


%-------------------------------------------------------------------------------
\clearpage
\section{Experimental and Simulated Data Samples}
\label{sec:samples}
%-------------------------------------------------------------------------------

\subsection{Data Samples}
\label{sec:datasamples}

The data used in this search correspond to the full Run 2 dataset, collected by the ATLAS detector between 2015-2018, using single-lepton triggers. This represents a total integrated luminosity of $139\ \text{fb}^{-1}$. This corresponds to: $3.2\ \text{fb}^{-1}$ of 2015 data; $33.0\ \text{fb}^{-1}$ of 2016 data; $44.3\ \text{fb}^{-1}$ of 2017 data; and $58.5\ \text{fb}^{-1}$ of 2018 data. This dataset is then processed using the \emph{HDBS3} derivation, in order to reduce the size of the dataset, while retaining tracking information, which is necessary to separate the light resonance from multijet production. The Good Run Lists (GRL) applied are provided in Table~\ref{tab:grls}. The datasets are listed in Appendix~\ref{app:datasamples}. Finally, it was noticed that some data events were duplicated in the final data files. These events are removed in the analysis-level software.

\begin{table}[!htbp]{\footnotesize\renewcommand{\arraystretch}{1.2}
    \begin{center}
      \begin{tabular}{|c|c|}
        \hline
        Year & GRL\\
        \hline
        2015 & All\_Good\_25ns \\
        2016 & All\_Good\_25ns \\
        2017 & All\_Good\_25ns\_Triggerno17e33prim \\
        2018 & All\_Good\_25ns\_Triggerno17e33prim \\
        %% 2015 & PHYS\_StandardGRL\_All\_Good\_25ns\_276262-284484\_OflLumi-13TeV-008.root \\
        %% 2016 & PHYS\_StandardGRL\_All\_Good\_25ns\_297730-311481\_OflLumi-13TeV-009.root \\
        %% 2017 & physics\_25ns\_Triggerno17e33prim.lumicalc.OflLumi-13TeV-010.root \\
        \hline
      \end{tabular}
      \caption{Good Run Lists applied to each year of data. 2015 only had 25 ns runs.}
      \label{tab:grls}
  \end{center}}
\end{table}


\subsection{Monte-Carlo Samples}
\label{sec:mcsamples}

Monte-Carlo (MC) simulation samples are used to guide the development of this analysis, and are used directly in the signal and background modelling. These MC samples are summarised in Table~\ref{tab:mcsamples}. The complete list of MC samples used in this analysis can be found in Appendices~\ref{app:sigmc} and \ref{app:bkgdmc}, for signal and background samples respectively. All MC samples used in this analysis have been processed using the \emph{FTAG2} derivation.

\begin{table}[!htbp]{\footnotesize\renewcommand{\arraystretch}{1.2}
    \begin{center}
      \begin{tabular}{|c|c|cc|c|c|}
        \hline
        & Process & Generator & Shower & DSID & Notes \\
        \hline
        \hline
        %& $gg\to H\to Z(\ell\ell)+\eta_c$ & \textsc{Powheg}+\textsc{Pythia}8 & \textsc{Pythia}8+\textsc{EvtGen} & 345906 & \\
        \cline{2-6}
        %\multirow{2}{*}{\parbox[t]{2mm}{\rotatebox[origin=c]{90}{Signal\hspace{5mm}}}} & $gg\to H\to Z(\ell\ell)+J/\psi$ & \textsc{Powheg}+\textsc{Pythia}8 & \textsc{Pythia}8+\textsc{EvtGen} & 450549 & \\
        \cline{2-6}
        & $gg\to H\to Z(\ell\ell)+a$ & \textsc{Powheg}+\textsc{Pythia}8 & \textsc{Pythia}8+\textsc{EvtGen} & \begin{tabular}{c}345907\\450550\\450551\\450552\\450553\\345908\\450554\\450555\\450556\\345909\\\end{tabular} & \begin{tabular}{c}$m_{a}=0.5$~GeV\\$m_{a}=0.75$~GeV\\$m_{a}=1$~GeV\\$m_{a}=1.5$~GeV\\$m_{a}=2$~GeV\\$m_{a}=2.5$~GeV\\$m_{a}=3$~GeV\\$m_{a}=3.5$~GeV\\$m_{a}=4$~GeV\\$m_{a}=8$~GeV\\\end{tabular}\\
        \hline
        \hline
        \multirow{2}{*}{\parbox[t]{2mm}{\rotatebox[origin=c]{90}{Background\hspace{2mm}}}} & $pp\to Z(\ell\ell)+jets$ & \multicolumn{2}{c|}{\textsc{Sherpa}2.2.1} & \begin{tabular}{c}364100-\\364141\\\end{tabular} & \begin{tabular}{c}$Max(H_T,p_\text{T}^V)$ slices\\and flavour filtered\end{tabular}\\
        \cline{2-6}
        & $gg\to Z(\ell\ell)+Z(q\bar q)$ & \multicolumn{2}{c|}{\textsc{Powheg}} & 364302 & \\
        \cline{2-6}
        & $pp\to Z(\ell\ell)+Z(q\bar q)$ & \multicolumn{2}{c|}{\textsc{Powheg}} & 363356 & \\
        \cline{2-6}
        & $pp\to Z(\ell\ell)+W(q\bar q)$ & \multicolumn{2}{c|}{\textsc{Powheg}} & 363358 & \\
        \cline{2-6}
        & $pp\to t\bar t$ & \textsc{Powheg} & \textsc{Pythia}8+\textsc{EvtGen} & 410503 & $\geq 2\ell$\\
        \hline
      \end{tabular}
      \caption{Signal and background processes simulated with MC for this analysis.}
      \label{tab:mcsamples}
  \end{center}}
\end{table}

Having the largest cross section, MC samples in which Higgs bosons are produced via gluon-gluon Fusion (ggF) are generated for use in this analysis. The Higgs boson is produced in \textsc{Powheg}~\cite{Nason:2004rx,Frixione:2007vw,Alioli:2010xd}, using the AZNLO tune~\cite{Aad:2014xaa}. The decay, hadronisation, parton shower and underlying event are modelled using \textsc{Pythia}8~\cite{Sjostrand:2007gs} (v8.212) and \textsc{EvtGen}~\cite{Ryd:2005zz}, interfaced to the CT10~\cite{Lai:2010vv} and CTEQ6L1 PDF sets. For the BSM signal hypotheses, the SM Higgs boson (pdgId=25) is replaced by the heavy neutral scalar Higgs from the 2HDM ($H^0$, pdgId=35), which is then decayed in \textsc{Pythia}8 to a $Z$ boson and a neutral pseudo-scalar $a$ (pdgId=36). The $a$ is allowed to decay to any final state, to ensure that any decay mode with a significant selection efficiency is considered, and the default \textsc{Pythia}8 2HDM $\tan\beta$ value of 1 is used to generate the decays of the BSM Higgs bosons. These resulting BRs are shown in Table~\ref{tab:a0brs}. Lastly, NNLO corrections are applied to the $p_\text{T}$ distribution of the Higgs boson.

\begin{table}[!htbp]{\footnotesize\renewcommand{\arraystretch}{1.2}
    \begin{center}
      \begin{tabular}{|c|c|}
        \hline
        $a$ Mass & Branching Ratio\\
        \hline
        0.5~GeV & $gg$ (92\%), $\mu^+\mu^-$ (8\%)\\
        0.75~GeV & $gg$ (88\%), $\mu^+\mu^-$ (12\%)\\
        1~GeV & $gg$ (88\%), $\mu^+\mu^-$ (12\%)\\
        1.5~GeV & $gg$ (76\%), $s\bar s$ (16\%), $\mu^+\mu^-$ (8\%)\\
        2~GeV & $gg$ (82\%), $s\bar s$ (13\%), $\mu^+\mu^-$ (5\%)\\
        2.5~GeV & $gg$ (88\%), $s\bar s$ (8\%), $\mu^+\mu^-$ (4\%)\\
        3~GeV & $gg$ (86\%), $s\bar s$ (9\%), $\mu^+\mu^-$ (4\%)\\
        3.5~GeV & $c\bar c$ (88\%), $gg$ (10\%), $s\bar s$ (1\%)\\
        4~GeV & $c\bar c$ (57\%), $\tau^+\tau^-$ (37\%), $gg$ (5\%)\\
        4.5~GeV & $c\bar c$ (52\%), $\tau^+\tau^-$ (43\%), $gg$ (4\%)\\
        5~GeV & $c\bar c$ (50\%), $\tau^+\tau^-$ (45\%), $gg$ (4\%)\\
        8~GeV & $\tau^+\tau^-$ (45\%), $c\bar c$ (40\%), $gg$ (14\%)\\
        12~GeV & $b\bar b$ (81\%), $\tau^+\tau^-$ (10\%), $c\bar c$ (7\%), $gg$ (2\%)\\
        \hline
      \end{tabular}
      \caption{Branching ratios of the main decay modes (BR$>1\%$), for various $a$ mass points. Values are determined in \textsc{Pythia}8 using the default \textsc{BSMHiggs} $\tan\beta$ value of 1.}
      \label{tab:a0brs}
  \end{center}}
\end{table}

The background for this analysis is dominated by $Z+\text{jets}$ events, which is modelled using the ATLAS recommendations~\cite{ATL-PHYS-PUB-2016-003}. This choice is further motivated by this sample having a NLO ME calculation, and showing the best data to MC agreement of the samples we have studied. The calculation of the hard scatter and parton shower tuning is done with \textsc{Sherpa}2.2.1~\cite{1126-6708-2009-02-007} interfaced to the NNPDF~\cite{Ball:2012cx} PDF sets. The samples are sliced according to the maximum of the scalar sum of the $p_\text{T}$ of all jets and leptons in the event ($H_\text{T}$), and the $p_\text{T}$ of the $Z$ boson, at generator-level. They are further split by the presence of heavy flavour partons at generator-level. The inclusive production cross sections are known to NNLO in QCD~\cite{Butterworth:1287902}. As per the ATLAS PMG recommendations, for all of the \textsc{Sherpa} samples used in this analysis anomalous high-weight (magnitude above 100) events have their event weights set to 1.

The $ZZ$ and $ZW$ processes constitute small ($<1\%$) backgrounds to this analysis. The diboson backgrounds are also modelled according to the ATLAS recommendations~\cite{ATL-PHYS-PUB-2017-005}. \textsc{Sherpa}2.2.1 is interfaced to the \textsc{NNPDF} 3.0 PDF set for the modelling of the hard interaction and parton shower.

The $t\bar t$ process constitutes are further, small ($<1\%$) background to this analysis. The hard interaction for the $t\bar t$ background is modelled using \textsc{Powheg}, while the decay, hadronisation, parton shower and underlying event are modelled using \textsc{Pythia}8 and \textsc{EvtGen}.

In addition to the nominal background MC samples, the dominant $Z+\text{jets}$ background is modelled using an alternative MC generator, in order to have a second estimate of the main background with which to cross check the first. This sample uses events generated from \textsc{MadGraph\_aMC@NLO}, \textsc{Pythia}8 and \textsc{EvtGen}. The generator tune is \textsc{A14}, and the PDF set is \textsc{NNPDF23LO}. The $Z\to e^+e^-$ and $Z\to \mu^+\mu^-$ samples are sliced based on $H_\text{T}$ and heavy flavour filters, while the $Z\to \tau^+\tau^-$ sample is slides based on the number of additional final state particles.

A full simulation of the ATLAS detector~\cite{Aad2010} in \textsc{Geant}4~\cite{AGOSTINELLI2003250} is used to estimate the response of the ATLAS detector in all of the above samples. Data-driven corrections are applied to the event level-trigger efficiency, the jet vertex tagging efficiency, the electron reconstruction, identification and isolation efficiencies, and the muon reconstruction, isolation and track-to-vertex association efficiencies.

Lastly, the design of the substructure-based selection was assisted using additional signal samples, with finely sampled $a$ mass values of: 0.4, 0.5, 0.75, 0.8, 1, 1.2, 1.5, 1.6, 2, 2.4, 2.5, 2.8, 3, 3.2, 3.5, 3.6 and 4~GeV, and a high-statistics $Z+$jets background sample of 20M events. These additional samples were generated using \textsc{Pythia}8, and the response of the ATLAS detector was simulated using \textsc{Delphes} with the ATLAS card. These samples were used to explore different strategies to maximise the substructure-based selection efficiency over the entire $a$ mass range (which was not possible with the limited number of full-simulation samples we had at hand), to ensure that the selection efficiency did not dip significantly for $a$ mass points not used in the optimisation, and as a second validation of the final selection.


%-------------------------------------------------------------------------------
\clearpage
\section{Event Selection}
\label{sec:selection}
%-------------------------------------------------------------------------------

The full event selection can be broken down into three main stages. First, an event-level pre-selection is applied, which targets a 2-lepton plus jet (used to reconstruct the target BSM or SM charmonium resonance) final state. This final state is contaminated by a large $Z+$jets background, and so track-based substructure techniques are used to discriminate between the signal resonances and the QCD background jets. The high resolution of the inner-tracker is required to do this, and so the second stage of the selection involved Ghost-Associating tracks to the reconstructed jet, and applying a loose track-selection in order to reject backgrounds from pileup, the underlying event, multiple parton interactions and fake tracks. Finally, the tracks surviving this track-selection are used to form substructure-based discriminants, which are given to a classification machine learning algorithm, to which we apply a requirement that discriminates signal from background.

\subsection{Event-Level Pre-Selection}
\label{sec:premlpselection}

Events are triggered for offline storage using the lowest unprescaled single lepton trigger for each period. These are listed in Table~\ref{tab:triggers}. The studies justifying this choice of trigger menu are detailed in Appendix~\ref{app:triggers}. The trigger decisions are implemented using the \emph{Trig::TrigDecisionTool} tool~\cite{TrigDecisionTool}. The trigger objects which fired the triggers are required to be matched to the offline leptons using the \emph{Trig::MatchingTool} tool~\cite{MatchingTool}, which implements a requirement of $\Delta R<0.1$ between the trigger object and the corresponding offline object.

\begin{table}[!htbp]{\footnotesize\renewcommand{\arraystretch}{1.2}
    \begin{center}
      \begin{tabular}{|c|c|}
        \hline
        Period & Triggers\\
        \hline
        \multirow{2}{*}{\parbox[t]{2mm}{\rotatebox[origin=c]{90}{2015\hspace{10mm}}}} & \emph{HLT\_mu20\_iloose\_L1MU15}\\
        & \emph{HLT\_mu40}\\
        & \emph{HLT\_e24\_lhmedium\_L1EM20VH}\\
        & \emph{HLT\_e60\_lhmedium}\\
        & \emph{HLT\_e120\_lhloose}\\
        \hline
        \multirow{2}{*}{\parbox[t]{2mm}{\rotatebox[origin=c]{90}{2016 A\hspace{6mm}}}} & \emph{HLT\_mu26\_ivarmedium}\\
        & \emph{HLT\_mu40}\\
        & \emph{HLT\_e26\_lhtight\_nod0\_ivarloose}\\
        & \emph{HLT\_e60\_lhmedium\_nod0}\\
        & \emph{HLT\_e140\_lhloose\_nod0}\\
        \hline
        \multirow{2}{*}{\parbox[t]{2mm}{\rotatebox[origin=c]{90}{2016 B-E\hspace{5mm}}}} & \emph{HLT\_mu26\_ivarmedium}\\
        & \emph{HLT\_mu50}\\
        & \emph{HLT\_e26\_lhtight\_nod0\_ivarloose}\\
        & \emph{HLT\_e60\_lhmedium\_nod0}\\
        & \emph{HLT\_e140\_lhloose\_nod0}\\
        \hline
        \multirow{2}{*}{\parbox[t]{2mm}{\rotatebox[origin=c]{90}{2016 F-L\hspace{5mm}}}} & \emph{HLT\_mu26\_ivarmedium}\\
        & \emph{HLT\_mu50}\\
        & \emph{HLT\_e26\_lhtight\_nod0\_ivarloose}\\
        & \emph{HLT\_e60\_lhmedium\_nod0}\\
        & \emph{HLT\_e140\_lhloose\_nod0}\\
        \hline
        \multirow{2}{*}{\parbox[t]{2mm}{\rotatebox[origin=c]{90}{2017\hspace{8mm}}}} & \emph{HLT\_mu26\_ivarmedium}\\
        & \emph{HLT\_mu50}\\
        & \emph{HLT\_e26\_lhtight\_nod0\_ivarloose}\\
        & \emph{HLT\_e60\_lhmedium\_nod0}\\
        & \emph{HLT\_e140\_lhloose\_nod0}\\
        \hline
        \multirow{2}{*}{\parbox[t]{2mm}{\rotatebox[origin=c]{90}{2018\hspace{8mm}}}} & \emph{HLT\_mu26\_ivarmedium}\\
        & \emph{HLT\_mu50}\\
        & \emph{HLT\_e26\_lhtight\_nod0\_ivarloose}\\
        & \emph{HLT\_e60\_lhmedium\_nod0}\\
        & \emph{HLT\_e140\_lhloose\_nod0}\\
        \hline
      \end{tabular}
      \caption{Triggers used to select events in data for the full Run 2 dataset. The letters correspond to the different run periods.}
      \label{tab:triggers}
  \end{center}}
\end{table}

Electron candidates are reconstructed offline by matching tracks in the inner detector to topological energy clusters in the electromagnetic calorimeter~\cite{Aad:2014fxa}. These electrons are then required to pass the \emph{Medium} likelihood-based identification working point, which requires that the shower profile be compatible with that of an electromagnetic shower, and is implemented using the \emph{DFCommonElectronsLHMedium} \emph{xAOD} decoration~\cite{DFCommonElectronsLHMedium}. The \emph{FixedCutLoose} isolation working point is also applied, requiring the electrons to have a transverse energy sum in a cone of $\Delta R<0.2$ around the electron of less than $20\%$ of its $p_\text{T}$, and the $p_\text{T}$ of tracks within a variable-width cone of $\Delta R<\text{Min}(0.2,\ 10\ \text{GeV}/p_\text{T})$ of the electron must be less than $15\%$ of its $p_\text{T}$, as implemented by the \emph{IsolationSelectionTool} and \emph{IsolationCloseByCorrectionTool}~\cite{IsolationSelectionTool}. Close-by leptons are removed from the cones. The electrons must also have a longitudinal impact parameter ($z_0sin\theta$) of less than 0.5 mm with respect to the reconstructed primary vertex, defined as the vertex with the highest sum of square track $p_\text{T}$. They are required to have a $p_\text{T}$ of at least 18~GeV due to the \emph{FTAG2} derivation\footnote{\label{footnote18gevleptonptcuts}This reduces the signal acceptance by approximately 5\%, while also reducing the background acceptance.}, and be found in the central body of the detector ($\vert\eta\vert<2.47$), but not the transition region ($1.37<\vert\eta\vert<1.52$). The leading lepton is required to have a $p_\text{T}$ of at least 27~GeV, due to the trigger requirement. The reconstruction, identification and isolation efficiencies of these electrons are calibrated using the \emph{AsgElectronEfficiencyCorrectionTool} tool~\cite{AsgElectronEfficiencyCorrectionTool}, and the momentum is calibrated using the \emph{EgammaCalibrationAndSmearingTool}~\cite{EgammaCalibrationAndSmearingTool}.

Muon candidates are most often reconstructed offline by matching tracks in the inner detector to complete or partial tracks in the muon spectrometers~\cite{Aad:2014rra}. If a complete track is present, the two momentum measurements are combined, else the inner detector track momentum measurement is used. In the absence of a track in the inner detector, a stand alone track in the muon spectrometer is used. In the center of the barrel of the detector ($\vert\eta\vert<0.1$) there is no muon spectrometer coverage, and so tracks with a $p_\text{T}>15$~GeV are reconstructed as muons if they are matched to calorimeter deposits compatible with a minimum ionising particle. Reconstructed muons are required to pass a \emph{Medium} quality working point, as implemented by the \emph{MuonSelectionTool}~\cite{MuonSelectionTool}. The \emph{FixedCutLoose} isolation working point is applied using the \emph{IsolationSelectionTool} and \emph{IsolationCloseByCorrectionTool}, requiring the transverse energy sum in a cone of $\Delta R<0.2$ around the muon of less than $30\%$ of its $p_\text{T}$, and the $p_\text{T}$ of tracks within a variable-width cone of $\Delta R<\text{Min}(0.15,\ 10\ \text{GeV}/p_\text{T})$ of the muon must be less than $15\%$ of its $p_\text{T}$. If an inner detector track is present, the muons must also have a longitudinal impact parameter ($z_0sin\theta$) of less than 0.5 mm, and a transverse impact parameter ($d_0$) of less than 1 mm, with respect to the primary vertex. They are required to have a $p_\text{T}$ of at least 18~GeV due to the \emph{FTAG2} derivation\textsuperscript{\ref{footnote18gevleptonptcuts}}, and be found within the acceptance of the muon spectrometers ($\vert\eta\vert<2.7$). The leading lepton is required to have a $p_\text{T}$ of at least 27~GeV, due to the trigger requirement. The reconstruction, isolation and track-to-vertex association efficiencies of these muons are calibrated using the \emph{MuonEfficiencyScaleFactors} tool~\cite{MuonEfficiencyScaleFactors}, and the momentum is calibrated using the \emph{MuonCalibrationAndSmearingTool}~\cite{MuonCalibrationAndSmearingTool}.

Due to the low mass of the resonance, and the relatively large kinetic energy imparted to it from the mass difference between the Higgs and $Z$ bosons, the resonance is highly boosted. The spread of the $a$ decay products is contained in a cone of width $\Delta R \approx 2 m_{a}/p_\text{T, }a$, which for our largest mass point (4~GeV) and lowest $p_\text{T}$ jet (20~GeV), gives a cone size of 0.4. For this reason, it is reconstructed as a single anti-$k_T$ jet with a radius parameter of 0.4, formed of topological calorimeter clusters and calibrated to the electromagnetic energy scale. This uses the \emph{AntiKt4EMTopoJets} \emph{xAOD} jet collection. Jet energies are corrected for contributions from pile-up interactions using a jet-area based technique, and calibrated using $p_\text{T}$- and $\eta$-dependent correction factors determined from simulation, with residual corrections from in situ measurements applied to data and internal jet properties, referred to as the Global Sequential Calibration (GSC). This calibration is implemented using the \emph{JetCalibrationTool}~\cite{JetCalibrationTool}. Jets are required to satisfy $p_\text{T}>20$~GeV, $\vert\eta\vert<2.5$ and the ``jet cleaning'' selection of Ref~\cite{ATLAS-CONF-2015-029}. To reject jets from pile-up interactions, jets with $p_\text{T}<60$~GeV and $\vert\eta\vert<2.4$ are required to have a ``Jet Vertex Tagger'' value in excess of 0.59.% The higher mass signal samples have a lower reconstructed $p_\text{T}$, because the higher mass resonance has more of its components reconstructed outside the calorimeter jet.


To avoid double counting, overlapping electrons, muons and jets are then removed according to the standard ATLAS procedure~\cite{Masubuchi:2235437}. This is implemented using the \emph{OverlapRemovalTool}~\cite{Adams:1743654}. The procedure is outlined as follows:
\begin{enumerate}
\item Any electron sharing an inner detector track with a muon is removed from the event.
\item Any jets within $\Delta R$ of 0.2 of electrons are removed.
\item Electrons within $min(0.4,\ 0.04 + 10\ \text{GeV}/p_\text{T}^\text{electron})$ of jets which satisfy $JVT > 0.59$, $p_\text{T}<60$~GeV and $\vert\eta\vert<2.4$ are removed.
\item Remove jets within $\Delta R$ of 0.2 of muons if the jets have less than three associated tracks, or if the muon has 70\% of the associated track momentum and less than 50\% of the jet momentum
\item Muons within $min(0.4,\ 0.04 + 10\ \text{GeV}/p_\text{T}^\text{muon})$ of jets passing $JVT > 0.59$, $p_\text{T}<60$~GeV and $\vert\eta\vert<2.4$ are removed.
\end{enumerate}

At least two leptons are required to survive the overlap removal procedure, two of which must be same-flavour opposite-sign leptons. These are required to have an invariant mass compatible with the $Z$ boson, $81<m_{\ell\ell}<101$~GeV. If multiple same flavour opposite sign lepton pairs mass these requirements, then the pairing with an invariant mass closest to that of the $Z$ boson is chosen.

The three body system is then required to have an invariant mass passing a loose pre-selection requirement: $m_{j\ell\ell}<250$~GeV. This variable is used to form the regions from which the background is estimated, thus the large high side window is to allow sufficient background to determine the background normalisation from data. Many events contain multiple jets which could be taken as the resonance candidate, in which case the jet with the highest $p_\text{T}$ is chosen. This criteria correctly selects the resonance in 82.1\% to 88.3\% of cases for the 0-4~GeV $a$ signal hypotheses, while not biasing the jet selection in such a way as to create a fake peak in the invariant mass of the three body system, as would be the case for a three body mass based selection. The fraction of events in which the correct jet was selected was also higher than with selecting the jet which gives a three body mass closest to 125~GeV.

Finally, the three body mass is required to be between 120~GeV and 135~GeV. This requirement was chosen to maximise the $S/\sqrt{B}$.

%% The three body system is then required to have an invariant mass, which enters either the Signal Region (SR) or one of the Validation Regions (VR) of the analysis: $m_{j\ell\ell}<250$~GeV. This variable is used in the fit used to extract the final result, thus the large high side window is to allow sufficient background to determine the background normalisation from data, and to populate a VR in the side-band of the variable. The 8~GeV $a$ signal sample fails this requirement more often, because the higher mass resonance has more of its components reconstructed outside the calorimeter jet. The majority of events contain multiple jets which could be taken as the resonance candidate, in which case the jet with the highest $p_\text{T}$ is chosen. This criteria correctly selects the resonance in most cases (see Table~\ref{tab:correctselection}), while not sculpting the background distribution of the invariant mass of the three body system. The correct reconstruction efficiency was also higher than in the three body mass case.

%% \begin{table}[!htbp]{\footnotesize\renewcommand{\arraystretch}{1.2}
%%     \begin{center}
%%       \begin{tabular}{|c|c|}
%%         \hline
%%         Signal Hypothesis & Correct Jet Efficiency \\
%%         \hline
%%         0.5~GeV $a$ & 85$\pm$3\%\\
%%         2.5~GeV $a$ & 89$\pm$3\%\\
%%         8.0~GeV $a$ & 71$\pm$3\%\\
%%         $\eta_c$ &  91$\pm$3\%\\
%%         \hline
%%       \end{tabular}
%%       \caption{Efficiency with which the correct resonance is reconstructed as the highest $p_\text{T}$ jet. Correct jet determined using a $\Delta R<0.3$ requirement from the generator level resonance. Only events in which the generator level signal resonance has a $p_\text{T}>20$~GeV, and $|\eta|<2.5$ are considered in this calculation.}
%%       \label{tab:correctselection}
%%   \end{center}}
%% \end{table}

Kinematic distributions for the selected calorimeter jet are shown in Figure~\ref{fig:calojetkinematics}. Kinematic distributions for the leptons chosen to reconstruct the $Z$ boson are shown Figure~\ref{fig:leptonkinematics}. Kinematic distributions for the reconstructed $Z$ boson (the sum of the 2 lepton 4-momenta) are shown in Figure~\ref{fig:zkinematics}. Kinematic distributions for the reconstructed Higgs boson (the sum of the 2 lepton and the jet 4-momenta) are shown in Figure~\ref{fig:higgskinematics}. Various event level angular distributions are shown in Figure~\ref{fig:drdistributions}.

The signal peak in Figure~\ref{fig:higgskinematics}(b) is not centred on 125~GeV due to the calibration of the calorimeter jets used in the construction of this variable being designed for QCD jets, rather than the jets produced from the decay of one of these light resonances. This is demonstrated by Figure~\ref{fig:a0calibration}. The central value shifts and resolution of the three body mass peak degrades for high $a$ masses, due to the presence of neutrinos from decaying charm-quarks and $\tau$-leptons produced in the decay of the resonance.

\begin{figure}[!htbp]
  \centering
  \subfigure[$p_\text{T}$]{\includegraphics[width=0.475\textwidth]{figures/Plots_09Feb20/All/CaloJetPt.pdf}}
  %% \subfigure[$m$]{\includegraphics[width=0.475\textwidth]{figures/Plots_09Feb20/All/CaloJetM.pdf}}\\
  \subfigure[$\eta$]{\includegraphics[width=0.475\textwidth]{figures/Plots_09Feb20/All/CaloJetEta.pdf}}\\
  \subfigure[$\phi$]{\includegraphics[width=0.475\textwidth]{figures/Plots_09Feb20/All/CaloJetPhi.pdf}}
  \caption{Kinematic distributions of the leading jet in the event, after the full event-level pre-selection. Data, signal and background distributions are shown. The background in these distributions has been reweighted as per Section~\ref{sec:bkgdrw}.}
  \label{fig:calojetkinematics}
\end{figure}

\begin{figure}[!htbp]
  \centering
  \subfigure[$p_\text{T}^{\text{lead\ lepton}}$]{\includegraphics[width=0.475\textwidth]{figures/Plots_09Feb20/All/LeptonLPt.pdf}}
  \subfigure[$p_\text{T}^{\text{sublead\ lepton}}$]{\includegraphics[width=0.475\textwidth]{figures/Plots_09Feb20/All/LeptonSLPt.pdf}}\\
  \subfigure[$\eta^{\text{lead\ lepton}}$]{\includegraphics[width=0.475\textwidth]{figures/Plots_09Feb20/All/LeptonLEta.pdf}}
  \subfigure[$\eta^{\text{sublead\ lepton}}$]{\includegraphics[width=0.475\textwidth]{figures/Plots_09Feb20/All/LeptonSLEta.pdf}}
  \caption{Kinematic distributions of the leptons used to reconstruct the $Z$ boson, after the full event-level pre-selection. Data, signal and background distributions are shown. Note the 18~GeV lepton requirement from the \emph{FTAG2} derivation. The background in these distributions has been reweighted as per Section~\ref{sec:bkgdrw}.}
  \label{fig:leptonkinematics}
\end{figure}

\begin{figure}[!htbp]
  \centering
  \subfigure[$p_\text{T}$]{\includegraphics[width=0.475\textwidth]{figures/Plots_09Feb20/All/ZPt.pdf}}
  \subfigure[$m$]{\includegraphics[width=0.475\textwidth]{figures/Plots_09Feb20/All/ZM.pdf}}\\
  \subfigure[$\eta$]{\includegraphics[width=0.475\textwidth]{figures/Plots_09Feb20/All/ZEta.pdf}}
  \subfigure[$\phi$]{\includegraphics[width=0.475\textwidth]{figures/Plots_09Feb20/All/ZPhi.pdf}}
  \caption{Kinematic distributions of the reconstructed $Z$ boson (the sum of the 2 lepton 4-momenta) in the event, after the full event-level pre-selection. Data, signal and background distributions are shown. The background in these distributions has been reweighted as per Section~\ref{sec:bkgdrw}.}
  \label{fig:zkinematics}
\end{figure}

\begin{figure}[!htbp]
  \centering
  \subfigure[$p_\text{T}$]{\includegraphics[width=0.475\textwidth]{figures/Plots_09Feb20/All/HPt.pdf}}
  \subfigure[$m$]{\includegraphics[width=0.475\textwidth]{figures/Plots_09Feb20/All/HM_SR.pdf}}\\
  \subfigure[$\eta$]{\includegraphics[width=0.475\textwidth]{figures/Plots_09Feb20/All/HEta.pdf}}
  \subfigure[$\phi$]{\includegraphics[width=0.475\textwidth]{figures/Plots_09Feb20/All/HPhi.pdf}}
  \caption{Kinematic distributions of the reconstructed Higgs boson (the sum of the 2 lepton and calo jet 4-momenta) in the event, after the full event-level pre-selection. Data, signal and background distributions are shown. The background in these distributions has been reweighted as per Section~\ref{sec:bkgdrw}.}
  \label{fig:higgskinematics}
\end{figure}

\begin{figure}[!htbp]
  \centering
  \subfigure[$\text{Max}(\Delta R^{\ell^-\text{jet}}, \Delta R^{\ell^+jet})$]{\includegraphics[width=0.475\textwidth]{figures/Plots_09Feb20/All/DeltaRLJetMax.pdf}}
  \subfigure[$\text{Min}(\Delta R^{\ell^-\text{jet}}, \Delta R^{\ell^+\text{jet}})$]{\includegraphics[width=0.475\textwidth]{figures/Plots_09Feb20/All/DeltaRLJetMin.pdf}}\\
  \subfigure[$\Delta R^{\ell^-\ell^+}$]{\includegraphics[width=0.475\textwidth]{figures/Plots_09Feb20/All/DeltaRLepLep.pdf}}
  \subfigure[$\Delta R^{Z\text{jet}}$]{\includegraphics[width=0.475\textwidth]{figures/Plots_09Feb20/All/DeltaRZJet.pdf}}
  \caption{Event level angular ($\Delta R$) distributions between the various objects in the event, after the full event-level pre-selection. Data, signal and background distributions are shown. The background in these distributions has been reweighted as per Section~\ref{sec:bkgdrw}.}
  \label{fig:drdistributions}
\end{figure}

\begin{figure}[!htbp]
  \centering
  \includegraphics[width=0.7\textwidth]{figures/CaloJetPtCalibration_09Feb20.pdf}
  \caption{$(p_\text{T, calo jet}-p_{\text{T, }a})/p_{\text{T, }a}$ distribution.}
  \label{fig:a0calibration}
\end{figure}


\subsection{Track Selection}
\label{sec:trackselection}

Ultimately, a track-based discriminant will be used to separate the signal resonance from the background QCD jets. These tracks are selected by a method know as Ghost-Association~\cite{Cacciari:2007fd}, in which the tracks in the event are assigned a negligible $p_\text{T}$, and then the jet reconstruction algorithm is re-run, on both the calorimeter clusters (as per the original algorithm) and these new Ghost-tracks; any tracks clustered in the jet are considered Ghost-Associated to it.

The majority of tracks Ghost-Associated to the jet come from pileup, the underlying event, multiple parton interactions and fake tracks. This is demonstrated by using the track to truth particle link provided in the \emph{xAOD} to identify the generator-level particle responsible for the track, and then following the truth record upwards to see if the particle originated from a decay of the signal resonance. The $p_\text{T}$ distributions of the tracks Ghost-Associated to the calorimeter jet, with and without truth matching, are shown in Figure~\ref{fig:truthmatchtracks}. The $p_\text{T}$ distributions of the tracks which are not truth matched to the signal resonance follow that of pileup, while the signal events have a significantly higher $p_\text{T}$ spectrum. This lower $p_\text{T}$ spectrum in fakes means that most of the variables constructed from these tracks, which are dependant on the $p_\text{T}$ of the jet, are somewhat resilient to change from the addition of fake tracks.

\begin{figure}[!htbp]
  \centering
  \subfigure[Generator-Level Matched]{\includegraphics[width=0.475\textwidth]{figures/TrackStudyPlotsRemade_09Feb20/trackPt_withTM.png}}
  \subfigure[Generator-Level Anti-Matched]{\includegraphics[width=0.475\textwidth]{figures/TrackStudyPlotsRemade_09Feb20/trackPt_noTM.png}}
  \caption{$p_\text{T}$ distributions of the tracks Ghost-Associated to the calorimeter jet, with and without truth matching. Only signal distributions are shown here.}
  \label{fig:truthmatchtracks}
\end{figure}

In order to reject the large contamination from fake tracks, the ATLAS \emph{Loose} Track Quality and \emph{Loose} track-to-vertex association (TTVA) requirements are applied, which are implemented using the \emph{InDetTrackSelectionTool}~\cite{InDetTrackSelectionTool} and \emph{TrackVertexAssociationTool}~\cite{TrackVertexAssociationTool}, respectively. The \emph{Loose} Track Quality working point requires that the track has $p_\text{T}>500$ MeV, $\vert\eta\vert<2.5$, at least 7 silicon hits, at most one shared module, at most 1 hole in the pixel, and at most 2 holes in the pixels or strips. The \emph{Loose} TTVA working point requires that $\vert d_0\vert <2$ mm and $\vert \Delta z_0sin\theta\vert <3$ mm, where $d_0$ and $z_0$ are the transverse and longitudinal impact parameters, respectively. The effect of this requirement on the track assisted mass are shown in Figure~\ref{fig:masseswithtrackselection}. The track assisted mass is defined as the mass of the track system multiplied by the ratio of the $p_\text{T}$ of the jet as measured in the calorimeter and tracker. This is not used later in the analysis, but provides a reasonable proxy for the mass, which illustrates the effectiveness of the track-selection.

\begin{figure}[!htbp]
  \centering
  \subfigure[Without Track Selection]{\includegraphics[width=0.475\textwidth]{figures/TrackStudyPlotsRemade_09Feb20/mTA_noQR.png}}
  \subfigure[With Track Selection]{\includegraphics[width=0.475\textwidth]{figures/TrackStudyPlotsRemade_09Feb20/mTA_withQR.png}}
  \caption{Track assisted mass distributions of the calorimeter jet, before and after the \emph{Loose} TTVA and track selection requirements. Data, signal and background distributions are shown.}
  \label{fig:masseswithtrackselection}
\end{figure}

The resulting discriminant (Section~\ref{sec:mlpselection}) relies on there being multiple tracks Ghost-Associated to the jet. As such, any events in which the selected jet has less than two Ghost-Associated tracks which passing the requirements detailed in this section are discarded.%% The requirements of the pre-selection (Section~\ref{sec:premlpselection}) and this track number requirement are summarised in Table~\ref{tab:selection}, which also gives example cutflows.

%% \begin{figure}[!htbp]
%%   \centering
%%   \subfigure[]{\includegraphics[width=0.475\textwidth]{}}
%%   \subfigure[]{\includegraphics[width=0.475\textwidth]{}}\\
%%   \subfigure[]{\includegraphics[width=0.475\textwidth]{}}
%%   \subfigure[]{\includegraphics[width=0.475\textwidth]{}}
%%   \caption{.}
%%   \label{figure:}
%% \end{figure}


\subsection{Track-Based Multi-Layer-Perceptron}
\label{sec:mlpselection}

A multi-layer perceptron (MLP) is used to discriminate between jets arising from hadronic $X$ decays, and background multijet production. \textsc{TMVA} is used to train the MLP. As these are thin ($R=0.4$) jets, the information from the inner tracker is expected to be more useful due to the improved angular and $p_\text{T}$ resolution. The reconstruction and modelling of the tracks are also better understood than that of the calorimeter objects. We therefore use the set of Ghost-Associated tracks, selected as per subsection~\ref{sec:trackselection}, as the basis for the inputs variables to this MLP.

%% Given a set of Ghost-Associated tracks, selected as per subsection~\ref{sec:trackselection}. Variables defined on these tracks are selected based on their ability to separate the various signals from the total background, accounting for correlations between them. These variables are summarised in Table~\ref{tab:mlpvars}, and displayed in Figures~\ref{fig:calojetkinematics},~\ref{fig:leadtrackvars},~\ref{fig:tauvars},~\ref{fig:modcfvars},~and~\ref{fig:angularity2}. Other variables considered are described in Appendix~\ref{app:nonmlpvars}. Further studies (Appendix~\ref{app:mvavarreductionstudies}) show that similar performance can be achieved without the $\eta^\text{calo\ jet}$ variable, which will be removed in the next data-processing.
Variables defined on these tracks are selected based on their ability to separate the various signals from the total background. These variables are summarised in Table~\ref{tab:mlpvars}, briefly described in the next paragraph, and displayed in Figures~\ref{fig:leadtrackvars},~\ref{fig:modcfvars},~and~\ref{fig:angularity2}. These variables are chosen to be dimensionless in order to reduce their correlation with event-level kinematic quantities; also minimising the correlation between the MVA output and the event-level kinematic quantities. The correlation coefficient between each of these variables and the three body mass is also shown in Table~\ref{tab:mlpvars}, with none having a correlation greater than 20\%. What correlation remains between the three body mass and these variables occurs primarily through the transverse momentum of the calorimeter jet.

\begin{table}[!htbp]{\footnotesize\renewcommand{\arraystretch}{1.2}
    \begin{center}
      \begin{tabular}{|c|c|c|}
        \hline
        Variable & Description & $m_{\ell\ell\text{j}}$ Correlation \\
        \hline
        $p_\text{T}^{\text{lead\ track}}/p_\text{T}^{\text{tracks}}$ & Ratio of transverse momentum of the leading track to total & 7.5\% \\
        $\Delta R^\text{lead\ track,\ calo\ jet}$ & $\Delta R$ between the leading track and the calorimeter jet axis & 19.6\% \\
        $\tau_{\text{2}}$ & NSubJettiness 2~\cite{Thaler:2010tr} & 1.0\% \\
        $U_1(0.7)$ & Modified energy correlation function (2, 1, 0.7)~\cite{Moult:2016cvt} & 9.4\% \\
        $M_2(0.3)$ & Ratio of modified energy correlation functions (3, 1, 0.3) and (2, 1, 0.3)~\cite{Moult:2016cvt} & 12.5\% \\
        $angularity(2)$ & Angularity 2~\cite{Almeida:2008yp} & 13.4\% \\
        \hline
      \end{tabular}
      \caption{Variables chosen to discriminate the resonance signals from the combined background. All tracks have been Ghost-Associated to the calorimeter jet as per subsection~\ref{sec:trackselection}. The Pearson correlation coefficient between each of these variables and the three body mass is shown in the last column for the background.}
      \label{tab:mlpvars}
  \end{center}}
\end{table}

\begin{figure}[!htbp]
  \centering
  \subfigure[$p_\text{T}^{\text{lead\ track}}/p_\text{T}^{\text{tracks}}$]{\includegraphics[width=0.475\textwidth]{figures/Plots_09Feb20/All/GhostTracks_leadTrackPtRatio.pdf}}
  \subfigure[$\Delta R^{\text{lead\ track}}$]{\includegraphics[width=0.475\textwidth]{figures/Plots_09Feb20/All/GhostTracks_deltaRLeadTrack.pdf}}\\
  \caption{MLP input distributions based on the leading track Ghost-Associated to the jet, after the full event-level pre-selection. Data, signal and background distributions are shown. The background in these distributions has been reweighted as per Section~\ref{sec:bkgdrw}.}
  \label{fig:leadtrackvars}
\end{figure}

%% \begin{figure}[!htbp]
%%   \centering
%%   \subfigure[$\tau_1$]{\includegraphics[width=0.475\textwidth]{figures/Plots_09Feb20/All/GhostTracks_tau1.pdf}}
%%   \subfigure[$\tau_2$]{\includegraphics[width=0.475\textwidth]{figures/Plots_09Feb20/All/GhostTracks_tau2.pdf}}
%%   \caption{MLP input distributions based on the NSubJettiness variable, after the full event-level pre-selection. The $\tau_1$ variable is given to the MLP trained on the 0.5~GeV $a$ signal sample, while the $\tau_2$ variable is used in all other cases. Data, signal and background distributions are shown.}
%%   \label{fig:tauvars}
%% \end{figure}

\begin{figure}[!htbp]
  \centering
  \subfigure[$U1(0.7)$]{\includegraphics[width=0.475\textwidth]{figures/Plots_09Feb20/All/GhostTracks_U1_0p7.pdf}}
  \subfigure[$M2(0.3)$]{\includegraphics[width=0.475\textwidth]{figures/Plots_09Feb20/All/GhostTracks_M2_0p3.pdf}}
  \caption{MLP input distributions based on the modified correlation functions, after the full event-level pre-selection. Data, signal and background distributions are shown. The background in these distributions has been reweighted as per Section~\ref{sec:bkgdrw}.}
  \label{fig:modcfvars}
\end{figure}

\begin{figure}[!htbp]
  \centering
  \subfigure[$\tau_2$]{\includegraphics[width=0.475\textwidth]{figures/Plots_09Feb20/All/GhostTracks_tau2.pdf}}
  \subfigure[$angularity(2)$]{\includegraphics[width=0.475\textwidth]{figures/Plots_09Feb20/All/GhostTracks_angularity_2.pdf}}
  \caption{MLP input distributions based on the NSubJettiness and angularity variables, after the full event-level pre-selection. Data, signal and background distributions are shown. $\tau_2$ is defined to be 0 for less than three tracks. The background in these distributions has been reweighted as per Section~\ref{sec:bkgdrw}.}
  \label{fig:angularity2}
\end{figure}



$p_\text{T}^{\text{lead\ track}}/p_\text{T}^{\text{tracks}}$ is the ratio of transverse momentum of the highest $p_\text{T}$ track to transverse momentum of the Ghost-Associated track system. $\Delta R^\text{lead\ track,\ calo\ jet}$ is the angular separation ($\Delta R$) between the highest $p_\text{T}$ track and the calorimeter jet axis. $\tau_{\text{2}}$ is the NSubJettiness 2 variable~\cite{Thaler:2010tr}, where NSubJettiness is a measure of how well the jet can be described in terms of two sub-jets. It is used due to the fact that most of the decays of the resonances of interest are to pairs of final state partons. It is defined as $\tau_2 = \Sigma_{t} p_{\text{T,}t}\text{min}(\Delta R_{1,t},\Delta R_{2,t})/\Sigma_{t}p_\text{T,}tR_0$, where the sums run over the Ghost-Associated tracks, the angles $\Delta R_{i,t}$ are between the track and one of two subjets, and $R_0$ is the radius parameter of the larger jet (0.4). The two subjets are chosen using the exclusive $k_T$ sub-jet algorithm~\cite{Thaler:2010tr}. $U_1(0.7)$ and $M_2(0.3)$ are both modified energy correlation functions~\cite{Moult:2016cvt}, designed for quark-gluon discrimination and to target 2-pronged substructure, respectively. $U_1(0.7)={}_1e^{(0.7)}_{2}$ and $M_2(0.3)={}_1e^{(0.3)}_{3}/{}_1e^{(0.3)}_{2}$, where ${}_1e^{(\beta)}_2 = \Sigma_{1\leq i < j \leq n} z_iz_j\theta^\beta_{ij}$ and ${}_1e^{(\beta)}_3 = \Sigma_{1\leq i < j < k \leq n} z_iz_jz_k\text{min}(\theta^\beta_{ij},\theta^\beta_{ik},\theta^\beta_{jk})$, and $z_i$ is the ratio of the $p_\text{T}$ of the track, to that of the track system. Lastly, $angularity(2)$ is an angularity variable, defined analogously to Ref.~\cite{Almeida:2008yp} as $angularity(2)=\Sigma_{t}p_{\text{T},t}\sin^2(\pi\theta_t/2R_0)(1-\cos(\pi\theta_t/2R))^{-1}$. The form of this variable is motivated by the different matrix elements for resonance- and QCD-induced jets, as detailed in Ref.~\cite{Almeida:2008yp}.


%% The NSubJettiness ($\tau$), $angularity$, and modified correlation function variables are detailed in References~\cite{Thaler:2010tr},~\cite{Almeida:2008yp},~and~\cite{Moult:2016cvt}, respectively.

This is not a typical classification task, as there is a continuous spectrum of signals being classified against a large background distribution. As such, before being given to a classification algorithm, the input variables are first given to a regression multi-layer-perceptron in \textsc{TMVA}. This MLP is trained on the 0.5, 0.75, 1, 1.5, 2, 2.5, 3, 3.5 and 4~GeV $a$ signal samples, and is designed to estimate the mass of the $a$ based on the input variables. Only events with $100<m^{\ell\ell\text{j}}<170$~GeV were used. It uses the default \textsc{TMVA} hyper-parameter values, and an architecture of 4 layers of 12 neurons. These default hyper-parameters are: NCycles=500, NeuronType=sigmoid, EstimatorType=Mean Square Estimator, NeuronInputType=sum, TrainingMethod=Back-Propagation, LearningRate=0.02, DecayRate=0.01, and TestRate=10. Negative weight events are ignored in the training of the MLPs. This regression is then given, along with the original input variables, to a classification MLP. This way, the classifier has been indirectly informed that it is searching for a spectrum of signals, rather than just a single signal; and it has some information with which to estimate which signal it is looking at in the event of a signal being present in the data. Various other architectures and hyper-parameters were tried, and these were found to be optimal. The regression output variable is shown in Figure~\ref{fig:mlpregression}.


\begin{figure}[!htbp]
  \centering
  \includegraphics[width=0.7\textwidth]{figures/PaperPlots_09Feb20/MLP_massRegression_coarse.pdf}
  \caption{Output of the regression MLP, for data, background and three signal hypotheses. Events are required to pass the complete event selection, including the $120~\text{GeV}<m_{\ell^+\ell^-\text{j}}<135~\text{GeV}$ requirement, but not the requirement on the classification MLP output variable. The background normalisation is set equal to that of the data, and the signal normalisations assume the SM Higgs production cross section and $\mathcal{B}(H\to Za)=100\%$ with the signal normalisation is scaled up by a factor of 100. The background in these distributions has been reweighted as per Section~\ref{sec:bkgdrw}. The error bars are the data statistical uncertainty, and the dashed bars are the MC statistical uncertainty.}
  \label{fig:mlpregression}
\end{figure}


This output variable of the regression MLP is then given, along with the original input variables, to a classification MLP. This MLP uses the default \textsc{TMVA} hyper-parameters (listed in the previous paragraph), and an architecture of 2 layers of 6 and 5 neurons, to separate the background from the sum of the following signals: 0.5, 1, 1.5, 2, 2.5, 3, 3.5 and 4~GeV $a$ signal samples. As with the regression MLP, various other architectures and hyper-parameters were tried, and these were found to be optimal. Also as with the regression MLP, only events with $100<m^{\ell\ell\text{j}}<170$~GeV were used. The 0.75~GeV $a$ signal sample was removed from the training because it was found to bias the classifier towards lower mass signal samples, resulting in a deterioration of the performance towards higher masses. The classification output variable is shown in Figure~\ref{fig:mlpclassification}, and the ROC curves for the testing and training events are given inclusively in Figure~\ref{fig:ROCs}, and separately in Figure~\ref{fig:ROCssep}. The ROC curves show good signal to background discrimination (average $\sim 50\%$ signal efficiency for $\sim 90\%$ background rejection), as well as little-to-no overtraining.% Finally, the signal purities in 15 MLP bins are shown for various signal hypotheses in Figure~\ref{fig:MLPSignalPurity}.


\begin{figure}[!htbp]
  \centering
  \includegraphics[width=0.7\textwidth]{figures/PaperPlots_09Feb20/MLP_regInID_coarse.pdf}
  \caption{Output of the classification MLP, for data, background and three signal hypotheses. Events are required to pass the complete event selection, including the $120~\text{GeV}<m_{\ell^+\ell^-\text{j}}<135~\text{GeV}$ requirement, but not the requirement on the classification MLP output variable. The background normalisation is set equal to that of the data, and the signal normalisations assume the SM Higgs production cross section and $\mathcal{B}(H\to Za)=100\%$. The background in these distributions has been reweighted as per Section~\ref{sec:bkgdrw}. The error bars are the data statistical uncertainty, and the dashed bars are the MC statistical uncertainty. The region to the right of the dashed line is the signal region.}
  \label{fig:mlpclassification}
\end{figure}


\begin{figure}[!htbp]
  \centering
  \includegraphics[width=0.8\textwidth]{figures/ROC_20Jan20.pdf}
  \caption{ROC curves for the testing events and the training events.}
  \label{fig:ROCs}
\end{figure}


\begin{figure}[!htbp]
  \centering
  \includegraphics[width=0.8\textwidth]{figures/IndividualROCs_11Feb20.pdf}
  \caption{ROC curves for the individual signal samples, with the background curve given for reference.}
  \label{fig:ROCssep}
\end{figure}


%% \begin{figure}[!htbp]
%%   \centering
%%   \includegraphics[width=0.7\textwidth]{figures/SignalPurity.png}
%%   \caption{Signal purities in 15 MLP bins are shown for various signal hypotheses. The MLP cut value is also shown by the red vertical dashed line. The background in these distributions has been reweighted as per Section~\ref{sec:bkgdrw}, and the signal is normalised assuming BR$(H\to Za)=100\%$.}
%%   \label{fig:SignalPuritys}
%% \end{figure}

The design of this MVA was assisted, and concepts validated, by the use of \textsc{Pythia}8 MC samples, with a \textsc{Delphes}~\cite{deFavereau:2013fsa} detector simulation using the ATLAS card. Signal samples were produced with $a$ masses of: 0.4, 0.5, 0.75, 0.8, 1, 1.2, 1.5, 1.6, 2, 2.4, 2.5, 2.8, 3, 3.2, 3.5, 3.6 and 4~GeV. Each signal sample had 100k signal events, and 20M $Z+$jets background events were simulated. These samples were not used in any of the results in this note directly, but were used to test the concepts which were then implemented using the nominal MC samples.

MLPs were used as the chosen MVA because they showed good interpolation potential. This was tested using both \textsc{Delphes}- and \textsc{Geant}4-based MC by removing a signal sample from the training of the MVA, and comparing the performance of the MVAs, with and without the exclusion of this signal sample in the training, on this signal sample. Removing the 2.2~GeV $a$ mass \textsc{Delphes} sample, the MLP was found to have an $S/\sqrt{B}$ 11\% lower than with the sample included in the training. This is to be compared to a 30\% loss in $S/\sqrt{B}$ improvement for a boosted decision tree (BDT) with the 2.2~GeV $a$ mass removed from the training, showing that the BDT looses much more performance when interpolating to mass points for which is was not trained. The MLP also showed greater overall performance and less overtraining as compared to the BDT.

%% These variables were then used to train a Multi-Layer-Perceptron (MLP) in \textsc{TMVA}, which has the default parameter values of: NTrees=800; MaxDepth=3; MinNodeSize=5\%; nCuts=20; BoostType=AdaBoost; AdaBoostR2Loss=Quadratic; AdaboostBeta=0.5. Other values of these parameters were tried, and were found to not offer significant improvement in the signal to background separation over the various signal samples. These hyper-parameters, and optimisation studies are described in Appendix~\ref{app:mlphyperparameters}. The output distributions (after the full non-MLP event selection) are shown in Figure~\ref{fig:mlpoutputdistributions}.

A single requirement on the resulting MLP is then used to reject background events. The requirement is chosen to maximise $S/\sqrt{B}$, as this minimises the uncertainty on the signal strength, and the limit, in the high stats limit. However, as there are multiple signal samples, the cut is chosen to maximise the average $S/\sqrt{B}$, with each signal sample given a weight. This weight is calculated as the $a$ mass phase space which it represents, i.e. the phase space which is closest to that mass hypothesis. For example, for the 1.5~GeV $a$ signal sample, because there is also a 1~GeV and 2~GeV $a$ signal sample, it represents 0.5~GeV of phase space. This also maximises the expected $S/\sqrt{B}$, assuming a flat Bayesian prior on the signal mass. This results in an MLP cut at 0.0524, which results in a background efficiency of 1.01\% for $110\ \text{GeV}\ <\ m^{\ell\ell\text{j}}\ <\ 170\ \text{GeV}$, and signal efficiencies as given in Table~\ref{tab:mlpwps}. The inclusion of the regression output variable in the classification MVA was found to result in an improvement in $S/\sqrt{B}$ of 13\%, when averaged across the signal samples with an $S/\sqrt{B}$ gain due to the MLP greater than unity. The MLP efficiency for the 8~GeV $a$ signal sample is less than that for the background because the MLP was trained to target the lower masses, for which the analysis is most sensitive. However, this is not an issue because the analysis will only target $a$ with masses below 4~GeV. % /sqrt(0.01013315918)

\begin{table}[!htbp]{\footnotesize\renewcommand{\arraystretch}{1.2}
  \begin{center}
    \footnotesize
    %% BDT Optimisation Results\\
    \begin{tabular}{|c|cc|}
      \hline
      $a$ mass /~GeV & MLP Eff (\%) & MLP $S/\sqrt{B}$ Gain \\
      \hline
      0.5 & $45.9 \pm 0.8$ & 5.26 \\
      0.75 & $42.1 \pm 0.8$ & 4.83 \\
      1 & $38.2 \pm 0.7$ & 4.38 \\
      1.5 & $31.5 \pm 0.6$ & 3.61 \\
      2 & $25.1 \pm 0.5$ & 2.87 \\
      2.5 & $15.4 \pm 0.4$ & 1.76 \\
      3 & $8.06 \pm 0.29$ & 0.924 \\
      3.5 & $5.70 \pm 0.24$ & 0.653 \\
      4 & $1.88 \pm 0.15$ & 0.216 \\
      \hline
      $\eta_c$ & $5.89 \pm 0.24$ & 0.675 \\
      $J/\psi$ & $6.66 \pm 0.26$ & 0.763 \\
      \hline
    \end{tabular}
    \caption{Efficiencies of the $MLP>0.0524$ requirement on each signal sample. This requirement results in a background efficiency of $(0.761\pm 0.020)\%$ for $120\ \text{GeV}<m^{\ell^+\ell^-\text{j}}<135\ \text{GeV}$. The $S/\sqrt{B}$ gains are defined as the ratio of the $S/\sqrt{B}$ after the MLP selection to before the MLP selection. \textsc{Pythia}8 $a$ BRs are assumed, using the default \textsc{BSMHiggs} $\tan\beta$ value of 1, as given in Table~\ref{tab:a0brs}.}
    \label{tab:mlpwps}
  \end{center}}
\end{table}


The performance of the MLP is then tested on both the testing and training samples to check for overtraining. No statistically significant overtraining is seen in any of the signal or background samples. Considering background events with $100\ \text{GeV}\ <\ m^{\ell\ell\text{j}}\ <\ 175\ \text{GeV}$, 3100 more events from the training sample than the validation sample pass the MLP cut. This is the opposite of what would be expected if overtraining were present, though it is consistent with equality within 0.42 times the statistical uncertainty. Taking the 1.5~GeV signal sample as an example, and considering events with $100\ \text{GeV}\ <\ m^{\ell\ell\text{j}}\ <\ 175\ \text{GeV}$, 43 more events from the training sample than the validation sample pass the MLP cut. This is consistent with equality within 0.48 times the statistical uncertainty.


\subsection{Full Selection}
\label{sec:finalselection}

The full selection is summarised in Table~\ref{tab:selection}, and a cutflow is provided in Appendix~\ref{app:cutflow}.

%% \begin{table}[!htbp]{\footnotesize\renewcommand{\arraystretch}{1.2}
%%     \begin{center}
%%       \begin{tabular}{|c|C{0.4\textwidth}|cccc|}
%%         \hline
%%         Cut & Details & 0.5~GeV $a$ & 2.5~GeV $a$ & 8~GeV $a$ & $\eta_c$ \\
%%         \hline
%%         %% None & & 100\% & 100\% & 100\% & 100\% \\
%%         Triggers & Lowest unprescaled single lepton triggers & 54\% & 53\% & 54\% & 54\% \\
%%         Leptons & $e\geq 2$ with $p_\text{T}>7\ \text{GeV}$ or $\mu\geq 2$~with~$p_\text{T}>5\ \text{GeV}^*$ & 35\% & 35\% & 35\% & 35\% \\
%%         $Z$ boson & 2 same-flavour opposite-sign leptons, with $|m^{ll}-m_Z|<10\ \text{GeV}$ and $p_\text{T}^{lead}>27\ \text{GeV}$ & 30\% & 30\% & 30\% & 30\% \\
%%         \hline
%%         \multicolumn{6}{|c|}{Select $X$-candidate as anti-$k_T$ 4 topo EM jet $(p_\text{T}^{jet}>20\ \text{GeV})$, with highest $p_\text{T}$, for which $m^{llj}<250\ \text{GeV}$}\\
%%         \hline
%%         %% Higgs boson & $100\ \text{GeV}<m^{llj}<200\ \text{GeV}$ & 20\% & 21\% & 17\% & 22\% \\
%%         $>2$ tracks & $\geq 2$ tracks ghost associated to the calo jet, surviving track selection & 19\% & 20\% & 16\% & 21\% \\
%%         \hline
%%       \end{tabular}
%%   \end{center}}
%%   \caption{Summary of full pre-MLP event selection (Section~\ref{sec:premlpselection}), and the track number requirement of Section~\ref{sec:trackselection}, with the running efficiency for various signal hypotheses.}
%%   \label{tab:selection}
%% \end{table}

\begin{table}[!htbp]{\footnotesize\renewcommand{\arraystretch}{1.2}
    \begin{center}
      \begin{tabular}{|c|c|}
        \hline
        Cut & Details \\
        \hline
        Triggers & Single lepton triggers requiring $p_\text{T}>27\ \text{GeV}$ \\
        Leptons & $e\text{ or }\mu\geq 2$ with $p_\text{T}>18\ \text{GeV}$ \\
        $Z$ boson & 2 same-flavour opposite-sign leptons, with $|m^{ll}-m_Z|<10\ \text{GeV}$ and $p_\text{T}^{lead}>27\ \text{GeV}$ \\
        \hline
        \multicolumn{2}{|c|}{Select $X$-candidate as anti-$k_T$ 4 topo EM jet $(p_\text{T}^{jet}>20\ \text{GeV})$, with highest $p_\text{T}$, for which $m^{llj}<250\ \text{GeV}$}\\
        \hline
        $>2$ tracks & $\geq 2$ tracks ghost associated to the calo jet, surviving track selection \\
        Higgs boson & $120\ \text{GeV}<m^{llj}<135\ \text{GeV}$ \\
        MLP & $MLP>0.0524$ \\
        \hline
      \end{tabular}
      \caption{Summary of full event selection.}
      \label{tab:selection}
  \end{center}}
\end{table}


\subsection{Generator-Level Acceptance}
\label{sec:truthacceptance}

To quantify the kinematic acceptance of this analysis to the different signal hypotheses under consideration, a generator-level acceptance is optimised to be independent of the detector performance. This selection closely follows the requirements of the full selection described in subsection~\ref{sec:finalselection}, but without the requirements placed on the track system, the MLP requirements, or the three body mass. This results in a generator-level acceptance which varies between 27\% and 29\% for the various signal hypotheses. The efficiency of the full (and fully reconstructed) selection efficiency, with these cuts relaxed varies between 16\% and 21\% for the various signal hypotheses. Thus the efficiency of the detector for signal events falling within its acceptance is approximately 65\%.


%-------------------------------------------------------------------------------
\clearpage
\section{Signal and Background Modelling}
\label{sec:modelling}
%-------------------------------------------------------------------------------

%% \subsection{Analysis Regions}
%% \label{sec:regions}

%% Before the results are extracted from the statistical model, two event-categories are defined. First, the Signal Region (SR), which is defined as events passing both the MLP cut (Section~\ref{sec:mlpselection}), and with $120<m^{\ell\ell\text{j}}<135$~GeV. The SR was designed to maximise $S/\sqrt{B}$. Second, the Control Region (CR), which is defined as events passing both the MLP cut (Section~\ref{sec:mlpselection}), and with either $100<m^{\ell\ell\text{j}}<110$~GeV or $155<m^{\ell\ell\text{j}}<175$~GeV. The CR was designed to contain $<1\%$ of the signal, while maximising the contained background, and minimising the distance from the SR. While the lower $m^{\ell\ell\text{j}}$ region of the CR contains only a low level of background, its inclusion symmetrises the CR about the SR, minimising the impact of systematic uncertainties on the relative yields in the SR and CR.


\subsection{Signal Modelling}
\label{sec:sigmodelling}

The signal efficiency for the selection is taken directly from MC. This scales the expected Higgs production yield, taken as the product of the luminosity (139~$\text{fb}^{-1}$) and the total SM Higgs production cross section (55.7~pb)\footnote{While the full inclusive Higgs production cross section is used in the normalisation, the MC samples were generated using the ggF production mode. A systematic uncertainty, described in Section~\ref{sec:syssigprodmodel}, is ascribed to account for this difference.}. The contributions to the total cross section are taken from the LHC Higgs Cross Section Working Group~\cite{xsecwg}. The total Higgs production cross section is taken as the sum of gluon fusion, vector boson fusion, $ZH$, $WH$, $b\bar bH$, $t\bar tH$ and $tH$ associated production. This is scaled by the branching fraction of the $Z$ boson to electrons, muons or tau-leptons, which is taken from the Particle Data Group to be 10.1\%~\cite{pdgZBRs}. Finally, this is scaled by the signal strength: $\mu = \sigma(H) \text{BR}(H\to Za) / \sigma_\text{SM}(H)$, to give the total number of expected signal events, assuming the default \textsc{Pythia} branching ratios given in Table~\ref{tab:a0brs}. $\mu$ is left free in the likelihood fit, as described in the subsection~\ref{sec:statmodel}. Table~\ref{tab:sigyields} shows the expected signal yields for each of the signal hypotheses considered, assuming $\mu=1$.%% A scale factor, and associated systematic uncertainty, accounting for a potential difference in acceptance between ggF and inclusive Higgs production may have to be assigned.

\begin{table}[!htbp]{\footnotesize\renewcommand{\arraystretch}{1.2}
  \begin{center}
    \footnotesize
    %% BDT Optimisation Results\\
    \begin{tabular}{|c|cc|}
      \hline
      $a$ mass /~GeV & Total Efficiency (\%) & Total Yield ($1000\times$) \\
      \hline
      0.5 & $3.27 \pm 0.06$ & $25.6 \pm 0.4$ \\
      0.75 & $2.76 \pm 0.05$ & $21.6 \pm 0.4$ \\
      1 & $2.86 \pm 0.05$ & $22.4 \pm 0.4$ \\
      1.5 & $2.50 \pm 0.05$ & $19.5 \pm 0.4$ \\
      2 & $2.00 \pm 0.04$ & $15.7 \pm 0.3$ \\
      2.5 & $1.30 \pm 0.03$ & $10.2 \pm 0.3$ \\
      3 & $0.692 \pm 0.025$ & $5.41 \pm 0.20$ \\
      3.5 & $0.505 \pm 0.021$ & $3.95 \pm 0.17$ \\
      4 & $0.140 \pm 0.011$ & $1.09 \pm 0.09$ \\
      \hline
      $\eta_c$ & $0.545 \pm 0.022$ & $4.26 \pm 0.17$ \\
      $J/\psi$ & $0.560 \pm 0.022$ & $4.37 \pm 0.17$ \\
      \hline
    \end{tabular}
    \caption{Efficiencies of the full selection (pre-selection and MLP requirement) and total expected signal yields (assuming $\mu=1$) for each signal sample. \textsc{Pythia}8 $a$ BRs are assumed, using the default \textsc{BSMHiggs} $\tan\beta$ value of 1, as given in Table~\ref{tab:a0brs}.}
    \label{tab:sigyields}
  \end{center}}
\end{table}

\subsection{Background Modelling}
\label{sec:bkgdmodelling}

A semi data-driven background model is used to estimate the SM background content in the signal region (SR), using three steps. First, the simulated background is reweighted to match the data to improve the modelling of key variables. Second, a fully data-driven ABCD estimate of the background in the SR is produced, which assumes no correlation between the three body mass and the MLP output variable. Third, the reweighted MC is used to correct the data-driven ABCD estimate for the correlation between the three body mass and the MLP output variable. Finally, this background estimation method is compared to data in 13 validation regions.

\subsubsection{Simulated Background Reweighting}
\label{sec:bkgdrw}

Before being used to construct the background estimate, the simulated data is reweighted to improve the modelling in the key variables used in the ABCD estimate correction. These variables are the three body mass, and the MLP output variable. The modelling of the MLP output variable is improved by improving the modelling of the input variables. All of the variables are reweighted against data in a blinded data region, consisting of the events passing the full selection except either the Higgs mass or the MLP requirements, but not passing both $110<m_{\ell\ell\text{j}}<155$~GeV and the MLP requirement.

It was observed that the three body mass is well modelled for each given number of ghost-associated tracks. Therefore, the mismodelling in the three body mass distribution is entirely due to the mismodelling in the ghost-associated track multiplicity. Hence, the ghost-associated track multiplicity is reweighted against data to improve the three body mass distribution.

Reweighting the simulated data based on the $U_1(0.7)$ variable was observed to improve the modelling of the other track-based substructure variables input to the MLP. In doing so, this improves the modelling of the regression and classification MLP output variables. However, this introduces a mismodelling in the $p_\text{T}$ distribution of the calorimeter jet and the ghost-associated track multiplicity, and thus in the three body mass.

To simultaneous improve the modelling in both the three body mass, and the MLP output variable, a fully-correlated 3D reweighting is applied based on $n_\text{tracks}$, $U_1(0.7)$, and $p_\text{T}^\text{jet}$. The reweighting is performed by applying corrections derived from the ratio of 3D histograms in data and background MC. Each value of $n_\text{tracks}$ between 2 and 6 has a dedicated bin in the reweighting, events with $n_\text{tracks}$ of 7 or 8 share a bin, and events with $n_\text{tracks}\geq 9$ share a bin. The $U_1(0.7)$ range 0 to 0.25 is split into 25 equal bins 0.01 wide, and one overflow bin is used for events with $U_1(0.7)>0.25$. The $p_\text{T}^\text{jet}$ region between 20~GeV and 50~GeV is split into 6 bins 5~GeV wide, the region 50~GeV to 60~GeV represents another bin, and the region above 60~GeV represents a final overflow bin. These three distributions before and after the reweighting procedure is applied are shown in Figure~\ref{fig:rwvars}. This results in significant improvements to the modelling of the three body mass, the MLP input variables and the MLP output variables, as shown in Figure~\ref{fig:rw3bm}, Figures~\ref{fig:rwmlpinputs1} and \ref{fig:rwmlpinputs2}, and Figure~\ref{fig:rwmlpoutputs}, respectively. As shown in Figure~\ref{fig:rwvars}(b), mismodelling remains for high values of the $n_\text{tracks}$ variable. This is not an issue because the MLP efficiency for these events is very low, so the impact on the background estimate is negligible.

\begin{figure}[!htbp]
  \centering
  \subfigure[$n_\text{tracks}$ Pre-Reweighting]{\includegraphics[width=0.375\textwidth]{figures/PreRW_09Feb20/All/GhostTracks_nTracks.png}}
  \subfigure[$n_\text{tracks}$ Post-Reweighting]{\includegraphics[width=0.375\textwidth]{figures/PostRW_09Feb20/All/GhostTracks_nTracks.png}}\\
  \subfigure[$U_1(0.7)$ Pre-Reweighting]{\includegraphics[width=0.375\textwidth]{figures/PreRW_09Feb20/All/GhostTracks_U1_0p7.png}}
  \subfigure[$U_1(0.7)$ Post-Reweighting]{\includegraphics[width=0.375\textwidth]{figures/PostRW_09Feb20/All/GhostTracks_U1_0p7.png}}\\
  \subfigure[$p_\text{T}^\text{jet}$ Pre-Reweighting]{\includegraphics[width=0.375\textwidth]{figures/PreRW_09Feb20/All/CaloJetPt.png}}
  \subfigure[$p_\text{T}^\text{jet}$ Post-Reweighting]{\includegraphics[width=0.375\textwidth]{figures/PostRW_09Feb20/All/CaloJetPt.png}}
  \caption{Distributions of the three variables used to reweight the background simulation, after the full event-level pre-selection, in data and background MC (both reweighted and not). These variables are the ghost-associated track multiplicity (a), the modified correlation variable $U_1(0.7)$ (b), and the transverse momentum of the calorimeter jet (c).}
  \label{fig:rwvars}
\end{figure}

\begin{figure}[!htbp]
  \centering
  \subfigure[$m_{\ell^+\ell^-\text{j}}$ Pre-Reweighting]{\includegraphics[width=0.475\textwidth]{figures/PreRW_09Feb20/All/HM_SR.png}}
  \subfigure[$m_{\ell^+\ell^-\text{j}}$ Post-Reweighting]{\includegraphics[width=0.475\textwidth]{figures/PostRW_09Feb20/All/HM_SR.png}}
  \caption{Distributions of the three body mass distribution, after the full event-level pre-selection, in data and background MC (both reweighted and not).}
  \label{fig:rw3bm}
\end{figure}

\begin{figure}[!htbp]
  \centering
  \subfigure[$p_\text{T}^{\text{lead\ track}}/p_\text{T}^{\text{tracks}}$ Pre-Reweighting]{\includegraphics[width=0.375\textwidth]{figures/PreRW_09Feb20/All/GhostTracks_leadTrackPtRatio.png}}
  \subfigure[$p_\text{T}^{\text{lead\ track}}/p_\text{T}^{\text{tracks}}$ Post-Reweighting]{\includegraphics[width=0.375\textwidth]{figures/PostRW_09Feb20/All/GhostTracks_leadTrackPtRatio.png}}\\
  \subfigure[$\Delta R^\text{lead\ track,\ calo\ jet}$ Pre-Reweighting]{\includegraphics[width=0.375\textwidth]{figures/PreRW_09Feb20/All/GhostTracks_deltaRLeadTrack.png}}
  \subfigure[$\Delta R^\text{lead\ track,\ calo\ jet}$ Post-Reweighting]{\includegraphics[width=0.375\textwidth]{figures/PostRW_09Feb20/All/GhostTracks_deltaRLeadTrack.png}}\\
  \subfigure[$\tau_{\text{2}}$ Pre-Reweighting]{\includegraphics[width=0.375\textwidth]{figures/PreRW_09Feb20/All/GhostTracks_tau2.png}}
  \subfigure[$\tau_{\text{2}}$ Post-Reweighting]{\includegraphics[width=0.375\textwidth]{figures/PostRW_09Feb20/All/GhostTracks_tau2.png}}
  \caption{Distributions of the variables input to the MLP, after the full event-level pre-selection, in data and background MC (both reweighted and not). The modified correlation variable $U_1(0.7)$ is also input to the MLP, but it is used in the reweighting, and shown in Figure~\ref{fig:rwvars}. In the case of exactly two tracks, the $p_\text{T}^{\text{lead\ track}}/p_\text{T}^{\text{tracks}}$ variable can not take values less than 0.5, leading to the spike around 0.5.}
  \label{fig:rwmlpinputs1}
\end{figure}

\begin{figure}[!htbp]
  \centering
  \subfigure[$M_2(0.3)$ Pre-Reweighting]{\includegraphics[width=0.375\textwidth]{figures/PreRW_09Feb20/All/GhostTracks_M2_0p3.png}}
  \subfigure[$M_2(0.3)$ Post-Reweighting]{\includegraphics[width=0.375\textwidth]{figures/PostRW_09Feb20/All/GhostTracks_M2_0p3.png}}\\
  \subfigure[$angularity(2)$ Pre-Reweighting]{\includegraphics[width=0.375\textwidth]{figures/PreRW_09Feb20/All/GhostTracks_angularity_2.png}}
  \subfigure[$angularity(2)$ Post-Reweighting]{\includegraphics[width=0.375\textwidth]{figures/PostRW_09Feb20/All/GhostTracks_angularity_2.png}}
  \caption{Distributions of the variables input to the MLP, after the full event-level pre-selection, in data and background MC (both reweighted and not). The modified correlation variable $U_1(0.7)$ is also input to the MLP, but it is used in the reweighting, and shown in Figure~\ref{fig:rwvars}.}
  \label{fig:rwmlpinputs2}
\end{figure}

\begin{figure}[!htbp]
  \centering
  \subfigure[Mass Regression MLP Output Variable Pre-Reweighting]{\includegraphics[width=0.475\textwidth]{figures/PreRW_09Feb20/All/MLP_massRegression.png}}
  \subfigure[Mass Regression MLP Output Variable Post-Reweighting]{\includegraphics[width=0.475\textwidth]{figures/PostRW_09Feb20/All/MLP_massRegression.png}}\\
  \subfigure[Classification MLP Output Variable Pre-Reweighting]{\includegraphics[width=0.475\textwidth]{figures/PreRW_09Feb20/All/MLP_regInID_zoomed.png}}
  \subfigure[Classification MLP Output Variable Post-Reweighting]{\includegraphics[width=0.475\textwidth]{figures/PostRW_09Feb20/All/MLP_regInID_zoomed.png}}
  \caption{Distributions of the output of the regression (a) and classification (b) MLPs, after the full event-level pre-selection, in data and background MC (both reweighted and not).}
  \label{fig:rwmlpoutputs}
\end{figure}


\subsubsection{ABCD-Based Background Estimation}
\label{sec:bkgdest}

A semi-data-driven estimate is used to estimate the background contribution to the SR. The first step towards this estimate is to calculate a fully data-driven ABCD estimate of the background contribution in the signal region. To do this, 4 regions are defined in the space of the three body mass and MLP classifier output variables. Region $A$ is the SR, Region $B$ shares the same three body mass requirement as the signal region but also requires that $0.0108<MLP<0.0524$, Region $C$ shares the MLP requirement of the SR but has $155 < m^{\ell\ell\text{j}} < 175$~GeV, and Region $D$ is defined by $0.0108<MLP<0.0524$ and $155 < m^{\ell\ell\text{j}} < 175$~GeV. The region $0.0108<MLP<0.0524$ is chosen to contain approximately 10\% of the background. An estimate of the background in the SR is then given by $A=BC/D$. This estimate is accurate if the MLP and three body mass variables are uncorrelated, and there is negligible signal contamination in regions $B$, $C$ and $D$.% Regions $B$, $C$ and $D$ have an $S/B$ less than 1.3\% that of the SR for the 1.5~GeV $a$ signal hypotheses, and so are not expected to be contaminated by the presence of signal events.

While the MLP input variables were selected to minimise the correlation with the three body mass, a non-negligible correlation remains. A correction factor is derived to account for this correlation, using the half of the MC events which were not used in the training of the MLP. This correction factor is defined as the ratio of the MC background events in the signal region, to the MC-based ABCD estimate in the SR: $A/(BC/D)$. This correction factor multiplies the data-driven ABCD estimate to produce the final background estimate:

$$ A^\text{ABCD Est.}_\text{SR} = \underbrace{\frac{B_\text{data} C_\text{data}}{D_\text{data}}}_\text{Data-driven ABCD Estimate} \times \underbrace{\frac{A_\text{MC}}{\frac{B_\text{MC} C_\text{MC}}{D_\text{MC}}}}_\text{MC-based ABCD Correction Factor} \text{.}$$

\noindent This results in an expectation of $82400 \pm 2900$ background events in the SR, where the uncertainty is derived from the statistical uncertainties in the MC and data inputs to the estimate.

This method allows an estimate of the background in the SR, in which only a double ratio of numbers of events are taken from MC. Only taking ratios of events from MC causes background normalisation uncertainties to fully cancel. While the double ratio ensures that any residual mismodelling in the shape of either of the ABCD variables will cancel insofar as the variables can be considered uncorrelated. The Pearson (Spearman) correlation coefficient between the three body mass and the MLP output variable plane, for post-reweighting background MC, is 6.54\% (12.7\%). %The distribution of events in the three body mass vs MLP output variable plane, for post-reweighting background MC, is shown in Figure~\ref{fig:2dmlpmh}, and the correlation coefficient between these variables is 9.73\%.

%% MG: Spearman Rank Correlation = 0.127199 and Pearson Correlation = 0.0653776

%% Blinded
%% cf_weight=3.2993e+07
%% cf_mean_MLP=0.00492543
%% cf_mean_mH=138.754
%% c_cov=489957
%% c_MLP=2403.42
%% c_mH=1.0557e+10
%% c_cov/cf_weight=0.0148503
%% c_MLP/cf_weight=7.28464e-05
%% c_mH/cf_weight=319.977
%% corr = 0.0972686

%% \begin{figure}[!htbp]
%%   \centering
%%   \includegraphics[width=0.7\textwidth]{figures/mH_MLP_3DRW.png}
%%   \caption{2D distribution of the three body mass and classification MLP output variables, after the full event-level pre-selection, in reweighted background MC.}
%%   \label{fig:2dmlpmh}
%% \end{figure}


%% With this method, the only part of the background estimate taken from MC, is the transfer factor, which was found to agree with data to 3.71\% with the MLP cut inverted. This level of non-closure is more than covered by the systematic uncertainties described in Sections~\ref{sec:systematicstheory} and \ref{sec:systematicsexperiment}. The only possibility for an inaccurate background estimate is for the application of the MLP cut to affect the relative normalisation of the SR and CR differently in data and MC. Due to the efforts made to decorrelate the MLP output from the event-level kinematic variables (such as $m^{\ell\ell\text{j}}$), described in Section~\ref{sec:mlpselection}, the change to the relative normalisation of the SR and CR from the application of the MLP should be minimal, and therefore so should any discrepancy between data and MC. Although, we account for any residual data to MC disagreement in the relative background contribution to the SR and CR, by assigned relevant systematic uncertainties, described in Sections~\ref{sec:systematicstheory} and \ref{sec:systematicsexperiment}.

%% The efficiency for the signal event to pass the selection is estimated in MC. The background normalisation is taken from data in the CR, corrected by the trasnfer factor, in order to account for a $2.96\%$ underestimation of the cross section in MC. The efficiency of the MLP is then applied, as evaluated in MC.  % 2.96% taken from ratio of MC/data in all MLP inverted regions

%% Various other background estimates were evaluated, in order to provide closure tests of the nominal transfer factor based method. First, a direct MC estimate. Second, using MC to extrapolate from an inverted MLP cut (with the SR $m^{\ell\ell\text{j}}$ cut applied), into the SR. Third, the flat adjusted ABCD method~\cite{FAABCD}, in which the standard ABCD estimate has a MC-based correction applied to account for correlations between the two variables of choice. Three other regions can be designed to provide additional closure tests for each of these methods, these are: the Gap Region (GR), the disconnected region between the SR and CR; the Validation Region (VR), an MLP-based validation region, defined to be as close as possible to SR in MLP containing the same amount of background (defined by $0.0341<MLP<0.0524$); and the VR-GR, defined as having the $m^{\ell\ell\text{j}}$ requirement of the GR, and the MLP requirement of the VR.


\subsubsection{Validation of Background Modelling}
\label{sec:bkgdval}

The background model is compared to data in 13 low-signal validation regions. 15 regions are defined by values of $m_{\ell\ell\text{j}}$ of 100-110~GeV, 110-120~GeV, 120-135~GeV, 135-150~GeV, 150-155~GeV, and the $MLP$ ranges of $>0.052$, $0.037-0.052$ and $0.026-0.037$. The two MLP validation regions are defined by the ranges in the $MLP$ output variable nearest the signal region, which contain equal amounts of background to the signal region. 2 of these 15 regions are blinded due to high signal concentration. The data in each of these regions is shown in Table~\ref{tab:vrsdata}, and the background estimates (calculated following the procedure described in subsection~\ref{sec:bkgdest}) for each of these regions is shown in Table~\ref{tab:vrsest}. This information is also provided graphically in Figure~\ref{fig:bkgdvrs}. Good agreement is seen between the data and the backgrounds estimates in these regions. The MC-based correction factors are presented in Table~\ref{tab:correctionfactors}.

%% Sherpa and MadGraph compatible between (87226.6-81069.4)/sqrt(2821.1*2821.1+1982.59*1982.59) = 1.7856884786 SDs of the statistical significance (stat error from data, and non-Zjets backgrounds incorrectly considered as uncorrelated, but should be a small effect)...

\begin{table}[!htbp]{\footnotesize\renewcommand{\arraystretch}{1.2}
    \begin{center}
      \begin{tabular}{c|c|c|c|c|c|c|}
        \cline{2-7}
        & \multicolumn{6}{|c|}{$m_{\ell\ell j}$ Range}\\
        \hline
        \multicolumn{1}{|c|}{MLP Range} & $100-110$~GeV & $110-120$~GeV & $120-135$~GeV & $135-150$~GeV & $150-155$~GeV & $155-175$~GeV \\
        \hline
        \multicolumn{1}{|c|}{SR (99-100\%)} & 2470 & 23002 & 82908 & 94674 & 29289 & 100408 \\
        \multicolumn{1}{|c|}{VR (98-99\%)} & 3912 & 32475 & 89919 & 82468 & 23410 & 76103 \\
        \multicolumn{1}{|c|}{VR (97-98\%)} & 4350 & 33958 & 88536 & 79054 & 22131 & 73705 \\
        \hline
      \end{tabular}
      \caption{Data in background estimate validation regions.}
      \label{tab:vrsdata}
  \end{center}}
\end{table}

\begin{table}[!htbp]{\footnotesize\renewcommand{\arraystretch}{1.2}
    \begin{center}
      \begin{tabular}{c|c|c|c|c|c|}
        \cline{2-6}
        & \multicolumn{5}{|c|}{$m_{\ell\ell j}$ Range}\\
        \hline
        \multicolumn{1}{|c|}{MLP Range} & $100-110$~GeV & $110-120$~GeV & $120-135$~GeV & $135-150$~GeV & $150-155$~GeV \\
        \hline
\multicolumn{1}{|c|}{(99-100\%)} & $\num[round-mode=figures,round-precision=3]{2177.94} \pm \num[round-mode=figures,round-precision=3]{12.4904} \pm \num[round-mode=figures,round-precision=3]{356.318}$ & $\num[round-mode=figures,round-precision=3]{20827.4} \pm \num[round-mode=figures,round-precision=3]{78.3621} \pm \num[round-mode=figures,round-precision=3]{1135.08}$ & $\num[round-mode=figures,round-precision=3]{82426} \pm \num[round-mode=figures,round-precision=3]{290.772} \pm \num[round-mode=figures,round-precision=3]{2854.77}$ & $\num[round-mode=figures,round-precision=3]{91063.1} \pm \num[round-mode=figures,round-precision=3]{322.939} \pm \num[round-mode=figures,round-precision=3]{3004.95}$ & $\num[round-mode=figures,round-precision=3]{31068.9} \pm \num[round-mode=figures,round-precision=3]{123.313} \pm \num[round-mode=figures,round-precision=3]{1435.62}$ \\
\multicolumn{1}{|c|}{(98-99\%)} & $\num[round-mode=figures,round-precision=3]{3786.1} \pm \num[round-mode=figures,round-precision=3]{22.6879} \pm \num[round-mode=figures,round-precision=3]{345.868}$ & $\num[round-mode=figures,round-precision=3]{32008.6} \pm \num[round-mode=figures,round-precision=3]{133.282} \pm \num[round-mode=figures,round-precision=3]{1361.88}$ & $\num[round-mode=figures,round-precision=3]{88744.7} \pm \num[round-mode=figures,round-precision=3]{350.965} \pm \num[round-mode=figures,round-precision=3]{2804.93}$ & $\num[round-mode=figures,round-precision=3]{77853} \pm \num[round-mode=figures,round-precision=3]{309.203} \pm \num[round-mode=figures,round-precision=3]{2537.37}$ & $\num[round-mode=figures,round-precision=3]{23809} \pm \num[round-mode=figures,round-precision=3]{103.702} \pm \num[round-mode=figures,round-precision=3]{1198.58}$ \\
\multicolumn{1}{|c|}{(97-98\%)} & $\num[round-mode=figures,round-precision=3]{4430.49} \pm \num[round-mode=figures,round-precision=3]{26.7311} \pm \num[round-mode=figures,round-precision=3]{381.741}$ & $\num[round-mode=figures,round-precision=3]{33954.7} \pm \num[round-mode=figures,round-precision=3]{143.156} \pm \num[round-mode=figures,round-precision=3]{1442.88}$ & $\num[round-mode=figures,round-precision=3]{86915.5} \pm \num[round-mode=figures,round-precision=3]{348.475} \pm \num[round-mode=figures,round-precision=3]{2919.42}$ & $\num[round-mode=figures,round-precision=3]{76093.3} \pm \num[round-mode=figures,round-precision=3]{306.313} \pm \num[round-mode=figures,round-precision=3]{2464.43}$ & $\num[round-mode=figures,round-precision=3]{21122.8} \pm \num[round-mode=figures,round-precision=3]{93.0346} \pm \num[round-mode=figures,round-precision=3]{1037.87}$ \\
        \hline
      \end{tabular}
      \caption{Background estimates (calculated following the procedure described in subsection~\ref{sec:bkgdest}) in background estimate validation regions. As the number of events in the $155-175$~GeV bins are used to calculate the background estimates, the background estimate method can not provide a prediction in these regions. The first and second quoted uncertainties are due to limited data and MC statistics, respectively.}
      \label{tab:vrsest}
  \end{center}}
\end{table}

\begin{figure}[!htbp]
  \centering
  \includegraphics[width=0.65\textwidth]{figures/VRPlots_09Feb20/VRsPlot_noSys.pdf}
  \includegraphics[width=0.65\textwidth]{figures/VRsPlot_withSys_03Mar20.pdf}
  \caption{Data and background estimates (calculated following the procedure described in subsection~\ref{sec:bkgdest}) in background estimate validation and signal regions. As the number of events in the $155-175$~GeV bins are used to calculate the background estimates, the background estimate method can not provide a prediction in these regions. The uncertainties in (a) represent just the data and MC statistical uncertainties, while the uncertainties in (b) represent the total statistical and systematic uncertainties in this analysis.}
  \label{fig:bkgdvrs}
\end{figure}

\begin{table}[!htbp]{\footnotesize\renewcommand{\arraystretch}{1.2}
    \begin{center}
      \begin{tabular}{c|c|c|c|c|c|}
        \cline{2-6}
        & \multicolumn{5}{|c|}{$m_{\ell\ell j}$ Range}\\
        \hline
        \multicolumn{1}{|c|}{MLP Range} & $100-110$~GeV & $110-120$~GeV & $120-135$~GeV & $135-150$~GeV & $150-155$~GeV \\
        \hline
        \multicolumn{1}{|c|}{(99-100\%)} & $\num[round-mode=figures,round-precision=3]{0.350061} \pm \num[round-mode=figures,round-precision=3]{0.057271}$ & $\num[round-mode=figures,round-precision=3]{0.443204} \pm \num[round-mode=figures,round-precision=3]{0.0241543}$ & $\num[round-mode=figures,round-precision=3]{0.703868} \pm \num[round-mode=figures,round-precision=3]{0.024378}$ & $\num[round-mode=figures,round-precision=3]{0.867058} \pm \num[round-mode=figures,round-precision=3]{0.0286116}$ & $\num[round-mode=figures,round-precision=3]{1.03051} \pm \num[round-mode=figures,round-precision=3]{0.0476174}$ \\
        \multicolumn{1}{|c|}{(98-99\%)} & $\num[round-mode=figures,round-precision=3]{0.795436} \pm \num[round-mode=figures,round-precision=3]{0.0726647}$ & $\num[round-mode=figures,round-precision=3]{0.896921} \pm \num[round-mode=figures,round-precision=3]{0.0381615}$ & $\num[round-mode=figures,round-precision=3]{1.00914} \pm \num[round-mode=figures,round-precision=3]{0.0318955}$ & $\num[round-mode=figures,round-precision=3]{0.987324} \pm \num[round-mode=figures,round-precision=3]{0.0321786}$ & $\num[round-mode=figures,round-precision=3]{1.04828} \pm \num[round-mode=figures,round-precision=3]{0.0527718}$ \\
        \multicolumn{1}{|c|}{(97-98\%)} & $\num[round-mode=figures,round-precision=3]{0.967317} \pm \num[round-mode=figures,round-precision=3]{0.0833463}$ & $\num[round-mode=figures,round-precision=3]{0.990249} \pm \num[round-mode=figures,round-precision=3]{0.04208}$ & $\num[round-mode=figures,round-precision=3]{1.03391} \pm \num[round-mode=figures,round-precision=3]{0.0347283}$ & $\num[round-mode=figures,round-precision=3]{1.00541} \pm \num[round-mode=figures,round-precision=3]{0.0325622}$ & $\num[round-mode=figures,round-precision=3]{0.964372} \pm \num[round-mode=figures,round-precision=3]{0.0473845}$ \\
        \hline
      \end{tabular}
      \caption{MC-based correction factors used in the calculation of the background estimates (calculated following the procedure described in subsection~\ref{sec:bkgdest}) in background estimate validation regions. As the number of events in the $155-175$~GeV bins are used to calculate the background estimates, the background estimate method can not provide a prediction in these regions. The quoted uncertainties are due to limited MC statistics.}
      \label{tab:correctionfactors}
  \end{center}}
\end{table}

%% \begin{table}[!htbp]{\footnotesize\renewcommand{\arraystretch}{1.2}
%%     \begin{center}
%%       \begin{tabular}{c|c|c|c|c|c|}
%%         \cline{2-6}
%%         & \multicolumn{5}{|c|}{$m_{\ell\ell j}$ Range}\\
%%         \hline
%%         \multicolumn{1}{|c|}{MLP Range} & $100-110$~GeV & $110-120$~GeV & $120-135$~GeV & $135-155$~GeV & $155-175$~GeV \\
%%         \hline
%%         \multicolumn{1}{|c|}{SR (99-100\%)} & $2330 \pm 370$ & $20900 \pm 1100$ & $81100 \pm 2800$ & $122700 \pm 3700$ & N/A \\
%%         \multicolumn{1}{|c|}{VR (98-99\%)} & $4030 \pm 360$ & $32200 \pm 1300$ & $88500 \pm 2700$ & $102600 \pm 3100$ & N/A \\
%%         \multicolumn{1}{|c|}{VR (97-98\%)} & $4700 \pm 390$ & $34300 \pm 1400$ & $87100 \pm 2800$ & $98600 \pm 2900$ & N/A \\
%%         \hline
%%       \end{tabular}
%%       \caption{Background estimates (calculated following the procedure described in subsection~\ref{sec:bkgdest}) in background estimate validation regions. As the number of events in the $155-175$~GeV bins are used to calculate the background estimates, the background estimate method can not provide a prediction in these regions. The quoted uncertainties are due to limited data and MC statistics.}
%%       \label{tab:vrsest}
%%   \end{center}}
%% \end{table}

As a further test of the background modelling strategy, these estimates are evaluated using a alternative $Z+\text{jets}$ MC generator: \textsc{MadGraph} (Appendix~\ref{app:bkgdmc}). The results of this test are given in Table~\ref{tab:vrsestmg}, and Figure~\ref{fig:bkgdvrsmg}. This sample has been reweighted using a procedure designed to mitigate the observed mismodelling in the \textsc{MadGraph} $Z+\text{jets}$ sample, using the $p_\text{T}$ of the calorimeter jet, the $p_\text{T}$ of the three body system, and the multiplicity of tracks Ghost-Associated to the calorimeter jet. Various distributions before and after this reweighting is applied are given in Appendix~\ref{app:MGRW}. %It should be noted that these results are produced using weights derived from the relevant \textsc{MadGraph} MC samples, but a reweighting procedure designed to mitigate the \textsc{Sherpa}2.2.1 mismodelling.

Finally, a 1.5~GeV $a$ produced with a $\text{BR}(H\to Za)=20\%$ results in the background estimate increasing by 0.53\%, demonstrating that this background estimation method is robust against significant signal contamination. % Finally, Appendix~\ref{app:abcdcontaminationstudy} demonstrates that none of the regions used in the ABCD estimate or its validation are susceptible to significant signal contamination.

\begin{table}[!htbp]{\footnotesize\renewcommand{\arraystretch}{1.2}
    \begin{center}
      \begin{tabular}{c|c|c|c|c|c|}
        \cline{2-6}
        & \multicolumn{5}{|c|}{$m_{\ell\ell j}$ Range}\\
        \hline
        \multicolumn{1}{|c|}{MLP Range} & $100-110$~GeV & $110-120$~GeV & $120-135$~GeV & $135-155$~GeV & $155-175$~GeV \\
        \hline
        \multicolumn{1}{|c|}{(99-100\%)} & $\num[round-mode=figures,round-precision=3]{2770.9} \pm \num[round-mode=figures,round-precision=3]{15.8909} \pm \num[round-mode=figures,round-precision=3]{281.319}$ & $\num[round-mode=figures,round-precision=3]{24525.7} \pm \num[round-mode=figures,round-precision=3]{92.2765} \pm \num[round-mode=figures,round-precision=3]{865.469}$ & $\num[round-mode=figures,round-precision=3]{85225} \pm \num[round-mode=figures,round-precision=3]{300.646} \pm \num[round-mode=figures,round-precision=3]{1965.04}$ & $\num[round-mode=figures,round-precision=3]{96053.6} \pm \num[round-mode=figures,round-precision=3]{340.637} \pm \num[round-mode=figures,round-precision=3]{2141.14}$ & $\num[round-mode=figures,round-precision=3]{31027.5} \pm \num[round-mode=figures,round-precision=3]{123.149}\pm \num[round-mode=figures,round-precision=3]{995.369}$ \\
        \multicolumn{1}{|c|}{(98-99\%)} & $\num[round-mode=figures,round-precision=3]{3344.4} \pm \num[round-mode=figures,round-precision=3]{20.0411} \pm \num[round-mode=figures,round-precision=3]{300.253}$ & $\num[round-mode=figures,round-precision=3]{30315.6} \pm \num[round-mode=figures,round-precision=3]{126.233} \pm \num[round-mode=figures,round-precision=3]{1036.73}$ & $\num[round-mode=figures,round-precision=3]{88071.7} \pm \num[round-mode=figures,round-precision=3]{348.303} \pm \num[round-mode=figures,round-precision=3]{2163}$ & $\num[round-mode=figures,round-precision=3]{79186.3} \pm \num[round-mode=figures,round-precision=3]{314.499} \pm \num[round-mode=figures,round-precision=3]{2000.28}$ & $\num[round-mode=figures,round-precision=3]{21586.9} \pm \num[round-mode=figures,round-precision=3]{94.0234} \pm \num[round-mode=figures,round-precision=3]{816.061}$ \\
        \multicolumn{1}{|c|}{(97-98\%)} & $\num[round-mode=figures,round-precision=3]{4559.06} \pm \num[round-mode=figures,round-precision=3]{27.5069} \pm \num[round-mode=figures,round-precision=3]{387.842}$ & $\num[round-mode=figures,round-precision=3]{35198.9} \pm \num[round-mode=figures,round-precision=3]{148.402} \pm \num[round-mode=figures,round-precision=3]{1182.7}$ & $\num[round-mode=figures,round-precision=3]{85916.2} \pm \num[round-mode=figures,round-precision=3]{344.468} \pm \num[round-mode=figures,round-precision=3]{2159.66}$ & $\num[round-mode=figures,round-precision=3]{79122} \pm \num[round-mode=figures,round-precision=3]{318.505} \pm \num[round-mode=figures,round-precision=3]{2035.67}$ & $\num[round-mode=figures,round-precision=3]{20906.5} \pm \num[round-mode=figures,round-precision=3]{92.0818} \pm \num[round-mode=figures,round-precision=3]{792.598}$ \\
        \hline
      \end{tabular}
      \caption{Background estimates (calculated following the procedure described in subsection~\ref{sec:bkgdest}) in background estimate validation regions, using the alternative \textsc{MadGraph} generator for the $Z+\text{jets}$ background. As the number of events in the $155-175$~GeV bins are used to calculate the background estimates, the background estimate method can not provide a prediction in these regions. The first and second quoted uncertainties are due to limited data and MC statistics, respectively.}
      \label{tab:vrsestmg}
  \end{center}}
\end{table}

\begin{figure}[!htbp]
  \centering
  \includegraphics[width=0.65\textwidth]{figures/VRPlots_09Feb20/VRsPlot_MGData.pdf}
  \includegraphics[width=0.65\textwidth]{figures/VRPlots_09Feb20/VRsPlot_MGSH.pdf}
  \caption{(a) Data and the alternative \textsc{MadGraph} background estimates and (b) alternative \textsc{MadGraph} and nominal background estimates (calculated following the procedure described in subsection~\ref{sec:bkgdest}) in background estimate validation and signal regions. As the number of events in the $155-175$~GeV bins are used to calculate the background estimates, the background estimate method can not provide a prediction in these regions. The uncertainties represent the data and MC statistical uncertainties.}
  \label{fig:bkgdvrsmg}
\end{figure}

%% \begin{table}[!htbp]{\footnotesize\renewcommand{\arraystretch}{1.2}
%%     \begin{center}
%%       \begin{tabular}{c|c|c|c|c|c|}
%%         \cline{2-6}
%%         & \multicolumn{5}{|c|}{$m_{\ell\ell j}$ Range}\\
%%         \hline
%%         \multicolumn{1}{|c|}{MLP Range} & $100-110$~GeV & $110-120$~GeV & $120-135$~GeV & $135-155$~GeV & $155-175$~GeV \\
%%         \hline
%%         \multicolumn{1}{|c|}{SR (99-100\%)} & $2490 \pm 250$ & $24110 \pm 840$ & $87200 \pm 2000$ & $130000 \pm 2700$ & N/A \\
%%         \multicolumn{1}{|c|}{VR (98-99\%)} & $3080 \pm 280$ & $30300 \pm 1000$ & $90300 \pm 2200$ & $103100 \pm 2400$ & N/A \\
%%         \multicolumn{1}{|c|}{VR (97-98\%)} & $4180 \pm 350$ & $34700 \pm 1100$ & $87700 \pm 2200$ & $101700 \pm 2400$ & N/A \\
%%         \hline
%%       \end{tabular}
%%       \caption{Background estimates (calculated following the procedure described in subsection~\ref{sec:bkgdest}) in background estimate validation regions, using the alternative \textsc{MadGraph} generator for the $Z+\text{jets}$ background. As the number of events in the $155-175$~GeV bins are used to calculate the background estimates, the background estimate method can not provide a prediction in these regions. The quoted uncertainties are due to limited data and MC statistics.}
%%       \label{tab:vrsestmg}
%%   \end{center}}
%% \end{table}


%-------------------------------------------------------------------------------
\clearpage
\section{Systematic Uncertainties}
\label{sec:systematics}
%-------------------------------------------------------------------------------

Systematic uncertainties are the dominant sources of uncertainty for this analysis. The systematic uncertainties relevant to this analysis have been implemented in the statistical model as nuisance parameters (NP). The systematic uncertainties are of two types: theoretical uncertainties, and experimental (detector and reconstruction) uncertainties. The following subsections describe the systematic uncertainties relevant to this analysis.


%-------------------------------------------------------------------------------
\subsection{Systematic Uncertainties: Modelling}
\label{sec:systematicstheory}
%-------------------------------------------------------------------------------

The following subsections describe the eight modelling systematic uncertainties relevant to this analysis in order of magnitude, where already evaluated.


%% \subsubsection{Background Modelling Uncertainty}
%% \label{sec:sysbkgdmodelling}

%% One of the largest uncertainties is the background modelling uncertainty. This is evaluated by comparing the nominal transfer factor (described in Section~\ref{sec:bkgdmodelling}) to one evaluated using an alternative generator. As it accounts for $>99\%$ of the total background, only the $Z+$jets process is used to evaluate this uncertainty. The nominal background sample uses \textsc{Sherpa}2.2.1, as described in Section~\ref{sec:mcsamples}. The alternative sample uses \textsc{MadGraph}, \textsc{Pythia}8 and \textsc{EvtGen}. This uncertainty is $9.50\%$ of the total background normalisation, including the MC statistical uncertainty on the two background samples used. We then subtract the MC statistical errors of the two sampes used to calculate this uncertainty, giving a final uncertainty of $8.89\%$.


\subsubsection{Statistical Uncertainty}
\label{sec:sysbkgdmcstat}

The dominant uncertainty for this analysis is the statistical uncertainty on the background estimate. This is due primarily to MC statistical uncertainty in the ABCD correction described in subsection~\ref{sec:bkgdest}. However, there is also a smaller contribution from the statistical uncertainty in data on the pre-correction ABCD estimate described in subsection~\ref{sec:bkgdest}. This results in a total uncertainty of 3.48\% on the background estimate.

The MC statistical uncertainty on the signal estimate varies between $1.69\%$ and $8.00\%$ of the total signal normalisation, for the 0-4~GeV $a$ signal samples, $4.06\%$ for the $\eta_c$ sample, and $3.98\%$ for the $J/\psi$ sample. This uncertainty is labelled \emph{MCSTAT} in the correlation and pull plots.


\subsubsection{Scale and PDF Uncertainties}
\label{sec:sysscale}

The renormalisation and factorisation scale, and PDF, uncertainties are investigated for both the signal and $Z+\text{jets}$ background samples. The half renormalisation scale uncertainty ($\mu_R=0.5$) is found to be the largest of these by far, and so is implemented in the fit for both the signal and $Z+\text{jets}$ background processes. This is illustrated in Table~\ref{tab:scalesys}. It is implemented using internal weights in the relevant MC samples. This uncertainty is labelled \emph{MUR} in the correlation and pull plots.

\begin{table}[!htbp]{\footnotesize\renewcommand{\arraystretch}{1.2}
  \begin{center}
    \footnotesize
    \begin{tabular}{|cc|}
      \hline
      Variation & Uncertainty \\
      \hline
      $\mu_R=0.5$ & 5.71\%\\
      $\mu_R=2$ & 0.731\%\\
      $\mu_F=0.5$ & 0.565\%\\
      $\mu_F=2$ & 3.68\%\\
      $\mu_R=0.5$ \& $\mu_F=0.5$ & 2.51\%\\
      $\mu_R=2$ \& $\mu_F=2$ & 1.14\%\\
      MMHT2014nnlo68cl & 1.19\%\\
      CT14nnlo & 1.06\%\\
      \hline
      \end{tabular}
    \caption{Scale and PDF uncertainties on the total background, evaluated by scaling the $Z+\text{jets}$ background and evaluating the change in the background estimate without a dedicated reweighting.}
    \label{tab:scalesys}
  \end{center}}
\end{table}

For the $Z+\text{jets}$ background sample, the systematic uncertainty is first estimated by the change in the background estimate after a dedicated reweighting is applied to the $Z+\text{jets}$ background sample. This dedicated reweighting is calculated analogously to the nominal reweighting, but the $Z+\text{jets}$ MC sample used in the calculation of the reweighting factors has the dominant renormalisation scale variation applied. The renormalisation scale uncertainty on the background normalisation was found to be 4.78\%; this was reduced from 5.71\% due to the dedicated reweighting, showing that the reweighting is successfully reducing the reliance of the background estimate on the chosen MC samples.

This value is found to be consistent with zero within the statistical uncertainty on its evaluation, and the agreement between the background estimates and data in the 14 VRs was greater than would be expected if this uncertainty was accurately estimated by this number, making it likely that it has a large statistical component. As such, this estimate is amended to evaluate the systematic in a region of the MLP variable expected to contain approximately two times more signal than the SR, reducing the statistical component of the estimate. The resulting value of the uncertainty was found to be 1.83\%. A linear fit to expanded MLP regions was also considered, as shown in Figure~\ref{fig:bkgdmodelmur}(b), but this resulted in an uncertainty of 0.685\%, and so the more conservative approach was adopted.

\begin{figure}[!htbp]
  \centering
  \subfigure[Inclusive Regions]{\includegraphics[width=0.475\textwidth]{figures/BackgroundModelling/muR_vs_sampleSize.pdf}}
  \subfigure[Exclusive Regions]{\includegraphics[width=0.475\textwidth]{figures/BackgroundModelling/muR_vs_sampleSize_exclusive.pdf}}
  \caption{Renormalisation scale uncertainty on the $Z+\text{jets}$ background, as determined using various expanded MLP requirements. The regions are defined to (a) contain and (b) not contain all of the more signal-like MLP regions. A linear fit is performed to both regions, and the fit to (b) is used to extract an estimate of this uncertainty.}
  \label{fig:bkgdmodelmur}
\end{figure}

For the signal samples, the systematic is taken from the change in the signal efficiency under the scaling. The internal weights used in the derivation of these uncertainties are not present in the first generation of MC signal samples: 0.5~GeV, 2.5~GeV and 8~GeV $a$, and the $\eta_c$ samples. As such, the renormalisation scale systematic uncertainty on the missing $a$ signal samples are interpolated if possible, else they are taken from the nearest signal sample. The renormalisation scale systematic uncertainty on the $\eta_c$ signal sample is taken from the $J/\psi$ signal sample. The renormalisation scale systematic uncertainty on the signal estimates are given in Table~\ref{tab:mursys}.%% varies between $21.0\%$ and $23.2\%$ of the total signal normalisation, for the 0-4~GeV $a$ signal samples, and is $22.0\%$ for the $\eta_c$ and $J/\psi$ samples.

\begin{table}[!htbp]{\footnotesize\renewcommand{\arraystretch}{1.2}
  \begin{center}
    \footnotesize
    \begin{tabular}{|c|ccccccccc|cc|}
      \hline
      $a$ mass /~GeV & 0.5 & 0.75 & 1 & 1.5 & 2 & 2.5 & 3 & 3.5 & 4 & $\eta_c$ & $J/\psi$ \\
      \hline
      Uncertainty (\%) & 0.0267 & 0.0267 & 0.140 & 0.345 & 0.0782 & 0.103 & 0.128 & 0.960 & 1.66 & 0.782 & 0.782 \\
      \hline
    \end{tabular}
    \caption{Renormalisation scale uncertainties for the various signal samples. They are evaluated for the $\mu_R=0.5$ variation.}
    \label{tab:mursys}
  \end{center}}
\end{table}


\subsubsection{Background Modelling Uncertainty}
\label{sec:sysbkgddecaymodel}

The hadronisation and ME uncertainties are evaluated for the dominant $Z+\text{jets}$ background, by comparing the background estimate derived with the nominal \textsc{Sherpa} MC sample to the background as estimated using an alternative \textsc{MadGraph} MC sample. The only difference in the estimation method is the reweighting, which uses different variables due to the different nature of the mismodelling in \textsc{MadGraph}. These three variables are the $p_\text{T}$ of the calorimeter jet, the $p_\text{T}$ of the three body system, and the multiplicity of tracks Ghost-Associated to the calorimeter jet. This results in an uncertainty of 3.40\% on the background normalisation. The key distributions in the derivation of this estimate are compared for \textsc{MadGraph} and \textsc{Sherpa} in Figure~\ref{fig:mgvssherpa}. This uncertainty is labelled \emph{MUR} in the correlation and pull plots. % 0.01579001199

\begin{figure}[!htbp]
  \centering
  \subfigure[$m_{\ell\ell\text{j}}$]{\includegraphics[width=0.475\textwidth]{figures/MGvsSherpa_09Feb20/All/HM_SR.png}}
  \subfigure[Classification MLP]{\includegraphics[width=0.475\textwidth]{figures/MGvsSherpa_09Feb20/All/MLP_regInID_log.png}}
  \caption{Distributions of the output of the three body mass (a) and classification MLP (b), for background MC, where the $Z+\text{jets}$ process is being modelled by \textsc{MadGraph} and \textsc{Sherpa}. A dedicated reweighting is applied for both backgrounds, which differs between backgrounds.}
  \label{fig:mgvssherpa}
\end{figure}

This value is found to be consistent with zero within the statistical uncertainty on its evaluation, and the agreement between the background estimates and data in the 14 VRs was greater than would be expected if this uncertainty was accurately estimated by this number, making it likely that it has a large statistical component. As such, this estimate is amended to evaluate the systematic using a linear fit to expanded MLP regions, as shown in Figure~\ref{fig:bkgdmodelmgvssh}(b), to reduce the statistical component of the estimate. The resulting value of this uncertainty was found to be 2.11\%. Expanding the MLP region by a factor of two was also considered, but this resulted in an uncertainty of 1.19\%, and so the more conservative approach was adopted.

\begin{figure}[!htbp]
  \centering
  \subfigure[Inclusive Regions]{\includegraphics[width=0.475\textwidth]{figures/BackgroundModelling/PSME_vs_sampleSize.pdf}}
  \subfigure[Exclusive Regions]{\includegraphics[width=0.475\textwidth]{figures/BackgroundModelling/PSME_vs_sampleSize_exclusive.pdf}}
  \caption{Background modelling uncertainty representing the hadronisation and ME uncertainties on the $Z+\text{jets}$ background was estimated, as determined using various expanded MLP requirements. The regions are defined to (a) contain and (b) not contain all of the more signal-like MLP regions. A linear fit is performed to both regions, and the fit to (b) is used to extract an estimate of this uncertainty.}
  \label{fig:bkgdmodelmgvssh}
\end{figure}


\subsubsection{Signal Hadronisation Uncertainty}
\label{sec:syssigdecaymodel}

The effect of the signal hadronisation modelling uncertainty on the MLP output has been evaluated by calculating the change in acceptance after reweighting events based on generator-level track multiplicity and $U1(0.7)$. These reweightings are derived from an alternative signal sample produced using \textsc{Herwig}7. This is based on the assumption that the largest impact of the modelling uncertainty is on the MLP via the track multiplicity. Due to technical limitations in \textsc{Herwig}7, only quark decays are used in the calculation of the scale factors, and the 1.5~GeV $a$ scale factors are used as a proxy for all lower masses. The hadronisation uncertainties on the signal estimates are given in Table~\ref{tab:sighadunctable}. This uncertainty is labelled \emph{PS} in the correlation and pull plots.

%% The hadronisation uncertainty on the signal estimate varies between $0.715\%$ and $17.6\%$ of the total signal normalisation, for the 0-4~GeV $a$ signal samples, $2.6\%$ for the $\eta_c$ sample, and $21\%$ for the $J/\psi$ sample. For the 3.5~GeV $a$ signal, this uncertainty was estimated to be much smaller than for the mass points surrounding it. In order to ensure that the estimate has not fluctuated to a small value for this mass point, the value of the systematic is conservatively interpolated from the surrounding mass points. This uncertainty is labelled \emph{PS} in the correlation and pull plots.
%% The hadronisation uncertainties on the signal estimates are given in Table~\ref{tab:sighadunctable}. For the 3.5~GeV $a$ signal, this uncertainty was estimated to be much smaller than for the mass points surrounding it, with a value of 0.716\%. In order to ensure that the estimate has not fluctuated to a small value for this mass point, the value of the systematic is conservatively interpolated from the mass points either side. This uncertainty is labelled \emph{PS} in the correlation and pull plots.

\begin{table}[!htbp]{\footnotesize\renewcommand{\arraystretch}{1.2}
  \begin{center}
    \footnotesize
    \begin{tabular}{|c|ccccccccc|cc|}
      \hline
      $a$ mass /~GeV & 0.5 & 0.75 & 1 & 1.5 & 2 & 2.5 & 3 & 3.5 & 4 & $\eta_c$ & $J/\psi$ \\
      \hline
      Uncertainty (\%) & 16.5 & 17.7 & 15.4 & 13.5 & 8.97 & 7.10 & 6.78 & 10.9 & 4.16 & 3.61 & 18.3 \\
      \hline
    \end{tabular}
    \caption{Hadronisation uncertainties for the various signal samples.}
    \label{tab:sighadunctable}
  \end{center}}
\end{table}


\subsubsection{Higgs Cross Section Uncertainty}
\label{sec:syshiggsxsec}

Theory (truncation, unknown N$^3$LO PDFs and unknown finite-mass effects), renormalisation and factorisation scale, combined PDF, $\alpha_s$, and flavour scheme uncertainties (only for $tH$ associated production) are applied to the various Higgs production mode cross sections. These are taken from the official recommendations of the CERN Higgs Cross Section Working Group~\cite{Dittmaier:2011ti,Dittmaier:2012vm}. The different types of uncertainty are summed in quadrature, then a the uncertainties on the various Higgs production processes are summed (weighted by the relevant cross sections) to calculate the total uncertainty. For all uncertainties, a Higgs boson mass of 125~GeV is assumed. The uncertainty on the ggF process is by far the largest contribution to the total uncertainty, with a combined theory, renormalisation and factorisation scale uncertainty of $^{+4.6}_{-6.7}\%$, a PDF uncertainty of $1.9\%$, and a $+\alpha_s$ uncertainty of $2.6\%$. The uncertainty on the inclusive cross section is $^{+9.4}_{-9.3}\%$. This uncertainty is labelled \emph{HXS} in the correlation and pull plots. % \footnote{\href{https://twiki.cern.ch/twiki/bin/view/LHCPhysics/CERNYellowReportPageAt1314TeV2014}{https://twiki.cern.ch/twiki/bin/view/LHCPhysics/CERNYellowReportPageAt1314TeV2014}}


\subsubsection{Signal Production Uncertainty}
\label{sec:syssigprodmodel}

The full inclusive Higgs production cross section is used to normalise the signal yields, while only the ggF production mode is used to model the signal samples. We account for this by applying a systematic uncertainty on the signal acceptance, derived from a comparison of generator-level MC. 100k ggF and 100k VBF events are generated using \textsc{Pythia}, the acceptance of these are compared in a generator-level fiducial acceptance. The Higgs boson is more highly boosted in the case of VBF production, which means the individual objects are more likely to pass the minimum $p_\text{T}$ requirements. However, the boost means that the angular separation between the objects is smaller, so they are more likely to fail the overlap removal for VBF production. This leads to the generator-level acceptances differing by just 0.941\%. This uncertainty is scaled down to 12.8\% of this value, which is the fraction of Higgs boson events in the SM which are not produced by ggF, leading to an overall systematic uncertainty of just 0.120\%. This uncertainty is labelled \emph{GGFACC} in the correlation and pull plots.


\subsubsection{Signal Data to MC Discrepancy Uncertainty}
\label{sec:syssigdatamc}

The uncertainty described in subsection~\ref{sec:syssigdecaymodel} estimates the effect of any mismodelling of the hadronisation of the signal resonance. However, it is possible that other sources of mismodelling in the signal samples may affect the estimates of the signal efficiencies. One such mechanism through which this may occur, is the known mismodelling in the length of the beamspot in MC, which is 30\% longer than data. The effect of this mismodelling is estimated by reweighting the event in MC based on the $z_0$ of the primary vertex to match data. For the 1.5~GeV $a$ signal sample, this results in the efficiency decreasing by $\sim 2.0\%$, which has a negligible impact on the final result.

Other similar affects may be present but unnoticed. A method to place an upper bound on the impact of any such affect is to apply the background reweighting scale factors, described in subsection~\ref{sec:bkgdrw}, to the signal. This will exaggerate the impact of this uncertainty, because it also factors in any mismodelling in the \textsc{Sherpa}2.2.1 $Z+\text{jets}$ distribution. This results in an uncertainty on the signal efficiency of $\sim 5.0\%$ for the 1.5~GeV $a$ signal sample. Including this uncertainty in the likelihood fit as a nuisance parameter, as described in subsection~\ref{sec:statmodel}, leads to the observed limit for the 1.5~GeV $a$ signal hypothesis changing by $\sim 0.2\%$. Due to the negligible impact of this uncertainty, even after it double-counts the mismodelling in the \textsc{Sherpa}2.2.1 $Z+\text{jets}$ distribution, this uncertainty is not included in the final statistical model.


\subsubsection{Reweighting Procedure Uncertainty}
\label{sec:sysrwproc}

If any uncertainty were to be introduced by the reweighting procedure, this should be covered by the uncertainty described in subsubsection~\ref{sec:sysbkgddecaymodel}, which uses a different reweighting procedure. However, any such uncertainty is conservatively estimated in a more direct way by reevaluating the background estimate without the reweighting, and evaluating the change in the background estimate. This is found to be 1.30\%, which is both small and highly conservative, and is therefore not included in the final fit.


%-------------------------------------------------------------------------------
\subsection{Systematic Uncertainties: Experimental}
\label{sec:systematicsexperiment}
%-------------------------------------------------------------------------------

The following subsections describe the seven experimental systematic uncertainties relevant to this analysis in order of magnitude, where already evaluated.


\subsubsection{Jet Energy Scale Uncertainties}
\label{sec:sysjet}

The standard jet energy scale uncertainties are expected to be among the largest experimental uncertainties for this analysis. They have been calculated using the \emph{JetUncertaintiesTool}~\cite{JetUncertaintiesTool}. These uncertainties have many sub-components, including those derived from: \emph{in-situ} analysis, $\eta$ calibration, high-$p_\text{T}$ jets, pileup, flavour composition, flavour response, $b$-jets and punch-through jets. The \emph{SIMPLE} configuration is used, in which a principle component analysis is used to combine the different components of the jet uncertainty, in such a way as to preserve correlations in certain jet-kinematic regions, resulting in 30 nuisance parameters. We are currently using \emph{Scenario1}. This results in an uncertainty that varies between $^{+0.449}_{-11.1}\%$ and $^{+3.99}_{-21.3}\%$ of the total signal normalisation for the 0-4~GeV $a$ signal hypotheses, an uncertainty of $^{+0.533}_{-20.5}\%$ for the $\eta_c$ signal sample normalisation, and an uncertainty of $^{+1.63}_{-13.0}\%$ for the $J/\psi$ signal sample normalisation.

The impact of the jet energy scale on the signal selection efficiencies is asymmetric due to the requirement applied on the three body mass. If the jet energy scale is increased (decreased), the signal peak in the three body mass distribution is shifted up (down), and more events are lost through the upper (lower) side of the window that are gained through the lower (upper). So a shift up or down in the jet energy scale causes a lowering of the signal efficiency. This causes in an asymmetric likelihood, as shown in Figure~\ref{fig:likelihood}. This asymmetry causes a deterioration in the expected limit in the event of a null observation, but has almost no affect on the discovery potential. This uncertainty is labelled \emph{JES} in the correlation and pull plots.


\subsubsection{Pileup Uncertainty}
\label{sec:sysprw}

The uncertainty on the pileup distribution affects many aspects of the event, especially the MLP input variables, which in turn affects the efficiency of the MLP requirement. This is modelled by reweighting the pileup distributions using the \emph{CP::PileupReweightingTool} tool~\cite{Buttinger:2014726}, using the actual-$\mu$ distribution for the 2017 data as per the recommendations. This uncertainty covers the discrepancy seen between predicted and measured inelastic cross-section in the fiducial volume defined by $M_X>13$~GeV, where $M_X$ is the mass of the non-diffractive hadronic system. This discrepancy arises because of mismodelling of the central activity by the MC tune, so can be incorporated into analysis as an uncertainty on the mean number of hard interactions per bunch crossing that a given MC event corresponds to. This results in a total uncertainty that varies between $^{-1.38}_{+1.39}\%$ and $^{-2.03}_{+1.84}\%$ of the total signal normalisation for the different 0-4~GeV $a$ signal hypotheses, an uncertainty of $^{-0.455}_{+0.0357}\%$ for the $\eta_c$ signal sample normalisation, and an uncertainty of $^{-1.19}_{+1.53}\%$ for the $J/\psi$ signal sample normalisation. The anti-correlation between the pileup NP and the signal normalisations is due to the fact that signal jets have less Ghost-Associated tracks, and so events with higher pileup are less likely to pass the MLP cut. This uncertainty is labelled \emph{PRW} in the correlation and pull plots.


\subsubsection{Luminosity Uncertainty}
\label{sec:syslumi}

The uncertainty in the combined 2015-2018 integrated luminosity is 1.7\%. It is derived from the calibration of the luminosity scale using $x$-$y$ beam-separation scans, following a methodology similar to that detailed in Ref.~\cite{DAPR-2013-01}, and using the LUCID-2 detector for the baseline luminosity measurements \cite{LUCID2}. This uncertainty is labelled \emph{LUMI} in the correlation and pull plots.

%% The integrated luminosity is used to calculate the total expected signal, and therefore the uncertainty on the luminosity affects all of the signal normalisations. The uncertainty on each componenet of the integrated luminosity is taken from the official ATLAS recommendations as 2\% for the 2018 data, 2.4\% for the 2017 data, 2.2\% for the 2016 data, and 2.1\% for the 2015 data. Weighting each component of the luminosity by the relevant uncertainty results in a total uncertainty on the integrated luminosity of 2.18\%.

%% A luminosity uncertainty of 2.18\% was used in the calculation of the results in this note, taken as the weighted average of the uncertainties for each period. This uncertainty will be updated in the statistical model for the next iteration of the note. It is not expected to change the results significantly.


\subsubsection{Lepton Uncertainties}
\label{sec:syslep}

Data-driven uncertainties on the reconstruction, identification and isolation of electrons, as well as the reconstruction, isolation and track-to-vertex association of muons are considered in this analysis. These uncertainties are derived using $Z\to \ell^+\ell^-$ events, in addition to $J/\psi\to \mu\mu$ events for muons. The uncertainties for electrons (muons) are all computed using the \emph{AsgElectronEfficiencyCorrectionTool}~\cite{AsgElectronEfficiencyCorrectionTool} (\emph{CP::MuonEfficiencyScaleFactors}~\cite{MuonEfficiencyScaleFactors}) tool, as recommended and provided by the egamma (muon) CP group. For the electron tool, the \emph{TOTAL} correlation model is used, in conjunction with the \emph{Moriond\_February2018\_v1} configuration files. As all of these lepton uncertainties small, for convenience they are added in quadrature as though they were uncorrelated, and evaluated as one total lepton uncertainty. This results in a total uncertainty that varies between $^{+0.943}_{-0.941}\%$ and $^{+1.08}_{-1.08}\%$ of the signal normalisation for the various 0-4~GeV $a$ signal hypotheses, an uncertainty of $^{+1.05}_{-1.05}\%$ for the $\eta_c$ signal sample normalisation, and an uncertainty of $^{+1.03}_{-1.02}\%$ for the $J/\psi$ signal sample normalisation. These uncertainties are labelled \emph{LEP} in the correlation and pull plots.%% These uncertainties are not exactly symmetric, but only look that way due to the precision with which they are quoted.
 %% The total lepton uncertainty on the total background normalisation is $^{+1.06}_{-1.06}\%$. These uncertainties are not exactly symmetric, but only look that way due to the precision with which they are quoted.


\subsubsection{Jet Vertex Tagging Uncertainties}
\label{sec:sysjvt}

Requirements are placed on the impact parameters of the jets from the primary vertex. These impact parameters have associated experimental uncertainties, which influence the efficiency of the jet vertex tagging cut. These efficiency scale factors are calculated using the \emph{JetUpdateJvt} tool~\cite{JetUpdateJvt}. These uncertainties are between about $0.2\%$ and $0.6\%$ for $\vert\eta\vert<2.4$, and between about $0.1\%$ and $0.2\%$ for $2.4<\vert\eta\vert<2.5$, depending on the $p_\text{T}$ of the jet. This results in an uncertainty that varies between $^{+0.584}_{-0.578}\%$ and $^{+0.647}_{-0.639}\%$ of the total signal normalisation for the 0-4~GeV $a$ signal hypotheses, an uncertainty of $^{+0.610}_{-0.604}\%$ for the $\eta_c$ signal sample normalisation, and an uncertainty of $^{+622}_{-0.614}\%$ for the $J/\psi$ signal sample normalisation. This uncertainty is labelled \emph{JVT} in the correlation and pull plots.


\subsubsection{Trigger Efficiency Uncertainty}
\label{sec:systrig}

Data-driven uncertainties on the trigger efficiencies are also considered in this analysis. Due to the multiple triggers used for this analysis, these uncertainties are computed using the \emph{TrigGlobalEfficiencyCorrectionTool}~\cite{TrigGlobalEfficiencyCorrectionTool}, taking the \emph{CP::MuonTriggerScaleFactors}~\cite{MuonTriggerScaleFactors} and \emph{AsgElectronEfficiencyCorrectionTool}~\cite{AsgElectronEfficiencyCorrectionTool} tools as inputs, as per the recommendations of the egamma and muons CP groups. For the electron trigger efficiency tools, the \emph{TOTAL} correlation model is used, in conjunction with the \emph{Consolidation\_September2018\_v1} configuration files. For the muon trigger efficiencies, both the statistical and systematic components of the uncertainty are calculated. As these are all fairly small uncertainties, for convenience they are added in quadrature as though they were uncorrelated, and evaluated as one total trigger uncertainty. This results in an uncertainty that varies between $^{+0.270}_{-0.115}\%$ and $^{+0.319}_{-0.134}\%$ of the total signal normalisation for the 0-4~GeV $a$ signal hypotheses, an uncertainty of $^{+0.344}_{-0.144}\%$ for the $\eta_c$ signal sample normalisation, and an uncertainty of $^{+0.328}_{-0.122}\%$ for the $J/\psi$ signal sample normalisation. This uncertainty is labelled \emph{TRIG} in the correlation and pull plots.%% The total uncertainty on the total background normalisation is $^{+0.398}_{-0.0945}\%$.% The background uncertainty is small because to the transfer-factor approach reduces the reliance of the background estimate on the . -0.115=-0.1147


%% \subsubsection{Jet Energy Scale}
%% \label{sec:sysjes}


%% \subsubsection{Jet Energy Resolution}
%% \label{sec:sysjer}


%% \subsubsection{Track Selection Uncertainties}
%% \label{sec:systrackselection}


\subsubsection{Tracking Uncertainties}
\label{sec:systrack}

The effect of the track smearing uncertainties on the MLP input variables has been estimated using the \emph{InDet::IInDetTrackSmearingTool}~\cite{tracksystematics}. This is propagated through to the MLP cut efficiency, and found to have a negligible impact on the signal efficiency, and as such is not included in the fit. The impact of these uncertainties has less than a $\sim 0.4\%$ effect on the signal for all systematic variations and all signal hypotheses.


%-------------------------------------------------------------------------------
\clearpage
\section{Statistical Interpretation}
\label{sec:stats}
%-------------------------------------------------------------------------------

\subsection{Statistical Model}
\label{sec:statmodel}

The statistical interpretation of the result is performed using a single-bin profile likelihood fit to the signal region, using the \textsc{RooStats} and \textsc{RooFit} frameworks. This uses a binned profile likelihood fit to extract the final results from the observed number of events, and the signal and background estimates described in the previous two sections. This likelihood is given by the product of the Poisson probability term for the signal region, and the Gaussian constraints on the various nuisance parameters which represent the systematic uncertainties described in Section~\ref{sec:systematics}:

$$\mathcal{L}=\text{Pois} \bigg(N^\text{D}_\text{SR}; \bm{\mu} \mathcal{L}_\text{int} \sigma_\text{SM}(H) \mathcal{B}(Z\to\ell^+\ell^-) \text{Eff}^\text{MC}_\text{SR} \prod_{i \epsilon S} \bm{\alpha_i} + A^\text{ABCD Est.}_\text{SR} \prod_{i \epsilon B} \bm{\alpha_i} \bigg) \times \prod_{i \epsilon S+B} \text{Gaus} \bigg(1; \bm{\alpha_i}, \sigma_i\bigg) \text{,}$$

where variables in bold are free in the fit. $N^\text{D}_\text{SR}$ is the observed number of data events in the signal region, $\mathcal{L}_\text{int}$ is the integrated luminosity, $\sigma_\text{SM}(H)$ is the SM Higgs production cross section, $\mathcal{B}(Z\to\ell^+\ell^-)$ is the BR of the $Z$~boson to electron or muon pairs and $\text{Eff}^\text{MC}_\text{SR}$ is the signal efficiency in the SR, as determined in MC. The parameter of interest $\mu$ scales the signal in the fit, and is left free. The $\alpha$ parameters represent the nuisance parameters, which model the effect of the systematic uncertainties, and are described in Sections~\ref{sec:systematicstheory} and \ref{sec:systematicsexperiment}. A likelihood ratio test statistic is then defined as

$$q_\mu = -2\ln(\mathcal{L}(\mu, \hat{\hat{\alpha}})/\mathcal{L}(\hat{\mu}, \hat{\alpha}))\text{,}$$

where $\hat{\mu}$ and $\hat{\alpha}$ are the values of the parameters which maximise the likelihood, and $\hat{\hat{\alpha}}$ are the values which maximise $\mathcal{L}$ given a certain value of $\mu$. This test statistic is used to measure the compatibility between the background-only model and the data. We then define the local $p_0$ value as the probability, assuming the background-only model, that we would have observed a test statistic at least as incompatible with the background-only model than the observed test statistic. This is then used to form exclusion intervals using the $CL_s$ method~\cite{Cowan:2010js,Read_2002}.


\subsection{Asimov Fits}
\label{sec:asimovfits}

Asimov datasets are defined based on the background hypothesis, and varying levels of signal. These are then fit with the signal plus background model, and the asymptotic approximation is used to produce uncertainties and limits. The uncertainties on the signal normalisation, along with 95\% $CL_s$ upper limits on both $\mu$ and $\sigma(H)\times \mathcal{B}(H\to Za)$, expected in the absence of a signal are summarised in Table~\ref{tab:asimovresults}.% The 95\% $CL_s$ upper limits on $\mu$ are also given in Figure~\ref{fig:asimovresultplot}.

%% Latest Results (With NNLO Signal)
\begin{table}[!htbp]{\footnotesize\renewcommand{\arraystretch}{1.2} %% Updated
  \begin{center}
    \footnotesize
    %% {\centering\small\BF{Expected Results}}\\
    %% \vspace{1mm}
    \begin{tabular}{|c|c|cc|}
      \cline{3-4}
      \multicolumn{2}{c}{} & \multicolumn{2}{|c|}{95\% $\text{CL}_s$ Limit}\\
      \hline
      $a$ mass /~GeV & $\Delta\mu$ (\%) & $\mu$ (\%) & $\sigma(H)\times \mathcal{B}(H\to Za)/\text{pb}$ \\
      \hline
      0.5 & 30.6 & $32.3 ^{+12.7}_{-9.0} $ & $18.0^{+7.1}_{-5.0}$ \\
      0.75 & 29.8 & $36.8 ^{+14.4}_{-10.3} $ & $20.3^{+8.0}_{-5.7}$ \\
      1 & 30.3 & $36.6 ^{+14.4}_{-10.2} $ & $20.0^{+7.9}_{-5.6}$ \\
      1.5 & 30.1 & $40.7 ^{+16.0}_{-11.4} $ & $22.5^{+8.8}_{-6.3}$ \\
      2 & 32.2 & $50.0 ^{+19.6}_{-14.0} $ & $27.5^{+10.8}_{-7.7}$ \\
      2.5 & 39.3 & $72.2 ^{+28.3}_{-20.2} $ & $39.9^{+15.6}_{-11.1}$ \\
      3 & 72.0 & $141 ^{+55}_{-40} $ & $78.3^{+30.7}_{-21.9}$ \\
      3.5 & 97.3 & $195 ^{+76}_{-54} $ & $108^{+42}_{-30}$ \\
      4 & 342 & $675 ^{+265}_{-189} $ & $374^{+146}_{-104}$ \\
      \hline
      $\eta_c$ & 91.0 & $184 ^{+72}_{-51} $ & $102^{+40}_{-28}$ \\
      $J/\psi$ & 89.8 & $180 ^{+71}_{-50} $ & $99.7^{+39.0}_{-27.8}$ \\
      \hline
    \end{tabular}
    \caption{Uncertainties on the signal strength parameters for $\mu=1$, and 95\% $\text{CL}_s$ upper limits on $\sigma(H) \times\mathcal{B}(H\to Za) / \sigma_\text{SM}(H)$ expected in the absence of a signal. $\Delta\mu$ is the post-fit uncertainty on the parameter of interest: $\sigma(H) \times\mathcal{B}(H\to Za) / \sigma_\text{SM}(H)$. The uncertainties, $\Delta\mu$, are the mean of the upward and downward \textsc{MINOS} uncertainties. While the 68\% intervals are given for all quoted limits. \textsc{Pythia}8 $a$ BRs are assumed, using the default \textsc{BSMHiggs} $\tan\beta$ value of 1, as given in Table~\ref{tab:a0brs}.}
    \label{tab:asimovresults}
  \end{center}}
\end{table}

%% %% Old Results
%% \begin{table}[!htbp]{\footnotesize\renewcommand{\arraystretch}{1.2}
%%   \begin{center}
%%     \footnotesize
%%     %% {\centering\small\BF{Expected Results}}\\
%%     %% \vspace{1mm}
%%     \begin{tabular}{|c|c|cc|}
%%       \cline{3-4}
%%       \multicolumn{2}{c}{} & \multicolumn{2}{|c|}{95\% $\text{CL}_s$ Limit}\\
%%       \hline
%%       $a$ mass /~GeV & $\Delta\mu$ (\%) & $\mu$ (\%) & $\sigma(H)\times \mathcal{B}(H\to Za)/\text{pb}$ \\
%%       \hline
%%       0.5 & 36.1 & $43.4 ^{+17.0}_{-12.1} $ & $24.0^{+9.4}_{-6.7}$ \\
%%       0.75 & 35.3 & $48.8 ^{+19.1}_{-13.6} $ & $27.0^{+10.6}_{-7.5}$ \\
%%       1 & 36.8 & $48.5 ^{+19.0}_{-13.6} $ & $26.8^{+10.5}_{-7.5}$ \\
%%       1.5 & 38.3 & $54.8 ^{+21.5}_{-15.3} $ & $30.3^{+11.9}_{-8.5}$ \\
%%       2 & 43.3 & $68.8 ^{+27.0}_{-19.2} $ & $38.0^{+14.9}_{-10.6}$ \\
%%       2.5 & 56.8 & $104 ^{+41}_{-29} $ & $57.5^{+22.5}_{-16.1}$ \\
%%       3 & 96.1 & $200 ^{+78}_{-56} $ & $111^{+43}_{-31}$ \\
%%       3.5 & 126 & $269 ^{+105}_{-75} $ & $149^{+58}_{-42}$ \\
%%       4 & 449 & $994 ^{+389}_{-278} $ & $551^{+216}_{-154}$ \\
%%       \hline
%%       $\eta_c$ & 119 & $251 ^{+98}_{-70} $ & $139^{+55}_{-39}$ \\
%%       $J/\psi$ & 117 & $250 ^{+98}_{-70} $ & $138^{+54}_{-39}$ \\
%%       \hline
%%     \end{tabular}
%%     \caption{Uncertainties on the signal strength parameters for $\mu=1$, and 95\% $\text{CL}_s$ upper limits on $\sigma(H) \times\mathcal{B}(H\to Za) / \sigma_\text{SM}(H)$ expected in the absence of a signal. $\Delta\mu$ is the post-fit uncertainty on the parameter of interest: $\sigma(H) \times\mathcal{B}(H\to Za) / \sigma_\text{SM}(H)$. The uncertainties, $\Delta\mu$, are the mean of the upward and downward \textsc{MINOS} uncertainties. While the 68\% intervals are given for all quoted limits. \textsc{Pythia}8 $a$ BRs are assumed, using the default \textsc{BSMHiggs} $\tan\beta$ value of 1, as given in Table~\ref{tab:a0brs}.}
%%     \label{tab:asimovresults}
%%   \end{center}}
%% \end{table}

%% \begin{figure}[!htbp]
%%   \centering
%%   \includegraphics[width=0.7\textwidth]{figures/LimitPlots/LimitPlot.png}
%%   \caption{95\% $\text{CL}_s$ median expected upper limits on $\sigma(H) \text{BR}(H\to Za) / \sigma_\text{SM}(H)$. The $1\sigma$ and $2\sigma$ bands are also shown.}
%%   \label{fig:asimovresultplot}
%% \end{figure}

The profiled likelihood curve, pull plot and correlation plot for the fit to the 1.5~GeV $a$ signal hypothesis, are shown in Figures~\ref{fig:likelihood}, \ref{fig:pulls} and \ref{fig:corr}, respectively. Signal injection tests are performed, in which various amounts of 1.5~GeV $a$ signal are injected into the Asimov dataset, and then fitted to recover the fitted level of signal. The fitted signal consistently recovers the injected signal, as shown in Figure~\ref{fig:injectiontest}. %% These plots are shown for all the other signal hypotheses in Appendix~\ref{app:asimovfits}.

%% \begin{table}[!htbp]{\footnotesize\renewcommand{\arraystretch}{1.2}
%%   \begin{center}
%%     \footnotesize
%%     {\centering\small\BF{Expected Stat-Only Results}}\\
%%     \vspace{1mm}
%%     \begin{tabular}{|c|cccccccccc|cc|}
%%       \hline
%%       $a$ mass /~GeV & 0.5 & 0.75 & 1 & 1.5 & 2 & 2.5 & 3 & 3.5 & 4 & 8 & $\eta_c$ & $J/\psi$ \\
%%       \hline
%%       $\Delta\mu_\text{Sig}$ (\%) & 1.57 & 1.79 & 1.78 & 2.04 & 2.59 & 3.96 & 7.23 & 9.73 & 38.8 & 142 & 9.24 & 8.97\\
%%       95\% CL Limit (\%)          & 3.08 & 3.51 & 3.49 & 4.00 & 5.08 & 7.75 & 14.2 & 19.1 & 76.0 & 277 & 18.1 & 17.6 \\
%%       \hline
%%     \end{tabular}
%%   \end{center}}
%%   \caption{Uncertainties on the signal normalisations, and limits on $BR(H\to ZX)\times BR(X\to \text{hadrons})$, expected in the absence of a signal. $\Delta\mu$ is the post-fit uncertainty on the parameter of interest: $BR(H\to ZX)\times BR(X\to \text{hadrons})$, and $\Delta\mu_{BKGD}$ is the fractional uncertainty on the background normalisation. No systematics have been included in these fits}
%%   \label{tab:asimovresults}
%% \end{table}

\begin{figure}[!htbp]
  \centering
  \subfigure[Without Systematics]{\includegraphics[width=0.475\textwidth]{figures/pll_a01p5_noSys.png}}
  \subfigure[With Systematics]{\includegraphics[width=0.475\textwidth]{figures/pll_a01p5.png}}
  \caption{The profiled likelihood curve for the fit to the Asimov dataset using the 1.5~GeV $a$ signal hypothesis, with (a) no systematics except the background MC statistical uncertainty and (b) with all systematics, for a dataset with $\mu=1$.}
  \label{fig:likelihood}
\end{figure}

\begin{figure}[!htbp]
  \centering
  \includegraphics[width=0.5\textwidth]{figures/SimpleFits_03Mar20/a01p5/Pull_data_wGamma.png}
  \caption{The pull plot for the fit to the Asimov dataset using the 1.5~GeV $a$ signal hypothesis, with systematics.}
  \label{fig:pulls}
\end{figure}

\begin{figure}[!htbp]
  \centering
  \includegraphics[width=0.7\textwidth]{figures/SimpleFits_03Mar20/a01p5/Corr_data.png}
  \caption{The correlation matrix for the fit to the Asimov dataset using the 1.5~GeV $a$ signal hypothesis, with systematics. Only entries with values greater than 0.5\% are plotted.}
  \label{fig:corr}
\end{figure}

\begin{figure}[!htbp]
  \centering
  \includegraphics[width=0.9\textwidth]{figures/InjectionTest_03Mar20.pdf}
  \caption{Plot of injected signal $\sigma(H)\times BR(H\to ZX)/\sigma_\text{SM}(H)$ against fitted signal $\sigma(H)\times BR(H\to ZX)/\sigma_\text{SM}(H)$ for the 1.5~GeV $a$ signal hypothesis.}
  \label{fig:injectiontest}
\end{figure}


%% \subsection{Fits to Toy Datasets}
%% \label{sec:toyfits}

%% Toy datasets are also generated from the background PDF. These are then each fit individually in the absence of systematics to produce a point in a statistical distribution of limits. The distribution of 95\% $CL_s$ upper limits on $\mu$ expected in the absence of a signal for the 1.5~GeV $a$ signal sample is shown in Figure~\ref{fig:toylimits}.


%% \begin{figure}[!htbp]
%%   \centering
%%   \subfigure[Without Systematics]{\includegraphics[width=0.475\textwidth]{figures/toyLimitsa01p5noSys.png}}
%%   \subfigure[With Systematics]{\includegraphics[width=0.475\textwidth]{figures/toyLimitsa01p5.png}}
%%   \caption{The distribution of 95\% $CL_s$ upper limits on $\mu$ expected in the absence of a signal for the 1.5~GeV $a$ signal sample, as determined to fits to toy datasets. Results are shown for (a) no systematics and (b) with all systematics.}
%%   \label{fig:toylimits}
%% \end{figure}


\subsection{Uncertainty Breakdown}
\label{sec:uncertaintybreakdown}

The breakdown of the uncertainties on $\mu$ are given for three example signal hypotheses in Table~\ref{tab:uncertaintybreakdown}. The total uncertainty is approximately 99.7\% systematic, the vast majority of which is due to the background modelling uncertainty. Due to the dominance of the background uncertainties, the pre-fit impact of the uncertainties described in Section~\ref{sec:systematics} on the background estimate, are summarised in Table~\ref{tab:prefitbkgdsys}.

\begin{table}[!htbp]{\footnotesize\renewcommand{\arraystretch}{1.25}
  \begin{center}
    \footnotesize
    %% BDT Optimisation Results\\
    \begin{tabular}{cccc}
      \toprule
      $a$ mass & 0.5 GeV & 1.5 GeV & 2.5 GeV \\
      \midrule
      Total Uncertainty & 8.3 (100\%) & 10.7 (100\%) & 20.3 (100\%) \\
      %% \midrule
      Total Statistical Uncertainty & 0.6 (7.6\%) & 0.8 (7.6\%) & 1.6 (7.7\%) \\
      %% \midrule
      Total Systematic Uncertainty & 8.2 (99.7\%) & 10.7 (99.7\%) & 20.2 (99.7\%) \\
      \midrule
      \multicolumn{4}{c}{Signal Systematic Uncertainties} \\
      \midrule
      Jet Energy Scale & 1.3 (16.0\%) & 1.5 (13.5\%) & 1.5 (7.5\%) \\
      Parton Shower & 1.4 (16.5\%) & 1.4 (13.4\%) & 1.4 (7.1\%) \\
      Luminosity, Pileup, Trigger, Leptons, \& JVT & 0.2 (2.7\%) & 0.3 (2.6\%) & 0.5 (2.4\%) \\
      MC Statistics & 0.2 (2.0\%) & 0.2 (1.9\%) & 0.6 (2.7\%) \\
      Renormalization Scale & 0.1 (1.0\%) & $<0.1$ ($<1.0\%$) & 0.2 (1.1\%) \\
      Acceptance & 0.1 (1.0\%) & $<0.1$ ($<1.0\%$) & 0.2 (0.8\%) \\
      \midrule
      \multicolumn{4}{c}{Background Systematic Uncertainties} \\
      \midrule
      MC Statistics & 6.4 (77.8\%) & 8.4 (77.8\%) & 15.8 (77.8\%) \\
      Parton Shower and ME & 3.9 (47.1\%) & 5.1 (47.1\%) & 9.6 (47.1\%) \\
      Renormalization Scale & 3.4 (40.9\%) & 4.4 (40.9\%) & 8.3 (40.9\%) \\
      \bottomrule
    \end{tabular}
    \caption{Breakdown of the impact of the various sources of uncertainty on the estimate of $\sigma(pp\to H)\text{BR}(H\to Za)/\text{pb}$ for three different $a$ mass hypotheses, derived from fits to the background-only Asimov dataset. The relative contribution of each source of uncertainty is given in parentheses. JVT and ME refer to the jet vertex tagging and matrix element uncertainties, respectively. The uncertainties are taken as the mean of the upward and downward \textsc{MINOS} uncertainties. The uncertainties are evaluated by removing them from the fit, and subtracting the overall uncertainty on $\mu$ without that parameter from with it in quadrature. The residual non-closure is due to correlations between the various nuisance parameters.}
  \label{tab:uncertaintybreakdown}
  \end{center}}
\end{table}


\begin{table}[!htbp]{\footnotesize\renewcommand{\arraystretch}{1.2}
  \begin{center}
    \footnotesize
    \begin{tabular}{|c|c|}
      \hline
      Source of Uncertainty & Impact on Background Estimate (\%) \\
      \hline
      MC Statistics & 3.48 \\
      Renormalisation Scale & 1.83 \\
      Parton Shower and ME & 2.11 \\
      \hline
    \end{tabular}
  \caption{Pre-fit impact of the uncertainties described in Section~\ref{sec:systematics} on the background estimate.}
  \label{tab:prefitbkgdsys}
  \end{center}}
\end{table}

%% \begin{table}[!htbp]{\footnotesize\renewcommand{\arraystretch}{1.2}
%%   \begin{center}
%%     \footnotesize
%%     %% BDT Optimisation Results\\
%%     \begin{tabular}{|c|ccc|}
%%       \hline
%%       $a$ mass & 0.5~GeV & 1.5~GeV & 2.5~GeV \\
%%       \hline
%%       %% 0.11057850 0.14319350 0.27125050
%%       \BF{Total Uncertainty} & 0.22 (100\%) & 0.28 (100\%) & 0.54 (100\%) \\
%%       \hline
%%       \BF{Total Statistical Uncertainty} & 0.011 (5.2\%) & 0.015 (5.2\%) & 0.028 (5.2\%) \\
%%       \hline
%%       \BF{Total Systematic Uncertainty} & 0.22 (99.9\%) & 0.28 (99.9\%) & 0.54 (99.9\%) \\
%%       \hline
%%       MC Statistics & 0.15 (66.6\%) & 0.19 (66.6\%) & 0.36 (66.6\%) \\
%%       Parton Shower and ME & 0.11 (51.4\%) & 0.15 (51.5\%) & 0.29 (53.1\%) \\
%%       Renormalisation Scale & 0.12 (56.6\%) & 0.16 (56.8\%) & 0.31 (56.7\%) \\
%%       Higgs Cross Section and Acceptance & 0.014 (6.2\%) & 0.017 (6.2\%) & 0.033 (6.2\%) \\
%%       \hline
%%       Jet Energy Scale & 0.034 (15.7\%) & 0.036 (12.7\%) & 0.039 (7.2\%) \\
%%       Luminosity, Pileup, Trigger, Leptons, \& JVT & 0.0064 (2.9\%) & 0.0075 (2.6\%) & 0.014 (2.6\%) \\
%%       \hline
%%     \end{tabular}
%%   \caption{Breakdown of the uncertainties on $\mu$ for 3 $a$ mass hypotheses, derived from fits to the background-only Asimov dataset. The fraction of the total uncertainty is given in parentheses. The uncertainties are taken as the mean of the upward and downward \textsc{MINOS} uncertainties. The uncertainties are evaluated by removing them from the fit, and subtracting the overall uncertainty on $\mu$ without that parameter from with it in quadrature. The residual non-closure is due to correlations between the various nuisance parameters.}
%%   \label{tab:uncertaintybreakdown}
%%   \end{center}}
%% \end{table}

%% \begin{table}[!htbp]{\footnotesize\renewcommand{\arraystretch}{1.2}
%%   \begin{center}
%%     \footnotesize
%%     %% BDT Optimisation Results\\
%%     \begin{tabular}{|c|ccc|}
%%       \hline
%%       $a$ mass & 0.5~GeV & 1.5~GeV & 2.5~GeV \\
%%       \hline
%%       %% 0.11057850 0.14319350 0.27125050
%%       \BF{Total Uncertainty} & 0.111 (100\%) & 0.143 (100\%) & 0.271 (100\%) \\
%%       \hline
%%       \BF{Total Statistical Uncertainty} & 0.0111 (10.0\%) & 0.0143 (10.0\%) & 0.0272 (10.0\%) \\
%%       \hline
%%       \BF{Total Systematic Uncertainty} & 0.110 (99.5\%) & 0.142 (99.5\%) & 0.270 (99.5\%) \\
%%       Background MC Statistics & 0.110 (99.5\%) & 0.142 (99.5\%) & 0.270 (99.5\%) \\
%%       Higgs Cross Section & 0.00703 (6.36\%) & 0.00911 (6.36\%) & 0.0172 (6.36\%) \\
%%       Pileup & 0.00210 (1.90\%) & 0.00224 (1.56\%) & 0.00280 (1.03\%) \\
%%       Signal MC Statistics & 0.00188 (1.70\%) & 0.00273 (1.91\%) & 0.00671 (2.47\%) \\
%%       Luminosity & 0.00188 (1.70\%) & 0.00242 (1.69\%) & 0.00393 (1.45\%) \\
%%       Leptons & 0.00124 (1.13\%) & 0.00156 (1.09\%) & 0.00173 (0.637\%) \\
%%       JVT & 0.000665 (0.601\%) & 0.000846 (0.591\%) & 0 (0\%) \\
%%       Trigger & 0.000333 (0.301\%) & 0.000378 (0.264\%) & 0 (0\%) \\
%%       \hline
%%     \end{tabular}
%%   \caption{Breakdown of the uncertainties on $\mu$ for 3 $a$ mass hypotheses. The fraction of the total uncertainty is given in parentheses. The uncertainties are taken as the mean of the upward and downward \textsc{MINOS} uncertainties. The uncertainties are evaluated by removing them from the fit, and subtracting the overall uncertainty on $\mu$ without that parameter from with it in quadrature. The residual non-closure is due to correlations between the various nuisance parameters.}
%%   \label{tab:uncertaintybreakdown}
%%   \end{center}}
%% \end{table}


\subsection{Model-Independent Interpretation}
\label{sec:modelindependentresults}

The nominal interpretation strategy assumes either a 2HDM(+s) or SM charmonium signal hypothesis, which although well motivated do not describe all possible processes which can produce this final state. The assumed BRs for these final states in each model leads to model-dependent limits which are only valid for the model under consideration, and not valid for other models.

To generalise the results of this search, model-independent results will be provided under the following assumptions. First, as the focus of this search, only hadronic decays are considered. Second, due to the Yukawa-ordering of the decays of Higgs bosons, only decays to gluons and the heaviest kinematically accessible quark will be considered. Third, due to the low masses of the first generation quarks, decays to these final states will not be considered. Fourth, the systematic uncertainties on each exclusive decay for any given sample, are the same as those for the inclusive decay for that sample. The motivation and justification for the fourth assumption is given in the next paragraph.

The efficiencies are then reevaluated for exclusive decays of the $a$ to gluons, and either $s$- or $c$-quarks, as shown in Table~\ref{tab:modelindependentefficiencies}. The decays to $c$-quarks occur only through charmonium resonances, for which the modelling is questionable. For this reason, the $a\to c\bar c$ results are not included in the paper. It can be seen that the efficiencies for quarks are consistently slightly higher than that for gluons. The limits from the inclusive search are then multiplied by the inclusive signal efficiency, and divided by the signal efficiencies to exclusive gluon or quark final states. These limits then represent the limits on $\sigma(H)\text{BR}(H\to Za)\times\text{BR}(a\to q/g)/\sigma_\text{SM}(H)$, under the assumption that the systematic uncertainties on the signal acceptance are the same for quark and gluon final states, and are given in Table~\ref{tab:modelindependentlimits}. Model-independent limits on $\sigma(H)\text{BR}(H\to Za)\times\text{BR}(a\to q/g)$ in Table~\ref{tab:modelindependentlimitsxsec}. This assumption allows the limit on any specific decay of the $a$ to be calculated, by a linear superposition of the two exclusive limits, which would not be possible if the systematics were considered exclusively. The impact of the systematic uncertainties on the signal, which may vary between gluon or quark final states, has just a $\sim 2.1\%$ impact on the $\Delta\mu$ for the 1.5~GeV $a$ signal hypothesis, justifying the assumption that these uncertainties can be approximated by their inclusive values. This only exception to this is for very low BR decays, such as the 8~GeV $a$ to $c\bar c$, in which the signal MC statistical uncertainty could be much larger than in the inclusive case. % , which is most likely due to the use of the $q/g$ tagger $U1$ in the MLP

\begin{table}[!htbp]{\footnotesize\renewcommand{\arraystretch}{1.2} %% updated
  \begin{center}
    \footnotesize
    \begin{tabular}{|c|ccc|}
      \hline
      $a$ mass /~GeV & $a\to gg$ & $a\to s\bar s$ & $a\to c\bar c$ \\
      \hline
      0.5 & $3.66 \pm 0.08$ & - & - \\
      0.75 & $2.97 \pm 0.06$ & - & - \\
      1 & $3.31 \pm 0.06$ & - & - \\
      1.5 & $2.74 \pm 0.06$ & $2.97 \pm 0.13$ & - \\
      2 & $2.13 \pm 0.05$ & $2.40 \pm 0.13$ & - \\
      2.5 & $1.36 \pm 0.04$ & $1.65 \pm 0.14$ & - \\
      3 & $0.727 \pm 0.028$ & $0.840 \pm 0.089$ & - \\
      3.5 & $0.475 \pm 0.066$ & - & $0.524 \pm 0.023$ \\
      4 & $0.162 \pm 0.054$ & - & $0.125 \pm 0.014$ \\
      \hline
    \end{tabular}
    \caption{Efficiencies of the full selection (pre-selection and MLP requirement) for exclusive gluon or quark decays of each signal sample. The selection efficiencies for $a\to q\bar q$ decays are generally higher than $a\to gg$ due to a higher MLP cut efficiency.}
    \label{tab:modelindependentefficiencies}
  \end{center}}
\end{table}

%% Sample              s-quark               c-quark               gluon
%% 0.5~GeV a0             -                     -              0.03611783233
%% 0.75~GeV a0            -                     -              0.02950960551
%% 1~GeV a0               -                     -              0.03306464164
%% 1.5~GeV a0      0.02957375352                -              0.02749549272
%% 2~GeV a0        0.02393102774                -              0.02140652393
%% 2.5~GeV a0      0.01657833101                -              0.01376095167
%% 3~GeV a0        0.008952631251               -              0.007364000392
%% 3.5~GeV a0             -              0.005434419575        0.004964709006
%% 4~GeV a0               -              0.001310382107        0.001696844779
%% 8~GeV a0               -              0.0002870243883       0.0001022038508


\begin{table}[!htbp]{\footnotesize\renewcommand{\arraystretch}{1.2} %% updated
  \begin{center}
    \footnotesize
    \begin{tabular}{|c|ccc|}
      \hline
      $a$ mass /~GeV & $a\to gg$ & $a\to s\bar s$ & $a\to c\bar c$ \\
      \hline
      0.5 & $28.9^{+11.3}_{-8.1}$ & - & - \\
      0.75 & $34.3^{+13.4}_{-9.6}$ & - & - \\
      1 & $31.6^{+12.4}_{-8.8}$ & - & - \\
      1.5 & $37.1^{+14.5}_{-10.4}$ & $34.2^{+13.4}_{-9.6}$ & - \\
      2 & $47.1^{+18.4}_{-13.2}$ & $41.7^{+16.3}_{-11.7}$ & - \\
      2.5 & $69.4^{+27.2}_{-19.4}$ & $57.2^{+22.4}_{-16.0}$ & - \\
      3 & $135^{+53}_{-38}$ & $117^{+46}_{-33}$ & - \\
      3.5 & $207^{+81}_{-58}$ & - & $188^{+74}_{-52}$ \\
      4 & $584^{+229}_{-163}$ & - & $755^{+296}_{-211}$ \\
      \hline
    \end{tabular}
    \caption{95\% $\text{CL}_s$ upper limits on $\sigma(H)\times\mathcal{B}(H\to Za)\times\mathcal{B}(a\to q/g)/\sigma_\text{SM}(H)$ expected in the absence of a signal. These results are derived for exclusive gluon or quark decays for each signal sample, using the Asimov dataset.}
    \label{tab:modelindependentlimits}
  \end{center}}
\end{table}

%% \begin{figure}[!htbp]
%%   \centering
%%   \subfigure[$a\to gg$]{\includegraphics[width=0.475\textwidth]{figures/LimitPlots/LimitPlot_gg.png}}
%%   \subfigure[$a\to q\bar q$]{\includegraphics[width=0.475\textwidth]{figures/LimitPlots/LimitPlot_qq.png}}
%%   \caption{Model-independent 95\% $\text{CL}_s$ median expected upper limits on $\sigma(H) \text{BR}(H\to Za) / \sigma_\text{SM}(H)$, for (a) $a\to gg$ and (b) $a\to q\bar q$ signals. The $1\sigma$ and $2\sigma$ bands are also shown.}
%%   \label{fig:asimovmiresultplot}
%% \end{figure}


\begin{table}[!htbp]{\footnotesize\renewcommand{\arraystretch}{1.2}
  \begin{center}
    \footnotesize
    \begin{tabular}{|c|ccc|}
      \hline
      $a$ mass /~GeV & $a\to gg$ & $a\to s\bar s$ & $a\to c\bar c$ \\
      \hline
      0.5 & $16.1^{+6.3}_{-4.5}$ & - & - \\
      0.75 & $18.9^{+7.4}_{-5.3}$ & - & - \\
      1 & $17.3^{+6.8}_{-4.8}$ & - & - \\
      1.5 & $20.4^{+8.0}_{-5.7}$ & $18.9^{+7.4}_{-5.3}$ & - \\
      2 & $25.9^{+10.1}_{-7.2}$ & $22.9^{+9.0}_{-6.4}$ & - \\
      2.5 & $38.3^{+15.0}_{-10.7}$ & $31.6^{+12.4}_{-8.8}$ & - \\
      3 & $74.6^{+29.2}_{-20.9}$ & $64.6^{+25.3}_{-18.0}$ & - \\
      3.5 & $115^{+45}_{-32}$ & - & $104^{+41}_{-29}$ \\
      4 & $323^{+127}_{-90}$ & - & $418^{+164}_{-117}$ \\
      \hline
    \end{tabular}
    \caption{95\% $\text{CL}_s$ upper limits on $\sigma(H)\times\mathcal{B}(H\to Za)\times\mathcal{B}(a\to q/g)/\text{pb}$ expected in the absence of a signal. These results are derived for exclusive gluon or quark decays for each signal sample, using the Asimov dataset.}
    \label{tab:modelindependentlimitsxsec}
  \end{center}}
\end{table}


%-------------------------------------------------------------------------------
\clearpage
\section{Validation Strategy}
\label{sec:validation}
%-------------------------------------------------------------------------------

A validation region is defined in the sideband of the MLP variable. It is as close as possible to the signal cut, containing the same amount of background as passes the signal cut. This includes all events with $0.0341<MLP<0.0524$. This region is used as a first validation of the analysis methods. This VR has an $S/\sqrt{B}$ improvement of less than 0.760 for all signal hypotheses, meaning that the full analysis can be performed in this VR as though it was the SR without unblinding the search.

89919 events were observed in this VR, to be compared with the background-only expectation of $88700 \pm 2800$ events. This region was fit with the full signal plus background hypothesis, including all the systematic uncertainties mentioned in Sections~\ref{sec:systematicstheory} and \ref{sec:systematicsexperiment}. The priors for the signal systematics were evaluated separately for this region. When fit with the 1.5~GeV $a$ signal and background hypotheses, the best fit number of signal events was 997, with a best fit number of background events of 88900. The corresponding best fit values for the signal strength parameter is: $\mu=0.160^{+0.667}_{-0.593}$. To test the relevant machinery, the 95\% $CL_s$ limit was found to be 131\%. Figures~\ref{fig:vrpulls} and \ref{fig:vrcorr} show the resulting pull plot and correlation matrix, respectively. %Figures~\ref{fig:vrlikelihood}, \ref{fig:vrpulls} and \ref{fig:vrcorr} show the resulting profiled likelihood, pull plot and correlation matrix, respectively.

%% \begin{figure}[!htbp]
%%   \centering
%%   \includegraphics[width=0.475\textwidth]{figures/pllVR_a01p5.png}
%%   \caption{The profiled likelihood curve for the fit to the data in the validation region using the 1.5~GeV $a$ signal hypothesis, with systematics.}
%%   \label{fig:vrlikelihood}
%% \end{figure}

\begin{figure}[!htbp]
  \centering
  \includegraphics[width=0.5\textwidth]{figures/SimpleFits_03Mar20/a01p5VR/Pull_data_wGamma.png}
  \caption{The pull plot for the fit to the data in the validation region using the 1.5~GeV $a$ signal hypothesis, with systematics.}
  \label{fig:vrpulls}
\end{figure}

\begin{figure}[!htbp]
  \centering
  \includegraphics[width=0.7\textwidth]{figures/SimpleFits_03Mar20/a01p5VR/Corr_data.png}
  \caption{The correlation matrix for the fit to the data in the validation region using the 1.5~GeV $a$ signal hypothesis, with systematics. Only entries with values greater than 0.5\% are plotted.}
  \label{fig:vrcorr}
\end{figure}

%% Total S/sqrt(B) improvement (all regions):
%% a00p5: 0.59382012
%% a00p75: 0.60710437
%% a01p0: 0.70254536
%% a01p5: 0.75745745
%% a02p0: 0.73258860
%% a02p5: 0.56623637
%% a03p0: 0.36472484
%% a03p5: 0.28752405
%% a04p0: 0.15033719
%% a08p0: 0.082403215
%% etac: 0.30063985
%% Jpsi: 0.29586778

%% More validation regions can be defined in the same way as the first, providing up to 97 checks of the analysis. However, these become significantly less valuable as the MLP output variable gets further from the signal region. After the analysis has been fully specified but before unblinding, a second validation region, defined as close as possible to the first while containing the same amount of signal, will be unblinded as a check of the analysis. If for some reason this pre-unblinding VR reveals an issue with the analysis, then this issue will be fixed, and the analysis tested in the next analogous VR before unblinding.


%-------------------------------------------------------------------------------
\clearpage
\section{Results}
\label{sec:result}
%-------------------------------------------------------------------------------

82908 events were observed in the signal region. This result is compatible with the SM background only expectation of $82400 \pm 3700\ (2900\oplus 2300)$ events, where the total uncertainty is followed by the uncertainty due to limited data and MC statistics, and then the systematic uncertainty. In the absence of a significant excess, 95\% $\text{CL}_s$ upper limits are set on $\sigma(H) \times\mathcal{B}(H\to Za) / \sigma_\text{SM}(H)$ and on $\sigma(H) \times\mathcal{B}(H\to Za)$, for both the nominal \textsc{Pythia}8 BRs, and the model independent interpretation described in Section~\ref{sec:modelindependentresults}. These are given in Table~\ref{tab:observedmodeldepedentresults}, and Tables~\ref{tab:observedmodelindepedentresults} and \ref{tab:observedmodelindepedentresultsxsec}, respectively. They are also given for the limits on $\sigma(H) \times\mathcal{B}(H\to Za) / \sigma_\text{SM}(H)$ in Figures~\ref{fig:observedmodeldepedentresults} and Figures~\ref{fig:observedmodelindepedentresults}, respectively, and for the limits on $\sigma(H) \times\mathcal{B}(H\to Za)$ in Figure~\ref{fig:observedmodelindepedentresultsxsecxbr}. In the absence of systematic uncertainties, these limits would range between $1.9~\text{pb}$ and $55~\text{pb}$. The pull and correlation plots for the fit to the observed data are given for the 1.5~GeV $a$ signal hypothesis in Figures~\ref{fig:observedpulls} and \ref{fig:observedcorr}, respectively. The regression MLP output and classification MLP outputs are given in Figure~\ref{fig:observedsrplots} for the events in the signal region. Finally, Figure~\ref{fig:observedmh} shows the three body mass distribution in the MLP signal region, without the three body mass cut applied.

\begin{table}[!htbp]{\footnotesize\renewcommand{\arraystretch}{1.2} %% updated
  \begin{center}
    \footnotesize
    %% {\centering\small\BF{Expected Results}}\\
    %% \vspace{1mm}
    \begin{tabular}{|c|c|cc|}
      \cline{3-4}
      \multicolumn{2}{c}{} & \multicolumn{2}{|c|}{95\% $\text{CL}_s$ Limit}\\
      \hline
      $a$ mass /~GeV & $\mu$ (\%) & $\mu$ (\%) & $\sigma(H)\times \mathcal{B}(H\to Za)/\text{pb}$ \\
      \hline
      0.5  & $0.036^{+0.132}_{-0.168}$ & 34.0 & 18.8 \\
      0.75 & $0.023^{+0.175}_{-0.176}$ & 38.8 & 21.5 \\
      1    & $0.022^{+0.169}_{-0.170}$ & 38.3 & 21.2 \\
      1.5  & $0.025^{+0.192}_{-0.195}$ & 42.8 & 23.7 \\
      2    & $0.031^{+0.239}_{-0.241}$ & 52.5 & 29.1 \\
      2.5  & $0.048^{+0.362}_{-0.368}$ & 75.7 & 41.9 \\
      3    & $0.088^{+0.688}_{-0.694}$ & 149 & 82.3  \\
      3.5  & $0.120^{+0.948}_{-0.952}$ & 206 & 113  \\ % (105.4)
      4    & $0.43^{+3.40}_{-3.43}$    & 710 & 392  \\ % 914.9
      \hline
      $\eta_c$ & $0.113^{+0.876}_{-0.883}$ & 193 & 107 \\
      $J/\psi$ & $0.107^{+0.867}_{-0.864}$ & 189 & 105 \\
      \hline
    \end{tabular}
    \caption{The observed model-dependent signal strength parameters and 95\% $\text{CL}_s$ upper limits on $\sigma(H) \times\mathcal{B}(H\to Za) / \sigma_\text{SM}(H)$ and $\sigma(H)\times \mathcal{B}(H\to Za)$ for the observed dataset. \textsc{Pythia}8 $a$ BRs are assumed, using the default \textsc{BSMHiggs} $\tan\beta$ value of 1, as given in Table~\ref{tab:a0brs}.}
    \label{tab:observedmodeldepedentresults}
  \end{center}}
\end{table}


\begin{table}[!htbp]{\footnotesize\renewcommand{\arraystretch}{1.2}
  \begin{center}
    \footnotesize
    \begin{tabular}{|c|ccc|}
      \hline
      $a$ mass /~GeV & $a\to gg$ & $a\to s\bar s$ & $a\to c\bar c$ \\
      \hline
      0.5  & 30.4 & -    & -    \\
      0.75 & 36.1 & -    & -    \\
      1    & 33.0 & -    & -    \\
      1.5  & 39.0 & 36.0 & -    \\
      2    & 49.4 & 43.8 & -    \\
      2.5  & 72.7 & 59.9 & -    \\
      3    & 142  & 123  & -    \\
      3.5  & 219  & -    & 198  \\
      4    & 614  & -    & 793  \\
      \hline
    \end{tabular}
    \caption{95\% $\text{CL}_s$ observed upper limits on $\sigma(H)\times\mathcal{B}(H\to Za)\times\mathcal{B}(a\to q/g)/\sigma_\text{SM}(H)$. These results are derived for exclusive gluon or quark decays for each signal sample.}
    \label{tab:observedmodelindepedentresults}
  \end{center}}
\end{table}


\begin{table}[!htbp]{\footnotesize\renewcommand{\arraystretch}{1.2}
  \begin{center}
    \footnotesize
    \begin{tabular}{|c|ccc|}
      \hline
      $a$ mass /~GeV & $a\to gg$ & $a\to s\bar s$ & $a\to c\bar c$ \\
      \hline
      0.5  & 16.8 & -    & -   \\
      0.75 & 20.0 & -    & -   \\
      1    & 18.2 & -    & -   \\
      1.5  & 21.6 & 19.9 & -   \\
      2    & 27.3 & 24.2 & -   \\
      2.5  & 40.2 & 33.1 & -   \\
      3    & 78.4 & 67.8  & -   \\
      3.5  & 121  & -    & 109 \\
      4    & 339  & -    & 438 \\
      \hline
    \end{tabular}
    \caption{95\% $\text{CL}_s$ observed upper limits on $\sigma(H)\times\mathcal{B}(H\to Za)\times\mathcal{B}(a\to q/g)/\text{pb}$. These results are derived for exclusive gluon or quark decays for each signal sample.}
    \label{tab:observedmodelindepedentresultsxsec}
  \end{center}}
\end{table}

\begin{figure}[!htbp]
  \centering
  \includegraphics[width=0.6\textwidth]{figures/LimitPlots_03Mar20/LimitPlot.pdf}
  \caption{95\% $\text{CL}_s$ observed, and median expected upper limits on $\sigma(H) \text{BR}(H\to Za) / \sigma_\text{SM}(H)$. The $1\sigma$ and $2\sigma$ bands are also shown. The limits are taken from linearly interpolating between the points for which MC signal samples are generated.}
  \label{fig:observedmodeldepedentresults}
\end{figure}

\begin{figure}[!htbp]
  \centering
  \subfigure[$a\to gg$]{\includegraphics[width=0.6\textwidth]{figures/LimitPlots_03Mar20/LimitPlot_gg.pdf}}
  \subfigure[$a\to q\bar q$]{\includegraphics[width=0.6\textwidth]{figures/LimitPlots_03Mar20/LimitPlot_ss.pdf}}
  \caption{Model-independent 95\% $\text{CL}_s$ observed, and median expected upper limits on $\sigma(H) \text{BR}(H\to Za) / \sigma_\text{SM}(H)$, for (a) $a\to gg$ and (b) $a\to q\bar q$ signals. The $1\sigma$ and $2\sigma$ bands are also shown. The limits are taken from linearly interpolating between the points for which MC signal samples are generated.}
  \label{fig:observedmodelindepedentresults}
\end{figure}

\begin{figure}[!htbp]
  \centering
  \subfigure[$a\to gg$]{\includegraphics[width=0.6\textwidth]{figures/LimitPlots_xsecxBR_03Mar20/LimitPlot_gg.pdf}}
  \subfigure[$a\to q\bar q$]{\includegraphics[width=0.6\textwidth]{figures/LimitPlots_xsecxBR_03Mar20/LimitPlot_ss.pdf}}
  \caption{Model-independent 95\% $\text{CL}_s$ observed, and median expected upper limits on $\sigma(H) \times\mathcal{B}(H\to Za)$, for (a) $a\to gg$ and (b) $a\to q\bar q$ signals. The $1\sigma$ and $2\sigma$ bands are also shown. The limits are taken from linearly interpolating between the points for which MC signal samples are generated.}
  \label{fig:observedmodelindepedentresultsxsecxbr}
\end{figure}


%% \begin{table}[!htbp]{\footnotesize\renewcommand{\arraystretch}{1.2}
%%   \begin{center}
%%     \footnotesize
%%     \begin{tabular}{|c|ccccccccc|}
%%       \hline
%%       $a$ mass /~GeV & 0.5 & 0.75 & 1 & 1.5 & 2 & 2.5 & 3 & 3.5 & 4 \\%& 8 \\
%%       \hline
%%       $a\to gg$      & 36.0 & 42.2 & 38.9 & 46.3 & 60.0 & 92.5 & 178 & 264 & 782  \\%& 734 \\
%%       $a\to s\bar s$ & -    & -    & -    & 43.1 & 53.7 & 76.7 & 146 & -   & -    \\%& - \\
%%       $a\to c\bar c$ & -    & -    & -    & -    & -    & -    & -   & 241 & 1010 \\%& 2620 \\
%%       \hline
%%     \end{tabular}
%%     \caption{95\% $\text{CL}_s$ observed upper limits on $\sigma(H)\times\mathcal{B}(H\to Za)\times\mathcal{B}(a\to q/g)/\sigma_\text{SM}(H)$. These results are derived for exclusive gluon or quark decays for each signal sample.}
%%     \label{tab:observedmodelindepedentresults}
%%   \end{center}}
%% \end{table}


\begin{figure}[!htbp]
  \centering
  \includegraphics[width=0.5\textwidth]{figures/SimpleFits_03Mar20/a01p5Obs/Pull_data_wGamma.png}
  \caption{The pull plot for the fit to the data in the signal region using the 1.5~GeV $a$ signal hypothesis, with systematics. \textsc{Pythia}8 $a$ BRs are assumed, using the default \textsc{BSMHiggs} $\tan\beta$ value of 1, as given in Table~\ref{tab:a0brs}.}
  \label{fig:observedpulls}
\end{figure}


\begin{figure}[!htbp]
  \centering
  \includegraphics[width=0.9\textwidth]{figures/SimpleFits_03Mar20/a01p5Obs/Corr_data.png}
  \caption{The correlation matrix for the fit to the data in the signal region using the 1.5~GeV $a$ signal hypothesis, with systematics. Only entries with values greater than 0.5\% are plotted. \textsc{Pythia}8 $a$ BRs are assumed, using the default \textsc{BSMHiggs} $\tan\beta$ value of 1, as given in Table~\ref{tab:a0brs}.}
  \label{fig:observedcorr}
\end{figure}


%% \begin{figure}[!htbp]
%%   \centering
%%   \subfigure[]{\includegraphics[width=0.475\textwidth]{figures/EverythingPlotsUnblinded_02Oct19/All/HM_SR.png}}\\
%%   \subfigure[]{\includegraphics[width=0.475\textwidth]{figures/EverythingPlotsUnblinded_02Oct19/All/MLP_regInID.png}}
%%   \subfigure[]{\includegraphics[width=0.475\textwidth]{figures/EverythingPlotsUnblinded_02Oct19/All/MLP_massRegression.png}}
%%   \caption{Distributions of the (a) three body mass, (b) regression MLP output and (c) classification MLP outputs for the unblinded data, reweighted background and various signal hypotheses, before the requirements on the three body mass and classification MLP outputs are applied.}
%%   \label{fig:observedeverythingplots}
%% \end{figure}


\begin{figure}[!htbp]
  \centering
  \subfigure[]{\includegraphics[width=0.475\textwidth]{figures/Plots_09Feb20/SR1/MLP_massRegression_coarse.pdf}}
  \subfigure[]{\includegraphics[width=0.475\textwidth]{figures/Plots_09Feb20/SR1/MLP_regInID_coarse.pdf}}
  \caption{Distributions of the (a) regression MLP output and (b) classification MLP outputs for the unblinded data, reweighted background and various signal hypotheses, in the signal region.}
  \label{fig:observedsrplots}
\end{figure}


\begin{figure}[!htbp]
  \centering
  \includegraphics[width=0.7\textwidth]{figures/PaperPlots_09Feb20/HM_SR_coarse.pdf}
  \caption{Invariant mass of the di-lepton plus jet system, for data, background and three signal hypotheses. Events are required to pass the the complete event selection, including the MLP output variable requirement, but not the $120~\text{GeV}<m_{\ell^+\ell^-\text{j}}<135~\text{GeV}$ requirement. The background normalisation is set equal to that of the data, and the signal normalisations assume the SM Higgs production cross section and $\mathcal{B}(H\to Za)=100\%$. The background in these distributions has been reweighted as per Section~\ref{sec:bkgdrw}. The error bars are the data statistical uncertainty, and the dashed bars are the MC statistical uncertainty. The region between the dashed lines is the signal region.}
  \label{fig:observedmh}
\end{figure}



%-------------------------------------------------------------------------------
\clearpage
\section{Conclusion}
\label{sec:conclusion}
%-------------------------------------------------------------------------------

A search has been performed, which probes Higgs boson decays to a $Z$~boson, and a $J/\psi$ or $\eta_c$ charmonium state, or an $a$. It is well-motivated both in the SM, and from its sensitivity to non-SM final states. No excess is observed. 95\% $\text{CL}_\text{s}$ upper limits are set on $\sigma(pp\to H)\mathcal{B}(H\to Z(a/\mathcal{Q}))$, with values starting from $16.8~\text{pb}$ for the signal hypothesis of a 0.5~GeV $a$ decaying to gluons, and $105~\text{pb}$ and $107~\text{pb}$ for the $J/\psi$ and $\eta_c$, respectively. This is the first direct limit on hadronic decays of a non-SM Higgs boson over this mass range. Due to the large value of $\mathcal{B}(a \to \text{hadrons})$ over the entire 2HDM($+$S) phase space, these limits represent the tightest direct constraints for low (high) $\tan\beta$ in the Type-II and type-II (Type-VI) 2HDM$+$S~\cite{ATL-PHYS-PUB-2018-045}.

%% A funcdamental limitation of this analysis strategy is that is is not sensitive to the mass of the light resonance. Initial studies have suggested that a regression MLP could provide sensitivity to the mass of the light resonance. The invariant mass of the three body system could then be given to the classification MLP as an input variable, along with various other event level kinematic variables, and the regression MLP output variable could be used in the fit. %% Give plots of regression mlp outputs


%-------------------------------------------------------------------------------
% If you use biblatex and either biber or bibtex to process the bibliography
% just say \printbibliography here
\printbibliography
% If you want to use the traditional BibTeX you need to use the syntax below.
% \bibliographystyle{obsolete/bst/atlasBibStyleWithTitle}
% \bibliography{HZXAnalysisSupportNote,bib/ATLAS,bib/CMS,bib/ConfNotes,bib/PubNotes}
%-------------------------------------------------------------------------------

%-------------------------------------------------------------------------------
% Print the list of contributors to the analysis
% The argument gives the fraction of the text width used for the names
%-------------------------------------------------------------------------------
%% \clearpage
%% The supporting notes for the analysis should also contain a list of contributors.
%% This information should usually be included in \texttt{mydocument-metadata.tex}.
%% The list should be printed either here or before the Table of Contents.
%% \PrintAtlasContribute{0.30}


%-------------------------------------------------------------------------------
\clearpage
\appendix
\part*{Appendices}
\addcontentsline{toc}{part}{Appendices}
%-------------------------------------------------------------------------------

%% In an ATLAS note, use the appendices to include all the technical details of your work
%% that are relevant for the ATLAS Collaboration only (e.g.\ dataset details, software release used).
%% This information should be printed after the Bibliography.

\clearpage
\section{Data Samples}
\label{app:datasamples}

Tables~\ref{tab:datasamples} shows the datasets used in this search.

\begin{table}[!htbp]{\tiny\renewcommand{\arraystretch}{1.2}
    \begin{center}
      \begin{tabular}{|c|}
        \hline
        2015 Samples\\
        \hline
        data15\_13TeV.periodD.physics\_Main.PhysCont.DAOD\_FTAG2.grp15\_v02\_p3704\\
        data15\_13TeV.periodE.physics\_Main.PhysCont.DAOD\_FTAG2.grp15\_v02\_p3704\\
        data15\_13TeV.periodF.physics\_Main.PhysCont.DAOD\_FTAG2.grp15\_v02\_p3704\\
        data15\_13TeV.periodG.physics\_Main.PhysCont.DAOD\_FTAG2.grp15\_v02\_p3704\\
        data15\_13TeV.periodH.physics\_Main.PhysCont.DAOD\_FTAG2.grp15\_v02\_p3704\\
        data15\_13TeV.periodJ.physics\_Main.PhysCont.DAOD\_FTAG2.grp15\_v02\_p3704\\
        \hline
        2016 Samples\\
        \hline
        data16\_13TeV.periodA.physics\_Main.PhysCont.DAOD\_FTAG2.grp16\_v02\_p3704\\
        data16\_13TeV.periodB.physics\_Main.PhysCont.DAOD\_FTAG2.grp16\_v02\_p3704\\
        data16\_13TeV.periodC.physics\_Main.PhysCont.DAOD\_FTAG2.grp16\_v02\_p3704\\
        data16\_13TeV.periodD.physics\_Main.PhysCont.DAOD\_FTAG2.grp16\_v02\_p3704\\
        data16\_13TeV.periodE.physics\_Main.PhysCont.DAOD\_FTAG2.grp16\_v02\_p3704\\
        data16\_13TeV.periodF.physics\_Main.PhysCont.DAOD\_FTAG2.grp16\_v02\_p3704\\
        data16\_13TeV.periodG.physics\_Main.PhysCont.DAOD\_FTAG2.grp16\_v02\_p3704\\
        data16\_13TeV.periodI.physics\_Main.PhysCont.DAOD\_FTAG2.grp16\_v02\_p3704\\
        data16\_13TeV.periodK.physics\_Main.PhysCont.DAOD\_FTAG2.grp16\_v02\_p3704\\
        data16\_13TeV.periodL.physics\_Main.PhysCont.DAOD\_FTAG2.grp16\_v02\_p3704\\
        \hline
        2017 Samples\\
        \hline
        data17\_13TeV.periodB.physics\_Main.PhysCont.DAOD\_FTAG2.grp17\_v02\_p3704\\
        data17\_13TeV.periodC.physics\_Main.PhysCont.DAOD\_FTAG2.grp17\_v02\_p3704\\
        data17\_13TeV.periodD.physics\_Main.PhysCont.DAOD\_FTAG2.grp17\_v02\_p3704\\
        data17\_13TeV.periodE.physics\_Main.PhysCont.DAOD\_FTAG2.grp17\_v02\_p3704\\
        data17\_13TeV.periodF.physics\_Main.PhysCont.DAOD\_FTAG2.grp17\_v02\_p3704\\
        data17\_13TeV.periodH.physics\_Main.PhysCont.DAOD\_FTAG2.grp17\_v02\_p3704\\
        data17\_13TeV.periodI.physics\_Main.PhysCont.DAOD\_FTAG2.grp17\_v02\_p3704\\
        data17\_13TeV.periodK.physics\_Main.PhysCont.DAOD\_FTAG2.grp17\_v02\_p3704\\
        \hline
        2018 Samples\\
        \hline
        data18\_13TeV.periodB.physics\_Main.PhysCont.DAOD\_FTAG2.grp18\_v04\_p3704\\
        data18\_13TeV.periodC.physics\_Main.PhysCont.DAOD\_FTAG2.grp18\_v04\_p3704\\
        data18\_13TeV.periodD.physics\_Main.PhysCont.DAOD\_FTAG2.grp18\_v04\_p3704\\
        data18\_13TeV.periodF.physics\_Main.PhysCont.DAOD\_FTAG2.grp18\_v04\_p3704\\
        data18\_13TeV.periodI.physics\_Main.PhysCont.DAOD\_FTAG2.grp18\_v04\_p3704\\
        data18\_13TeV.periodK.physics\_Main.PhysCont.DAOD\_FTAG2.grp18\_v04\_p3704\\
        data18\_13TeV.periodL.physics\_Main.PhysCont.DAOD\_FTAG2.grp18\_v04\_p3704\\
        data18\_13TeV.periodM.physics\_Main.PhysCont.DAOD\_FTAG2.grp18\_v04\_p3704\\
        data18\_13TeV.periodO.physics\_Main.PhysCont.DAOD\_FTAG2.grp18\_v04\_p3704\\
        data18\_13TeV.periodQ.physics\_Main.PhysCont.DAOD\_FTAG2.grp18\_v04\_p3704\\
        \hline
      \end{tabular}
      \caption{Full list of Run 2 data samples.}
      \label{tab:datasamples}
  \end{center}}
\end{table}


\clearpage
\section{Monte-Carlo Simulation Signal Samples}
\label{app:sigmc}

Tables~\ref{tab:mcsignalsamplesmc16a}, \ref{tab:mcsignalsamplesmc16d} and \ref{tab:mcsignalsamplesmc16e} list the MC signal samples used in this analysis, for MC16a, MC16d and MC16e conditions, respectively.

\begin{table}[!htbp]{\tiny\renewcommand{\arraystretch}{1.2}
    \begin{center}
      \begin{tabular}{|c|}
        \hline
        Sample\\
        \hline
        mc16\_13TeV.345906.PowhegPythia8EvtGen\_CT10\_AZNLOCTEQ6L1\_ggH125\_EtacZll.merge.AOD.e6591\_e5984\_s3126\_r9364\_r9315\\
        mc16\_13TeV.450549.PowhegPythia8EvtGen\_CT10\_AZNLOCTEQ6L1\_ggH125\_JpsiZll.merge.AOD.e7242\_e5984\_s3126\_r9364\_r9315\\
        mc16\_13TeV.345907.PowhegPythia8EvtGen\_CT10\_AZNLOCTEQ6L1\_ggH125\_a0Zll\_0p5GeVa0.merge.AOD.e6591\_e5984\_s3126\_r9364\_r9315\\
        mc16\_13TeV.450550.PowhegPythia8EvtGen\_CT10\_AZNLOCTEQ6L1\_ggH125\_a0Zll\_0p75GeVa0.merge.AOD.e7242\_e5984\_s3126\_r9364\_r9315\\
        mc16\_13TeV.450551.PowhegPythia8EvtGen\_CT10\_AZNLOCTEQ6L1\_ggH125\_a0Zll\_1p0GeVa0.merge.AOD.e7242\_e5984\_s3126\_r9364\_r9315\\
        mc16\_13TeV.450552.PowhegPythia8EvtGen\_CT10\_AZNLOCTEQ6L1\_ggH125\_a0Zll\_1p5GeVa0.merge.AOD.e7242\_e5984\_s3126\_r9364\_r9315\\
        mc16\_13TeV.450553.PowhegPythia8EvtGen\_CT10\_AZNLOCTEQ6L1\_ggH125\_a0Zll\_2p0GeVa0.merge.AOD.e7242\_e5984\_s3126\_r9364\_r9315\\
        mc16\_13TeV.345908.PowhegPythia8EvtGen\_CT10\_AZNLOCTEQ6L1\_ggH125\_a0Zll\_2p5GeVa0.merge.AOD.e6591\_e5984\_s3126\_r9364\_r9315\\
        mc16\_13TeV.450554.PowhegPythia8EvtGen\_CT10\_AZNLOCTEQ6L1\_ggH125\_a0Zll\_3p0GeVa0.merge.AOD.e7242\_e5984\_s3126\_r9364\_r9315\\
        mc16\_13TeV.450555.PowhegPythia8EvtGen\_CT10\_AZNLOCTEQ6L1\_ggH125\_a0Zll\_3p5GeVa0.merge.AOD.e7242\_e5984\_s3126\_r9364\_r9315\\
        mc16\_13TeV.450556.PowhegPythia8EvtGen\_CT10\_AZNLOCTEQ6L1\_ggH125\_a0Zll\_4p0GeVa0.merge.AOD.e7242\_e5984\_s3126\_r9364\_r9315\\
        mc16\_13TeV.345909.PowhegPythia8EvtGen\_CT10\_AZNLOCTEQ6L1\_ggH125\_a0Zll\_8GeVa0.merge.AOD.e6591\_e5984\_s3126\_r9364\_r9315\\
        \hline
      \end{tabular}
      \caption{MC16a MC background samples used in this analysis. MC16a files correspond to 2015 and 2016 data conditions.}
      \label{tab:mcsignalsamplesmc16a}
  \end{center}}
\end{table}

\begin{table}[!htbp]{\tiny\renewcommand{\arraystretch}{1.2}
    \begin{center}
      \begin{tabular}{|c|}
        \hline
        Sample\\
        \hline
        mc16\_13TeV.345906.PowhegPythia8EvtGen\_CT10\_AZNLOCTEQ6L1\_ggH125\_EtacZll.merge.AOD.e6591\_e5984\_s3126\_r10201\_r10210\\
        mc16\_13TeV.450549.PowhegPythia8EvtGen\_CT10\_AZNLOCTEQ6L1\_ggH125\_JpsiZll.merge.AOD.e7242\_e5984\_s3126\_r10201\_r10210\\
        mc16\_13TeV.345907.PowhegPythia8EvtGen\_CT10\_AZNLOCTEQ6L1\_ggH125\_a0Zll\_0p5GeVa0.merge.AOD.e6591\_e5984\_s3126\_r10201\_r10210\\
        mc16\_13TeV.450550.PowhegPythia8EvtGen\_CT10\_AZNLOCTEQ6L1\_ggH125\_a0Zll\_0p75GeVa0.merge.AOD.e7242\_e5984\_s3126\_r10201\_r10210\\
        mc16\_13TeV.450551.PowhegPythia8EvtGen\_CT10\_AZNLOCTEQ6L1\_ggH125\_a0Zll\_1p0GeVa0.merge.AOD.e7242\_e5984\_s3126\_r10201\_r10210\\
        mc16\_13TeV.450552.PowhegPythia8EvtGen\_CT10\_AZNLOCTEQ6L1\_ggH125\_a0Zll\_1p5GeVa0.merge.AOD.e7242\_e5984\_s3126\_r10201\_r10210\\
        mc16\_13TeV.450553.PowhegPythia8EvtGen\_CT10\_AZNLOCTEQ6L1\_ggH125\_a0Zll\_2p0GeVa0.merge.AOD.e7242\_e5984\_s3126\_r10201\_r10210\\
        mc16\_13TeV.345908.PowhegPythia8EvtGen\_CT10\_AZNLOCTEQ6L1\_ggH125\_a0Zll\_2p5GeVa0.merge.AOD.e6591\_e5984\_s3126\_r10201\_r10210\\
        mc16\_13TeV.450554.PowhegPythia8EvtGen\_CT10\_AZNLOCTEQ6L1\_ggH125\_a0Zll\_3p0GeVa0.merge.AOD.e7242\_e5984\_s3126\_r10201\_r10210\\
        mc16\_13TeV.450555.PowhegPythia8EvtGen\_CT10\_AZNLOCTEQ6L1\_ggH125\_a0Zll\_3p5GeVa0.merge.AOD.e7242\_e5984\_s3126\_r10201\_r10210\\
        mc16\_13TeV.450556.PowhegPythia8EvtGen\_CT10\_AZNLOCTEQ6L1\_ggH125\_a0Zll\_4p0GeVa0.merge.AOD.e7242\_e5984\_s3126\_r10201\_r10210\\
        mc16\_13TeV.345909.PowhegPythia8EvtGen\_CT10\_AZNLOCTEQ6L1\_ggH125\_a0Zll\_8GeVa0.merge.AOD.e6591\_e5984\_s3126\_r10201\_r10210\\
        \hline
      \end{tabular}
      \caption{MC16d MC background samples used in this analysis. MC16d files correspond to 2017 data conditions.}
      \label{tab:mcsignalsamplesmc16d}
  \end{center}}
\end{table}

\begin{table}[!htbp]{\tiny\renewcommand{\arraystretch}{1.2}
    \begin{center}
      \begin{tabular}{|c|}
        \hline
        Sample\\
        \hline
        mc16\_13TeV.345906.PowhegPythia8EvtGen\_CT10\_AZNLOCTEQ6L1\_ggH125\_EtacZll.merge.AOD.e6591\_e5984\_s3126\_r10724\_r10726\\
        mc16\_13TeV.450549.PowhegPythia8EvtGen\_CT10\_AZNLOCTEQ6L1\_ggH125\_JpsiZll.merge.AOD.e7242\_e5984\_s3126\_r10724\_r10726\\
        mc16\_13TeV.345907.PowhegPythia8EvtGen\_CT10\_AZNLOCTEQ6L1\_ggH125\_a0Zll\_0p5GeVa0.merge.AOD.e6591\_e5984\_s3126\_r10724\_r10726\\
        mc16\_13TeV.450550.PowhegPythia8EvtGen\_CT10\_AZNLOCTEQ6L1\_ggH125\_a0Zll\_0p75GeVa0.merge.AOD.e7242\_e5984\_s3126\_r10724\_r10726\\
        mc16\_13TeV.450551.PowhegPythia8EvtGen\_CT10\_AZNLOCTEQ6L1\_ggH125\_a0Zll\_1p0GeVa0.merge.AOD.e7242\_e5984\_s3126\_r10724\_r10726\\
        mc16\_13TeV.450552.PowhegPythia8EvtGen\_CT10\_AZNLOCTEQ6L1\_ggH125\_a0Zll\_1p5GeVa0.merge.AOD.e7242\_e5984\_s3126\_r10724\_r10726\\
        mc16\_13TeV.450553.PowhegPythia8EvtGen\_CT10\_AZNLOCTEQ6L1\_ggH125\_a0Zll\_2p0GeVa0.merge.AOD.e7242\_e5984\_s3126\_r10724\_r10726\\
        mc16\_13TeV.345908.PowhegPythia8EvtGen\_CT10\_AZNLOCTEQ6L1\_ggH125\_a0Zll\_2p5GeVa0.merge.AOD.e6591\_e5984\_s3126\_r10724\_r10726\\
        mc16\_13TeV.450554.PowhegPythia8EvtGen\_CT10\_AZNLOCTEQ6L1\_ggH125\_a0Zll\_3p0GeVa0.merge.AOD.e7242\_e5984\_s3126\_r10724\_r10726\\
        mc16\_13TeV.450555.PowhegPythia8EvtGen\_CT10\_AZNLOCTEQ6L1\_ggH125\_a0Zll\_3p5GeVa0.merge.AOD.e7242\_e5984\_s3126\_r10724\_r10726\\
        mc16\_13TeV.450556.PowhegPythia8EvtGen\_CT10\_AZNLOCTEQ6L1\_ggH125\_a0Zll\_4p0GeVa0.merge.AOD.e7242\_e5984\_s3126\_r10724\_r10726\\
        mc16\_13TeV.345909.PowhegPythia8EvtGen\_CT10\_AZNLOCTEQ6L1\_ggH125\_a0Zll\_8GeVa0.merge.AOD.e6591\_e5984\_s3126\_r10724\_r10726\\
        \hline
      \end{tabular}
      \caption{MC16e MC background samples used in this analysis. MC16e files correspond to 2018 data conditions.}
      \label{tab:mcsignalsamplesmc16e}
  \end{center}}
\end{table}


\clearpage
\section{Monte-Carlo Simulation Background Samples}
\label{app:bkgdmc}

Tables~\ref{tab:mcbkgdsamplesmc16a}, \ref{tab:mcbkgdsamplesmc16d} and \ref{tab:mcbkgdsamplesmc16e} list the nominal background MC samples used in this analysis, for MC16a, MC16d and MC16e conditions, respectively. Tables~\ref{tab:mcaltbkgdsamplesmc16a}, \ref{tab:mcaltbkgdsamplesmc16d} and \ref{tab:mcaltbkgdsamplesmc16e} list the alternative $Z+\text{jets}$ background MC samples used, for MC16a, MC16d and MC16e conditions, respectively.

\begin{table}[!htbp]{\tiny\renewcommand{\arraystretch}{1.2}
    \begin{center}
      \begin{tabular}{|c|}
        \hline
        Sample\\
        \hline
        $t\bar t$\\
        \hline
        mc16\_13TeV.410472.PhPy8EG\_A14\_ttbar\_hdamp258p75\_dil.deriv.DAOD\_FTAG2.e6348\_s3126\_r9364\_p3703\\
        \hline
        Diboson\\
        \hline
        mc16\_13TeV.363356.Sherpa\_221\_NNPDF30NNLO\_ZqqZll.deriv.DAOD\_FTAG2.e5525\_s3126\_r9364\_p3703\\
        mc16\_13TeV.363358.Sherpa\_221\_NNPDF30NNLO\_WqqZll.deriv.DAOD\_FTAG2.e5525\_s3126\_r9364\_p3703\\
        mc16\_13TeV.364302.Sherpa\_222\_NNPDF30NNLO\_ggZllZqq.deriv.DAOD\_FTAG2.e6273\_s3126\_r9364\_p3703\\
        \hline
        $Z+\text{jets}$\\
        \hline
        mc16\_13TeV.364100.Sherpa\_221\_NNPDF30NNLO\_Zmumu\_MAXHTPTV0\_70\_CVetoBVeto.deriv.DAOD\_FTAG2.e5271\_s3126\_r9364\_p3703\\
        mc16\_13TeV.364101.Sherpa\_221\_NNPDF30NNLO\_Zmumu\_MAXHTPTV0\_70\_CFilterBVeto.deriv.DAOD\_FTAG2.e5271\_s3126\_r9364\_p3703\\
        mc16\_13TeV.364102.Sherpa\_221\_NNPDF30NNLO\_Zmumu\_MAXHTPTV0\_70\_BFilter.deriv.DAOD\_FTAG2.e5271\_s3126\_r9364\_p3703\\
        mc16\_13TeV.364103.Sherpa\_221\_NNPDF30NNLO\_Zmumu\_MAXHTPTV70\_140\_CVetoBVeto.deriv.DAOD\_FTAG2.e5271\_s3126\_r9364\_p3703\\
        mc16\_13TeV.364104.Sherpa\_221\_NNPDF30NNLO\_Zmumu\_MAXHTPTV70\_140\_CFilterBVeto.deriv.DAOD\_FTAG2.e5271\_s3126\_r9364\_p3703\\
        mc16\_13TeV.364105.Sherpa\_221\_NNPDF30NNLO\_Zmumu\_MAXHTPTV70\_140\_BFilter.deriv.DAOD\_FTAG2.e5271\_s3126\_r9364\_p3703\\
        mc16\_13TeV.364106.Sherpa\_221\_NNPDF30NNLO\_Zmumu\_MAXHTPTV140\_280\_CVetoBVeto.deriv.DAOD\_FTAG2.e5271\_s3126\_r9364\_p3703\\
        mc16\_13TeV.364107.Sherpa\_221\_NNPDF30NNLO\_Zmumu\_MAXHTPTV140\_280\_CFilterBVeto.deriv.DAOD\_FTAG2.e5271\_s3126\_r9364\_p3703\\
        mc16\_13TeV.364108.Sherpa\_221\_NNPDF30NNLO\_Zmumu\_MAXHTPTV140\_280\_BFilter.deriv.DAOD\_FTAG2.e5271\_s3126\_r9364\_p3703\\
        mc16\_13TeV.364109.Sherpa\_221\_NNPDF30NNLO\_Zmumu\_MAXHTPTV280\_500\_CVetoBVeto.deriv.DAOD\_FTAG2.e5271\_s3126\_r9364\_p3703\\
        mc16\_13TeV.364110.Sherpa\_221\_NNPDF30NNLO\_Zmumu\_MAXHTPTV280\_500\_CFilterBVeto.deriv.DAOD\_FTAG2.e5271\_s3126\_r9364\_p3703\\
        mc16\_13TeV.364111.Sherpa\_221\_NNPDF30NNLO\_Zmumu\_MAXHTPTV280\_500\_BFilter.deriv.DAOD\_FTAG2.e5271\_s3126\_r9364\_p3703\\
        mc16\_13TeV.364112.Sherpa\_221\_NNPDF30NNLO\_Zmumu\_MAXHTPTV500\_1000.deriv.DAOD\_FTAG2.e5271\_s3126\_r9364\_p3703\\
        mc16\_13TeV.364113.Sherpa\_221\_NNPDF30NNLO\_Zmumu\_MAXHTPTV1000\_E\_CMS.deriv.DAOD\_FTAG2.e5271\_s3126\_r9364\_p3703\\
        mc16\_13TeV.364114.Sherpa\_221\_NNPDF30NNLO\_Zee\_MAXHTPTV0\_70\_CVetoBVeto.deriv.DAOD\_FTAG2.e5299\_s3126\_r9364\_p3703\\
        mc16\_13TeV.364115.Sherpa\_221\_NNPDF30NNLO\_Zee\_MAXHTPTV0\_70\_CFilterBVeto.deriv.DAOD\_FTAG2.e5299\_s3126\_r9364\_p3703\\
        mc16\_13TeV.364116.Sherpa\_221\_NNPDF30NNLO\_Zee\_MAXHTPTV0\_70\_BFilter.deriv.DAOD\_FTAG2.e5299\_s3126\_r9364\_p3703\\
        mc16\_13TeV.364117.Sherpa\_221\_NNPDF30NNLO\_Zee\_MAXHTPTV70\_140\_CVetoBVeto.deriv.DAOD\_FTAG2.e5299\_s3126\_r9364\_p3703\\
        mc16\_13TeV.364118.Sherpa\_221\_NNPDF30NNLO\_Zee\_MAXHTPTV70\_140\_CFilterBVeto.deriv.DAOD\_FTAG2.e5299\_s3126\_r9364\_p3703\\
        mc16\_13TeV.364119.Sherpa\_221\_NNPDF30NNLO\_Zee\_MAXHTPTV70\_140\_BFilter.deriv.DAOD\_FTAG2.e5299\_s3126\_r9364\_p3703\\
        mc16\_13TeV.364120.Sherpa\_221\_NNPDF30NNLO\_Zee\_MAXHTPTV140\_280\_CVetoBVeto.deriv.DAOD\_FTAG2.e5299\_s3126\_r9364\_p3703\\
        mc16\_13TeV.364121.Sherpa\_221\_NNPDF30NNLO\_Zee\_MAXHTPTV140\_280\_CFilterBVeto.deriv.DAOD\_FTAG2.e5299\_s3126\_r9364\_p3703\\
        mc16\_13TeV.364122.Sherpa\_221\_NNPDF30NNLO\_Zee\_MAXHTPTV140\_280\_BFilter.deriv.DAOD\_FTAG2.e5299\_s3126\_r9364\_p3703\\
        mc16\_13TeV.364123.Sherpa\_221\_NNPDF30NNLO\_Zee\_MAXHTPTV280\_500\_CVetoBVeto.deriv.DAOD\_FTAG2.e5299\_s3126\_r9364\_p3703\\
        mc16\_13TeV.364124.Sherpa\_221\_NNPDF30NNLO\_Zee\_MAXHTPTV280\_500\_CFilterBVeto.deriv.DAOD\_FTAG2.e5299\_s3126\_r9364\_p3703\\
        mc16\_13TeV.364125.Sherpa\_221\_NNPDF30NNLO\_Zee\_MAXHTPTV280\_500\_BFilter.deriv.DAOD\_FTAG2.e5299\_s3126\_r9364\_p3703\\
        mc16\_13TeV.364126.Sherpa\_221\_NNPDF30NNLO\_Zee\_MAXHTPTV500\_1000.deriv.DAOD\_FTAG2.e5299\_s3126\_r9364\_p3703\\
        mc16\_13TeV.364127.Sherpa\_221\_NNPDF30NNLO\_Zee\_MAXHTPTV1000\_E\_CMS.deriv.DAOD\_FTAG2.e5299\_s3126\_r9364\_p3703\\
        mc16\_13TeV.364128.Sherpa\_221\_NNPDF30NNLO\_Ztautau\_MAXHTPTV0\_70\_CVetoBVeto.deriv.DAOD\_FTAG2.e5307\_s3126\_r9364\_p3703\\
        mc16\_13TeV.364129.Sherpa\_221\_NNPDF30NNLO\_Ztautau\_MAXHTPTV0\_70\_CFilterBVeto.deriv.DAOD\_FTAG2.e5307\_s3126\_r9364\_p3703\\
        mc16\_13TeV.364130.Sherpa\_221\_NNPDF30NNLO\_Ztautau\_MAXHTPTV0\_70\_BFilter.deriv.DAOD\_FTAG2.e5307\_s3126\_r9364\_p3703\\
        mc16\_13TeV.364131.Sherpa\_221\_NNPDF30NNLO\_Ztautau\_MAXHTPTV70\_140\_CVetoBVeto.deriv.DAOD\_FTAG2.e5307\_s3126\_r9364\_p3703\\
        mc16\_13TeV.364132.Sherpa\_221\_NNPDF30NNLO\_Ztautau\_MAXHTPTV70\_140\_CFilterBVeto.deriv.DAOD\_FTAG2.e5307\_s3126\_r9364\_p3703\\
        mc16\_13TeV.364133.Sherpa\_221\_NNPDF30NNLO\_Ztautau\_MAXHTPTV70\_140\_BFilter.deriv.DAOD\_FTAG2.e5307\_s3126\_r9364\_p3703\\
        mc16\_13TeV.364134.Sherpa\_221\_NNPDF30NNLO\_Ztautau\_MAXHTPTV140\_280\_CVetoBVeto.deriv.DAOD\_FTAG2.e5307\_s3126\_r9364\_p3703\\
        mc16\_13TeV.364135.Sherpa\_221\_NNPDF30NNLO\_Ztautau\_MAXHTPTV140\_280\_CFilterBVeto.deriv.DAOD\_FTAG2.e5307\_s3126\_r9364\_p3703\\
        mc16\_13TeV.364136.Sherpa\_221\_NNPDF30NNLO\_Ztautau\_MAXHTPTV140\_280\_BFilter.deriv.DAOD\_FTAG2.e5307\_s3126\_r9364\_p3703\\
        mc16\_13TeV.364137.Sherpa\_221\_NNPDF30NNLO\_Ztautau\_MAXHTPTV280\_500\_CVetoBVeto.deriv.DAOD\_FTAG2.e5307\_s3126\_r9364\_p3703\\
        mc16\_13TeV.364138.Sherpa\_221\_NNPDF30NNLO\_Ztautau\_MAXHTPTV280\_500\_CFilterBVeto.deriv.DAOD\_FTAG2.e5313\_s3126\_r9364\_p3703\\
        mc16\_13TeV.364139.Sherpa\_221\_NNPDF30NNLO\_Ztautau\_MAXHTPTV280\_500\_BFilter.deriv.DAOD\_FTAG2.e5313\_s3126\_r9364\_p3703\\
        mc16\_13TeV.364140.Sherpa\_221\_NNPDF30NNLO\_Ztautau\_MAXHTPTV500\_1000.deriv.DAOD\_FTAG2.e5307\_s3126\_r9364\_p3703\\
        mc16\_13TeV.364141.Sherpa\_221\_NNPDF30NNLO\_Ztautau\_MAXHTPTV1000\_E\_CMS.deriv.DAOD\_FTAG2.e5307\_s3126\_r9364\_p3703\\
        \hline
      \end{tabular}
      \caption{MC16a MC background samples used in this analysis. MC16a files correspond to 2015 and 2016 data conditions.}
      \label{tab:mcbkgdsamplesmc16a}
  \end{center}}
\end{table}

\begin{table}[!htbp]{\tiny\renewcommand{\arraystretch}{1.2}
    \begin{center}
      \begin{tabular}{|c|}
        \hline
        Sample\\
        \hline
        $t\bar t$\\
        \hline
        mc16\_13TeV.410472.PhPy8EG\_A14\_ttbar\_hdamp258p75\_dil.deriv.DAOD\_FTAG2.e6348\_s3126\_r10201\_p3703\\
        \hline
        Diboson\\
        \hline
        mc16\_13TeV.363356.Sherpa\_221\_NNPDF30NNLO\_ZqqZll.deriv.DAOD\_FTAG2.e5525\_s3126\_r10201\_p3703\\
        mc16\_13TeV.363358.Sherpa\_221\_NNPDF30NNLO\_WqqZll.deriv.DAOD\_FTAG2.e5525\_s3126\_r10201\_p3703\\
        mc16\_13TeV.364302.Sherpa\_222\_NNPDF30NNLO\_ggZllZqq.deriv.DAOD\_FTAG2.e6273\_s3126\_r10201\_p3703\\
        \hline
        $Z+\text{jets}$\\
        \hline
        mc16\_13TeV.364100.Sherpa\_221\_NNPDF30NNLO\_Zmumu\_MAXHTPTV0\_70\_CVetoBVeto.deriv.DAOD\_FTAG2.e5271\_s3126\_r10201\_p3703\\
        mc16\_13TeV.364101.Sherpa\_221\_NNPDF30NNLO\_Zmumu\_MAXHTPTV0\_70\_CFilterBVeto.deriv.DAOD\_FTAG2.e5271\_s3126\_r10201\_p3703\\
        mc16\_13TeV.364102.Sherpa\_221\_NNPDF30NNLO\_Zmumu\_MAXHTPTV0\_70\_BFilter.deriv.DAOD\_FTAG2.e5271\_s3126\_r10201\_p3703\\
        mc16\_13TeV.364103.Sherpa\_221\_NNPDF30NNLO\_Zmumu\_MAXHTPTV70\_140\_CVetoBVeto.deriv.DAOD\_FTAG2.e5271\_s3126\_r10201\_p3703\\
        mc16\_13TeV.364104.Sherpa\_221\_NNPDF30NNLO\_Zmumu\_MAXHTPTV70\_140\_CFilterBVeto.deriv.DAOD\_FTAG2.e5271\_s3126\_r10201\_p3703\\
        mc16\_13TeV.364105.Sherpa\_221\_NNPDF30NNLO\_Zmumu\_MAXHTPTV70\_140\_BFilter.deriv.DAOD\_FTAG2.e5271\_s3126\_r10201\_p3703\\
        mc16\_13TeV.364106.Sherpa\_221\_NNPDF30NNLO\_Zmumu\_MAXHTPTV140\_280\_CVetoBVeto.deriv.DAOD\_FTAG2.e5271\_s3126\_r10201\_p3703\\
        mc16\_13TeV.364107.Sherpa\_221\_NNPDF30NNLO\_Zmumu\_MAXHTPTV140\_280\_CFilterBVeto.deriv.DAOD\_FTAG2.e5271\_s3126\_r10201\_p3703\\
        mc16\_13TeV.364108.Sherpa\_221\_NNPDF30NNLO\_Zmumu\_MAXHTPTV140\_280\_BFilter.deriv.DAOD\_FTAG2.e5271\_s3126\_r10201\_p3703\\
        mc16\_13TeV.364109.Sherpa\_221\_NNPDF30NNLO\_Zmumu\_MAXHTPTV280\_500\_CVetoBVeto.deriv.DAOD\_FTAG2.e5271\_s3126\_r10201\_p3703\\
        mc16\_13TeV.364110.Sherpa\_221\_NNPDF30NNLO\_Zmumu\_MAXHTPTV280\_500\_CFilterBVeto.deriv.DAOD\_FTAG2.e5271\_s3126\_r10201\_p3703\\
        mc16\_13TeV.364111.Sherpa\_221\_NNPDF30NNLO\_Zmumu\_MAXHTPTV280\_500\_BFilter.deriv.DAOD\_FTAG2.e5271\_s3126\_r10201\_p3703\\
        mc16\_13TeV.364112.Sherpa\_221\_NNPDF30NNLO\_Zmumu\_MAXHTPTV500\_1000.deriv.DAOD\_FTAG2.e5271\_s3126\_r10201\_p3703\\
        mc16\_13TeV.364113.Sherpa\_221\_NNPDF30NNLO\_Zmumu\_MAXHTPTV1000\_E\_CMS.deriv.DAOD\_FTAG2.e5271\_s3126\_r10201\_p3703\\
        mc16\_13TeV.364114.Sherpa\_221\_NNPDF30NNLO\_Zee\_MAXHTPTV0\_70\_CVetoBVeto.deriv.DAOD\_FTAG2.e5299\_s3126\_r10201\_p3703\\
        mc16\_13TeV.364115.Sherpa\_221\_NNPDF30NNLO\_Zee\_MAXHTPTV0\_70\_CFilterBVeto.deriv.DAOD\_FTAG2.e5299\_s3126\_r10201\_p3703\\
        mc16\_13TeV.364116.Sherpa\_221\_NNPDF30NNLO\_Zee\_MAXHTPTV0\_70\_BFilter.deriv.DAOD\_FTAG2.e5299\_s3126\_r10201\_p3703\\
        mc16\_13TeV.364117.Sherpa\_221\_NNPDF30NNLO\_Zee\_MAXHTPTV70\_140\_CVetoBVeto.deriv.DAOD\_FTAG2.e5299\_s3126\_r10201\_p3703\\
        mc16\_13TeV.364118.Sherpa\_221\_NNPDF30NNLO\_Zee\_MAXHTPTV70\_140\_CFilterBVeto.deriv.DAOD\_FTAG2.e5299\_s3126\_r10201\_p3703\\
        mc16\_13TeV.364119.Sherpa\_221\_NNPDF30NNLO\_Zee\_MAXHTPTV70\_140\_BFilter.deriv.DAOD\_FTAG2.e5299\_s3126\_r10201\_p3703\\
        mc16\_13TeV.364120.Sherpa\_221\_NNPDF30NNLO\_Zee\_MAXHTPTV140\_280\_CVetoBVeto.deriv.DAOD\_FTAG2.e5299\_s3126\_r10201\_p3703\\
        mc16\_13TeV.364121.Sherpa\_221\_NNPDF30NNLO\_Zee\_MAXHTPTV140\_280\_CFilterBVeto.deriv.DAOD\_FTAG2.e5299\_s3126\_r10201\_p3703\\
        mc16\_13TeV.364122.Sherpa\_221\_NNPDF30NNLO\_Zee\_MAXHTPTV140\_280\_BFilter.deriv.DAOD\_FTAG2.e5299\_s3126\_r10201\_p3703\\
        mc16\_13TeV.364123.Sherpa\_221\_NNPDF30NNLO\_Zee\_MAXHTPTV280\_500\_CVetoBVeto.deriv.DAOD\_FTAG2.e5299\_s3126\_r10201\_p3703\\
        mc16\_13TeV.364124.Sherpa\_221\_NNPDF30NNLO\_Zee\_MAXHTPTV280\_500\_CFilterBVeto.deriv.DAOD\_FTAG2.e5299\_s3126\_r10201\_p3703\\
        mc16\_13TeV.364125.Sherpa\_221\_NNPDF30NNLO\_Zee\_MAXHTPTV280\_500\_BFilter.deriv.DAOD\_FTAG2.e5299\_s3126\_r10201\_p3703\\
        mc16\_13TeV.364126.Sherpa\_221\_NNPDF30NNLO\_Zee\_MAXHTPTV500\_1000.deriv.DAOD\_FTAG2.e5299\_s3126\_r10201\_p3703\\
        mc16\_13TeV.364127.Sherpa\_221\_NNPDF30NNLO\_Zee\_MAXHTPTV1000\_E\_CMS.deriv.DAOD\_FTAG2.e5299\_s3126\_r10201\_p3703\\
        mc16\_13TeV.364128.Sherpa\_221\_NNPDF30NNLO\_Ztautau\_MAXHTPTV0\_70\_CVetoBVeto.deriv.DAOD\_FTAG2.e5307\_s3126\_r10201\_p3703\\
        mc16\_13TeV.364129.Sherpa\_221\_NNPDF30NNLO\_Ztautau\_MAXHTPTV0\_70\_CFilterBVeto.deriv.DAOD\_FTAG2.e5307\_s3126\_r10201\_p3703\\
        mc16\_13TeV.364130.Sherpa\_221\_NNPDF30NNLO\_Ztautau\_MAXHTPTV0\_70\_BFilter.deriv.DAOD\_FTAG2.e5307\_s3126\_r10201\_p3703\\
        mc16\_13TeV.364131.Sherpa\_221\_NNPDF30NNLO\_Ztautau\_MAXHTPTV70\_140\_CVetoBVeto.deriv.DAOD\_FTAG2.e5307\_s3126\_r10201\_p3703\\
        mc16\_13TeV.364132.Sherpa\_221\_NNPDF30NNLO\_Ztautau\_MAXHTPTV70\_140\_CFilterBVeto.deriv.DAOD\_FTAG2.e5307\_s3126\_r10201\_p3703\\
        mc16\_13TeV.364133.Sherpa\_221\_NNPDF30NNLO\_Ztautau\_MAXHTPTV70\_140\_BFilter.deriv.DAOD\_FTAG2.e5307\_s3126\_r10201\_p3703\\
        mc16\_13TeV.364134.Sherpa\_221\_NNPDF30NNLO\_Ztautau\_MAXHTPTV140\_280\_CVetoBVeto.deriv.DAOD\_FTAG2.e5307\_s3126\_r10201\_p3703\\
        mc16\_13TeV.364135.Sherpa\_221\_NNPDF30NNLO\_Ztautau\_MAXHTPTV140\_280\_CFilterBVeto.deriv.DAOD\_FTAG2.e5307\_s3126\_r10201\_p3703\\
        mc16\_13TeV.364136.Sherpa\_221\_NNPDF30NNLO\_Ztautau\_MAXHTPTV140\_280\_BFilter.deriv.DAOD\_FTAG2.e5307\_s3126\_r10201\_p3703\\
        mc16\_13TeV.364137.Sherpa\_221\_NNPDF30NNLO\_Ztautau\_MAXHTPTV280\_500\_CVetoBVeto.deriv.DAOD\_FTAG2.e5307\_s3126\_r10201\_p3703\\
        mc16\_13TeV.364138.Sherpa\_221\_NNPDF30NNLO\_Ztautau\_MAXHTPTV280\_500\_CFilterBVeto.deriv.DAOD\_FTAG2.e5313\_s3126\_r10201\_p3703\\
        mc16\_13TeV.364139.Sherpa\_221\_NNPDF30NNLO\_Ztautau\_MAXHTPTV280\_500\_BFilter.deriv.DAOD\_FTAG2.e5313\_s3126\_r10201\_p3703\\
        mc16\_13TeV.364140.Sherpa\_221\_NNPDF30NNLO\_Ztautau\_MAXHTPTV500\_1000.deriv.DAOD\_FTAG2.e5307\_s3126\_r10201\_p3703\\
        mc16\_13TeV.364141.Sherpa\_221\_NNPDF30NNLO\_Ztautau\_MAXHTPTV1000\_E\_CMS.deriv.DAOD\_FTAG2.e5307\_s3126\_r10201\_p3703\\
        \hline
      \end{tabular}
      \caption{MC16d MC background samples used in this analysis. MC16a files correspond to 2017 data conditions.}
      \label{tab:mcbkgdsamplesmc16d}
  \end{center}}
\end{table}

\begin{table}[!htbp]{\tiny\renewcommand{\arraystretch}{1.2}
    \begin{center}
      \begin{tabular}{|c|}
        \hline
        Sample\\
        \hline
        $t\bar t$\\
        \hline
        mc16\_13TeV.410472.PhPy8EG\_A14\_ttbar\_hdamp258p75\_dil.deriv.DAOD\_FTAG2.e6348\_s3126\_r10724\_p3703\\
        \hline
        Diboson\\
        \hline
        mc16\_13TeV.363356.Sherpa\_221\_NNPDF30NNLO\_ZqqZll.deriv.DAOD\_FTAG2.e5525\_s3126\_r10724\_p3703\\
        mc16\_13TeV.363358.Sherpa\_221\_NNPDF30NNLO\_WqqZll.deriv.DAOD\_FTAG2.e5525\_s3126\_r10724\_p3703\\
        mc16\_13TeV.364302.Sherpa\_222\_NNPDF30NNLO\_ggZllZqq.deriv.DAOD\_FTAG2.e6273\_s3126\_r10724\_p3703\\
        \hline
        $Z+\text{jets}$\\
        \hline
        mc16\_13TeV.364100.Sherpa\_221\_NNPDF30NNLO\_Zmumu\_MAXHTPTV0\_70\_CVetoBVeto.deriv.DAOD\_FTAG2.e5271\_s3126\_r10724\_p3703\\
        mc16\_13TeV.364101.Sherpa\_221\_NNPDF30NNLO\_Zmumu\_MAXHTPTV0\_70\_CFilterBVeto.deriv.DAOD\_FTAG2.e5271\_s3126\_r10724\_p3703\\
        mc16\_13TeV.364102.Sherpa\_221\_NNPDF30NNLO\_Zmumu\_MAXHTPTV0\_70\_BFilter.deriv.DAOD\_FTAG2.e5271\_s3126\_r10724\_p3703\\
        mc16\_13TeV.364103.Sherpa\_221\_NNPDF30NNLO\_Zmumu\_MAXHTPTV70\_140\_CVetoBVeto.deriv.DAOD\_FTAG2.e5271\_s3126\_r10724\_p3703\\
        mc16\_13TeV.364104.Sherpa\_221\_NNPDF30NNLO\_Zmumu\_MAXHTPTV70\_140\_CFilterBVeto.deriv.DAOD\_FTAG2.e5271\_s3126\_r10724\_p3703\\
        mc16\_13TeV.364105.Sherpa\_221\_NNPDF30NNLO\_Zmumu\_MAXHTPTV70\_140\_BFilter.deriv.DAOD\_FTAG2.e5271\_s3126\_r10724\_p3703\\
        mc16\_13TeV.364106.Sherpa\_221\_NNPDF30NNLO\_Zmumu\_MAXHTPTV140\_280\_CVetoBVeto.deriv.DAOD\_FTAG2.e5271\_s3126\_r10724\_p3703\\
        mc16\_13TeV.364107.Sherpa\_221\_NNPDF30NNLO\_Zmumu\_MAXHTPTV140\_280\_CFilterBVeto.deriv.DAOD\_FTAG2.e5271\_s3126\_r10724\_p3703\\
        mc16\_13TeV.364108.Sherpa\_221\_NNPDF30NNLO\_Zmumu\_MAXHTPTV140\_280\_BFilter.deriv.DAOD\_FTAG2.e5271\_s3126\_r10724\_p3703\\
        mc16\_13TeV.364109.Sherpa\_221\_NNPDF30NNLO\_Zmumu\_MAXHTPTV280\_500\_CVetoBVeto.deriv.DAOD\_FTAG2.e5271\_s3126\_r10724\_p3703\\
        mc16\_13TeV.364110.Sherpa\_221\_NNPDF30NNLO\_Zmumu\_MAXHTPTV280\_500\_CFilterBVeto.deriv.DAOD\_FTAG2.e5271\_s3126\_r10724\_p3703\\
        mc16\_13TeV.364111.Sherpa\_221\_NNPDF30NNLO\_Zmumu\_MAXHTPTV280\_500\_BFilter.deriv.DAOD\_FTAG2.e5271\_s3126\_r10724\_p3703\\
        mc16\_13TeV.364112.Sherpa\_221\_NNPDF30NNLO\_Zmumu\_MAXHTPTV500\_1000.deriv.DAOD\_FTAG2.e5271\_s3126\_r10724\_p3703\\
        mc16\_13TeV.364113.Sherpa\_221\_NNPDF30NNLO\_Zmumu\_MAXHTPTV1000\_E\_CMS.deriv.DAOD\_FTAG2.e5271\_s3126\_r10724\_p3703\\
        mc16\_13TeV.364114.Sherpa\_221\_NNPDF30NNLO\_Zee\_MAXHTPTV0\_70\_CVetoBVeto.deriv.DAOD\_FTAG2.e5299\_s3126\_r10724\_p3703\\
        mc16\_13TeV.364115.Sherpa\_221\_NNPDF30NNLO\_Zee\_MAXHTPTV0\_70\_CFilterBVeto.deriv.DAOD\_FTAG2.e5299\_s3126\_r10724\_p3703\\
        mc16\_13TeV.364116.Sherpa\_221\_NNPDF30NNLO\_Zee\_MAXHTPTV0\_70\_BFilter.deriv.DAOD\_FTAG2.e5299\_s3126\_r10724\_p3703\\
        mc16\_13TeV.364117.Sherpa\_221\_NNPDF30NNLO\_Zee\_MAXHTPTV70\_140\_CVetoBVeto.deriv.DAOD\_FTAG2.e5299\_s3126\_r10724\_p3703\\
        mc16\_13TeV.364118.Sherpa\_221\_NNPDF30NNLO\_Zee\_MAXHTPTV70\_140\_CFilterBVeto.deriv.DAOD\_FTAG2.e5299\_s3126\_r10724\_p3703\\
        mc16\_13TeV.364119.Sherpa\_221\_NNPDF30NNLO\_Zee\_MAXHTPTV70\_140\_BFilter.deriv.DAOD\_FTAG2.e5299\_s3126\_r10724\_p3703\\
        mc16\_13TeV.364120.Sherpa\_221\_NNPDF30NNLO\_Zee\_MAXHTPTV140\_280\_CVetoBVeto.deriv.DAOD\_FTAG2.e5299\_s3126\_r10724\_p3703\\
        mc16\_13TeV.364121.Sherpa\_221\_NNPDF30NNLO\_Zee\_MAXHTPTV140\_280\_CFilterBVeto.deriv.DAOD\_FTAG2.e5299\_s3126\_r10724\_p3703\\
        mc16\_13TeV.364122.Sherpa\_221\_NNPDF30NNLO\_Zee\_MAXHTPTV140\_280\_BFilter.deriv.DAOD\_FTAG2.e5299\_s3126\_r10724\_p3703\\
        mc16\_13TeV.364123.Sherpa\_221\_NNPDF30NNLO\_Zee\_MAXHTPTV280\_500\_CVetoBVeto.deriv.DAOD\_FTAG2.e5299\_s3126\_r10724\_p3703\\
        mc16\_13TeV.364124.Sherpa\_221\_NNPDF30NNLO\_Zee\_MAXHTPTV280\_500\_CFilterBVeto.deriv.DAOD\_FTAG2.e5299\_s3126\_r10724\_p3703\\
        mc16\_13TeV.364125.Sherpa\_221\_NNPDF30NNLO\_Zee\_MAXHTPTV280\_500\_BFilter.deriv.DAOD\_FTAG2.e5299\_s3126\_r10724\_p3703\\
        mc16\_13TeV.364126.Sherpa\_221\_NNPDF30NNLO\_Zee\_MAXHTPTV500\_1000.deriv.DAOD\_FTAG2.e5299\_s3126\_r10724\_p3703\\
        mc16\_13TeV.364127.Sherpa\_221\_NNPDF30NNLO\_Zee\_MAXHTPTV1000\_E\_CMS.deriv.DAOD\_FTAG2.e5299\_s3126\_r10724\_p3703\\
        mc16\_13TeV.364128.Sherpa\_221\_NNPDF30NNLO\_Ztautau\_MAXHTPTV0\_70\_CVetoBVeto.deriv.DAOD\_FTAG2.e5307\_s3126\_r10724\_p3703\\
        mc16\_13TeV.364129.Sherpa\_221\_NNPDF30NNLO\_Ztautau\_MAXHTPTV0\_70\_CFilterBVeto.deriv.DAOD\_FTAG2.e5307\_s3126\_r10724\_p3703\\
        mc16\_13TeV.364130.Sherpa\_221\_NNPDF30NNLO\_Ztautau\_MAXHTPTV0\_70\_BFilter.deriv.DAOD\_FTAG2.e5307\_s3126\_r10724\_p3703\\
        mc16\_13TeV.364131.Sherpa\_221\_NNPDF30NNLO\_Ztautau\_MAXHTPTV70\_140\_CVetoBVeto.deriv.DAOD\_FTAG2.e5307\_s3126\_r10724\_p3703\\
        mc16\_13TeV.364132.Sherpa\_221\_NNPDF30NNLO\_Ztautau\_MAXHTPTV70\_140\_CFilterBVeto.deriv.DAOD\_FTAG2.e5307\_s3126\_r10724\_p3703\\
        mc16\_13TeV.364133.Sherpa\_221\_NNPDF30NNLO\_Ztautau\_MAXHTPTV70\_140\_BFilter.deriv.DAOD\_FTAG2.e5307\_s3126\_r10724\_p3703\\
        mc16\_13TeV.364134.Sherpa\_221\_NNPDF30NNLO\_Ztautau\_MAXHTPTV140\_280\_CVetoBVeto.deriv.DAOD\_FTAG2.e5307\_s3126\_r10724\_p3703\\
        mc16\_13TeV.364135.Sherpa\_221\_NNPDF30NNLO\_Ztautau\_MAXHTPTV140\_280\_CFilterBVeto.deriv.DAOD\_FTAG2.e5307\_s3126\_r10724\_p3703\\
        mc16\_13TeV.364136.Sherpa\_221\_NNPDF30NNLO\_Ztautau\_MAXHTPTV140\_280\_BFilter.deriv.DAOD\_FTAG2.e5307\_s3126\_r10724\_p3703\\
        mc16\_13TeV.364137.Sherpa\_221\_NNPDF30NNLO\_Ztautau\_MAXHTPTV280\_500\_CVetoBVeto.deriv.DAOD\_FTAG2.e5307\_s3126\_r10724\_p3703\\
        mc16\_13TeV.364138.Sherpa\_221\_NNPDF30NNLO\_Ztautau\_MAXHTPTV280\_500\_CFilterBVeto.deriv.DAOD\_FTAG2.e5313\_s3126\_r10724\_p3703\\
        mc16\_13TeV.364139.Sherpa\_221\_NNPDF30NNLO\_Ztautau\_MAXHTPTV280\_500\_BFilter.deriv.DAOD\_FTAG2.e5313\_s3126\_r10724\_p3703\\
        mc16\_13TeV.364140.Sherpa\_221\_NNPDF30NNLO\_Ztautau\_MAXHTPTV500\_1000.deriv.DAOD\_FTAG2.e5307\_s3126\_r10724\_p3703\\
        mc16\_13TeV.364141.Sherpa\_221\_NNPDF30NNLO\_Ztautau\_MAXHTPTV1000\_E\_CMS.deriv.DAOD\_FTAG2.e5307\_s3126\_r10724\_p3703\\
        \hline
      \end{tabular}
      \caption{MC16e MC background samples used in this analysis. MC16a files correspond to 2017 data conditions.}
      \label{tab:mcbkgdsamplesmc16e}
  \end{center}}
\end{table}

\begin{table}[!htbp]{\tiny\renewcommand{\arraystretch}{1.2}
    \begin{center}
      \begin{tabular}{|c|}
        \hline
        Sample\\
        \hline
        $Z+\text{jets}$ (\textsc{MadGraph})\\
        \hline
        mc16\_13TeV.363123.MGPy8EG\_N30NLO\_Zmumu\_Ht0\_70\_CVetoBVeto.deriv.DAOD\_FTAG2.e4649\_s3126\_r9364\_p3703\\
        mc16\_13TeV.363124.MGPy8EG\_N30NLO\_Zmumu\_Ht0\_70\_CFilterBVeto.deriv.DAOD\_FTAG2.e4649\_s3126\_r9364\_p3703\\
        mc16\_13TeV.363125.MGPy8EG\_N30NLO\_Zmumu\_Ht0\_70\_BFilter.deriv.DAOD\_FTAG2.e4649\_s3126\_r9364\_p3703\\
        mc16\_13TeV.363126.MGPy8EG\_N30NLO\_Zmumu\_Ht70\_140\_CVetoBVeto.deriv.DAOD\_FTAG2.e4649\_s3126\_r9364\_p3703\\
        mc16\_13TeV.363127.MGPy8EG\_N30NLO\_Zmumu\_Ht70\_140\_CFilterBVeto.deriv.DAOD\_FTAG2.e4649\_s3126\_r9364\_p3703\\
        mc16\_13TeV.363128.MGPy8EG\_N30NLO\_Zmumu\_Ht70\_140\_BFilter.deriv.DAOD\_FTAG2.e4649\_s3126\_r9364\_p3703\\
        mc16\_13TeV.363129.MGPy8EG\_N30NLO\_Zmumu\_Ht140\_280\_CVetoBVeto.deriv.DAOD\_FTAG2.e4649\_s3126\_r9364\_p3703\\
        mc16\_13TeV.363130.MGPy8EG\_N30NLO\_Zmumu\_Ht140\_280\_CFilterBVeto.deriv.DAOD\_FTAG2.e4649\_s3126\_r9364\_p3703\\
        mc16\_13TeV.363131.MGPy8EG\_N30NLO\_Zmumu\_Ht140\_280\_BFilter.deriv.DAOD\_FTAG2.e4649\_s3126\_r9364\_p3703\\
        mc16\_13TeV.363132.MGPy8EG\_N30NLO\_Zmumu\_Ht280\_500\_CVetoBVeto.deriv.DAOD\_FTAG2.e4649\_s3126\_r9364\_p3703\\
        mc16\_13TeV.363133.MGPy8EG\_N30NLO\_Zmumu\_Ht280\_500\_CFilterBVeto.deriv.DAOD\_FTAG2.e4649\_s3126\_r9364\_p3703\\
        mc16\_13TeV.363134.MGPy8EG\_N30NLO\_Zmumu\_Ht280\_500\_BFilter.deriv.DAOD\_FTAG2.e4649\_s3126\_r9364\_p3703\\
        mc16\_13TeV.363135.MGPy8EG\_N30NLO\_Zmumu\_Ht500\_700\_CVetoBVeto.deriv.DAOD\_FTAG2.e4649\_s3126\_r9364\_p3703\\
        mc16\_13TeV.363136.MGPy8EG\_N30NLO\_Zmumu\_Ht500\_700\_CFilterBVeto.deriv.DAOD\_FTAG2.e4649\_s3126\_r9364\_p3703\\
        mc16\_13TeV.363137.MGPy8EG\_N30NLO\_Zmumu\_Ht500\_700\_BFilter.deriv.DAOD\_FTAG2.e4649\_s3126\_r9364\_p3703\\
        mc16\_13TeV.363138.MGPy8EG\_N30NLO\_Zmumu\_Ht700\_1000\_CVetoBVeto.deriv.DAOD\_FTAG2.e4649\_s3126\_r9364\_p3703\\
        mc16\_13TeV.363139.MGPy8EG\_N30NLO\_Zmumu\_Ht700\_1000\_CFilterBVeto.deriv.DAOD\_FTAG2.e4649\_s3126\_r9364\_p3703\\
        mc16\_13TeV.363140.MGPy8EG\_N30NLO\_Zmumu\_Ht700\_1000\_BFilter.deriv.DAOD\_FTAG2.e4649\_s3126\_r9364\_p3703\\
        mc16\_13TeV.363141.MGPy8EG\_N30NLO\_Zmumu\_Ht1000\_2000\_CVetoBVeto.deriv.DAOD\_FTAG2.e4649\_s3126\_r9364\_p3703\\
        mc16\_13TeV.363142.MGPy8EG\_N30NLO\_Zmumu\_Ht1000\_2000\_CFilterBVeto.deriv.DAOD\_FTAG2.e4649\_s3126\_r9364\_p3703\\
        mc16\_13TeV.363143.MGPy8EG\_N30NLO\_Zmumu\_Ht1000\_2000\_BFilter.deriv.DAOD\_FTAG2.e4649\_s3126\_r9364\_p3703\\
        mc16\_13TeV.363144.MGPy8EG\_N30NLO\_Zmumu\_Ht2000\_E\_CMS\_CVetoBVeto.deriv.DAOD\_FTAG2.e4649\_s3126\_r9364\_p3703\\
        mc16\_13TeV.363145.MGPy8EG\_N30NLO\_Zmumu\_Ht2000\_E\_CMS\_CFilterBVeto.deriv.DAOD\_FTAG2.e4649\_s3126\_r9364\_p3703\\
        mc16\_13TeV.363146.MGPy8EG\_N30NLO\_Zmumu\_Ht2000\_E\_CMS\_BFilter.deriv.DAOD\_FTAG2.e4649\_s3126\_r9364\_p3703\\
        mc16\_13TeV.363147.MGPy8EG\_N30NLO\_Zee\_Ht0\_70\_CVetoBVeto.deriv.DAOD\_FTAG2.e4866\_s3126\_r9364\_p3703\\
        mc16\_13TeV.363148.MGPy8EG\_N30NLO\_Zee\_Ht0\_70\_CFilterBVeto.deriv.DAOD\_FTAG2.e4866\_s3126\_r9364\_p3703\\
        mc16\_13TeV.363149.MGPy8EG\_N30NLO\_Zee\_Ht0\_70\_BFilter.deriv.DAOD\_FTAG2.e4866\_s3126\_r9364\_p3703\\
        mc16\_13TeV.363150.MGPy8EG\_N30NLO\_Zee\_Ht70\_140\_CVetoBVeto.deriv.DAOD\_FTAG2.e4866\_s3126\_r9364\_p3703\\
        mc16\_13TeV.363151.MGPy8EG\_N30NLO\_Zee\_Ht70\_140\_CFilterBVeto.deriv.DAOD\_FTAG2.e4866\_s3126\_r9364\_p3703\\
        mc16\_13TeV.363152.MGPy8EG\_N30NLO\_Zee\_Ht70\_140\_BFilter.deriv.DAOD\_FTAG2.e4866\_s3126\_r9364\_p3703\\
        mc16\_13TeV.363153.MGPy8EG\_N30NLO\_Zee\_Ht140\_280\_CVetoBVeto.deriv.DAOD\_FTAG2.e4866\_s3126\_r9364\_p3703\\
        mc16\_13TeV.363154.MGPy8EG\_N30NLO\_Zee\_Ht140\_280\_CFilterBVeto.deriv.DAOD\_FTAG2.e4866\_s3126\_r9364\_p3703\\
        mc16\_13TeV.363155.MGPy8EG\_N30NLO\_Zee\_Ht140\_280\_BFilter.deriv.DAOD\_FTAG2.e4866\_s3126\_r9364\_p3703\\
        mc16\_13TeV.363156.MGPy8EG\_N30NLO\_Zee\_Ht280\_500\_CVetoBVeto.deriv.DAOD\_FTAG2.e4866\_s3126\_r9364\_p3703\\
        mc16\_13TeV.363157.MGPy8EG\_N30NLO\_Zee\_Ht280\_500\_CFilterBVeto.deriv.DAOD\_FTAG2.e4866\_s3126\_r9364\_p3703\\
        mc16\_13TeV.363158.MGPy8EG\_N30NLO\_Zee\_Ht280\_500\_BFilter.deriv.DAOD\_FTAG2.e4866\_s3126\_r9364\_p3703\\
        mc16\_13TeV.363159.MGPy8EG\_N30NLO\_Zee\_Ht500\_700\_CVetoBVeto.deriv.DAOD\_FTAG2.e4866\_s3126\_r9364\_p3703\\
        mc16\_13TeV.363160.MGPy8EG\_N30NLO\_Zee\_Ht500\_700\_CFilterBVeto.deriv.DAOD\_FTAG2.e4866\_s3126\_r9364\_p3703\\
        mc16\_13TeV.363161.MGPy8EG\_N30NLO\_Zee\_Ht500\_700\_BFilter.deriv.DAOD\_FTAG2.e4866\_s3126\_r9364\_p3703\\
        mc16\_13TeV.363162.MGPy8EG\_N30NLO\_Zee\_Ht700\_1000\_CVetoBVeto.deriv.DAOD\_FTAG2.e4866\_s3126\_r9364\_p3703\\
        mc16\_13TeV.363163.MGPy8EG\_N30NLO\_Zee\_Ht700\_1000\_CFilterBVeto.deriv.DAOD\_FTAG2.e4866\_s3126\_r9364\_p3703\\
        mc16\_13TeV.363164.MGPy8EG\_N30NLO\_Zee\_Ht700\_1000\_BFilter.deriv.DAOD\_FTAG2.e4866\_s3126\_r9364\_p3703\\
        mc16\_13TeV.363165.MGPy8EG\_N30NLO\_Zee\_Ht1000\_2000\_CVetoBVeto.deriv.DAOD\_FTAG2.e4866\_s3126\_r9364\_p3703\\
        mc16\_13TeV.363166.MGPy8EG\_N30NLO\_Zee\_Ht1000\_2000\_CFilterBVeto.deriv.DAOD\_FTAG2.e4866\_s3126\_r9364\_p3703\\
        mc16\_13TeV.363167.MGPy8EG\_N30NLO\_Zee\_Ht1000\_2000\_BFilter.deriv.DAOD\_FTAG2.e4866\_s3126\_r9364\_p3703\\
        mc16\_13TeV.363168.MGPy8EG\_N30NLO\_Zee\_Ht2000\_E\_CMS\_CVetoBVeto.deriv.DAOD\_FTAG2.e4866\_s3126\_r9364\_p3703\\
        mc16\_13TeV.363169.MGPy8EG\_N30NLO\_Zee\_Ht2000\_E\_CMS\_CFilterBVeto.deriv.DAOD\_FTAG2.e4866\_s3126\_r9364\_p3703\\
        mc16\_13TeV.363170.MGPy8EG\_N30NLO\_Zee\_Ht2000\_E\_CMS\_BFilter.deriv.DAOD\_FTAG2.e4866\_s3126\_r9364\_p3703\\
        mc16\_13TeV.361510.MadGraphPythia8EvtGen\_A14NNPDF23LO\_Ztautau\_Np0.deriv.DAOD\_FTAG2.e3898\_s3126\_r9364\_p3703\\
        mc16\_13TeV.361511.MadGraphPythia8EvtGen\_A14NNPDF23LO\_Ztautau\_Np1.deriv.DAOD\_FTAG2.e3898\_s3126\_r9364\_p3703\\
        mc16\_13TeV.361512.MadGraphPythia8EvtGen\_A14NNPDF23LO\_Ztautau\_Np2.deriv.DAOD\_FTAG2.e3898\_s3126\_r9364\_p3703\\
        mc16\_13TeV.361513.MadGraphPythia8EvtGen\_A14NNPDF23LO\_Ztautau\_Np3.deriv.DAOD\_FTAG2.e3898\_s3126\_r9364\_p3703\\
        mc16\_13TeV.361514.MadGraphPythia8EvtGen\_A14NNPDF23LO\_Ztautau\_Np4.deriv.DAOD\_FTAG2.e3898\_s3126\_r9364\_p3703\\
        \hline
      \end{tabular}
      \caption{MC16a MC background samples used in this analysis. MC16a files correspond to 2015 and 2016 data conditions.}
      \label{tab:mcaltbkgdsamplesmc16a}
  \end{center}}
\end{table}

\begin{table}[!htbp]{\tiny\renewcommand{\arraystretch}{1.2}
    \begin{center}
      \begin{tabular}{|c|}
        \hline
        Sample\\
        \hline
        $Z+\text{jets}$ (\textsc{MadGraph})\\
        \hline
        mc16\_13TeV.363123.MGPy8EG\_N30NLO\_Zmumu\_Ht0\_70\_CVetoBVeto.deriv.DAOD\_FTAG2.e4649\_s3126\_r10201\_p3703\\
        mc16\_13TeV.363124.MGPy8EG\_N30NLO\_Zmumu\_Ht0\_70\_CFilterBVeto.deriv.DAOD\_FTAG2.e4649\_s3126\_r10201\_p3703\\
        mc16\_13TeV.363125.MGPy8EG\_N30NLO\_Zmumu\_Ht0\_70\_BFilter.deriv.DAOD\_FTAG2.e4649\_s3126\_r10201\_p3703\\
        mc16\_13TeV.363126.MGPy8EG\_N30NLO\_Zmumu\_Ht70\_140\_CVetoBVeto.deriv.DAOD\_FTAG2.e4649\_s3126\_r10201\_p3703\\
        mc16\_13TeV.363127.MGPy8EG\_N30NLO\_Zmumu\_Ht70\_140\_CFilterBVeto.deriv.DAOD\_FTAG2.e4649\_s3126\_r10201\_p3703\\
        mc16\_13TeV.363128.MGPy8EG\_N30NLO\_Zmumu\_Ht70\_140\_BFilter.deriv.DAOD\_FTAG2.e4649\_s3126\_r10201\_p3703\\
        mc16\_13TeV.363129.MGPy8EG\_N30NLO\_Zmumu\_Ht140\_280\_CVetoBVeto.deriv.DAOD\_FTAG2.e4649\_s3126\_r10201\_p3703\\
        mc16\_13TeV.363130.MGPy8EG\_N30NLO\_Zmumu\_Ht140\_280\_CFilterBVeto.deriv.DAOD\_FTAG2.e4649\_s3126\_r10201\_p3703\\
        mc16\_13TeV.363131.MGPy8EG\_N30NLO\_Zmumu\_Ht140\_280\_BFilter.deriv.DAOD\_FTAG2.e4649\_s3126\_r10201\_p3703\\
        mc16\_13TeV.363132.MGPy8EG\_N30NLO\_Zmumu\_Ht280\_500\_CVetoBVeto.deriv.DAOD\_FTAG2.e4649\_s3126\_r10201\_p3703\\
        mc16\_13TeV.363133.MGPy8EG\_N30NLO\_Zmumu\_Ht280\_500\_CFilterBVeto.deriv.DAOD\_FTAG2.e4649\_s3126\_r10201\_p3703\\
        mc16\_13TeV.363134.MGPy8EG\_N30NLO\_Zmumu\_Ht280\_500\_BFilter.deriv.DAOD\_FTAG2.e4649\_s3126\_r10201\_p3703\\
        mc16\_13TeV.363135.MGPy8EG\_N30NLO\_Zmumu\_Ht500\_700\_CVetoBVeto.deriv.DAOD\_FTAG2.e4649\_s3126\_r10201\_p3703\\
        mc16\_13TeV.363136.MGPy8EG\_N30NLO\_Zmumu\_Ht500\_700\_CFilterBVeto.deriv.DAOD\_FTAG2.e4649\_s3126\_r10201\_p3703\\
        mc16\_13TeV.363137.MGPy8EG\_N30NLO\_Zmumu\_Ht500\_700\_BFilter.deriv.DAOD\_FTAG2.e4649\_s3126\_r10201\_p3703\\
        mc16\_13TeV.363138.MGPy8EG\_N30NLO\_Zmumu\_Ht700\_1000\_CVetoBVeto.deriv.DAOD\_FTAG2.e4649\_s3126\_r10201\_p3703\\
        mc16\_13TeV.363139.MGPy8EG\_N30NLO\_Zmumu\_Ht700\_1000\_CFilterBVeto.deriv.DAOD\_FTAG2.e4649\_s3126\_r10201\_p3703\\
        mc16\_13TeV.363140.MGPy8EG\_N30NLO\_Zmumu\_Ht700\_1000\_BFilter.deriv.DAOD\_FTAG2.e4649\_s3126\_r10201\_p3703\\
        mc16\_13TeV.363141.MGPy8EG\_N30NLO\_Zmumu\_Ht1000\_2000\_CVetoBVeto.deriv.DAOD\_FTAG2.e4649\_s3126\_r10201\_p3703\\
        mc16\_13TeV.363142.MGPy8EG\_N30NLO\_Zmumu\_Ht1000\_2000\_CFilterBVeto.deriv.DAOD\_FTAG2.e4649\_s3126\_r10201\_p3703\\
        mc16\_13TeV.363143.MGPy8EG\_N30NLO\_Zmumu\_Ht1000\_2000\_BFilter.deriv.DAOD\_FTAG2.e4649\_s3126\_r10201\_p3703\\
        mc16\_13TeV.363144.MGPy8EG\_N30NLO\_Zmumu\_Ht2000\_E\_CMS\_CVetoBVeto.deriv.DAOD\_FTAG2.e4649\_s3126\_r10201\_p3703\\
        mc16\_13TeV.363145.MGPy8EG\_N30NLO\_Zmumu\_Ht2000\_E\_CMS\_CFilterBVeto.deriv.DAOD\_FTAG2.e4649\_s3126\_r10201\_p3703\\
        mc16\_13TeV.363146.MGPy8EG\_N30NLO\_Zmumu\_Ht2000\_E\_CMS\_BFilter.deriv.DAOD\_FTAG2.e4649\_s3126\_r10201\_p3703\\
        mc16\_13TeV.363147.MGPy8EG\_N30NLO\_Zee\_Ht0\_70\_CVetoBVeto.deriv.DAOD\_FTAG2.e4866\_s3126\_r10201\_p3703\\
        mc16\_13TeV.363148.MGPy8EG\_N30NLO\_Zee\_Ht0\_70\_CFilterBVeto.deriv.DAOD\_FTAG2.e4866\_s3126\_r10201\_p3703\\
        mc16\_13TeV.363149.MGPy8EG\_N30NLO\_Zee\_Ht0\_70\_BFilter.deriv.DAOD\_FTAG2.e4866\_s3126\_r10201\_p3703\\
        mc16\_13TeV.363150.MGPy8EG\_N30NLO\_Zee\_Ht70\_140\_CVetoBVeto.deriv.DAOD\_FTAG2.e4866\_s3126\_r10201\_p3703\\
        mc16\_13TeV.363151.MGPy8EG\_N30NLO\_Zee\_Ht70\_140\_CFilterBVeto.deriv.DAOD\_FTAG2.e4866\_s3126\_r10201\_p3703\\
        mc16\_13TeV.363152.MGPy8EG\_N30NLO\_Zee\_Ht70\_140\_BFilter.deriv.DAOD\_FTAG2.e4866\_s3126\_r10201\_p3703\\
        mc16\_13TeV.363153.MGPy8EG\_N30NLO\_Zee\_Ht140\_280\_CVetoBVeto.deriv.DAOD\_FTAG2.e4866\_s3126\_r10201\_p3703\\
        mc16\_13TeV.363154.MGPy8EG\_N30NLO\_Zee\_Ht140\_280\_CFilterBVeto.deriv.DAOD\_FTAG2.e4866\_s3126\_r10201\_p3703\\
        mc16\_13TeV.363155.MGPy8EG\_N30NLO\_Zee\_Ht140\_280\_BFilter.deriv.DAOD\_FTAG2.e4866\_s3126\_r10201\_p3703\\
        mc16\_13TeV.363156.MGPy8EG\_N30NLO\_Zee\_Ht280\_500\_CVetoBVeto.deriv.DAOD\_FTAG2.e4866\_s3126\_r10201\_p3703\\
        mc16\_13TeV.363157.MGPy8EG\_N30NLO\_Zee\_Ht280\_500\_CFilterBVeto.deriv.DAOD\_FTAG2.e4866\_s3126\_r10201\_p3703\\
        mc16\_13TeV.363158.MGPy8EG\_N30NLO\_Zee\_Ht280\_500\_BFilter.deriv.DAOD\_FTAG2.e4866\_s3126\_r10201\_p3703\\
        mc16\_13TeV.363159.MGPy8EG\_N30NLO\_Zee\_Ht500\_700\_CVetoBVeto.deriv.DAOD\_FTAG2.e4866\_s3126\_r10201\_p3703\\
        mc16\_13TeV.363160.MGPy8EG\_N30NLO\_Zee\_Ht500\_700\_CFilterBVeto.deriv.DAOD\_FTAG2.e4866\_s3126\_r10201\_p3703\\
        mc16\_13TeV.363161.MGPy8EG\_N30NLO\_Zee\_Ht500\_700\_BFilter.deriv.DAOD\_FTAG2.e4866\_s3126\_r10201\_p3703\\
        mc16\_13TeV.363162.MGPy8EG\_N30NLO\_Zee\_Ht700\_1000\_CVetoBVeto.deriv.DAOD\_FTAG2.e4866\_s3126\_r10201\_p3703\\
        mc16\_13TeV.363163.MGPy8EG\_N30NLO\_Zee\_Ht700\_1000\_CFilterBVeto.deriv.DAOD\_FTAG2.e4866\_s3126\_r10201\_p3703\\
        mc16\_13TeV.363164.MGPy8EG\_N30NLO\_Zee\_Ht700\_1000\_BFilter.deriv.DAOD\_FTAG2.e4866\_s3126\_r10201\_p3703\\
        mc16\_13TeV.363165.MGPy8EG\_N30NLO\_Zee\_Ht1000\_2000\_CVetoBVeto.deriv.DAOD\_FTAG2.e4866\_s3126\_r10201\_p3703\\
        mc16\_13TeV.363166.MGPy8EG\_N30NLO\_Zee\_Ht1000\_2000\_CFilterBVeto.deriv.DAOD\_FTAG2.e4866\_s3126\_r10201\_p3703\\
        mc16\_13TeV.363167.MGPy8EG\_N30NLO\_Zee\_Ht1000\_2000\_BFilter.deriv.DAOD\_FTAG2.e4866\_s3126\_r10201\_p3703\\
        mc16\_13TeV.363168.MGPy8EG\_N30NLO\_Zee\_Ht2000\_E\_CMS\_CVetoBVeto.deriv.DAOD\_FTAG2.e4866\_s3126\_r10201\_p3703\\
        mc16\_13TeV.363169.MGPy8EG\_N30NLO\_Zee\_Ht2000\_E\_CMS\_CFilterBVeto.deriv.DAOD\_FTAG2.e4866\_s3126\_r10201\_p3703\\
        mc16\_13TeV.363170.MGPy8EG\_N30NLO\_Zee\_Ht2000\_E\_CMS\_BFilter.deriv.DAOD\_FTAG2.e4866\_s3126\_r10201\_p3703\\
        mc16\_13TeV.361510.MadGraphPythia8EvtGen\_A14NNPDF23LO\_Ztautau\_Np0.deriv.DAOD\_FTAG2.e3898\_s3126\_r10201\_p3703\\
        mc16\_13TeV.361511.MadGraphPythia8EvtGen\_A14NNPDF23LO\_Ztautau\_Np1.deriv.DAOD\_FTAG2.e3898\_s3126\_r10201\_p3703\\
        mc16\_13TeV.361512.MadGraphPythia8EvtGen\_A14NNPDF23LO\_Ztautau\_Np2.deriv.DAOD\_FTAG2.e3898\_s3126\_r10201\_p3703\\
        mc16\_13TeV.361513.MadGraphPythia8EvtGen\_A14NNPDF23LO\_Ztautau\_Np3.deriv.DAOD\_FTAG2.e3898\_s3126\_r10201\_p3703\\
        mc16\_13TeV.361514.MadGraphPythia8EvtGen\_A14NNPDF23LO\_Ztautau\_Np4.deriv.DAOD\_FTAG2.e3898\_s3126\_r10201\_p3703\\
        \hline
      \end{tabular}
      \caption{MC16d MC background samples used in this analysis. MC16a files correspond to 2017 data conditions.}
      \label{tab:mcaltbkgdsamplesmc16d}
  \end{center}}
\end{table}

\begin{table}[!htbp]{\tiny\renewcommand{\arraystretch}{1.2}
    \begin{center}
      \begin{tabular}{|c|}
        \hline
        Sample\\
        \hline
        $Z+\text{jets}$ (\textsc{MadGraph})\\
        \hline
        mc16\_13TeV.363123.MGPy8EG\_N30NLO\_Zmumu\_Ht0\_70\_CVetoBVeto.deriv.DAOD\_FTAG2.e4649\_s3126\_r10724\_p3703\\
        mc16\_13TeV.363124.MGPy8EG\_N30NLO\_Zmumu\_Ht0\_70\_CFilterBVeto.deriv.DAOD\_FTAG2.e4649\_s3126\_r10724\_p3703\\
        mc16\_13TeV.363125.MGPy8EG\_N30NLO\_Zmumu\_Ht0\_70\_BFilter.deriv.DAOD\_FTAG2.e4649\_s3126\_r10724\_p3703\\
        mc16\_13TeV.363126.MGPy8EG\_N30NLO\_Zmumu\_Ht70\_140\_CVetoBVeto.deriv.DAOD\_FTAG2.e4649\_s3126\_r10724\_p3703\\
        mc16\_13TeV.363127.MGPy8EG\_N30NLO\_Zmumu\_Ht70\_140\_CFilterBVeto.deriv.DAOD\_FTAG2.e4649\_s3126\_r10724\_p3703\\
        mc16\_13TeV.363128.MGPy8EG\_N30NLO\_Zmumu\_Ht70\_140\_BFilter.deriv.DAOD\_FTAG2.e4649\_s3126\_r10724\_p3703\\
        mc16\_13TeV.363129.MGPy8EG\_N30NLO\_Zmumu\_Ht140\_280\_CVetoBVeto.deriv.DAOD\_FTAG2.e4649\_s3126\_r10724\_p3703\\
        mc16\_13TeV.363130.MGPy8EG\_N30NLO\_Zmumu\_Ht140\_280\_CFilterBVeto.deriv.DAOD\_FTAG2.e4649\_s3126\_r10724\_p3703\\
        mc16\_13TeV.363131.MGPy8EG\_N30NLO\_Zmumu\_Ht140\_280\_BFilter.deriv.DAOD\_FTAG2.e4649\_s3126\_r10724\_p3703\\
        mc16\_13TeV.363132.MGPy8EG\_N30NLO\_Zmumu\_Ht280\_500\_CVetoBVeto.deriv.DAOD\_FTAG2.e4649\_s3126\_r10724\_p3703\\
        mc16\_13TeV.363133.MGPy8EG\_N30NLO\_Zmumu\_Ht280\_500\_CFilterBVeto.deriv.DAOD\_FTAG2.e4649\_s3126\_r10724\_p3703\\
        mc16\_13TeV.363134.MGPy8EG\_N30NLO\_Zmumu\_Ht280\_500\_BFilter.deriv.DAOD\_FTAG2.e4649\_s3126\_r10724\_p3703\\
        mc16\_13TeV.363135.MGPy8EG\_N30NLO\_Zmumu\_Ht500\_700\_CVetoBVeto.deriv.DAOD\_FTAG2.e4649\_s3126\_r10724\_p3703\\
        mc16\_13TeV.363136.MGPy8EG\_N30NLO\_Zmumu\_Ht500\_700\_CFilterBVeto.deriv.DAOD\_FTAG2.e4649\_s3126\_r10724\_p3703\\
        mc16\_13TeV.363137.MGPy8EG\_N30NLO\_Zmumu\_Ht500\_700\_BFilter.deriv.DAOD\_FTAG2.e4649\_s3126\_r10724\_p3703\\
        mc16\_13TeV.363138.MGPy8EG\_N30NLO\_Zmumu\_Ht700\_1000\_CVetoBVeto.deriv.DAOD\_FTAG2.e4649\_s3126\_r10724\_p3703\\
        mc16\_13TeV.363139.MGPy8EG\_N30NLO\_Zmumu\_Ht700\_1000\_CFilterBVeto.deriv.DAOD\_FTAG2.e4649\_s3126\_r10724\_p3703\\
        mc16\_13TeV.363140.MGPy8EG\_N30NLO\_Zmumu\_Ht700\_1000\_BFilter.deriv.DAOD\_FTAG2.e4649\_s3126\_r10724\_p3703\\
        mc16\_13TeV.363141.MGPy8EG\_N30NLO\_Zmumu\_Ht1000\_2000\_CVetoBVeto.deriv.DAOD\_FTAG2.e4649\_s3126\_r10724\_p3703\\
        mc16\_13TeV.363142.MGPy8EG\_N30NLO\_Zmumu\_Ht1000\_2000\_CFilterBVeto.deriv.DAOD\_FTAG2.e4649\_s3126\_r10724\_p3703\\
        mc16\_13TeV.363143.MGPy8EG\_N30NLO\_Zmumu\_Ht1000\_2000\_BFilter.deriv.DAOD\_FTAG2.e4649\_s3126\_r10724\_p3703\\
        mc16\_13TeV.363144.MGPy8EG\_N30NLO\_Zmumu\_Ht2000\_E\_CMS\_CVetoBVeto.deriv.DAOD\_FTAG2.e4649\_s3126\_r10724\_p3703\\
        mc16\_13TeV.363145.MGPy8EG\_N30NLO\_Zmumu\_Ht2000\_E\_CMS\_CFilterBVeto.deriv.DAOD\_FTAG2.e4649\_s3126\_r10724\_p3703\\
        mc16\_13TeV.363146.MGPy8EG\_N30NLO\_Zmumu\_Ht2000\_E\_CMS\_BFilter.deriv.DAOD\_FTAG2.e4649\_s3126\_r10724\_p3703\\
        mc16\_13TeV.363147.MGPy8EG\_N30NLO\_Zee\_Ht0\_70\_CVetoBVeto.deriv.DAOD\_FTAG2.e4866\_s3126\_r10724\_p3703\\
        mc16\_13TeV.363148.MGPy8EG\_N30NLO\_Zee\_Ht0\_70\_CFilterBVeto.deriv.DAOD\_FTAG2.e4866\_s3126\_r10724\_p3703\\
        mc16\_13TeV.363149.MGPy8EG\_N30NLO\_Zee\_Ht0\_70\_BFilter.deriv.DAOD\_FTAG2.e4866\_s3126\_r10724\_p3703\\
        mc16\_13TeV.363150.MGPy8EG\_N30NLO\_Zee\_Ht70\_140\_CVetoBVeto.deriv.DAOD\_FTAG2.e4866\_s3126\_r10724\_p3703\\
        mc16\_13TeV.363151.MGPy8EG\_N30NLO\_Zee\_Ht70\_140\_CFilterBVeto.deriv.DAOD\_FTAG2.e4866\_s3126\_r10724\_p3703\\
        mc16\_13TeV.363152.MGPy8EG\_N30NLO\_Zee\_Ht70\_140\_BFilter.deriv.DAOD\_FTAG2.e4866\_s3126\_r10724\_p3703\\
        mc16\_13TeV.363153.MGPy8EG\_N30NLO\_Zee\_Ht140\_280\_CVetoBVeto.deriv.DAOD\_FTAG2.e4866\_s3126\_r10724\_p3703\\
        mc16\_13TeV.363154.MGPy8EG\_N30NLO\_Zee\_Ht140\_280\_CFilterBVeto.deriv.DAOD\_FTAG2.e4866\_s3126\_r10724\_p3703\\
        mc16\_13TeV.363155.MGPy8EG\_N30NLO\_Zee\_Ht140\_280\_BFilter.deriv.DAOD\_FTAG2.e4866\_s3126\_r10724\_p3703\\
        mc16\_13TeV.363156.MGPy8EG\_N30NLO\_Zee\_Ht280\_500\_CVetoBVeto.deriv.DAOD\_FTAG2.e4866\_s3126\_r10724\_p3703\\
        mc16\_13TeV.363157.MGPy8EG\_N30NLO\_Zee\_Ht280\_500\_CFilterBVeto.deriv.DAOD\_FTAG2.e4866\_s3126\_r10724\_p3703\\
        mc16\_13TeV.363158.MGPy8EG\_N30NLO\_Zee\_Ht280\_500\_BFilter.deriv.DAOD\_FTAG2.e4866\_s3126\_r10724\_p3703\\
        mc16\_13TeV.363159.MGPy8EG\_N30NLO\_Zee\_Ht500\_700\_CVetoBVeto.deriv.DAOD\_FTAG2.e4866\_s3126\_r10724\_p3703\\
        mc16\_13TeV.363160.MGPy8EG\_N30NLO\_Zee\_Ht500\_700\_CFilterBVeto.deriv.DAOD\_FTAG2.e4866\_s3126\_r10724\_p3703\\
        mc16\_13TeV.363161.MGPy8EG\_N30NLO\_Zee\_Ht500\_700\_BFilter.deriv.DAOD\_FTAG2.e4866\_s3126\_r10724\_p3703\\
        mc16\_13TeV.363162.MGPy8EG\_N30NLO\_Zee\_Ht700\_1000\_CVetoBVeto.deriv.DAOD\_FTAG2.e4866\_s3126\_r10724\_p3703\\
        mc16\_13TeV.363163.MGPy8EG\_N30NLO\_Zee\_Ht700\_1000\_CFilterBVeto.deriv.DAOD\_FTAG2.e4866\_s3126\_r10724\_p3703\\
        mc16\_13TeV.363164.MGPy8EG\_N30NLO\_Zee\_Ht700\_1000\_BFilter.deriv.DAOD\_FTAG2.e4866\_s3126\_r10724\_p3703\\
        mc16\_13TeV.363165.MGPy8EG\_N30NLO\_Zee\_Ht1000\_2000\_CVetoBVeto.deriv.DAOD\_FTAG2.e4866\_s3126\_r10724\_p3703\\
        mc16\_13TeV.363166.MGPy8EG\_N30NLO\_Zee\_Ht1000\_2000\_CFilterBVeto.deriv.DAOD\_FTAG2.e4866\_s3126\_r10724\_p3703\\
        mc16\_13TeV.363167.MGPy8EG\_N30NLO\_Zee\_Ht1000\_2000\_BFilter.deriv.DAOD\_FTAG2.e4866\_s3126\_r10724\_p3703\\
        mc16\_13TeV.363168.MGPy8EG\_N30NLO\_Zee\_Ht2000\_E\_CMS\_CVetoBVeto.deriv.DAOD\_FTAG2.e4866\_s3126\_r10724\_p3703\\
        mc16\_13TeV.363169.MGPy8EG\_N30NLO\_Zee\_Ht2000\_E\_CMS\_CFilterBVeto.deriv.DAOD\_FTAG2.e4866\_s3126\_r10724\_p3703\\
        mc16\_13TeV.363170.MGPy8EG\_N30NLO\_Zee\_Ht2000\_E\_CMS\_BFilter.deriv.DAOD\_FTAG2.e4866\_s3126\_r10724\_p3703\\
        mc16\_13TeV.361510.MadGraphPythia8EvtGen\_A14NNPDF23LO\_Ztautau\_Np0.deriv.DAOD\_FTAG2.e3898\_s3126\_r10724\_p3703\\
        mc16\_13TeV.361511.MadGraphPythia8EvtGen\_A14NNPDF23LO\_Ztautau\_Np1.deriv.DAOD\_FTAG2.e3898\_s3126\_r10724\_p3703\\
        mc16\_13TeV.361512.MadGraphPythia8EvtGen\_A14NNPDF23LO\_Ztautau\_Np2.deriv.DAOD\_FTAG2.e3898\_s3126\_r10724\_p3703\\
        mc16\_13TeV.361513.MadGraphPythia8EvtGen\_A14NNPDF23LO\_Ztautau\_Np3.deriv.DAOD\_FTAG2.e3898\_s3126\_r10724\_p3703\\
        mc16\_13TeV.361514.MadGraphPythia8EvtGen\_A14NNPDF23LO\_Ztautau\_Np4.deriv.DAOD\_FTAG2.e3898\_s3126\_r10724\_p3703\\
        \hline
      \end{tabular}
      \caption{MC16e MC background samples used in this analysis. MC16a files correspond to 2017 data conditions.}
      \label{tab:mcaltbkgdsamplesmc16e}
  \end{center}}
\end{table}


\clearpage
\section{Trigger Studies}
\label{app:triggers}

The trigger efficiencies after pre-selection are shown in Table~\ref{tab:trigeffs}. The inclusion of di-lepton triggers was found to result in just a $\sim 2\%$ gain in signal efficiency, while letting in more background and increasing the uncertainty on the trigger scale factors (also making the \emph{TrigGlobalEfficiencyCorrectionTool} tool insufficient for the evaluation of these uncertainties). For this reason these di-lepton triggers were not included.


\begin{table}[!htbp]{\footnotesize\renewcommand{\arraystretch}{1.2}
    \begin{center}
      \begin{tabular}{|c|c|c|}
        \hline
        Sample & Single Trigger Efficiency & Fractional Efficiency Gain \\
        \hline
        0.5~GeV $a$ & $93.6 \pm 0.7\%$ & $2.4\pm 0.8\%$ \\
        2.5~GeV $a$ & $93.2 \pm 0.7\%$ & $2.6\pm 0.7\%$ \\
        8~GeV $a$ & $93.9\pm 0.9\%$ & $2.5\pm 1\%$ \\
        $\eta_c$ & $93.4\pm 0.7\%$ & $2.8\pm 0.7\%$ \\
        \hline
      \end{tabular}
      \caption{Trigger efficiencies for the nominal single lepton triggers after pre-selection, and fractional gain in trigger efficiency by introducing the dilepton triggers as well.}
      \label{tab:trigeffs}
  \end{center}}
\end{table}


%% \section{$X$ Identification Variables Considered}
%% \label{app:nonmlpvars}

%% Other variables which were considered, but were found to be less discriminant that the chosen MLP input variables are listed in this Appendix.

%% Ghost-associated track based variables ($R=0.4$):
%% \begin{multicols}{2}
%%   \begin{itemize}
%%   \item D2
%%   \item N2
%%   \item N3
%%   \item R2
%%   \item U2
%%   \item V2
%%   \item c2
%%   \item c3
%%   \item cc2
%%   \item m
%%   \item nTracks
%%   \item planarFlow
%%   \item pt
%%   \item tauRatio21
%%   \end{itemize}
%% \end{multicols}

%% Ghost-associated track based variables ($R=0.2$):
%% \begin{multicols}{2}
%%   \begin{itemize}
%%   \item deltaRLeadTrack
%%   \item leadTrackPt
%%   \item m
%%   \item mTA
%%   \item meanDeltaR
%%   \item nTracks
%%   \item pt
%%   \end{itemize}
%% \end{multicols}

%% Ghost-associated track ratio based variables ($R=0.4/R=0.2$):
%% \begin{multicols}{2}
%%   \begin{itemize}
%%   \item deltaRLeadTrack
%%   \item leadTrackPt
%%   \item m
%%   \item mTA
%%   \item meanDeltaR
%%   \item nTracks
%%   \item pt
%%   \end{itemize}
%% \end{multicols}

%% Flavour Tagging Variables:
%% \begin{multicols}{2}
%%   \begin{itemize}
%%   \item DL1mupb
%%   \item DL1mupc
%%   \item DL1mupu
%%   \item DL1pb
%%   \item DL1pc
%%   \item DL1pu
%%   \item DL1rnn\_pb
%%   \item DL1rnn\_pc
%%   \item DL1rnn\_pu
%%   \item IP2D\_loglikelihoodratio
%%   \item IP2D\_pb
%%   \item IP2D\_pc
%%   \item IP2D\_pu
%%   \item IP3D\_loglikelihoodratio
%%   \item IP3D\_pb
%%   \item IP3D\_pc
%%   \item IP3D\_pu
%%   \item JetFitter\_N2Tpair
%%   \item JetFitter\_chi2
%%   \item JetFitter\_dRFlightDir
%%   \item JetFitter\_deltaeta
%%   \item JetFitter\_deltaphi
%%   \item JetFitter\_energyFraction
%%   \item JetFitter\_loglikelihoodratio
%%   \item JetFitter\_mass
%%   \item JetFitter\_massUncorr
%%   \item JetFitter\_nSingleTracks
%%   \item JetFitter\_nTracksAtVtx
%%   \item JetFitter\_nVTX
%%   \item JetFitter\_ndof
%%   \item JetFitter\_pb
%%   \item JetFitter\_pc
%%   \item JetFitter\_pu
%%   \item JetFitter\_significance3d
%%   \item JetVertexCharge\_discriminant
%%   \item MSV\_N2Tpair
%%   \item MSV\_energyTrkInJet
%%   \item MSV\_normdist
%%   \item MSV\_nvsec
%%   \item MV2c100\_discriminant
%%   \item MV2c10\_discriminant
%%   \item MultiSVbb1\_discriminant
%%   \item MultiSVbb2\_discriminant
%%   \item SMT\_ID\_qOverP
%%   \item SMT\_dR
%%   \item SMT\_mombalsignif
%%   \item SMT\_mu\_d0
%%   \item SMT\_mu\_pt
%%   \item SMT\_mu\_z0
%%   \item SMT\_pTrel
%%   \item SMT\_qOverPratio
%%   \item SMT\_scatneighsignif
%%   \item SV1\_L3d
%%   \item SV1\_Lxy
%%   \item SV1\_N2Tpair
%%   \item SV1\_NGTinSvx
%%   \item SV1\_deltaR
%%   \item SV1\_dstToMatLay
%%   \item SV1\_efracsvx
%%   \item SV1\_energyTrkInJet
%%   \item SV1\_loglikelihoodratio
%%   \item SV1\_masssvx
%%   \item SV1\_normdist
%%   \item SV1\_pb
%%   \item SV1\_pc
%%   \item SV1\_pu
%%   \item SV1\_significance3d
%%   \item SV1plusIP3D\_discriminant
%%   \item rnnip\_pb
%%   \item rnnip\_pc
%%   \item rnnip\_ptau
%%   \item rnnip\_pu
%%   \item trkSum\_SPt
%%   \item trkSum\_VEta
%%   \item trkSum\_VPt
%%   \item trkSum\_ntrk
%%   \end{itemize}
%% \end{multicols}


\clearpage
\section{Track Selection Studies}
\label{app:trackselectionstudies}

In order to reject the large contamination from pileup tracks, two track selection requirements are considered. One is a combination of the ATLAS \emph{Loose} Track Quality and \emph{Loose} track-to-vertex association (TTVA) requirements, which are implemented using the \emph{InDetTrackSelectionTool}~\cite{InDetTrackSelectionTool} and \emph{TrackVertexAssociationTool}~\cite{TrackVertexAssociationTool}, respectively. The other requirement considered is a combination of the ATLAS Loose Track Quality and specifically optimised vertexing requirements (summarised in Table~\ref{tab:trackselection}). While the custom requirements did result in a better signal to background ratio, then also resulted in a greater dependence on the signal hypothesis mass, as shown in Table~\ref{tab:trackselection}. As such, the former requirement is chosen.

\begin{table}[!htbp]{\footnotesize\renewcommand{\arraystretch}{1.2} % Redo using FTAG2 samples
    \begin{center}
      \begin{tabular}{|c|c|c|c|c|}
        \hline
        \multirow{2}{*}{Signal Sample} & \multicolumn{2}{c|}{\BF{Loose TTVA ($\vert d_0\vert <2$, $\vert \Delta z_0sin\theta\vert <3$)}} & \multicolumn{2}{c|}{$\vert d_0\vert <2$, $\vert \Delta z_0sin\theta\vert <1$, $p_\text{T}>1$~GeV}\\
        \cline{2-5}
        & Signal Efficiency & Pileup Efficiency & Signal Efficiency & Pileup Efficiency\\
        \hline
        0.5~GeV $a$ & $96\pm 7\%$ & $41\pm 3\%$ & $94\pm 7\%$ & $26\pm 2\%$\\
        2.5~GeV $a$ & $95\pm 6\%$ & $38\pm 2\%$ & $90\pm 5\%$ & $24\pm 2\%$\\
        8~GeV $a$ & $94\pm 6\%$ & $47\pm 3\%$ & $83\pm 5\%$ & $32\pm 2\%$\\
        $\eta_c$ & $95\pm 4\%$ & $37\pm 2\%$ & $88\pm 4\%$ & $23\pm 2\%$\\
        \hline
      \end{tabular}
      \caption{Comparison of efficiencies of two track selection working points. Both working points include the \emph{Loose} track selection working point. The \emph{Loose} \textsc{TMVA} working point (in bold) is adopted for the analysis, while the other requirement was optimised specifically for the analysis.}
      \label{tab:trackselection}
  \end{center}}
\end{table}



\clearpage
\section{MVA Input Variable Reduction Studies}
\label{app:mvavarreductionstudies}


A BDT analogous to the one described in this section was trained without $\eta$, $angularity(2)$, and both. The signal efficiencies for a 1\% background efficiency are shown in Table~\ref{tab:altbdtvars}. The efficiency loss from removing the $angularity(2)$ variable was on the order of 2-12\% (depending on the signal hypothesis), which warrants keeping the variable. However, the $\eta$ variable resulted in a negligible loss in efficiency. As such, the next iteration of the analysis is expected to use a BDT without the $\eta$ variable.

\begin{table}[!htbp]{\footnotesize\renewcommand{\arraystretch}{1.2}
    \begin{center}
      \begin{tabular}{|c|c|c|c|c|}
        \hline
        Sample & Nominal BDT & No $\eta$ & No $angularity(2)$ & No $\eta$ or $angularity(2)$ \\
        \hline
        0.5~GeV $a$ & 46.1\% & 46.1\% & 45.7\% & 45.2\%\\
        2.5~GeV $a$ & 17.7\% & 17.6\% & 15.5\% & 15.6\%\\
        8~GeV $a$ & 3.3\% & 3.4\% & 3.2\% & 3.0\%\\
        $\eta_c$ & 10.3\% & 10.0\% & 9.3\% & 9.3\%\\
        \hline
      \end{tabular}
      \caption{Signal efficiencies for each of the BDTs described in subsubsection~\ref{app:mvavarreductionstudies}, for a constant background efficiency of 1\%.}
      \label{tab:altbdtvars}
  \end{center}}
\end{table}


\clearpage
\section{MVA Hyper-Parameter Studies}
\label{app:bdthyperparameters}

The variables used to definite the BDT with which the $X$ particle is identified are: NTrees, the number of trees in the random forest; MaxDepth, the maximum depth of each decision tree; MinNodeSize, the minimum fraction of evens that may be contained in a tree node; nCuts, the number of cut values which are tested in each variable. Various values of these parameters were tested before settling on the default \textsc{TMVA} parameters. These tested values included: NTrees=400; NTrees=1200; MaxDepth=4; MinNodeSize=2\%; MinNodeSize=2\% and nCuts=50.


%% \section{Background Estimate Signal Contamination Studies}
%% \label{app:abcdcontaminationstudy}

%% The signal and background estimated by simulation in each of the regions used in the ABCD-based background estimate, and in the background validation, are shown in Tables~\ref{tab:sigconbkgd} to \ref{tab:sigconjpsi}. The signal contamination in each of the regions used in the ABCD estimate is below 0.68\%, assuming a BR$(H\to ZX)=100\%$. In reality, should such a signal exist, its BR will be much lower, making any signal contamination negligible on the scale of the MC statistical uncertainties on the background estimate.

%% \begin{table}[!htbp]{\footnotesize\renewcommand{\arraystretch}{1.2}
%%     \begin{center}
%%       \begin{tabular}{c|c|c|c|c|c|}
%%         \cline{2-6}
%%         & \multicolumn{5}{|c|}{$m_{\ell\ell j}$ Range}\\
%%         \hline
%%         \multicolumn{1}{|c|}{MLP Range} & $100-110$~GeV & $110-120$~GeV & $120-135$~GeV & $135-155$~GeV & $155-175$~GeV \\
%%         \hline
%%         \multicolumn{1}{|c|}{SR (99-100\%)} & $2020\pm 320$ & $18900\pm 920$ & $72300\pm 1940$ & $109000\pm 2170$ & $89500\pm 1940$ \\
%%         \multicolumn{1}{|c|}{VR (98-99\%)} & $3600\pm 308$ & $29800\pm 1030$ & $81000\pm 1710$ & $93400\pm 1860$ & $69600\pm 1500$ \\
%%         \multicolumn{1}{|c|}{VR (97-98\%)} & $4170\pm 334$ & $31700\pm 1090$ & $79500\pm 1870$ & $89400\pm 1730$ & $67200\pm 1470$ \\
%%         \multicolumn{1}{|c|}{R (0-97\%)} & $675000\pm 8130$ & $4640000\pm 12700$ & $9250000\pm 26000$ & $9650000\pm 25400$ & $6900000\pm 15700$ \\
%%         \hline
%%       \end{tabular}
%%       \caption{Background yield as estimated in MC simulation.}
%%       \label{tab:sigconbkgd}
%%   \end{center}}
%% \end{table}

%% \begin{table}[!htbp]{\footnotesize\renewcommand{\arraystretch}{1.2}
%%     \begin{center}
%%       \begin{tabular}{c|c|c|c|c|c|}
%%         \cline{2-6}
%%         & \multicolumn{5}{|c|}{$m_{\ell\ell j}$ Range}\\
%%         \hline
%%         \multicolumn{1}{|c|}{MLP Range} & $100-110$~GeV & $110-120$~GeV & $120-135$~GeV & $135-155$~GeV & $155-175$~GeV \\
%%         \hline
%%         \multicolumn{1}{|c|}{SR (99-100\%)} & $\num[round-mode=figures,round-precision=3]{42.8078} \pm \num[round-mode=figures,round-precision=3]{17.6005}$ & $\num[round-mode=figures,round-precision=3]{1362.53} \pm \num[round-mode=figures,round-precision=3]{97.0789}$ & $\num[round-mode=figures,round-precision=3]{25444.8} \pm \num[round-mode=figures,round-precision=3]{424.908}$ & $\num[round-mode=figures,round-precision=3]{8366.23} \pm \num[round-mode=figures,round-precision=3]{242.739}$ & $\num[round-mode=figures,round-precision=3]{174.002} \pm \num[round-mode=figures,round-precision=3]{34.6818}$ \\
%%         \multicolumn{1}{|c|}{VR (98-99\%)} & $\num[round-mode=figures,round-precision=3]{6.16006} \pm \num[round-mode=figures,round-precision=3]{6.16006}$ & $\num[round-mode=figures,round-precision=3]{224.793} \pm \num[round-mode=figures,round-precision=3]{39.548}$ & $\num[round-mode=figures,round-precision=3]{3910.29} \pm \num[round-mode=figures,round-precision=3]{166.76}$ & $\num[round-mode=figures,round-precision=3]{2191.27} \pm \num[round-mode=figures,round-precision=3]{125.028}$ & $\num[round-mode=figures,round-precision=3]{175.898} \pm \num[round-mode=figures,round-precision=3]{34.6782}$ \\
%%         \multicolumn{1}{|c|}{VR (97-98\%)} & $\num[round-mode=figures,round-precision=3]{12.6322} \pm \num[round-mode=figures,round-precision=3]{9.02941}$ & $\num[round-mode=figures,round-precision=3]{136.178} \pm \num[round-mode=figures,round-precision=3]{30.4342}$ & $\num[round-mode=figures,round-precision=3]{3169.36} \pm \num[round-mode=figures,round-precision=3]{149.637}$ & $\num[round-mode=figures,round-precision=3]{1884.12} \pm \num[round-mode=figures,round-precision=3]{116.597}$ & $\num[round-mode=figures,round-precision=3]{117.309} \pm \num[round-mode=figures,round-precision=3]{28.6256}$ \\
%%         \multicolumn{1}{|c|}{R (0-97\%)} & $\num[round-mode=figures,round-precision=3]{535.7} \pm \num[round-mode=figures,round-precision=3]{61.6877}$ & $\num[round-mode=figures,round-precision=3]{4086.98} \pm \num[round-mode=figures,round-precision=3]{169.541}$ & $\num[round-mode=figures,round-precision=3]{22994} \pm \num[round-mode=figures,round-precision=3]{402.071}$ & $\num[round-mode=figures,round-precision=3]{26566.4} \pm \num[round-mode=figures,round-precision=3]{431.761}$ & $\num[round-mode=figures,round-precision=3]{13277.1} \pm \num[round-mode=figures,round-precision=3]{305.484}$ \\
%%         \hline
%%       \end{tabular}
%%       \caption{0.5~GeV $a$ signal yield (assuming $\tan\beta=1$ and BR$(H\to Za)=100\%$) as estimated in MC simulation.}
%%       \label{tab:sigcona00p5}
%%   \end{center}}
%% \end{table}

%% \begin{table}[!htbp]{\footnotesize\renewcommand{\arraystretch}{1.2}
%%     \begin{center}
%%       \begin{tabular}{c|c|c|c|c|c|}
%%         \cline{2-6}
%%         & \multicolumn{5}{|c|}{$m_{\ell\ell j}$ Range}\\
%%         \hline
%%         \multicolumn{1}{|c|}{MLP Range} & $100-110$~GeV & $110-120$~GeV & $120-135$~GeV & $135-155$~GeV & $155-175$~GeV \\
%%         \hline
%%         \multicolumn{1}{|c|}{SR (99-100\%)} & $\num[round-mode=figures,round-precision=3]{29.1667} \pm \num[round-mode=figures,round-precision=3]{13.5828}$ & $\num[round-mode=figures,round-precision=3]{1338.5} \pm \num[round-mode=figures,round-precision=3]{96.1683}$ & $\num[round-mode=figures,round-precision=3]{21591.2} \pm \num[round-mode=figures,round-precision=3]{391.816}$ & $\num[round-mode=figures,round-precision=3]{6636.68} \pm \num[round-mode=figures,round-precision=3]{216.75}$ & $\num[round-mode=figures,round-precision=3]{219.268} \pm \num[round-mode=figures,round-precision=3]{38.7558}$ \\
%%         \multicolumn{1}{|c|}{VR (98-99\%)} & $\num[round-mode=figures,round-precision=3]{15.8508} \pm \num[round-mode=figures,round-precision=3]{11.213}$ & $\num[round-mode=figures,round-precision=3]{218.094} \pm \num[round-mode=figures,round-precision=3]{37.8792}$ & $\num[round-mode=figures,round-precision=3]{4138.5} \pm \num[round-mode=figures,round-precision=3]{171.536}$ & $\num[round-mode=figures,round-precision=3]{2073.88} \pm \num[round-mode=figures,round-precision=3]{121.151}$ & $\num[round-mode=figures,round-precision=3]{112.976} \pm \num[round-mode=figures,round-precision=3]{28.0154}$ \\
%%         \multicolumn{1}{|c|}{VR (97-98\%)} & $\num[round-mode=figures,round-precision=3]{0} \pm \num[round-mode=figures,round-precision=3]{0}$ & $\num[round-mode=figures,round-precision=3]{193.52} \pm \num[round-mode=figures,round-precision=3]{36.773}$ & $\num[round-mode=figures,round-precision=3]{2984.39} \pm \num[round-mode=figures,round-precision=3]{145.086}$ & $\num[round-mode=figures,round-precision=3]{1760.71} \pm \num[round-mode=figures,round-precision=3]{110.839}$ & $\num[round-mode=figures,round-precision=3]{97.2927} \pm \num[round-mode=figures,round-precision=3]{26.5039}$ \\
%%         \multicolumn{1}{|c|}{R (0-97\%)} & $\num[round-mode=figures,round-precision=3]{617.322} \pm \num[round-mode=figures,round-precision=3]{65.8159}$ & $\num[round-mode=figures,round-precision=3]{4169.44} \pm \num[round-mode=figures,round-precision=3]{171.762}$ & $\num[round-mode=figures,round-precision=3]{22657.5} \pm \num[round-mode=figures,round-precision=3]{398.389}$ & $\num[round-mode=figures,round-precision=3]{26773.8} \pm \num[round-mode=figures,round-precision=3]{433.447}$ & $\num[round-mode=figures,round-precision=3]{13114.8} \pm \num[round-mode=figures,round-precision=3]{303.462}$ \\
%%         \hline
%%       \end{tabular}
%%       \caption{0.75~GeV $a$ signal yield (assuming $\tan\beta=1$ and BR$(H\to Za)=100\%$) as estimated in MC simulation.}
%%       \label{tab:sigcona00p75}
%%   \end{center}}
%% \end{table}

%% \begin{table}[!htbp]{\footnotesize\renewcommand{\arraystretch}{1.2}
%%     \begin{center}
%%       \begin{tabular}{c|c|c|c|c|c|}
%%         \cline{2-6}
%%         & \multicolumn{5}{|c|}{$m_{\ell\ell j}$ Range}\\
%%         \hline
%%         \multicolumn{1}{|c|}{MLP Range} & $100-110$~GeV & $110-120$~GeV & $120-135$~GeV & $135-155$~GeV & $155-175$~GeV \\
%%         \hline
%%         \multicolumn{1}{|c|}{SR (99-100\%)} & $\num[round-mode=figures,round-precision=3]{49.8649} \pm \num[round-mode=figures,round-precision=3]{18.9886}$ & $\num[round-mode=figures,round-precision=3]{1632.1} \pm \num[round-mode=figures,round-precision=3]{107.846}$ & $\num[round-mode=figures,round-precision=3]{22421.6} \pm \num[round-mode=figures,round-precision=3]{400.641}$ & $\num[round-mode=figures,round-precision=3]{6191.73} \pm \num[round-mode=figures,round-precision=3]{209.724}$ & $\num[round-mode=figures,round-precision=3]{155.737} \pm \num[round-mode=figures,round-precision=3]{33.9495}$ \\
%%         \multicolumn{1}{|c|}{VR (98-99\%)} & $\num[round-mode=figures,round-precision=3]{10.8569} \pm \num[round-mode=figures,round-precision=3]{7.7717}$ & $\num[round-mode=figures,round-precision=3]{211.757} \pm \num[round-mode=figures,round-precision=3]{39.1437}$ & $\num[round-mode=figures,round-precision=3]{4924.14} \pm \num[round-mode=figures,round-precision=3]{186.632}$ & $\num[round-mode=figures,round-precision=3]{2620.47} \pm \num[round-mode=figures,round-precision=3]{135.706}$ & $\num[round-mode=figures,round-precision=3]{126.219} \pm \num[round-mode=figures,round-precision=3]{29.4}$ \\
%%         \multicolumn{1}{|c|}{VR (97-98\%)} & $\num[round-mode=figures,round-precision=3]{8.12009} \pm \num[round-mode=figures,round-precision=3]{8.12009}$ & $\num[round-mode=figures,round-precision=3]{217.804} \pm \num[round-mode=figures,round-precision=3]{39.2697}$ & $\num[round-mode=figures,round-precision=3]{3722.48} \pm \num[round-mode=figures,round-precision=3]{161.089}$ & $\num[round-mode=figures,round-precision=3]{1764.76} \pm \num[round-mode=figures,round-precision=3]{111.061}$ & $\num[round-mode=figures,round-precision=3]{144.323} \pm \num[round-mode=figures,round-precision=3]{31.9958}$ \\
%%         \multicolumn{1}{|c|}{R (0-97\%)} & $\num[round-mode=figures,round-precision=3]{665.743} \pm \num[round-mode=figures,round-precision=3]{68.7893}$ & $\num[round-mode=figures,round-precision=3]{3702.17} \pm \num[round-mode=figures,round-precision=3]{161.777}$ & $\num[round-mode=figures,round-precision=3]{27412.5} \pm \num[round-mode=figures,round-precision=3]{440.398}$ & $\num[round-mode=figures,round-precision=3]{28939.2} \pm \num[round-mode=figures,round-precision=3]{451.431}$ & $\num[round-mode=figures,round-precision=3]{13223} \pm \num[round-mode=figures,round-precision=3]{304.965}$ \\
%%         \hline
%%       \end{tabular}
%%       \caption{1~GeV $a$ signal yield (assuming $\tan\beta=1$ and BR$(H\to Za)=100\%$) as estimated in MC simulation.}
%%       \label{tab:sigcona01p0}
%%   \end{center}}
%% \end{table}

%% \begin{table}[!htbp]{\footnotesize\renewcommand{\arraystretch}{1.2}
%%     \begin{center}
%%       \begin{tabular}{c|c|c|c|c|c|}
%%         \cline{2-6}
%%         & \multicolumn{5}{|c|}{$m_{\ell\ell j}$ Range}\\
%%         \hline
%%         \multicolumn{1}{|c|}{MLP Range} & $100-110$~GeV & $110-120$~GeV & $120-135$~GeV & $135-155$~GeV & $155-175$~GeV \\
%%         \hline
%%         \multicolumn{1}{|c|}{SR (99-100\%)} & $\num[round-mode=figures,round-precision=3]{22.7946} \pm \num[round-mode=figures,round-precision=3]{13.1952}$ & $\num[round-mode=figures,round-precision=3]{1336.97} \pm \num[round-mode=figures,round-precision=3]{96.6339}$ & $\num[round-mode=figures,round-precision=3]{19659.7} \pm \num[round-mode=figures,round-precision=3]{374.611}$ & $\num[round-mode=figures,round-precision=3]{4569.22} \pm \num[round-mode=figures,round-precision=3]{179.556}$ & $\num[round-mode=figures,round-precision=3]{67.2091} \pm \num[round-mode=figures,round-precision=3]{23.0373}$ \\
%%         \multicolumn{1}{|c|}{VR (98-99\%)} & $\num[round-mode=figures,round-precision=3]{0} \pm \num[round-mode=figures,round-precision=3]{0}$ & $\num[round-mode=figures,round-precision=3]{369.203} \pm \num[round-mode=figures,round-precision=3]{51.1582}$ & $\num[round-mode=figures,round-precision=3]{6054.63} \pm \num[round-mode=figures,round-precision=3]{206.77}$ & $\num[round-mode=figures,round-precision=3]{2351.8} \pm \num[round-mode=figures,round-precision=3]{128.485}$ & $\num[round-mode=figures,round-precision=3]{78.1924} \pm \num[round-mode=figures,round-precision=3]{22.5934}$ \\
%%         \multicolumn{1}{|c|}{VR (97-98\%)} & $\num[round-mode=figures,round-precision=3]{21.3697} \pm \num[round-mode=figures,round-precision=3]{12.4264}$ & $\num[round-mode=figures,round-precision=3]{188.36} \pm \num[round-mode=figures,round-precision=3]{36.9243}$ & $\num[round-mode=figures,round-precision=3]{4124.13} \pm \num[round-mode=figures,round-precision=3]{172.032}$ & $\num[round-mode=figures,round-precision=3]{1546.02} \pm \num[round-mode=figures,round-precision=3]{103.734}$ & $\num[round-mode=figures,round-precision=3]{119.501} \pm \num[round-mode=figures,round-precision=3]{29.0257}$ \\
%%         \multicolumn{1}{|c|}{R (0-97\%)} & $\num[round-mode=figures,round-precision=3]{581.92} \pm \num[round-mode=figures,round-precision=3]{63.5985}$ & $\num[round-mode=figures,round-precision=3]{3883.4} \pm \num[round-mode=figures,round-precision=3]{165.642}$ & $\num[round-mode=figures,round-precision=3]{31766.1} \pm \num[round-mode=figures,round-precision=3]{473.129}$ & $\num[round-mode=figures,round-precision=3]{30212.2} \pm \num[round-mode=figures,round-precision=3]{460.22}$ & $\num[round-mode=figures,round-precision=3]{13157.3} \pm \num[round-mode=figures,round-precision=3]{304.528}$ \\
%%         \hline
%%       \end{tabular}
%%       \caption{1.5~GeV $a$ signal yield (assuming $\tan\beta=1$ and BR$(H\to Za)=100\%$) as estimated in MC simulation.}
%%       \label{tab:sigcona01p5}
%%   \end{center}}
%% \end{table}

%% \begin{table}[!htbp]{\footnotesize\renewcommand{\arraystretch}{1.2}
%%     \begin{center}
%%       \begin{tabular}{c|c|c|c|c|c|}
%%         \cline{2-6}
%%         & \multicolumn{5}{|c|}{$m_{\ell\ell j}$ Range}\\
%%         \hline
%%         \multicolumn{1}{|c|}{MLP Range} & $100-110$~GeV & $110-120$~GeV & $120-135$~GeV & $135-155$~GeV & $155-175$~GeV \\
%%         \hline
%%         \multicolumn{1}{|c|}{SR (99-100\%)} & $\num[round-mode=figures,round-precision=3]{40.4889} \pm \num[round-mode=figures,round-precision=3]{16.5824}$ & $\num[round-mode=figures,round-precision=3]{1215.28} \pm \num[round-mode=figures,round-precision=3]{92.2023}$ & $\num[round-mode=figures,round-precision=3]{15806.2} \pm \num[round-mode=figures,round-precision=3]{335.058}$ & $\num[round-mode=figures,round-precision=3]{3405.01} \pm \num[round-mode=figures,round-precision=3]{155.433}$ & $\num[round-mode=figures,round-precision=3]{114.75} \pm \num[round-mode=figures,round-precision=3]{29.1768}$ \\
%%         \multicolumn{1}{|c|}{VR (98-99\%)} & $\num[round-mode=figures,round-precision=3]{6.22424} \pm \num[round-mode=figures,round-precision=3]{6.22424}$ & $\num[round-mode=figures,round-precision=3]{381.156} \pm \num[round-mode=figures,round-precision=3]{51.5237}$ & $\num[round-mode=figures,round-precision=3]{5925.79} \pm \num[round-mode=figures,round-precision=3]{205.223}$ & $\num[round-mode=figures,round-precision=3]{2217.02} \pm \num[round-mode=figures,round-precision=3]{125.349}$ & $\num[round-mode=figures,round-precision=3]{91.0803} \pm \num[round-mode=figures,round-precision=3]{25.8574}$ \\
%%         \multicolumn{1}{|c|}{VR (97-98\%)} & $\num[round-mode=figures,round-precision=3]{7.72296} \pm \num[round-mode=figures,round-precision=3]{7.72296}$ & $\num[round-mode=figures,round-precision=3]{205.278} \pm \num[round-mode=figures,round-precision=3]{37.6388}$ & $\num[round-mode=figures,round-precision=3]{3804.67} \pm \num[round-mode=figures,round-precision=3]{164.602}$ & $\num[round-mode=figures,round-precision=3]{1661.37} \pm \num[round-mode=figures,round-precision=3]{108.261}$ & $\num[round-mode=figures,round-precision=3]{74.5553} \pm \num[round-mode=figures,round-precision=3]{22.1803}$ \\
%%         \multicolumn{1}{|c|}{R (0-97\%)} & $\num[round-mode=figures,round-precision=3]{542.968} \pm \num[round-mode=figures,round-precision=3]{62.1928}$ & $\num[round-mode=figures,round-precision=3]{4203.12} \pm \num[round-mode=figures,round-precision=3]{171.925}$ & $\num[round-mode=figures,round-precision=3]{36543.6} \pm \num[round-mode=figures,round-precision=3]{507.726}$ & $\num[round-mode=figures,round-precision=3]{32780.2} \pm \num[round-mode=figures,round-precision=3]{479.729}$ & $\num[round-mode=figures,round-precision=3]{12845.6} \pm \num[round-mode=figures,round-precision=3]{299.885}$ \\
%%         \hline
%%       \end{tabular}
%%       \caption{2~GeV $a$ signal yield (assuming $\tan\beta=1$ and BR$(H\to Za)=100\%$) as estimated in MC simulation.}
%%       \label{tab:sigcona02p0}
%%   \end{center}}
%% \end{table}

%% \begin{table}[!htbp]{\footnotesize\renewcommand{\arraystretch}{1.2}
%%     \begin{center}
%%       \begin{tabular}{c|c|c|c|c|c|}
%%         \cline{2-6}
%%         & \multicolumn{5}{|c|}{$m_{\ell\ell j}$ Range}\\
%%         \hline
%%         \multicolumn{1}{|c|}{MLP Range} & $100-110$~GeV & $110-120$~GeV & $120-135$~GeV & $135-155$~GeV & $155-175$~GeV \\
%%         \hline
%%         \multicolumn{1}{|c|}{SR (99-100\%)} & $\num[round-mode=figures,round-precision=3]{0} \pm \num[round-mode=figures,round-precision=3]{0}$ & $\num[round-mode=figures,round-precision=3]{913.602} \pm \num[round-mode=figures,round-precision=3]{79.7074}$ & $\num[round-mode=figures,round-precision=3]{10374.6} \pm \num[round-mode=figures,round-precision=3]{272.009}$ & $\num[round-mode=figures,round-precision=3]{1832.11} \pm \num[round-mode=figures,round-precision=3]{114.24}$ & $\num[round-mode=figures,round-precision=3]{99.392} \pm \num[round-mode=figures,round-precision=3]{27.4869}$ \\
%%         \multicolumn{1}{|c|}{VR (98-99\%)} & $\num[round-mode=figures,round-precision=3]{6.50949} \pm \num[round-mode=figures,round-precision=3]{6.50949}$ & $\num[round-mode=figures,round-precision=3]{334.697} \pm \num[round-mode=figures,round-precision=3]{47.9336}$ & $\num[round-mode=figures,round-precision=3]{4895.13} \pm \num[round-mode=figures,round-precision=3]{186.013}$ & $\num[round-mode=figures,round-precision=3]{1411.23} \pm \num[round-mode=figures,round-precision=3]{99.5082}$ & $\num[round-mode=figures,round-precision=3]{50.7706} \pm \num[round-mode=figures,round-precision=3]{18.6949}$ \\
%%         \multicolumn{1}{|c|}{VR (97-98\%)} & $\num[round-mode=figures,round-precision=3]{8.11353} \pm \num[round-mode=figures,round-precision=3]{6.09652}$ & $\num[round-mode=figures,round-precision=3]{235.21} \pm \num[round-mode=figures,round-precision=3]{40.5432}$ & $\num[round-mode=figures,round-precision=3]{3250.55} \pm \num[round-mode=figures,round-precision=3]{151.302}$ & $\num[round-mode=figures,round-precision=3]{1159.63} \pm \num[round-mode=figures,round-precision=3]{89.9659}$ & $\num[round-mode=figures,round-precision=3]{93.8334} \pm \num[round-mode=figures,round-precision=3]{25.914}$ \\
%%         \multicolumn{1}{|c|}{R (0-97\%)} & $\num[round-mode=figures,round-precision=3]{492.904} \pm \num[round-mode=figures,round-precision=3]{58.8138}$ & $\num[round-mode=figures,round-precision=3]{5047.35} \pm \num[round-mode=figures,round-precision=3]{188.368}$ & $\num[round-mode=figures,round-precision=3]{47372.1} \pm \num[round-mode=figures,round-precision=3]{578.575}$ & $\num[round-mode=figures,round-precision=3]{33663.5} \pm \num[round-mode=figures,round-precision=3]{486.75}$ & $\num[round-mode=figures,round-precision=3]{12630.8} \pm \num[round-mode=figures,round-precision=3]{297.278}$ \\
%%         \hline
%%       \end{tabular}
%%       \caption{2.5~GeV $a$ signal yield (assuming $\tan\beta=1$ and BR$(H\to Za)=100\%$) as estimated in MC simulation.}
%%       \label{tab:sigcona02p5}
%%   \end{center}}
%% \end{table}

%% \begin{table}[!htbp]{\footnotesize\renewcommand{\arraystretch}{1.2}
%%     \begin{center}
%%       \begin{tabular}{c|c|c|c|c|c|}
%%         \cline{2-6}
%%         & \multicolumn{5}{|c|}{$m_{\ell\ell j}$ Range}\\
%%         \hline
%%         \multicolumn{1}{|c|}{MLP Range} & $100-110$~GeV & $110-120$~GeV & $120-135$~GeV & $135-155$~GeV & $155-175$~GeV \\
%%         \hline
%%         \multicolumn{1}{|c|}{SR (99-100\%)} & $\num[round-mode=figures,round-precision=3]{14.3784} \pm \num[round-mode=figures,round-precision=3]{10.3151}$ & $\num[round-mode=figures,round-precision=3]{542.354} \pm \num[round-mode=figures,round-precision=3]{62.46}$ & $\num[round-mode=figures,round-precision=3]{5547.93} \pm \num[round-mode=figures,round-precision=3]{197.996}$ & $\num[round-mode=figures,round-precision=3]{839.408} \pm \num[round-mode=figures,round-precision=3]{76.8712}$ & $\num[round-mode=figures,round-precision=3]{57.4535} \pm \num[round-mode=figures,round-precision=3]{19.4231}$ \\
%%         \multicolumn{1}{|c|}{VR (98-99\%)} & $\num[round-mode=figures,round-precision=3]{0.289539} \pm \num[round-mode=figures,round-precision=3]{0.289539}$ & $\num[round-mode=figures,round-precision=3]{346.928} \pm \num[round-mode=figures,round-precision=3]{49.1418}$ & $\num[round-mode=figures,round-precision=3]{3230.2} \pm \num[round-mode=figures,round-precision=3]{151.37}$ & $\num[round-mode=figures,round-precision=3]{830.631} \pm \num[round-mode=figures,round-precision=3]{75.7274}$ & $\num[round-mode=figures,round-precision=3]{51.7127} \pm \num[round-mode=figures,round-precision=3]{19.1828}$ \\
%%         \multicolumn{1}{|c|}{VR (97-98\%)} & $\num[round-mode=figures,round-precision=3]{6.77369} \pm \num[round-mode=figures,round-precision=3]{6.77369}$ & $\num[round-mode=figures,round-precision=3]{151.17} \pm \num[round-mode=figures,round-precision=3]{32.4105}$ & $\num[round-mode=figures,round-precision=3]{2219.49} \pm \num[round-mode=figures,round-precision=3]{124.746}$ & $\num[round-mode=figures,round-precision=3]{665.8} \pm \num[round-mode=figures,round-precision=3]{68.0882}$ & $\num[round-mode=figures,round-precision=3]{63.1515} \pm \num[round-mode=figures,round-precision=3]{20.0698}$ \\
%%         \multicolumn{1}{|c|}{R (0-97\%)} & $\num[round-mode=figures,round-precision=3]{514.685} \pm \num[round-mode=figures,round-precision=3]{60.7961}$ & $\num[round-mode=figures,round-precision=3]{6193.9} \pm \num[round-mode=figures,round-precision=3]{208.941}$ & $\num[round-mode=figures,round-precision=3]{55719.3} \pm \num[round-mode=figures,round-precision=3]{628.293}$ & $\num[round-mode=figures,round-precision=3]{35489.1} \pm \num[round-mode=figures,round-precision=3]{500.084}$ & $\num[round-mode=figures,round-precision=3]{12691.1} \pm \num[round-mode=figures,round-precision=3]{298.777}$ \\
%%         \hline
%%       \end{tabular}
%%       \caption{3~GeV $a$ signal yield (assuming $\tan\beta=1$ and BR$(H\to Za)=100\%$) as estimated in MC simulation.}
%%       \label{tab:sigcona03p0}
%%   \end{center}}
%% \end{table}

%% \begin{table}[!htbp]{\footnotesize\renewcommand{\arraystretch}{1.2}
%%     \begin{center}
%%       \begin{tabular}{c|c|c|c|c|c|}
%%         \cline{2-6}
%%         & \multicolumn{5}{|c|}{$m_{\ell\ell j}$ Range}\\
%%         \hline
%%         \multicolumn{1}{|c|}{MLP Range} & $100-110$~GeV & $110-120$~GeV & $120-135$~GeV & $135-155$~GeV & $155-175$~GeV \\
%%         \hline
%%         \multicolumn{1}{|c|}{SR (99-100\%)} & $\num[round-mode=figures,round-precision=3]{15.732} \pm \num[round-mode=figures,round-precision=3]{11.1379}$ & $\num[round-mode=figures,round-precision=3]{484.449} \pm \num[round-mode=figures,round-precision=3]{57.8816}$ & $\num[round-mode=figures,round-precision=3]{4122.87} \pm \num[round-mode=figures,round-precision=3]{170.91}$ & $\num[round-mode=figures,round-precision=3]{769.556} \pm \num[round-mode=figures,round-precision=3]{73.1568}$ & $\num[round-mode=figures,round-precision=3]{67.7582} \pm \num[round-mode=figures,round-precision=3]{21.668}$ \\
%%         \multicolumn{1}{|c|}{VR (98-99\%)} & $\num[round-mode=figures,round-precision=3]{0} \pm \num[round-mode=figures,round-precision=3]{0}$ & $\num[round-mode=figures,round-precision=3]{180.325} \pm \num[round-mode=figures,round-precision=3]{35.5585}$ & $\num[round-mode=figures,round-precision=3]{2580.86} \pm \num[round-mode=figures,round-precision=3]{134.481}$ & $\num[round-mode=figures,round-precision=3]{703.59} \pm \num[round-mode=figures,round-precision=3]{69.8432}$ & $\num[round-mode=figures,round-precision=3]{54.7273} \pm \num[round-mode=figures,round-precision=3]{19.7307}$ \\
%%         \multicolumn{1}{|c|}{VR (97-98\%)} & $\num[round-mode=figures,round-precision=3]{6.17166} \pm \num[round-mode=figures,round-precision=3]{6.17166}$ & $\num[round-mode=figures,round-precision=3]{171.266} \pm \num[round-mode=figures,round-precision=3]{34.9779}$ & $\num[round-mode=figures,round-precision=3]{2057.89} \pm \num[round-mode=figures,round-precision=3]{120.872}$ & $\num[round-mode=figures,round-precision=3]{635.518} \pm \num[round-mode=figures,round-precision=3]{67.2921}$ & $\num[round-mode=figures,round-precision=3]{69.0999} \pm \num[round-mode=figures,round-precision=3]{22.3368}$ \\
%%         \multicolumn{1}{|c|}{R (0-97\%)} & $\num[round-mode=figures,round-precision=3]{503.974} \pm \num[round-mode=figures,round-precision=3]{58.988}$ & $\num[round-mode=figures,round-precision=3]{7011.12} \pm \num[round-mode=figures,round-precision=3]{221.342}$ & $\num[round-mode=figures,round-precision=3]{60230.1} \pm \num[round-mode=figures,round-precision=3]{652.341}$ & $\num[round-mode=figures,round-precision=3]{34419.2} \pm \num[round-mode=figures,round-precision=3]{491.419}$ & $\num[round-mode=figures,round-precision=3]{12576.7} \pm \num[round-mode=figures,round-precision=3]{296.861}$ \\
%%         \hline
%%       \end{tabular}
%%       \caption{3.5~GeV $a$ signal yield (assuming $\tan\beta=1$ and BR$(H\to Za)=100\%$) as estimated in MC simulation.}
%%       \label{tab:sigcona03p5}
%%   \end{center}}
%% \end{table}

%% \begin{table}[!htbp]{\footnotesize\renewcommand{\arraystretch}{1.2}
%%     \begin{center}
%%       \begin{tabular}{c|c|c|c|c|c|}
%%         \cline{2-6}
%%         & \multicolumn{5}{|c|}{$m_{\ell\ell j}$ Range}\\
%%         \hline
%%         \multicolumn{1}{|c|}{MLP Range} & $100-110$~GeV & $110-120$~GeV & $120-135$~GeV & $135-155$~GeV & $155-175$~GeV \\
%%         \hline
%%         \multicolumn{1}{|c|}{SR (99-100\%)} & $\num[round-mode=figures,round-precision=3]{62.4656} \pm \num[round-mode=figures,round-precision=3]{20.756}$ & $\num[round-mode=figures,round-precision=3]{547.931} \pm \num[round-mode=figures,round-precision=3]{62.2176}$ & $\num[round-mode=figures,round-precision=3]{1130.22} \pm \num[round-mode=figures,round-precision=3]{89.6202}$ & $\num[round-mode=figures,round-precision=3]{117.073} \pm \num[round-mode=figures,round-precision=3]{28.6776}$ & $\num[round-mode=figures,round-precision=3]{59.9014} \pm \num[round-mode=figures,round-precision=3]{21.2828}$ \\
%%         \multicolumn{1}{|c|}{VR (98-99\%)} & $\num[round-mode=figures,round-precision=3]{65.0847} \pm \num[round-mode=figures,round-precision=3]{20.9524}$ & $\num[round-mode=figures,round-precision=3]{329.366} \pm \num[round-mode=figures,round-precision=3]{48.4097}$ & $\num[round-mode=figures,round-precision=3]{1025.75} \pm \num[round-mode=figures,round-precision=3]{86.8166}$ & $\num[round-mode=figures,round-precision=3]{188.711} \pm \num[round-mode=figures,round-precision=3]{36.2117}$ & $\num[round-mode=figures,round-precision=3]{125.547} \pm \num[round-mode=figures,round-precision=3]{30.2452}$ \\
%%         \multicolumn{1}{|c|}{VR (97-98\%)} & $\num[round-mode=figures,round-precision=3]{25.5954} \pm \num[round-mode=figures,round-precision=3]{13.1278}$ & $\num[round-mode=figures,round-precision=3]{311.337} \pm \num[round-mode=figures,round-precision=3]{47.5351}$ & $\num[round-mode=figures,round-precision=3]{815.256} \pm \num[round-mode=figures,round-precision=3]{76.1587}$ & $\num[round-mode=figures,round-precision=3]{134.392} \pm \num[round-mode=figures,round-precision=3]{31.3819}$ & $\num[round-mode=figures,round-precision=3]{55.6009} \pm \num[round-mode=figures,round-precision=3]{18.6595}$ \\
%%         \multicolumn{1}{|c|}{R (0-97\%)} & $\num[round-mode=figures,round-precision=3]{897.551} \pm \num[round-mode=figures,round-precision=3]{79.1268}$ & $\num[round-mode=figures,round-precision=3]{11860.9} \pm \num[round-mode=figures,round-precision=3]{289.198}$ & $\num[round-mode=figures,round-precision=3]{54710.6} \pm \num[round-mode=figures,round-precision=3]{622.798}$ & $\num[round-mode=figures,round-precision=3]{26662} \pm \num[round-mode=figures,round-precision=3]{432.395}$ & $\num[round-mode=figures,round-precision=3]{13291.2} \pm \num[round-mode=figures,round-precision=3]{305.752}$ \\
%%         \hline
%%       \end{tabular}
%%       \caption{4~GeV $a$ signal yield (assuming $\tan\beta=1$ and BR$(H\to Za)=100\%$) as estimated in MC simulation.}
%%       \label{tab:sigcona04p0}
%%   \end{center}}
%% \end{table}

%% \begin{table}[!htbp]{\footnotesize\renewcommand{\arraystretch}{1.2}
%%     \begin{center}
%%       \begin{tabular}{c|c|c|c|c|c|}
%%         \cline{2-6}
%%         & \multicolumn{5}{|c|}{$m_{\ell\ell j}$ Range}\\
%%         \hline
%%         \multicolumn{1}{|c|}{MLP Range} & $100-110$~GeV & $110-120$~GeV & $120-135$~GeV & $135-155$~GeV & $155-175$~GeV \\
%%         \hline
%%         \multicolumn{1}{|c|}{SR (99-100\%)} & $\num[round-mode=figures,round-precision=3]{12.0561} \pm \num[round-mode=figures,round-precision=3]{8.56829}$ & $\num[round-mode=figures,round-precision=3]{183.906} \pm \num[round-mode=figures,round-precision=3]{36.0908}$ & $\num[round-mode=figures,round-precision=3]{272.346} \pm \num[round-mode=figures,round-precision=3]{43.2661}$ & $\num[round-mode=figures,round-precision=3]{78.4231} \pm \num[round-mode=figures,round-precision=3]{23.3355}$ & $\num[round-mode=figures,round-precision=3]{62.3083} \pm \num[round-mode=figures,round-precision=3]{21.2115}$ \\
%%         \multicolumn{1}{|c|}{VR (98-99\%)} & $\num[round-mode=figures,round-precision=3]{18.694} \pm \num[round-mode=figures,round-precision=3]{11.3202}$ & $\num[round-mode=figures,round-precision=3]{233.845} \pm \num[round-mode=figures,round-precision=3]{41.7172}$ & $\num[round-mode=figures,round-precision=3]{338.574} \pm \num[round-mode=figures,round-precision=3]{48.6583}$ & $\num[round-mode=figures,round-precision=3]{147.826} \pm \num[round-mode=figures,round-precision=3]{32.2506}$ & $\num[round-mode=figures,round-precision=3]{20.9438} \pm \num[round-mode=figures,round-precision=3]{12.3392}$ \\
%%         \multicolumn{1}{|c|}{VR (97-98\%)} & $\num[round-mode=figures,round-precision=3]{23.0884} \pm \num[round-mode=figures,round-precision=3]{13.4592}$ & $\num[round-mode=figures,round-precision=3]{206.859} \pm \num[round-mode=figures,round-precision=3]{39.0356}$ & $\num[round-mode=figures,round-precision=3]{323.694} \pm \num[round-mode=figures,round-precision=3]{48.4359}$ & $\num[round-mode=figures,round-precision=3]{96.2491} \pm \num[round-mode=figures,round-precision=3]{25.7203}$ & $\num[round-mode=figures,round-precision=3]{65.1694} \pm \num[round-mode=figures,round-precision=3]{20.7762}$ \\
%%         \multicolumn{1}{|c|}{R (0-97\%)} & $\num[round-mode=figures,round-precision=3]{1162.11} \pm \num[round-mode=figures,round-precision=3]{90.0934}$ & $\num[round-mode=figures,round-precision=3]{13789} \pm \num[round-mode=figures,round-precision=3]{310.985}$ & $\num[round-mode=figures,round-precision=3]{43967.1} \pm \num[round-mode=figures,round-precision=3]{556.931}$ & $\num[round-mode=figures,round-precision=3]{21354.5} \pm \num[round-mode=figures,round-precision=3]{385.795}$ & $\num[round-mode=figures,round-precision=3]{12991.4} \pm \num[round-mode=figures,round-precision=3]{302.717}$ \\
%%         \hline
%%       \end{tabular}
%%       \caption{8~GeV $a$ signal yield (assuming $\tan\beta=1$ and BR$(H\to Za)=100\%$) as estimated in MC simulation.}
%%       \label{tab:sigcona08p0}
%%   \end{center}}
%% \end{table}

%% \begin{table}[!htbp]{\footnotesize\renewcommand{\arraystretch}{1.2}
%%     \begin{center}
%%       \begin{tabular}{c|c|c|c|c|c|}
%%         \cline{2-6}
%%         & \multicolumn{5}{|c|}{$m_{\ell\ell j}$ Range}\\
%%         \hline
%%         \multicolumn{1}{|c|}{MLP Range} & $100-110$~GeV & $110-120$~GeV & $120-135$~GeV & $135-155$~GeV & $155-175$~GeV \\
%%         \hline
%%         \multicolumn{1}{|c|}{SR (99-100\%)} & $\num[round-mode=figures,round-precision=3]{13.3775} \pm \num[round-mode=figures,round-precision=3]{9.48292}$ & $\num[round-mode=figures,round-precision=3]{424.787} \pm \num[round-mode=figures,round-precision=3]{53.9355}$ & $\num[round-mode=figures,round-precision=3]{4363.12} \pm \num[round-mode=figures,round-precision=3]{175.895}$ & $\num[round-mode=figures,round-precision=3]{749.357} \pm \num[round-mode=figures,round-precision=3]{72.6877}$ & $\num[round-mode=figures,round-precision=3]{96.3125} \pm \num[round-mode=figures,round-precision=3]{25.8453}$ \\
%%         \multicolumn{1}{|c|}{VR (98-99\%)} & $\num[round-mode=figures,round-precision=3]{12.8594} \pm \num[round-mode=figures,round-precision=3]{9.0931}$ & $\num[round-mode=figures,round-precision=3]{286.295} \pm \num[round-mode=figures,round-precision=3]{44.4674}$ & $\num[round-mode=figures,round-precision=3]{2657.33} \pm \num[round-mode=figures,round-precision=3]{136.719}$ & $\num[round-mode=figures,round-precision=3]{745.833} \pm \num[round-mode=figures,round-precision=3]{72.496}$ & $\num[round-mode=figures,round-precision=3]{72.9983} \pm \num[round-mode=figures,round-precision=3]{21.962}$ \\
%%         \multicolumn{1}{|c|}{VR (97-98\%)} & $\num[round-mode=figures,round-precision=3]{0} \pm \num[round-mode=figures,round-precision=3]{0}$ & $\num[round-mode=figures,round-precision=3]{111.883} \pm \num[round-mode=figures,round-precision=3]{27.5553}$ & $\num[round-mode=figures,round-precision=3]{2085.32} \pm \num[round-mode=figures,round-precision=3]{122.026}$ & $\num[round-mode=figures,round-precision=3]{483.449} \pm \num[round-mode=figures,round-precision=3]{58.6145}$ & $\num[round-mode=figures,round-precision=3]{79.5802} \pm \num[round-mode=figures,round-precision=3]{23.5173}$ \\
%%         \multicolumn{1}{|c|}{R (0-97\%)} & $\num[round-mode=figures,round-precision=3]{516.77} \pm \num[round-mode=figures,round-precision=3]{59.8764}$ & $\num[round-mode=figures,round-precision=3]{7017.69} \pm \num[round-mode=figures,round-precision=3]{221.639}$ & $\num[round-mode=figures,round-precision=3]{62531.3} \pm \num[round-mode=figures,round-precision=3]{664.551}$ & $\num[round-mode=figures,round-precision=3]{35041.6} \pm \num[round-mode=figures,round-precision=3]{496.077}$ & $\num[round-mode=figures,round-precision=3]{12036.8} \pm \num[round-mode=figures,round-precision=3]{290.505}$ \\
%%         \hline
%%       \end{tabular}
%%       \caption{$\eta_c$ signal yield (BR$(H\to Z\eta_c)=100\%$) as estimated in MC simulation.}
%%       \label{tab:sigconetac}
%%   \end{center}}
%% \end{table}

%% \begin{table}[!htbp]{\footnotesize\renewcommand{\arraystretch}{1.2}
%%     \begin{center}
%%       \begin{tabular}{c|c|c|c|c|c|}
%%         \cline{2-6}
%%         & \multicolumn{5}{|c|}{$m_{\ell\ell j}$ Range}\\
%%         \hline
%%         \multicolumn{1}{|c|}{MLP Range} & $100-110$~GeV & $110-120$~GeV & $120-135$~GeV & $135-155$~GeV & $155-175$~GeV \\
%%         \hline
%%         \multicolumn{1}{|c|}{SR (99-100\%)} & $\num[round-mode=figures,round-precision=3]{6.20365} \pm \num[round-mode=figures,round-precision=3]{6.20365}$ & $\num[round-mode=figures,round-precision=3]{376.304} \pm \num[round-mode=figures,round-precision=3]{51.0312}$ & $\num[round-mode=figures,round-precision=3]{4526.12} \pm \num[round-mode=figures,round-precision=3]{178.135}$ & $\num[round-mode=figures,round-precision=3]{1028.59} \pm \num[round-mode=figures,round-precision=3]{85.4183}$ & $\num[round-mode=figures,round-precision=3]{117.749} \pm \num[round-mode=figures,round-precision=3]{28.465}$ \\
%%         \multicolumn{1}{|c|}{VR (98-99\%)} & $\num[round-mode=figures,round-precision=3]{5.51057} \pm \num[round-mode=figures,round-precision=3]{5.51057}$ & $\num[round-mode=figures,round-precision=3]{140.006} \pm \num[round-mode=figures,round-precision=3]{32.442}$ & $\num[round-mode=figures,round-precision=3]{2380.1} \pm \num[round-mode=figures,round-precision=3]{128.734}$ & $\num[round-mode=figures,round-precision=3]{869.496} \pm \num[round-mode=figures,round-precision=3]{78.0622}$ & $\num[round-mode=figures,round-precision=3]{66.1156} \pm \num[round-mode=figures,round-precision=3]{21.3905}$ \\
%%         \multicolumn{1}{|c|}{VR (97-98\%)} & $\num[round-mode=figures,round-precision=3]{5.57536} \pm \num[round-mode=figures,round-precision=3]{5.57536}$ & $\num[round-mode=figures,round-precision=3]{170.644} \pm \num[round-mode=figures,round-precision=3]{34.3566}$ & $\num[round-mode=figures,round-precision=3]{1973.28} \pm \num[round-mode=figures,round-precision=3]{118.428}$ & $\num[round-mode=figures,round-precision=3]{667.271} \pm \num[round-mode=figures,round-precision=3]{68.9166}$ & $\num[round-mode=figures,round-precision=3]{64.1711} \pm \num[round-mode=figures,round-precision=3]{20.616}$ \\
%%         \multicolumn{1}{|c|}{R (0-97\%)} & $\num[round-mode=figures,round-precision=3]{552.815} \pm \num[round-mode=figures,round-precision=3]{62.5991}$ & $\num[round-mode=figures,round-precision=3]{6367.54} \pm \num[round-mode=figures,round-precision=3]{212.003}$ & $\num[round-mode=figures,round-precision=3]{56406.5} \pm \num[round-mode=figures,round-precision=3]{631.707}$ & $\num[round-mode=figures,round-precision=3]{34751.5} \pm \num[round-mode=figures,round-precision=3]{495.047}$ & $\num[round-mode=figures,round-precision=3]{13288.6} \pm \num[round-mode=figures,round-precision=3]{304.771}$ \\
%%         \hline
%%       \end{tabular}
%%       \caption{$J/\psi$ signal yield (BR$(H\to ZJ/\psi)=100\%$) as estimated in MC simulation.}
%%       \label{tab:sigconjpsi}
%%   \end{center}}
%% \end{table}


%-------------------------------------------------------------------------------
\clearpage
\section{MLP Pileup Dependence}
\label{app:mlppileup}
%-------------------------------------------------------------------------------

Figures \ref{fig:pileup0to25}, \ref{fig:pileup25to40} and \ref{fig:pileup40toINF} show the MLP distributions for events with 0-25, 25-40 and $>40$ interactions per bunch crossing, respectively.

\begin{figure}[!htbp]
  \centering
  \subfigure[Classification MLP]{\includegraphics[width=0.475\textwidth]{figures/PileupDependance/pileup0to25/MLP_regInID.png}}
  \subfigure[Classification MLP (Zoomed)]{\includegraphics[width=0.475\textwidth]{figures/PileupDependance/pileup0to25/MLP_regInID_zoomed.png}}\\
  \subfigure[Regression MLP]{\includegraphics[width=0.475\textwidth]{figures/PileupDependance/pileup0to25/MLP_massRegression.png}}
  \caption{Classification MLP with a nominal (a) and zoomed (b) axis range, and regression MLP (c), for events with 0-25 interactions per bunch crossing.}
  \label{fig:pileup0to25}
\end{figure}

\begin{figure}[!htbp]
  \centering
  \subfigure[Classification MLP]{\includegraphics[width=0.475\textwidth]{figures/PileupDependance/pileup25to40/MLP_regInID.png}}
  \subfigure[Classification MLP (Zoomed)]{\includegraphics[width=0.475\textwidth]{figures/PileupDependance/pileup25to40/MLP_regInID_zoomed.png}}\\
  \subfigure[Regression MLP]{\includegraphics[width=0.475\textwidth]{figures/PileupDependance/pileup25to40/MLP_massRegression.png}}
  \caption{Classification MLP with a nominal (a) and zoomed (b) axis range, and regression MLP (c), for events with 25-40 interactions per bunch crossing.}
  \label{fig:pileup25to40}
\end{figure}

\begin{figure}[!htbp]
  \centering
  \subfigure[Classification MLP]{\includegraphics[width=0.475\textwidth]{figures/PileupDependance/pileup40toINF/MLP_regInID.png}}
  \subfigure[Classification MLP (Zoomed)]{\includegraphics[width=0.475\textwidth]{figures/PileupDependance/pileup40toINF/MLP_regInID_zoomed.png}}\\
  \subfigure[Regression MLP]{\includegraphics[width=0.475\textwidth]{figures/PileupDependance/pileup40toINF/MLP_massRegression.png}}
  \caption{Classification MLP with a nominal (a) and zoomed (b) axis range, and regression MLP (c), for events with $>40$ interactions per bunch crossing.}
  \label{fig:pileup40toINF}
\end{figure}


%-------------------------------------------------------------------------------
\clearpage
\section{\textsc{MadGraph} Reweighting Studies}
\label{app:MGRW}
%-------------------------------------------------------------------------------

The \textsc{MadGraph} $Z+\text{jets}$ sample has been reweighted using a procedure designed to mitigate the observed mismodelling in the \textsc{MadGraph} $Z+\text{jets}$ sample, using the $p_\text{T}$ of the calorimeter jet, the $p_\text{T}$ of the three body system, and the multiplicity of tracks Ghost-Associated to the calorimeter jet. Various distributions before and after this reweighting is applied are shown in Figures~\ref{fig:mgrwvars} to \ref{fig:mgrwmlpoutputs}.

\begin{figure}[!htbp]
  \centering
  \subfigure[$n_\text{tracks}$ Pre-Reweighting]{\includegraphics[width=0.375\textwidth]{figures/MGvsData_Blinded_NoRW_08Aug19/GhostTracks_nTracks.png}}
  \subfigure[$n_\text{tracks}$ Post-Reweighting]{\includegraphics[width=0.375\textwidth]{figures/MGvsData_Blinded_DedicatedMGRW_16Aug19/GhostTracks_nTracks.png}}\\
  \subfigure[$p_\text{T}^\text{jet}$ Pre-Reweighting]{\includegraphics[width=0.375\textwidth]{figures/MGvsData_Blinded_NoRW_08Aug19/CaloJetPt.png}}
  \subfigure[$p_\text{T}^\text{jet}$ Post-Reweighting]{\includegraphics[width=0.375\textwidth]{figures/MGvsData_Blinded_DedicatedMGRW_16Aug19/CaloJetPt.png}}\\
  \subfigure[$p_\text{T}^\text{jet}$ Pre-Reweighting]{\includegraphics[width=0.375\textwidth]{figures/MGvsData_Blinded_NoRW_08Aug19/HPt.png}}
  \subfigure[$p_\text{T}^\text{jet}$ Post-Reweighting]{\includegraphics[width=0.375\textwidth]{figures/MGvsData_Blinded_DedicatedMGRW_16Aug19/HPt.png}}
  \caption{Distributions of the three variables used to reweight the \textsc{MadGraph} $Z+\text{jets}$ background simulation, after the full event-level pre-selection, in data and background MC (both reweighted and not). These variables are the ghost-associated track multiplicity (a and b), the transverse momentum of the calorimeter jet (c and d), and the transverse momentum of the three body system (e and f).}
  \label{fig:mgrwvars}
\end{figure}

\begin{figure}[!htbp]
  \centering
  \subfigure[$m_{\ell^+\ell^-\text{j}}$ Pre-Reweighting]{\includegraphics[width=0.475\textwidth]{figures/MGvsData_Blinded_NoRW_08Aug19/HM_SR.png}}
  \subfigure[$m_{\ell^+\ell^-\text{j}}$ Post-Reweighting]{\includegraphics[width=0.475\textwidth]{figures/MGvsData_Blinded_DedicatedMGRW_16Aug19/HM_SR.png}}
  \caption{Distributions of the three body mass distribution, after the full event-level pre-selection, in data and \textsc{MadGraph} $Z+\text{jets}$ background MC (both reweighted and not).}
  \label{fig:mgrw3bm}
\end{figure}

\begin{figure}[!htbp]
  \centering
  \subfigure[$p_\text{T}^{\text{lead\ track}}/p_\text{T}^{\text{tracks}}$ Pre-Reweighting]{\includegraphics[width=0.375\textwidth]{figures/MGvsData_Blinded_NoRW_08Aug19/GhostTracks_leadTrackPtRatio.png}}
  \subfigure[$p_\text{T}^{\text{lead\ track}}/p_\text{T}^{\text{tracks}}$ Post-Reweighting]{\includegraphics[width=0.375\textwidth]{figures/MGvsData_Blinded_DedicatedMGRW_16Aug19/GhostTracks_leadTrackPtRatio.png}}\\
  \subfigure[$\Delta R^\text{lead\ track,\ calo\ jet}$ Pre-Reweighting]{\includegraphics[width=0.375\textwidth]{figures/MGvsData_Blinded_NoRW_08Aug19/GhostTracks_deltaRLeadTrack.png}}
  \subfigure[$\Delta R^\text{lead\ track,\ calo\ jet}$ Post-Reweighting]{\includegraphics[width=0.375\textwidth]{figures/MGvsData_Blinded_DedicatedMGRW_16Aug19/GhostTracks_deltaRLeadTrack.png}}\\
  \caption{Distributions of the variables input to the MLP, after the full event-level pre-selection, in data and \textsc{MadGraph} $Z+\text{jets}$ background MC (both reweighted and not). In the case of exactly two tracks, the $p_\text{T}^{\text{lead\ track}}/p_\text{T}^{\text{tracks}}$ variable can not take values less than 0.5, leading to the spike around 0.5.}
  \label{fig:mgrwmlpinputs1}
\end{figure}

\begin{figure}[!htbp]
  \centering
  \subfigure[$p_\text{T}^{\text{lead\ track}}/p_\text{T}^{\text{tracks}}$ Pre-Reweighting]{\includegraphics[width=0.375\textwidth]{figures/MGvsData_Blinded_NoRW_08Aug19/GhostTracks_leadTrackPtRatio.png}}
  \subfigure[$p_\text{T}^{\text{lead\ track}}/p_\text{T}^{\text{tracks}}$ Post-Reweighting]{\includegraphics[width=0.375\textwidth]{figures/MGvsData_Blinded_DedicatedMGRW_16Aug19/GhostTracks_leadTrackPtRatio.png}}\\
  \subfigure[$\Delta R^\text{lead\ track,\ calo\ jet}$ Pre-Reweighting]{\includegraphics[width=0.375\textwidth]{figures/MGvsData_Blinded_NoRW_08Aug19/GhostTracks_deltaRLeadTrack.png}}
  \subfigure[$\Delta R^\text{lead\ track,\ calo\ jet}$ Post-Reweighting]{\includegraphics[width=0.375\textwidth]{figures/MGvsData_Blinded_DedicatedMGRW_16Aug19/GhostTracks_deltaRLeadTrack.png}}\\
  \caption{Distributions of the variables input to the MLP, after the full event-level pre-selection, in data and \textsc{MadGraph} $Z+\text{jets}$ background MC (both reweighted and not). In the case of exactly two tracks, the $p_\text{T}^{\text{lead\ track}}/p_\text{T}^{\text{tracks}}$ variable can not take values less than 0.5, leading to the spike around 0.5.}
  \label{fig:mgrwmlpinputs2}
\end{figure}

\begin{figure}[!htbp]
  \centering
  \subfigure[$U_1(0.7)$ Pre-Reweighting]{\includegraphics[width=0.375\textwidth]{figures/MGvsData_Blinded_NoRW_08Aug19/GhostTracks_U1_0p7.png}}
  \subfigure[$U_1(0.7)$ Post-Reweighting]{\includegraphics[width=0.375\textwidth]{figures/MGvsData_Blinded_DedicatedMGRW_16Aug19/GhostTracks_U1_0p7.png}}\\
  \subfigure[$M_2(0.3)$ Pre-Reweighting]{\includegraphics[width=0.375\textwidth]{figures/MGvsData_Blinded_NoRW_08Aug19/GhostTracks_M2_0p3.png}}
  \subfigure[$M_2(0.3)$ Post-Reweighting]{\includegraphics[width=0.375\textwidth]{figures/MGvsData_Blinded_DedicatedMGRW_16Aug19/GhostTracks_M2_0p3.png}}
  \caption{Distributions of the variables input to the MLP, after the full event-level pre-selection, in data and \textsc{MadGraph} $Z+\text{jets}$ background MC (both reweighted and not).}
  \label{fig:mgrwmlpinputs3}
\end{figure}

\begin{figure}[!htbp]
  \centering
  \subfigure[Mass Regression MLP Output Variable Pre-Reweighting]{\includegraphics[width=0.475\textwidth]{figures/MGvsData_Blinded_NoRW_08Aug19/MLP_massRegression.png}}
  \subfigure[Mass Regression MLP Output Variable Post-Reweighting]{\includegraphics[width=0.475\textwidth]{figures/MGvsData_Blinded_DedicatedMGRW_16Aug19/MLP_massRegression.png}}\\
  \subfigure[Classification MLP Output Variable Pre-Reweighting]{\includegraphics[width=0.475\textwidth]{figures/MGvsData_Blinded_NoRW_08Aug19/MLP_regInID_zoomed.png}}
  \subfigure[Classification MLP Output Variable Post-Reweighting]{\includegraphics[width=0.475\textwidth]{figures/MGvsData_Blinded_DedicatedMGRW_16Aug19/MLP_regInID_zoomed.png}}
  \caption{Distributions of the output of the regression (a) and classification (b) MLPs, after the full event-level pre-selection, in data and \textsc{MadGraph} $Z+\text{jets}$ background MC (both reweighted and not).}
  \label{fig:mgrwmlpoutputs}
\end{figure}


%-------------------------------------------------------------------------------
\clearpage
\section{MLP Inputs in SR}
\label{app:mlpinputsinsr}
%-------------------------------------------------------------------------------

The MLP inputs in the SR are given in Figure~\ref{fig:mlpinputsinsr}.

\begin{figure}[!htbp]
  \centering
  \subfigure[$p_\text{T}^{\text{lead\ track}}/p_\text{T}^{\text{tracks}}$]{\includegraphics[width=0.375\textwidth]{figures/SR1Plots_13Oct19/GhostTracks_leadTrackPtRatio.png}}
  \subfigure[$\Delta R^\text{lead\ track,\ calo\ jet}$]{\includegraphics[width=0.375\textwidth]{figures/SR1Plots_13Oct19/GhostTracks_deltaRLeadTrack.png}}\\
  \subfigure[$U_1(0.7)$]{\includegraphics[width=0.375\textwidth]{figures/SR1Plots_13Oct19/GhostTracks_U1_0p7.png}}
  \subfigure[$M_2(0.3)$]{\includegraphics[width=0.375\textwidth]{figures/SR1Plots_13Oct19/GhostTracks_M2_0p3.png}}\\
  \subfigure[$\tau_{\text{2}}$]{\includegraphics[width=0.375\textwidth]{figures/SR1Plots_13Oct19/GhostTracks_tau2.png}}
  \subfigure[$angularity(2)$]{\includegraphics[width=0.375\textwidth]{figures/SR1Plots_13Oct19/GhostTracks_angularity_2.png}}\\
  \caption{Distributions of the variables input to the MLP, after the full selection, in data and background MC (reweighted).}
  \label{fig:mlpinputsinsr}
\end{figure}


%-------------------------------------------------------------------------------
\clearpage
\section{Full Selection Cutflow}
\label{app:cutflow}
%-------------------------------------------------------------------------------

A summary of the full event selection, with an example cutflow is given in Figure~\ref{tab:selectioncutflow}.

\begin{table}[!htbp]{\footnotesize\renewcommand{\arraystretch}{1.2}
    \begin{center}
      \begin{tabular}{|c|c|c|}
        \hline
        Cut & Details & 1.5 GeV $a$ Cutflow \\
        \hline
        FTAG2 & Event must enter \textsc{FTAG2} derivation & 44.0\% \\ % 0.44006176
        Triggers & Single lepton triggers requiring $p_\text{T}>27\ \text{GeV}$ & 39.2\% \\ % 0.39238824
        Event Cleaning & Event Cleaning & 39.1\% \\ % 0.39050000
        Leptons & $e\text{ or }\mu\geq 2$ & 32.2\% \\ % 0.32231176
        $Z$ boson & 2 same-flavour opposite-sign leptons, with $|m^{ll}-m_Z|<10\ \text{GeV}$ and $p_\text{T}^{lead}>27\ \text{GeV}$ & 27.9\% \\ % 0.27855294
        Trigger Matching & One of the $Z$~boson leptons must have triggered the event & 27.8\% \\ % 0.27848529
        \hline
        \multicolumn{2}{|c|}{Select $X$-candidate as anti-$k_T$ 4 topo EM jet $(p_\text{T}^{jet}>20\ \text{GeV})$, with highest $p_\text{T}$, for which $m^{llj}<250\ \text{GeV}$} & 21.4\% \\ % 0.21365294
        \hline
        FTAG2 Harmonisation Cuts & Lepton $p_\text{T}>18$ GeV & 21.3\% \\ % 0.212529
        $>2$ tracks & $\geq 2$ tracks ghost associated to the calo jet, surviving track selection & 20.3\% \\ % 0.203312
        Higgs boson & $120\ \text{GeV}<m^{llj}<135\ \text{GeV}$ & 8.34\% \\ % 0.0834471
        MLP & $MLP>0.0524$ & 2.62\% \\ % 0.0262441
        \hline
      \end{tabular}
      \caption{Summary of full event selection. The cutflow for the 1.5 GeV $a$ signal sample is shown for the unweighted MC, relative to the initial number of events in the \textsc{FTAG2} DxAOD.}
      \label{tab:selectioncutflow}
  \end{center}}
\end{table}



\end{document}


%% %-------------------------------------------------------------------------------
%% \section{Signal and Background Modelling}
%% \label{sec:models}
%% %-------------------------------------------------------------------------------

%% Fits to one dimensional signal and background models of the three body mass, and its distribution in data, are used to extract the final result; providing strong signal to background discrimination, and an easily visually interpret-able result. Other modelling strategies have been considered, and are described in Appendices~\ref{app:kdemodelling} and \ref{app:2dmodelling}.

%% \subsection{Signal Model}
%% \label{sec:signalmodel}

%% The signal is modelled as a Crystal Ball function, with a tail on the high side. The tail is due to the presence of events with incorrectly matched jets. This can be seen from a comparison of the size of the tail for samples with tight (0.5 and 2.5~GeV $a$) and loose (8~GeV $a$ and $\eta_c$) MLP cuts, as the tight cut removes events with mis-identified resonance candidates, and thus the high side tail. The fits used to derive the parameters for each signal sample are shown in Figure~\ref{fig:1dsignalmodels}, while the extracted parameters are listed in Table~\ref{tab:1dsignalfitpars}.

%% \begin{figure}[!htbp]
%%   \centering
%%   \subfigure[0.5~GeV $a$]{\includegraphics[width=0.475\textwidth]{figures/FitPlots_30Nov18/Signal/HZX_a00p5.png}}
%%   \subfigure[2.5~GeV $a$]{\includegraphics[width=0.475\textwidth]{figures/FitPlots_30Nov18/Signal/HZX_a02p5.png}}\\
%%   \subfigure[8~GeV $a$]{\includegraphics[width=0.475\textwidth]{figures/FitPlots_30Nov18/Signal/HZX_a08p0.png}}
%%   \subfigure[$\eta_c$]{\includegraphics[width=0.475\textwidth]{figures/FitPlots_30Nov18/Signal/HZX_etac.png}}
%%   \caption{One dimensional three body mass distributions for the signal, after the full selection, including the MLPs trained on the displayed signal samples. Each is fit with a Crystal Ball function.}
%%   \label{fig:1dsignalmodels}
%% \end{figure}

%% \begin{table}[!htbp]{\footnotesize\renewcommand{\arraystretch}{1.2}
%%     \begin{center}
%%       \begin{tabular}{|c|cccc|}
%%      \hline
%%      Signal Hypothesis & Mean /~GeV & Sigma /~GeV & Alpha & Slope \\
%%      \hline
%%         0.5~GeV $a$ & 130 & 5.78 & -2.09 & 2.34 \\
%%         2.5~GeV $a$ & 131 & 5.81 & -1.51 & 2.02 \\
%%         8~GeV $a$ & 125 & 7.33 & -1.10 & 4.38 \\
%%         $\eta_c$ & 130 & 6.67 & -1.42 & 0.564 \\
%%      \hline
%%       \end{tabular}
%%   \end{center}}
%%   \caption{Crystal ball fit parameters for the one dimensional signal fits.}
%%   \label{tab:1dsignalfitpars}
%% \end{table}


%% \subsection{Background Model}
%% \label{sec:1dbackgroundmodel}

%% The two dimensional background model is a Landau function, multiplied by two 'scaling functions':

%% \[
%% f_s^i(m) = \left\{
%% \begin{array}{lr}
%%   1 & \text{for}\ m < c^i_1\\
%%   1+c^i_2(m/c^i_1-1)^{1.5} & \text{for}\ m \ge c^i_1
%% \end{array}
%% \right.
%% ,
%% \]

%% %% $f^i_s(x; c^i_1, c^i_2)) = x<c^i_1\ ?\ 1\ :\ 1+c^i_2(x/c^i_1-1)^{1.5}$, with $i=1,2$.

%% for $i=1,2$. It is further required that: $c_1^1<c_1^2$, $c_2^1>0$ and $c_2^2<0$. These functions scale the tail of the landau distribution such that it has the flexibility to model the binned MC background data, while ensuring a continuous background model with a continuous first derivative.The exponent of $1.5$ was chosen because lower values led to very high first derivative (which would have caused difficulties in fit convergence), while greater value did not fit the MC data.

%% The fits used to derive the parameters for the combined background after the application of each MLP are shown in Figure~\ref{fig:1dbackgroundmodels}, while the extracted parameters are listed in Table~\ref{tab:1dbackgroundfitpars}. Appendix~\ref{app:6orderpolybkgdmodel} demonstrates that this background model succeeds for reasons beyond the six free parameters it is afforded.

%% \begin{figure}[!htbp]
%%   \centering
%%   \subfigure[0.5~GeV $a$]{\includegraphics[width=0.475\textwidth]{figures/FitPlots_30Nov18/Bkgd/a00p5.png}}
%%   \subfigure[2.5~GeV $a$]{\includegraphics[width=0.475\textwidth]{figures/FitPlots_30Nov18/Bkgd/a02p5.png}}\\
%%   \subfigure[8~GeV $a$]{\includegraphics[width=0.475\textwidth]{figures/FitPlots_30Nov18/Bkgd/a08p0.png}}
%%   \subfigure[$\eta_c$]{\includegraphics[width=0.475\textwidth]{figures/FitPlots_30Nov18/Bkgd/etac.png}}
%%   \caption{One dimensional three body mass distributions for the total background, after the full selection, including the MLPs trained on the various signal samples. Each is fit with the modified Landau function described in subsection~\ref{sec:1dbackgroundmodel}.}
%%   \label{fig:1dbackgroundmodels}
%% \end{figure}

%% \begin{table}[!htbp]{\footnotesize\renewcommand{\arraystretch}{1.2}
%%     \begin{center}
%%       \begin{tabular}{|c|cccccc|}
%%      \hline
%%      MLP & Mean /~GeV & Sigma /~GeV & $\uparrow$ Scale Begin /~GeV & $\uparrow$ Scale Coef. & $\downarrow$ Scale Begin /~GeV & $\downarrow$ Scale Coef. \\
%%      \hline
%%         0.5~GeV $a$ & 118 & 6.58 & 112 & 135 & 164 & 1.73 \\
%%         2.5~GeV $a$ & 123 & 7.34 & 111 & 126 & 135 & 1.05 \\
%%         8~GeV $a$ & 114 & 4.99 & 106 & 128 & 136 & 0.777 \\
%%         $\eta_c$ & 119 & 6.62 & 105 & 59.4 & 147 & 0.781 \\
%%      \hline
%%       \end{tabular}
%%   \end{center}}
%%   \caption{Crystal ball fit parameters for the one dimensional background fits, described in subsection~\ref{sec:1dbackgroundmodel}.}
%%   \label{tab:1dbackgroundfitpars}
%% \end{table}





%% \section{KDE Based Modelling}
%% \label{app:kdemodelling}

%% Signal and background models based on Gaussian Kernel Density Estimation (KDE) techniques were also attempted using the \textsc{RooKeysPdf} class. When given the raw events, this was found to be too computationally slow. However, this could be resolved by collecting the events in finely binned histograms, and then calling each bin a 'pseudo event', with a value equal to the bin center, and a weight equal to the bin height. When these 'pseudo events' are given to \textsc{RooKeysPdf}, a reasonable background model is obtained. The background passing the BDT trained on the 2.5~GeV $a$ signal sample is modelled using this method in Figure~\ref{fig:kdebkgd}. The downward inflections at the edges of the distribution are caused by leakage of the kernels outside the region of interest. These can be mitigated for by reflecting the kernels at the boundaries. However, this causes some mismodelling in other regions of the distribution. A more acceptable solution to this issue is to use a wider variable range in the construction of the KDE than is required in the fit.


%% \begin{figure}[!htbp]
%%   \centering
%%   \includegraphics[width=0.65\textwidth]{figures/KDE_a02p5.png}
%%   \caption{KDE model of the complete background passing the cut on the BDT trained on the 2.5~GeV $a$ signal sample. The model was created using the methods described in Appendix~\ref{app:kdemodelling}.}
%%   \label{fig:kdebkgd}
%% \end{figure}


%% \section{2D Three Body Mass and Resonance Mass Modelling}
%% \label{app:2dmodelling}

%% Two dimensional signal and background models were considered because the three body invariant mass alone does not offer significant discrimination to discriminate between the different mass hypotheses searched for in this analysis. Three proxy variables are under consideration for the jet mass: the calorimeter jet mass; the track assisted jet mass; and the $U1(0.7)$ modified correlation function variable. The $U1(0.7)$ variable has been used in the BDT, and while it is not intended as a proxy for the mass, it has similar properties (depending on the $p_\text{T}$ and angles of the Ghost-Associated tracks), and can be observed to provide clear discrimination between each $a$ signal mass hypothesis.


%% \subsection{Signal Model}
%% \label{app:2dsignalmodel}

%% The two dimensional signal distributions before the application of the BDT are shown in Figures~\ref{fig:2dsignalcalomassprebdt},~\ref{fig:2dsignaltamassprebdt}~and~\ref{fig:2dsignalu1prebdt}, for the calorimeter mass, track assisted mass, and $U1(0.7)$ modified correlation variables, respectively. The two dimensional signal distributions after the application of the BDT are shown in Figures~\ref{fig:2dsignalcalomasspostbdt},~\ref{fig:2dsignaltamasspostbdt}~and~\ref{fig:2dsignalu1postbdt}, for the calorimeter mass, track assisted mass, and $U1(0.7)$ modified correlation variables, respectively.

%% \begin{figure}[!htbp]
%%   \centering
%%   \subfigure[0.5~GeV $a$]{\includegraphics[width=0.475\textwidth]{figures/SignalAndBackground_22Jan19/All/HM__CaloJetM__a00p5.eps}}
%%   \subfigure[2.5~GeV $a$]{\includegraphics[width=0.475\textwidth]{figures/SignalAndBackground_22Jan19/All/HM__CaloJetM__a02p5.eps}}\\
%%   \subfigure[8~GeV $a$]{\includegraphics[width=0.475\textwidth]{figures/SignalAndBackground_22Jan19/All/HM__CaloJetM__a08p0.eps}}
%%   \subfigure[$\eta_c$]{\includegraphics[width=0.475\textwidth]{figures/SignalAndBackground_22Jan19/All/HM__CaloJetM__etac.eps}}
%%   \caption{Two dimensional signal distributions, using the calorimeter mass as the resonance mass proxy, after the full selection, not including the BDT.}
%%   \label{fig:2dsignalcalomassprebdt}
%% \end{figure}


%% \begin{figure}[!htbp]
%%   \centering
%%   \subfigure[0.5~GeV $a$]{\includegraphics[width=0.475\textwidth]{figures/SignalAndBackground_22Jan19/All/HM__GhostTracks_mTA__a00p5.eps}}
%%   \subfigure[2.5~GeV $a$]{\includegraphics[width=0.475\textwidth]{figures/SignalAndBackground_22Jan19/All/HM__GhostTracks_mTA__a02p5.eps}}\\
%%   \subfigure[8~GeV $a$]{\includegraphics[width=0.475\textwidth]{figures/SignalAndBackground_22Jan19/All/HM__GhostTracks_mTA__a08p0.eps}}
%%   \subfigure[$\eta_c$]{\includegraphics[width=0.475\textwidth]{figures/SignalAndBackground_22Jan19/All/HM__GhostTracks_mTA__etac.eps}}
%%   \caption{Two dimensional signal distributions, using the track assisted mass as the resonance mass proxy, after the full selection, not including the BDT.}
%%   \label{fig:2dsignaltamassprebdt}
%% \end{figure}


%% \begin{figure}[!htbp]
%%   \centering
%%   \subfigure[0.5~GeV $a$]{\includegraphics[width=0.475\textwidth]{figures/SignalAndBackground_22Jan19/All/Fit_HM_GhostTracks_U1_0p7__a00p5.eps}}
%%   \subfigure[2.5~GeV $a$]{\includegraphics[width=0.475\textwidth]{figures/SignalAndBackground_22Jan19/All/Fit_HM_GhostTracks_U1_0p7__a02p5.eps}}\\
%%   \subfigure[8~GeV $a$]{\includegraphics[width=0.475\textwidth]{figures/SignalAndBackground_22Jan19/All/Fit_HM_GhostTracks_U1_0p7__a08p0.eps}}
%%   \subfigure[$\eta_c$]{\includegraphics[width=0.475\textwidth]{figures/SignalAndBackground_22Jan19/All/Fit_HM_GhostTracks_U1_0p7__etac.eps}}
%%   \caption{Two dimensional signal distributions, using $U1(0.7)$ as the resonance mass proxy, after the full selection, not including the BDT.}
%%   \label{fig:2dsignalu1prebdt}
%% \end{figure}

%% \begin{figure}[!htbp]
%%   \centering
%%   \subfigure[0.5~GeV $a$]{\includegraphics[width=0.475\textwidth]{figures/SignalAndBackground_22Jan19/PassedBDT_a00p5/HM__CaloJetM__a00p5.eps}}
%%   \subfigure[2.5~GeV $a$]{\includegraphics[width=0.475\textwidth]{figures/SignalAndBackground_22Jan19/PassedBDT_a02p5/HM__CaloJetM__a02p5.eps}}\\
%%   \subfigure[8~GeV $a$]{\includegraphics[width=0.475\textwidth]{figures/SignalAndBackground_22Jan19/PassedBDT_a08p0/HM__CaloJetM__a08p0.eps}}
%%   \subfigure[$\eta_c$]{\includegraphics[width=0.475\textwidth]{figures/SignalAndBackground_22Jan19/PassedBDT_etac/HM__CaloJetM__etac.eps}}
%%   \caption{Two dimensional signal distributions, using the calorimeter mass as the resonance mass proxy, after the full selection, including the BDT trained on the displayed signal sample.}
%%   \label{fig:2dsignalcalomasspostbdt}
%% \end{figure}


%% \begin{figure}[!htbp]
%%   \centering
%%   \subfigure[0.5~GeV $a$]{\includegraphics[width=0.475\textwidth]{figures/SignalAndBackground_22Jan19/PassedBDT_a00p5/HM__GhostTracks_mTA__a00p5.eps}}
%%   \subfigure[2.5~GeV $a$]{\includegraphics[width=0.475\textwidth]{figures/SignalAndBackground_22Jan19/PassedBDT_a02p5/HM__GhostTracks_mTA__a02p5.eps}}\\
%%   \subfigure[8~GeV $a$]{\includegraphics[width=0.475\textwidth]{figures/SignalAndBackground_22Jan19/PassedBDT_a08p0/HM__GhostTracks_mTA__a08p0.eps}}
%%   \subfigure[$\eta_c$]{\includegraphics[width=0.475\textwidth]{figures/SignalAndBackground_22Jan19/PassedBDT_etac/HM__GhostTracks_mTA__etac.eps}}
%%   \caption{Two dimensional signal distributions, using the track assisted mass as the resonance mass proxy, after the full selection, including the BDT trained on the displayed signal sample.}
%%   \label{fig:2dsignaltamasspostbdt}
%% \end{figure}


%% \begin{figure}[!htbp]
%%   \centering
%%   \subfigure[0.5~GeV $a$]{\includegraphics[width=0.475\textwidth]{figures/SignalAndBackground_22Jan19/PassedBDT_a00p5/Fit_HM_GhostTracks_U1_0p7__a00p5.eps}}
%%   \subfigure[2.5~GeV $a$]{\includegraphics[width=0.475\textwidth]{figures/SignalAndBackground_22Jan19/PassedBDT_a02p5/Fit_HM_GhostTracks_U1_0p7__a02p5.eps}}\\
%%   \subfigure[8~GeV $a$]{\includegraphics[width=0.475\textwidth]{figures/SignalAndBackground_22Jan19/PassedBDT_a08p0/Fit_HM_GhostTracks_U1_0p7__a08p0.eps}}
%%   \subfigure[$\eta_c$]{\includegraphics[width=0.475\textwidth]{figures/SignalAndBackground_22Jan19/PassedBDT_etac/Fit_HM_GhostTracks_U1_0p7__etac.eps}}
%%   \caption{Two dimensional signal distributions, using $U1(0.7)$ as the resonance mass proxy, after the full selection, including the BDT trained on the displayed signal sample.}
%%   \label{fig:2dsignalu1postbdt}
%% \end{figure}


%% \subsection{Background Model}
%% \label{app:2dbackgroundmodel}

%% The two dimensional background distributions before the application of the BDT are shown in Figures~\ref{fig:2dbackgroundcalomassprebdt},~\ref{fig:2dbackgroundtamassprebdt}~and~\ref{fig:2dbackgroundu1prebdt}, for the calorimeter mass, track assisted mass, and $U1(0.7)$ modified correlation variables, respectively. The two dimensional background distributions after the application of the BDT trained on the 0.5~GeV $a$ are shown in Figures~\ref{fig:2dbackgroundcalomassposta00p5bdt},~\ref{fig:2dbackgroundtamassposta00p5bdt}~and~\ref{fig:2dbackgroundu1posta00p5bdt}, for the calorimeter mass, track assisted mass, and $U1(0.7)$ modified correlation variables, respectively. The two dimensional background distributions after the application of the BDT trained on the 2.5~GeV $a$ are shown in Figures~\ref{fig:2dbackgroundcalomassposta02p5bdt},~\ref{fig:2dbackgroundtamassposta02p5bdt}~and~\ref{fig:2dbackgroundu1posta02p5bdt}, for the calorimeter mass, track assisted mass, and $U1(0.7)$ modified correlation variables, respectively. The two dimensional background distributions after the application of the BDT trained on the 8~GeV $a$ are shown in Figures~\ref{fig:2dbackgroundcalomassposta08p0bdt},~\ref{fig:2dbackgroundtamassposta08p0bdt}~and~\ref{fig:2dbackgroundu1posta08p0bdt}, for the calorimeter mass, track assisted mass, and $U1(0.7)$ modified correlation variables, respectively. The two dimensional background distributions after the application of the BDT trained on the $eta_c$ are shown in Figures~\ref{fig:2dbackgroundcalomasspostetacbdt},~\ref{fig:2dbackgroundtamasspostetacbdt}~and~\ref{fig:2dbackgroundu1postetacbdt}, for the calorimeter mass, track assisted mass, and $U1(0.7)$ modified correlation variables, respectively.

%% \begin{figure}[!htbp]
%%   \centering
%%   \subfigure[0.5~GeV $a$]{\includegraphics[width=0.475\textwidth]{figures/SignalAndBackground_22Jan19/All/HM__CaloJetM__a00p5.eps}}
%%   \subfigure[2.5~GeV $a$]{\includegraphics[width=0.475\textwidth]{figures/SignalAndBackground_22Jan19/All/HM__CaloJetM__a02p5.eps}}\\
%%   \subfigure[8~GeV $a$]{\includegraphics[width=0.475\textwidth]{figures/SignalAndBackground_22Jan19/All/HM__CaloJetM__a08p0.eps}}
%%   \subfigure[$\eta_c$]{\includegraphics[width=0.475\textwidth]{figures/SignalAndBackground_22Jan19/All/HM__CaloJetM__etac.eps}}
%%   \caption{Two dimensional background distributions, using the calorimeter mass as the resonance mass proxy, after the full selection, not including the BDT.}
%%   \label{fig:2dbackgroundcalomassprebdt}
%% \end{figure}


%% \begin{figure}[!htbp]
%%   \centering
%%   \subfigure[0.5~GeV $a$]{\includegraphics[width=0.475\textwidth]{figures/SignalAndBackground_22Jan19/All/HM__GhostTracks_mTA__a00p5.eps}}
%%   \subfigure[2.5~GeV $a$]{\includegraphics[width=0.475\textwidth]{figures/SignalAndBackground_22Jan19/All/HM__GhostTracks_mTA__a02p5.eps}}\\
%%   \subfigure[8~GeV $a$]{\includegraphics[width=0.475\textwidth]{figures/SignalAndBackground_22Jan19/All/HM__GhostTracks_mTA__a08p0.eps}}
%%   \subfigure[$\eta_c$]{\includegraphics[width=0.475\textwidth]{figures/SignalAndBackground_22Jan19/All/HM__GhostTracks_mTA__etac.eps}}
%%   \caption{Two dimensional background distributions, using the track assisted mass as the resonance mass proxy, after the full selection, not including the BDT.}
%%   \label{fig:2dbackgroundtamassprebdt}
%% \end{figure}


%% \begin{figure}[!htbp]
%%   \centering
%%   \subfigure[0.5~GeV $a$]{\includegraphics[width=0.475\textwidth]{figures/SignalAndBackground_22Jan19/All/Fit_HM_GhostTracks_U1_0p7__a00p5.eps}}
%%   \subfigure[2.5~GeV $a$]{\includegraphics[width=0.475\textwidth]{figures/SignalAndBackground_22Jan19/All/Fit_HM_GhostTracks_U1_0p7__a02p5.eps}}\\
%%   \subfigure[8~GeV $a$]{\includegraphics[width=0.475\textwidth]{figures/SignalAndBackground_22Jan19/All/Fit_HM_GhostTracks_U1_0p7__a08p0.eps}}
%%   \subfigure[$\eta_c$]{\includegraphics[width=0.475\textwidth]{figures/SignalAndBackground_22Jan19/All/Fit_HM_GhostTracks_U1_0p7__etac.eps}}
%%   \caption{Two dimensional background distributions, using $U1(0.7)$ as the resonance mass proxy, after the full selection, not including the BDT.}
%%   \label{fig:2dbackgroundu1prebdt}
%% \end{figure}

%% \begin{figure}[!htbp]
%%   \centering
%%   \subfigure[0.5~GeV $a$]{\includegraphics[width=0.475\textwidth]{figures/SignalAndBackground_22Jan19/PassedBDT_a00p5/HM__CaloJetM__a00p5.eps}}
%%   \subfigure[2.5~GeV $a$]{\includegraphics[width=0.475\textwidth]{figures/SignalAndBackground_22Jan19/PassedBDT_a00p5/HM__CaloJetM__a02p5.eps}}\\
%%   \subfigure[8~GeV $a$]{\includegraphics[width=0.475\textwidth]{figures/SignalAndBackground_22Jan19/PassedBDT_a00p5/HM__CaloJetM__a08p0.eps}}
%%   \subfigure[$\eta_c$]{\includegraphics[width=0.475\textwidth]{figures/SignalAndBackground_22Jan19/PassedBDT_a00p5/HM__CaloJetM__etac.eps}}
%%   \caption{Two dimensional background distributions, using the calorimeter mass as the resonance mass proxy, after the full selection, including the BDT trained on the 0.5~GeV $a$ signal sample.}
%%   \label{fig:2dbackgroundcalomassposta00p5bdt}
%% \end{figure}


%% \begin{figure}[!htbp]
%%   \centering
%%   \subfigure[0.5~GeV $a$]{\includegraphics[width=0.475\textwidth]{figures/SignalAndBackground_22Jan19/PassedBDT_a00p5/HM__GhostTracks_mTA__a00p5.eps}}
%%   \subfigure[2.5~GeV $a$]{\includegraphics[width=0.475\textwidth]{figures/SignalAndBackground_22Jan19/PassedBDT_a00p5/HM__GhostTracks_mTA__a02p5.eps}}\\
%%   \subfigure[8~GeV $a$]{\includegraphics[width=0.475\textwidth]{figures/SignalAndBackground_22Jan19/PassedBDT_a00p5/HM__GhostTracks_mTA__a08p0.eps}}
%%   \subfigure[$\eta_c$]{\includegraphics[width=0.475\textwidth]{figures/SignalAndBackground_22Jan19/PassedBDT_a00p5/HM__GhostTracks_mTA__etac.eps}}
%%   \caption{Two dimensional background distributions, using the track assisted mass as the resonance mass proxy, after the full selection, including the BDT trained on the 0.5~GeV $a$ signal sample.}
%%   \label{fig:2dbackgroundtamassposta00p5bdt}
%% \end{figure}


%% \begin{figure}[!htbp]
%%   \centering
%%   \subfigure[0.5~GeV $a$]{\includegraphics[width=0.475\textwidth]{figures/SignalAndBackground_22Jan19/PassedBDT_a00p5/Fit_HM_GhostTracks_U1_0p7__a00p5.eps}}
%%   \subfigure[2.5~GeV $a$]{\includegraphics[width=0.475\textwidth]{figures/SignalAndBackground_22Jan19/PassedBDT_a00p5/Fit_HM_GhostTracks_U1_0p7__a02p5.eps}}\\
%%   \subfigure[8~GeV $a$]{\includegraphics[width=0.475\textwidth]{figures/SignalAndBackground_22Jan19/PassedBDT_a00p5/Fit_HM_GhostTracks_U1_0p7__a08p0.eps}}
%%   \subfigure[$\eta_c$]{\includegraphics[width=0.475\textwidth]{figures/SignalAndBackground_22Jan19/PassedBDT_a00p5/Fit_HM_GhostTracks_U1_0p7__etac.eps}}
%%   \caption{Two dimensional background distributions, using $U1(0.7)$ as the resonance mass proxy, after the full selection, including the BDT trained on the 0.5~GeV $a$ signal sample.}
%%   \label{fig:2dbackgroundu1posta00p5bdt}
%% \end{figure}

%% \begin{figure}[!htbp]
%%   \centering
%%   \subfigure[0.5~GeV $a$]{\includegraphics[width=0.475\textwidth]{figures/SignalAndBackground_22Jan19/PassedBDT_a02p5/HM__CaloJetM__a00p5.eps}}
%%   \subfigure[2.5~GeV $a$]{\includegraphics[width=0.475\textwidth]{figures/SignalAndBackground_22Jan19/PassedBDT_a02p5/HM__CaloJetM__a02p5.eps}}\\
%%   \subfigure[8~GeV $a$]{\includegraphics[width=0.475\textwidth]{figures/SignalAndBackground_22Jan19/PassedBDT_a02p5/HM__CaloJetM__a08p0.eps}}
%%   \subfigure[$\eta_c$]{\includegraphics[width=0.475\textwidth]{figures/SignalAndBackground_22Jan19/PassedBDT_a02p5/HM__CaloJetM__etac.eps}}
%%   \caption{Two dimensional background distributions, using the calorimeter mass as the resonance mass proxy, after the full selection, including the BDT trained on the 2.5~GeV $a$ signal sample.}
%%   \label{fig:2dbackgroundcalomassposta02p5bdt}
%% \end{figure}


%% \begin{figure}[!htbp]
%%   \centering
%%   \subfigure[0.5~GeV $a$]{\includegraphics[width=0.475\textwidth]{figures/SignalAndBackground_22Jan19/PassedBDT_a02p5/HM__GhostTracks_mTA__a00p5.eps}}
%%   \subfigure[2.5~GeV $a$]{\includegraphics[width=0.475\textwidth]{figures/SignalAndBackground_22Jan19/PassedBDT_a02p5/HM__GhostTracks_mTA__a02p5.eps}}\\
%%   \subfigure[8~GeV $a$]{\includegraphics[width=0.475\textwidth]{figures/SignalAndBackground_22Jan19/PassedBDT_a02p5/HM__GhostTracks_mTA__a08p0.eps}}
%%   \subfigure[$\eta_c$]{\includegraphics[width=0.475\textwidth]{figures/SignalAndBackground_22Jan19/PassedBDT_a02p5/HM__GhostTracks_mTA__etac.eps}}
%%   \caption{Two dimensional background distributions, using the track assisted mass as the resonance mass proxy, after the full selection, including the BDT trained on the 2.5~GeV $a$ signal sample.}
%%   \label{fig:2dbackgroundtamassposta02p5bdt}
%% \end{figure}


%% \begin{figure}[!htbp]
%%   \centering
%%   \subfigure[0.5~GeV $a$]{\includegraphics[width=0.475\textwidth]{figures/SignalAndBackground_22Jan19/PassedBDT_a02p5/Fit_HM_GhostTracks_U1_0p7__a00p5.eps}}
%%   \subfigure[2.5~GeV $a$]{\includegraphics[width=0.475\textwidth]{figures/SignalAndBackground_22Jan19/PassedBDT_a02p5/Fit_HM_GhostTracks_U1_0p7__a02p5.eps}}\\
%%   \subfigure[8~GeV $a$]{\includegraphics[width=0.475\textwidth]{figures/SignalAndBackground_22Jan19/PassedBDT_a02p5/Fit_HM_GhostTracks_U1_0p7__a08p0.eps}}
%%   \subfigure[$\eta_c$]{\includegraphics[width=0.475\textwidth]{figures/SignalAndBackground_22Jan19/PassedBDT_a02p5/Fit_HM_GhostTracks_U1_0p7__etac.eps}}
%%   \caption{Two dimensional background distributions, using $U1(0.7)$ as the resonance mass proxy, after the full selection, including the BDT trained on the 2.5~GeV $a$ signal sample.}
%%   \label{fig:2dbackgroundu1posta02p5bdt}
%% \end{figure}

%% \begin{figure}[!htbp]
%%   \centering
%%   \subfigure[0.5~GeV $a$]{\includegraphics[width=0.475\textwidth]{figures/SignalAndBackground_22Jan19/PassedBDT_a08p0/HM__CaloJetM__a00p5.eps}}
%%   \subfigure[2.5~GeV $a$]{\includegraphics[width=0.475\textwidth]{figures/SignalAndBackground_22Jan19/PassedBDT_a08p0/HM__CaloJetM__a02p5.eps}}\\
%%   \subfigure[8~GeV $a$]{\includegraphics[width=0.475\textwidth]{figures/SignalAndBackground_22Jan19/PassedBDT_a08p0/HM__CaloJetM__a08p0.eps}}
%%   \subfigure[$\eta_c$]{\includegraphics[width=0.475\textwidth]{figures/SignalAndBackground_22Jan19/PassedBDT_a08p0/HM__CaloJetM__etac.eps}}
%%   \caption{Two dimensional background distributions, using the calorimeter mass as the resonance mass proxy, after the full selection, including the BDT trained on the 8~GeV $a$ signal sample.}
%%   \label{fig:2dbackgroundcalomassposta08p0bdt}
%% \end{figure}


%% \begin{figure}[!htbp]
%%   \centering
%%   \subfigure[0.5~GeV $a$]{\includegraphics[width=0.475\textwidth]{figures/SignalAndBackground_22Jan19/PassedBDT_a08p0/HM__GhostTracks_mTA__a00p5.eps}}
%%   \subfigure[2.5~GeV $a$]{\includegraphics[width=0.475\textwidth]{figures/SignalAndBackground_22Jan19/PassedBDT_a08p0/HM__GhostTracks_mTA__a02p5.eps}}\\
%%   \subfigure[8~GeV $a$]{\includegraphics[width=0.475\textwidth]{figures/SignalAndBackground_22Jan19/PassedBDT_a08p0/HM__GhostTracks_mTA__a08p0.eps}}
%%   \subfigure[$\eta_c$]{\includegraphics[width=0.475\textwidth]{figures/SignalAndBackground_22Jan19/PassedBDT_a08p0/HM__GhostTracks_mTA__etac.eps}}
%%   \caption{Two dimensional background distributions, using the track assisted mass as the resonance mass proxy, after the full selection, including the BDT trained on the 8~GeV $a$ signal sample.}
%%   \label{fig:2dbackgroundtamassposta08p0bdt}
%% \end{figure}


%% \begin{figure}[!htbp]
%%   \centering
%%   \subfigure[0.5~GeV $a$]{\includegraphics[width=0.475\textwidth]{figures/SignalAndBackground_22Jan19/PassedBDT_a08p0/Fit_HM_GhostTracks_U1_0p7__a00p5.eps}}
%%   \subfigure[2.5~GeV $a$]{\includegraphics[width=0.475\textwidth]{figures/SignalAndBackground_22Jan19/PassedBDT_a08p0/Fit_HM_GhostTracks_U1_0p7__a02p5.eps}}\\
%%   \subfigure[8~GeV $a$]{\includegraphics[width=0.475\textwidth]{figures/SignalAndBackground_22Jan19/PassedBDT_a08p0/Fit_HM_GhostTracks_U1_0p7__a08p0.eps}}
%%   \subfigure[$\eta_c$]{\includegraphics[width=0.475\textwidth]{figures/SignalAndBackground_22Jan19/PassedBDT_a08p0/Fit_HM_GhostTracks_U1_0p7__etac.eps}}
%%   \caption{Two dimensional background distributions, using $U1(0.7)$ as the resonance mass proxy, after the full selection, including the BDT trained on the 8~GeV $a$ signal sample.}
%%   \label{fig:2dbackgroundu1posta08p0bdt}
%% \end{figure}

%% \begin{figure}[!htbp]
%%   \centering
%%   \subfigure[0.5~GeV $a$]{\includegraphics[width=0.475\textwidth]{figures/SignalAndBackground_22Jan19/PassedBDT_etac/HM__CaloJetM__a00p5.eps}}
%%   \subfigure[2.5~GeV $a$]{\includegraphics[width=0.475\textwidth]{figures/SignalAndBackground_22Jan19/PassedBDT_etac/HM__CaloJetM__a02p5.eps}}\\
%%   \subfigure[8~GeV $a$]{\includegraphics[width=0.475\textwidth]{figures/SignalAndBackground_22Jan19/PassedBDT_etac/HM__CaloJetM__a08p0.eps}}
%%   \subfigure[$\eta_c$]{\includegraphics[width=0.475\textwidth]{figures/SignalAndBackground_22Jan19/PassedBDT_etac/HM__CaloJetM__etac.eps}}
%%   \caption{Two dimensional background distributions, using the calorimeter mass as the resonance mass proxy, after the full selection, including the BDT trained on the $\eta_c$ signal sample.}
%%   \label{fig:2dbackgroundcalomasspostetacbdt}
%% \end{figure}


%% \begin{figure}[!htbp]
%%   \centering
%%   \subfigure[0.5~GeV $a$]{\includegraphics[width=0.475\textwidth]{figures/SignalAndBackground_22Jan19/PassedBDT_etac/HM__GhostTracks_mTA__a00p5.eps}}
%%   \subfigure[2.5~GeV $a$]{\includegraphics[width=0.475\textwidth]{figures/SignalAndBackground_22Jan19/PassedBDT_etac/HM__GhostTracks_mTA__a02p5.eps}}\\
%%   \subfigure[8~GeV $a$]{\includegraphics[width=0.475\textwidth]{figures/SignalAndBackground_22Jan19/PassedBDT_etac/HM__GhostTracks_mTA__a08p0.eps}}
%%   \subfigure[$\eta_c$]{\includegraphics[width=0.475\textwidth]{figures/SignalAndBackground_22Jan19/PassedBDT_etac/HM__GhostTracks_mTA__etac.eps}}
%%   \caption{Two dimensional background distributions, using the track assisted mass as the resonance mass proxy, after the full selection, including the BDT trained on the $\eta_c$ signal sample.}
%%   \label{fig:2dbackgroundtamasspostetacbdt}
%% \end{figure}


%% \begin{figure}[!htbp]
%%   \centering
%%   \subfigure[0.5~GeV $a$]{\includegraphics[width=0.475\textwidth]{figures/SignalAndBackground_22Jan19/PassedBDT_etac/Fit_HM_GhostTracks_U1_0p7__a00p5.eps}}
%%   \subfigure[2.5~GeV $a$]{\includegraphics[width=0.475\textwidth]{figures/SignalAndBackground_22Jan19/PassedBDT_etac/Fit_HM_GhostTracks_U1_0p7__a02p5.eps}}\\
%%   \subfigure[8~GeV $a$]{\includegraphics[width=0.475\textwidth]{figures/SignalAndBackground_22Jan19/PassedBDT_etac/Fit_HM_GhostTracks_U1_0p7__a08p0.eps}}
%%   \subfigure[$\eta_c$]{\includegraphics[width=0.475\textwidth]{figures/SignalAndBackground_22Jan19/PassedBDT_etac/Fit_HM_GhostTracks_U1_0p7__etac.eps}}
%%   \caption{Two dimensional background distributions, using $U1(0.7)$ as the resonance mass proxy, after the full selection, including the BDT trained on the $\eta_c$ signal sample.}
%%   \label{fig:2dbackgroundu1postetacbdt}
%% \end{figure}


%% \section{Background Model Study}
%% \label{app:6orderpolybkgdmodel}

%% A six-order polynomial was also fit to the background, in order to confirm that the background model succeeds for reasons beyond the six free parameters it is afforded. The polynomial did not model the background accurately, as can be seen in Figure~\ref{fig:6orderpolybkgdmodel}. This confirms that the model is not just accurate due to its six free parameters.

%% \begin{figure}[!htbp]
%%   \centering
%%   \includegraphics[width=0.475\textwidth]{figures/FitPlots_30Nov18/Bkgd/a02p5_6thOrderPoly.png}
%%   \caption{One dimensional three body mass distributions for the total background, after the full selection, including the BDT trained on the 2.5~GeV $a$ signal sample. The distribution is fit with a sixth order polynomial.}
%%   \label{fig:6orderpolybkgdmodel}
%% \end{figure}


%% \section{Toy Systematic Uncertainty Studies}
%% \label{app:toysystematics}

%% Toy systematics were implemented in the fitting code, in preparation for the addition of the real systematics. This Appendix described the results of these toys in the context of the 2.5~GeV $a$ signal sample. The dataset used was the Asimov dataset described in Section~\ref{sec:asimovfits}.

%% As default, the background normalisation is left free in the fit, resulting in $\Delta\mu = 3.8\%$ and $\Delta\mu_{BKGD} = 0.10\%$, where $\Delta\mu$ is the uncertainty on the parameter of interest, and $\Delta\mu_{BKGD}$ is the uncertainty on the fitted background normalisation. When the background normalisation is fixed to its nominal value, the uncertainty on the parameter of interest reduces to: $\Delta\mu = 3.1\%$. Various fractional Gaussian uncertainties on the background normalisation are trialled, and the resulting uncertainties on $\Delta\mu$ and $\Delta\mu_{BKGD}$ are given in Table~\ref{tab:toysysbkgd}. It can be seen that when the background normalisation uncertainty is large, it gets constrained by the data, resulting in a $\Delta\mu$ compatible with the free background normalisation fit. Conversely, when the background uncertainty is much smaller than the constrain possible from data, the nuisance parameter does not become constrained, and $\Delta\mu$ remains compatible with the fixed background normalisation fit result.

%% \begin{table}[!htbp]{\footnotesize\renewcommand{\arraystretch}{1.2}
%%     \begin{center}
%%     \begin{tabular}{|l|l|l|}
%%       \hline
%%       NP Magnitude & Post-Fit $\sigma(\alpha_{BKGD})$ & $\Delta\mu$ \\
%%       \hline
%%       $10\%$     & $1.0\%$ & $3.8\%$ \\
%%       $1\%$      & $10\%$ & $3.8\%$ \\
%%       $0.1\%$    & $72\%$ & $3.5\%$ \\
%%       $0.01\%$   & $100\%$  & $3.2\%$ \\
%%       $0.001\%$  & $100\%$  & $3.1\%$ \\
%%       \hline
%%     \end{tabular}
%%     \caption{Uncertainties on $\Delta\mu$ and $\Delta\mu_{BKGD}$ in fit to 2.5~GeV $a$ signal sample, when the background is constrained using a Gaussian nuisance parameter of various magnitudes.}
%%     \label{tab:toysysbkgd}
%%   \end{center}}
%% \end{table}

%% Adding a luminosity uncertainty of 2\% increases $\Delta\mu$ to $4.3\%$, and $\Delta\mu_{BKGD}$ to $2.0\%$, while remaining unconstrained by the data at $\sigma(\alpha_\text{Lumi})=100\%$.

%% Adding a $\pm 5\%$ tilt shape uncertainty on the background results in $\Delta\mu = 5.5\%$ and $\Delta\mu_{BKGD} = 0.12\%$, with a prior constrained to 5.6\%. This shows that the background data statistics is capable of constraining a sizeable shape systematic.

%% Finally, the impact of signal normalisation systematics on the final result are considered. The results are in Table~\ref{tab:toysyssignal}.

%% \begin{table}[!htbp]{\footnotesize\renewcommand{\arraystretch}{1.2}
%%     \begin{center}
%%     \begin{tabular}{|l|l|l|}
%%       \hline
%%       NP Magnitude & Post-Fit $\sigma(\alpha_{SIG})$ & $\Delta\mu$ \\
%%       \hline
%%       $100\%$   & $48.8\%$ & $41\%$  \\
%%       $10\%$    & $100\%$  & $11\%$  \\
%%       $1\%$     & $100\%$  & $4.0\%$  \\
%%       $0.1\%$   & $100\%$  & $3.8\%$  \\
%%       $0.01\%$  & $100\%$  & $3.8\%$  \\
%%       \hline
%%     \end{tabular}
%%     \caption{Uncertainties on $\Delta\mu$ in fits to 2.5~GeV $a$ signal sample, when the background normalisation is fixed, and the signal is assigned an uncertainty using a Gaussian nuisance parameter of various magnitudes.}
%%     \label{tab:toysyssignal}
%%   \end{center}}
%% \end{table}


%% \section{Analytic Cut and Count Limit Comparison}
%% \label{app:cutandcountcomparison}

%% Old BDT used for event numbers in this subsection.

%% In order to validate the conclusion of the \textsc{RooFit} shape-based limit, an analytic cut and count limit is set by considering the 2.5~GeV $a$ signal sample in the region $100 < m_{\ell^+\ell^-a} < 150$~GeV. The background and signal expectations and fixed in this estimate. The number of signal and background events are $\approx 4800$ and $\approx 31000$ respectively (as smaller $m_{\ell^+\ell^-a$ range). This gives a limit of: $BR(H\to Za)\times BR(a\to jet) \lesssim 7.3\%$. This can be compared with the limit taken from the fit to the $m_{l^+l^-a$ mass distribution with fixed background normalisation, which is 5.6\%. The difference is due to the shape information in the fit.

%% The remainder of this subsection outlines a derivation of the formulae used in the analytic calculation of the expected limit. Assuming a counting experiment in region with fixed background and signal expectations, we have

%% $\mathcal{L}(\mu;\ N_{obs},\ N_{S},\ N_{B})=\frac{(\mu N_{S}+N_{B})^{N_{obs}}e^{-(\mu N_{S}+N_{B})}}{N_{obs}!}$.

%% For an expected limit calculation $N_{obs}=N_{B}$, meaning

%% $\mathcal{L}(\mu;\ N_{S},\ N_{B})=\frac{(\mu N_{S}+N_{B})^{N_{B}}e^{-(\mu N_{S}+N_{B})}}{N_{B}!}$.

%% Then, using Wilk's theorem, we have

%% $\chi^2(\mu;\ N_{S},\ N_{B})=-2ln\frac{\mathcal{L}(\mu;\ N_{S},\ N_{B})}{\mathcal{L}(0;\ N_{S},\ N_{B})}=2\mu N_S-2N_Bln(1+\mu N_S/N_B)$.

%% By Taylor expanding in $\mu N_{S}/N_{B}$, we then have

%% $\chi^2(\mu;\ N_{S},\ N_{B})=2\mu N_S-2N_B(\mu N_S/N_B-(\mu N_S/N_B)^2/2)=\frac{\mu^2N_S^2}{N_B}$,

%% $\chi^2(\mu;\ N_{S},\ N_{B})=\frac{\mu^2N_S^2}{N_B}$.

%% Giving the $95\%$ upper limit as

%% $\mu_{95\%}=\sqrt{\frac{\chi^2_{95\%}N_B}{N_S^2}}$,

%% $\chi^2_{95\%}=3.84$: $\mu_{95\%}=\sqrt{\frac{3.84N_B}{N_S^2}}$.
